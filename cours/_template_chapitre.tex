% Template pour les chapitres du cours ML
% À copier et adapter pour chaque chapitre

\documentclass[11pt,a4paper]{article}

% ===== PACKAGES =====
\usepackage[utf8]{inputenc}
\usepackage[T1]{fontenc}
\usepackage[french]{babel}
\usepackage{lmodern}

% Mathématiques
\usepackage{amsmath, amssymb, amsthm}
\usepackage{mathtools}

% Mise en page
\usepackage[margin=2.5cm]{geometry}
\usepackage{parskip}
\usepackage{setspace}
\setstretch{1.15}

% Graphiques et couleurs
\usepackage{graphicx}
\usepackage{xcolor}
\usepackage{tikz}
\usetikzlibrary{arrows.meta, positioning, shapes.geometric}

% Tableaux
\usepackage{booktabs}
\usepackage{longtable}
\usepackage{multirow}
\usepackage{tabularx}
\usepackage{colortbl}

% Code et algorithmes
\usepackage{listings}
\usepackage{algorithm}
\usepackage{algorithmic}

% Hyperliens
\usepackage{hyperref}
\hypersetup{
    colorlinks=true,
    linkcolor=blue,
    filecolor=magenta,
    urlcolor=cyan,
    citecolor=green,
    pdftitle={Chapitre XX - Titre},
    pdfauthor={Cours ML},
}

% Boxes colorées
\usepackage{tcolorbox}
\tcbuselibrary{skins, breakable}

% En-têtes et pieds de page
\usepackage{fancyhdr}
\pagestyle{fancy}
\fancyhf{}
\fancyhead[L]{\small Chapitre XX - Titre du Chapitre}
\fancyhead[R]{\small Cours Machine Learning}
\fancyfoot[C]{\thepage}

% ===== CONFIGURATION LISTINGS (code Python) =====
\definecolor{codegreen}{rgb}{0,0.6,0}
\definecolor{codegray}{rgb}{0.5,0.5,0.5}
\definecolor{codepurple}{rgb}{0.58,0,0.82}
\definecolor{backcolour}{rgb}{0.95,0.95,0.92}

\lstdefinestyle{pythonstyle}{
    language=Python,
    backgroundcolor=\color{backcolour},
    commentstyle=\color{codegreen},
    keywordstyle=\color{blue},
    numberstyle=\tiny\color{codegray},
    stringstyle=\color{codepurple},
    basicstyle=\ttfamily\small,
    breakatwhitespace=false,
    breaklines=true,
    captionpos=b,
    keepspaces=true,
    numbers=left,
    numbersep=5pt,
    showspaces=false,
    showstringspaces=false,
    showtabs=false,
    tabsize=4,
    frame=single,
    rulecolor=\color{black}
}
\lstset{style=pythonstyle}

% ===== CONFIGURATION TCOLORBOX =====
% Box pour définitions
\newtcolorbox{definition}[1]{
    colback=blue!5!white,
    colframe=blue!75!black,
    fonttitle=\bfseries,
    title=Définition: #1,
    breakable
}

% Box pour théorèmes
\newtcolorbox{theoreme}[1]{
    colback=green!5!white,
    colframe=green!75!black,
    fonttitle=\bfseries,
    title=Théorème: #1,
    breakable
}

% Box pour exemples
\newtcolorbox{exemple}[1]{
    colback=orange!5!white,
    colframe=orange!75!black,
    fonttitle=\bfseries,
    title=Exemple: #1,
    breakable
}

% Box pour attention/warning
\newtcolorbox{attention}{
    colback=red!5!white,
    colframe=red!75!black,
    fonttitle=\bfseries,
    title=⚠️ Attention,
    breakable
}

% Box pour astuce/tips
\newtcolorbox{astuce}{
    colback=yellow!10!white,
    colframe=yellow!75!black,
    fonttitle=\bfseries,
    title=💡 Astuce,
    breakable
}

% ===== COMMANDES PERSONNALISÉES =====
\newcommand{\vect}[1]{\mathbf{#1}}  % Vecteur
\newcommand{\mat}[1]{\mathbf{#1}}   % Matrice
\newcommand{\R}{\mathbb{R}}         % Réels
\newcommand{\N}{\mathbb{N}}         % Naturels
\newcommand{\argmin}{\operatorname{argmin}}
\newcommand{\argmax}{\operatorname{argmax}}

% ===== DÉBUT DU DOCUMENT =====
\begin{document}

% ===== PAGE DE TITRE =====
\begin{titlepage}
    \centering
    \vspace*{2cm}

    {\Huge\bfseries Cours Machine Learning}\\[0.5cm]

    \vspace{1cm}

    {\LARGE Chapitre XX}\\[0.3cm]
    {\LARGE\bfseries Titre du Chapitre}\\[2cm]

    \vfill

    {\large
    \textbf{Objectifs d'apprentissage :}\\[0.5cm]
    \begin{itemize}
        \item Objectif 1
        \item Objectif 2
        \item Objectif 3
    \end{itemize}
    }

    \vfill

    {\large
    \textbf{Prérequis :} Chapitres XX, YY\\[0.3cm]
    \textbf{Durée estimée :} X-Y heures\\[0.3cm]
    \textbf{Notebooks :} \texttt{XX\_demo\_*.ipynb}
    }

    \vfill

    {\large Cours ML - Sandbox-ML\\
    Version 1.0 - 2026}
\end{titlepage}

% ===== TABLE DES MATIÈRES =====
\tableofcontents
\newpage

% ===== SECTION 1: MOTIVATION =====
\section{Motivation}

% Commencer par un problème concret qui illustre pourquoi ce chapitre est important
% Exemple : Pour un chapitre sur la régression, présenter un problème de prédiction de prix

\begin{exemple}{Problème illustratif}
Description d'un problème concret qui va motiver l'étude du contenu du chapitre.
\end{exemple}

% ===== SECTION 2: FONDEMENTS THÉORIQUES =====
\section{Fondements Théoriques}

\subsection{Concept Principal 1}

\begin{definition}{Nom du concept}
Définition formelle et rigoureuse du concept principal.
\end{definition}

Explication en langage naturel, intuition géométrique ou visuelle.

\subsubsection{Formulation mathématique}

Équations et développements mathématiques :

\begin{equation}
    f(\vect{x}) = \vect{w}^T \vect{x} + b
\end{equation}

où :
\begin{itemize}
    \item $\vect{x} \in \R^d$ : vecteur de features
    \item $\vect{w} \in \R^d$ : vecteur de poids
    \item $b \in \R$ : biais (intercept)
\end{itemize}

\subsection{Concept Principal 2}

% Répéter la structure ci-dessus pour chaque concept important

% ===== SECTION 3: ALGORITHMES =====
\section{Algorithmes}

\subsection{Algorithme Principal}

\begin{algorithm}[H]
\caption{Nom de l'Algorithme}
\label{alg:principal}
\begin{algorithmic}[1]
\REQUIRE Données d'entrée $X$, labels $y$
\ENSURE Modèle entraîné $\theta$
\STATE Initialiser $\theta$ aléatoirement
\FOR{$epoch = 1$ \TO $n_{epochs}$}
    \FOR{chaque batch $(X_{batch}, y_{batch})$}
        \STATE Calculer prédictions : $\hat{y} = f(X_{batch}; \theta)$
        \STATE Calculer loss : $L = \text{loss}(\hat{y}, y_{batch})$
        \STATE Calculer gradient : $\nabla_\theta L$
        \STATE Mettre à jour : $\theta \leftarrow \theta - \alpha \nabla_\theta L$
    \ENDFOR
\ENDFOR
\RETURN $\theta$
\end{algorithmic}
\end{algorithm}

\subsubsection{Complexité}

\begin{itemize}
    \item \textbf{Temps :} $O(...)$
    \item \textbf{Espace :} $O(...)$
\end{itemize}

% ===== SECTION 4: IMPLÉMENTATION =====
\section{Implémentation}

\subsection{Implémentation NumPy/scikit-learn}

Exemple de code Python :

\begin{lstlisting}[language=Python, caption=Implémentation de l'algorithme]
import numpy as np
from sklearn.base import BaseEstimator

class MonAlgorithme(BaseEstimator):
    def __init__(self, param1=1.0, param2=10):
        self.param1 = param1
        self.param2 = param2

    def fit(self, X, y):
        # Implémentation de l'entraînement
        n_samples, n_features = X.shape
        self.coef_ = np.zeros(n_features)

        for epoch in range(self.param2):
            # ... logique d'entraînement
            pass

        return self

    def predict(self, X):
        # Prédictions
        return X @ self.coef_
\end{lstlisting}

% ===== SECTION 5: VISUALISATIONS =====
\section{Visualisations et Interprétation}

% Inclure des figures si disponibles
% \begin{figure}[h]
%     \centering
%     \includegraphics[width=0.7\textwidth]{chemin/vers/image.png}
%     \caption{Description de la figure}
%     \label{fig:exemple}
% \end{figure}

Description des visualisations typiques pour ce type de modèle.

% ===== SECTION 6: AVANTAGES ET LIMITES =====
\section{Avantages et Limites}

\subsection{Avantages}
\begin{itemize}
    \item ✅ Avantage 1
    \item ✅ Avantage 2
    \item ✅ Avantage 3
\end{itemize}

\subsection{Limites}
\begin{itemize}
    \item ❌ Limite 1
    \item ❌ Limite 2
    \item ❌ Limite 3
\end{itemize}

\subsection{Quand utiliser cet algorithme ?}

\begin{astuce}
Cet algorithme est particulièrement adapté quand :
\begin{itemize}
    \item Condition 1
    \item Condition 2
    \item Condition 3
\end{itemize}
\end{astuce}

\begin{attention}
Éviter cet algorithme dans les cas suivants :
\begin{itemize}
    \item Cas problématique 1
    \item Cas problématique 2
\end{itemize}
\end{attention}

% ===== SECTION 7: HYPERPARAMÈTRES =====
\section{Hyperparamètres et Tuning}

\begin{table}[h]
\centering
\caption{Hyperparamètres principaux}
\label{tab:hyperparams}
\begin{tabular}{lll}
\toprule
\textbf{Paramètre} & \textbf{Valeurs typiques} & \textbf{Impact} \\
\midrule
learning\_rate & $10^{-4}$ à $10^{-1}$ & Vitesse convergence \\
n\_iterations & $10$ à $1000$ & Qualité ajustement \\
regularization & $10^{-3}$ à $10$ & Compromis biais-variance \\
\bottomrule
\end{tabular}
\end{table}

% ===== SECTION 8: VARIANTES =====
\section{Variantes et Extensions}

\subsection{Variante 1}
Description de la première variante de l'algorithme.

\subsection{Variante 2}
Description de la deuxième variante.

% ===== SECTION 9: APPLICATIONS =====
\section{Applications Pratiques}

\begin{enumerate}
    \item \textbf{Application 1} : Description et domaine d'usage
    \item \textbf{Application 2} : Description et domaine d'usage
    \item \textbf{Application 3} : Description et domaine d'usage
\end{enumerate}

% ===== SECTION 10: RÉSUMÉ =====
\section{Résumé du Chapitre}

\subsection{Points Clés}
\begin{itemize}
    \item \textbf{Point 1} : Résumé concis
    \item \textbf{Point 2} : Résumé concis
    \item \textbf{Point 3} : Résumé concis
\end{itemize}

\subsection{Formules Essentielles}

\begin{tcolorbox}[colback=blue!5!white, colframe=blue!75!black, title=Formules à retenir]
\begin{align}
    \text{Formule 1:} & \quad y = f(x) \\
    \text{Formule 2:} & \quad L = \text{loss}(y, \hat{y})
\end{align}
\end{tcolorbox}

% ===== SECTION 11: EXERCICES =====
\section{Exercices}

\subsection{Questions de compréhension}
\begin{enumerate}
    \item Question théorique 1 ?
    \item Question théorique 2 ?
    \item Question théorique 3 ?
\end{enumerate}

\subsection{Exercices pratiques}
\begin{enumerate}
    \item \textbf{Exercice 1} : Implémentation from scratch
    \begin{itemize}
        \item Implémenter l'algorithme en NumPy pur
        \item Comparer avec scikit-learn
    \end{itemize}

    \item \textbf{Exercice 2} : Application sur dataset réel
    \begin{itemize}
        \item Charger le dataset XYZ
        \item Appliquer l'algorithme
        \item Analyser les résultats
    \end{itemize}
\end{enumerate}

\textit{Solutions disponibles dans} \texttt{XX\_exercices\_solutions.ipynb}

% ===== SECTION 12: POUR ALLER PLUS LOIN =====
\section{Pour Aller Plus Loin}

\subsection{Lectures Recommandées}
\begin{itemize}
    \item Article/Papier 1 (lien)
    \item Livre chapitre XX
    \item Blog post
\end{itemize}

\subsection{Ressources en Ligne}
\begin{itemize}
    \item Documentation scikit-learn: \url{https://scikit-learn.org/}
    \item Tutoriel PyTorch: \url{https://pytorch.org/tutorials/}
\end{itemize}

\subsection{Prochaines Étapes}
Chapitre suivant recommandé : \textbf{Chapitre XX+1 - Titre}

% ===== BIBLIOGRAPHIE =====
\section*{Références}
\begin{enumerate}
    \item Auteur A. (Année). \textit{Titre du Livre}. Éditeur.
    \item Auteur B. et al. (Année). "Titre de l'Article". \textit{Journal}, vol(issue), pages.
\end{enumerate}

\end{document}
