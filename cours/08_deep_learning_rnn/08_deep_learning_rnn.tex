\documentclass[11pt,a4paper]{article}

% ===== PACKAGES =====
\usepackage[utf8]{inputenc}
\usepackage[T1]{fontenc}
\usepackage[french]{babel}
\usepackage{lmodern}

% Mathématiques
\usepackage{amsmath, amssymb, amsthm}
\usepackage{mathtools}

% Mise en page
\usepackage[margin=2.5cm]{geometry}
\usepackage{parskip}
\usepackage{setspace}
\setstretch{1.15}

% Graphiques et couleurs
\usepackage{graphicx}
\usepackage{xcolor}

% ===== UNICODE CHARACTERS SUPPORT =====
\usepackage{newunicodechar}

% Emojis et symboles
\newunicodechar{✅}{\textcolor{green!60!black}{$\checkmark$}}
\newunicodechar{❌}{\textcolor{red!60!black}{$\times$}}
\newunicodechar{✓}{\textcolor{green!60!black}{$\checkmark$}}
\newunicodechar{✗}{\textcolor{red!60!black}{$\times$}}
\newunicodechar{⚠}{\textcolor{orange!80!black}{\textbf{/!\textbackslash}}}
\newunicodechar{💡}{\textcolor{blue!70!black}{\textbf{(i)}}}
\newunicodechar{🎯}{\textcolor{purple!70!black}{\textbf{$\star$}}}
\newunicodechar{📊}{\textcolor{blue!70!black}{\textbf{[=]}}}

% Étoiles (pour tableaux)
\newunicodechar{★}{\textcolor{orange!80!black}{$\star$}}
\newunicodechar{☆}{\textcolor{gray!50}{$\star$}}

% Flèches
\newunicodechar{→}{$\rightarrow$}
\newunicodechar{←}{$\leftarrow$}
\newunicodechar{↑}{$\uparrow$}
\newunicodechar{↓}{$\downarrow$}

\usepackage{tikz}
\usetikzlibrary{arrows.meta, positioning, shapes.geometric, calc, matrix}

% Tableaux
\usepackage{booktabs}
\usepackage{longtable}
\usepackage{multirow}
\usepackage{tabularx}
\usepackage{colortbl}

% Code et algorithmes
\usepackage{listings}
\usepackage{algorithm}
\usepackage{algorithmic}

% Hyperliens
\usepackage{hyperref}
\hypersetup{
    colorlinks=true,
    linkcolor=blue,
    filecolor=magenta,
    urlcolor=cyan,
    citecolor=green,
    pdftitle={Chapitre 08 - Deep Learning : RNN et Transformers},
    pdfauthor={Cours ML},
}

% Boxes colorées
\usepackage{tcolorbox}
\tcbuselibrary{skins, breakable}


% ===== TCOLORBOX AVEC EMOJIS =====
\newtcolorbox{attention}{
    colback=red!5!white,
    colframe=red!75!black,
    fonttitle=\bfseries,
    title=⚠ Attention,
    breakable
}

\newtcolorbox{definition}{
    colback=blue!5!white,
    colframe=blue!75!black,
    fonttitle=\bfseries,
    title=Définition,
    breakable
}

\newtcolorbox{astuce}{
    colback=green!5!white,
    colframe=green!60!black,
    fonttitle=\bfseries,
    title=💡 Astuce,
    breakable
}

\newtcolorbox{remarque}{
    colback=yellow!5!white,
    colframe=orange!75!black,
    fonttitle=\bfseries,
    title=💡 Remarque,
    breakable
}

\newtcolorbox{important}{
    colback=purple!5!white,
    colframe=purple!75!black,
    fonttitle=\bfseries,
    title=⚠ Important,
    breakable
}

\newtcolorbox{exemple}{
    colback=gray!5!white,
    colframe=gray!75!black,
    fonttitle=\bfseries,
    title=Exemple,
    breakable
}

% En-têtes et pieds de page
\usepackage{fancyhdr}
\pagestyle{fancy}
\fancyhf{}
\fancyhead[L]{\small Chapitre 08 - Deep Learning : RNN et Transformers}
\fancyhead[R]{\small Cours Machine Learning}
\fancyfoot[C]{\thepage}

% ===== CONFIGURATION LISTINGS =====
\definecolor{codegreen}{rgb}{0,0.6,0}
\definecolor{codegray}{rgb}{0.5,0.5,0.5}
\definecolor{codepurple}{rgb}{0.58,0,0.82}
\definecolor{backcolour}{rgb}{0.95,0.95,0.92}

\lstdefinestyle{pythonstyle}{
    language=Python,
    backgroundcolor=\color{backcolour},
    commentstyle=\color{codegreen},
    keywordstyle=\color{blue},
    numberstyle=\tiny\color{codegray},
    stringstyle=\color{codepurple},
    basicstyle=\ttfamily\small,
    breakatwhitespace=false,
    breaklines=true,
    captionpos=b,
    keepspaces=true,
    numbers=left,
    numbersep=5pt,
    showspaces=false,
    showstringspaces=false,
    showtabs=false,
    tabsize=4,
    frame=single,
    rulecolor=\color{black}
}
\lstset{style=pythonstyle}

% ===== CONFIGURATION TCOLORBOX =====


\newtcolorbox{theoreme}[1]{
    colback=green!5!white,
    colframe=green!75!black,
    fonttitle=\bfseries,
    title=Théorème: #1,
    breakable
}







% ===== COMMANDES PERSONNALISÉES =====
\newcommand{\vect}[1]{\mathbf{#1}}
\newcommand{\mat}[1]{\mathbf{#1}}
\newcommand{\R}{\mathbb{R}}
\newcommand{\N}{\mathbb{N}}
\newcommand{\argmin}{\operatorname{argmin}}
\newcommand{\argmax}{\operatorname{argmax}}
% \tanh est déjà défini dans amsmath
\newcommand{\sigmoid}{\operatorname{sigmoid}}
\newcommand{\softmax}{\operatorname{softmax}}

\begin{document}

% ===== PAGE DE TITRE =====
\begin{titlepage}
    \centering
    \vspace*{2cm}

    {\Huge\bfseries Cours Machine Learning}\\[0.5cm]

    \vspace{1cm}

    {\LARGE Chapitre 08}\\[0.3cm]
    {\LARGE\bfseries Deep Learning : RNN et Transformers}\\[2cm]

    \vfill

    {\large
    \textbf{Objectifs d'apprentissage :}\\[0.5cm]
    \begin{itemize}
        \item Comprendre les RNN (Recurrent Neural Networks) pour les séquences
        \item Maîtriser LSTM et GRU pour résoudre les problèmes de vanishing gradient
        \item Découvrir le mécanisme d'attention et son importance
        \item Étudier l'architecture Transformer et son fonctionnement
        \item Appliquer ces modèles au NLP et aux séries temporelles
    \end{itemize}
    }

    \vfill

    {\large
    \textbf{Prérequis :} Chapitre 06 (Réseaux de Neurones Fondamentaux)\\[0.3cm]
    \textbf{Durée estimée :} 8-10 heures\\[0.3cm]
    \textbf{Notebooks :} \texttt{08\_demo\_*.ipynb}
    }

    \vfill

    {\large Cours ML - Sandbox-ML\\
    Version 1.0 - 2026}
\end{titlepage}

\tableofcontents
\newpage

% ===== SECTION 1: INTRODUCTION =====
\section{Introduction aux Données Séquentielles}

\subsection{Qu'est-ce qu'une séquence ?}

Une \textbf{séquence} est une suite ordonnée d'éléments où l'ordre a une importance cruciale.

\begin{exemple}{Exemples de séquences}
\begin{itemize}
    \item \textbf{Texte} : "Le chat mange la souris" (ordre des mots → sens)
    \item \textbf{Audio} : Signal sonore échantillonné dans le temps
    \item \textbf{Vidéo} : Séquence d'images (frames)
    \item \textbf{Séries temporelles} : Cours de bourse, température, trafic web
    \item \textbf{ADN} : Séquence de nucléotides (A, C, G, T)
\end{itemize}
\end{exemple}

\subsection{Limitations des MLP et CNN pour les séquences}

\textbf{MLP :}
\begin{itemize}
    \item ❌ Taille d'entrée fixe (impossible pour séquences de longueur variable)
    \item ❌ Pas de mémoire du contexte précédent
    \item ❌ Nombre de paramètres explose avec la longueur
\end{itemize}

\textbf{CNN :}
\begin{itemize}
    \item ✓ Peut traiter des séquences avec Conv1D
    \item ❌ Champ récepteur limité (contexte local uniquement)
    \item ❌ Pas de mémoire à long terme
\end{itemize}

\subsection{Types de tâches séquentielles}

\begin{table}[h]
\centering
\caption{Architectures séquence-to-X}
\label{tab:seq_tasks}
\begin{tabular}{lll}
\toprule
\textbf{Type} & \textbf{Description} & \textbf{Exemple} \\
\midrule
One-to-One & Entrée fixe → Sortie fixe & Classification d'image (CNN) \\
One-to-Many & Entrée fixe → Séquence & Image captioning \\
Many-to-One & Séquence → Sortie fixe & Sentiment analysis \\
Many-to-Many & Séquence → Séquence & Traduction, génération texte \\
 & (même longueur) & Étiquetage de séquences (NER) \\
 & (longueur différente) & Traduction automatique \\
\bottomrule
\end{tabular}
\end{table}

% ===== SECTION 2: RNN =====
\section{Recurrent Neural Networks (RNN)}

\subsection{Architecture}

\begin{definition}{RNN}
Un RNN traite une séquence $(\vect{x}_1, \vect{x}_2, \ldots, \vect{x}_T)$ en maintenant un \textbf{état caché} $\vect{h}_t$ qui se propage dans le temps :
\begin{align}
    \vect{h}_t &= \tanh(\mat{W}_{hh} \vect{h}_{t-1} + \mat{W}_{xh} \vect{x}_t + \vect{b}_h) \\
    \vect{y}_t &= \mat{W}_{hy} \vect{h}_t + \vect{b}_y
\end{align}
où :
\begin{itemize}
    \item $\vect{h}_t \in \R^h$ : état caché au temps $t$
    \item $\vect{x}_t \in \R^d$ : entrée au temps $t$
    \item $\vect{y}_t \in \R^k$ : sortie au temps $t$
    \item $\mat{W}_{hh} \in \R^{h \times h}$ : poids récurrents
    \item $\mat{W}_{xh} \in \R^{h \times d}$ : poids d'entrée
    \item $\mat{W}_{hy} \in \R^{k \times h}$ : poids de sortie
\end{itemize}
\end{definition}

\textbf{Caractéristique clé :} Les mêmes poids $\mat{W}_{hh}, \mat{W}_{xh}, \mat{W}_{hy}$ sont partagés à chaque pas de temps → capacité à traiter des séquences de longueur variable.

\subsection{Déroulement dans le temps (Unfolding)}

On peut "dérouler" le RNN dans le temps :
\begin{align*}
\vect{h}_1 &= \tanh(\mat{W}_{hh} \vect{h}_0 + \mat{W}_{xh} \vect{x}_1 + \vect{b}_h) \\
\vect{h}_2 &= \tanh(\mat{W}_{hh} \vect{h}_1 + \mat{W}_{xh} \vect{x}_2 + \vect{b}_h) \\
\vect{h}_3 &= \tanh(\mat{W}_{hh} \vect{h}_2 + \mat{W}_{xh} \vect{x}_3 + \vect{b}_h) \\
&\vdots
\end{align*}

Initialisation : $\vect{h}_0 = \vect{0}$ (ou appris)

\begin{figure}[h]
\centering
\begin{tikzpicture}[scale=0.85, every node/.style={font=\small}]
    % Time step 0 (initial state)
    \node[draw, circle, minimum size=1.2cm, fill=gray!20] (h0) at (0,0) {$\vect{h}_0$};
    \node[below, font=\tiny] at (h0.south) {État initial};

    % Time step 1
    \node[draw, rectangle, minimum width=1.5cm, minimum height=1.2cm, fill=blue!20] (rnn1) at (3,0) {RNN};
    \node[draw, circle, minimum size=1cm, fill=green!20] (x1) at (3,-2.5) {$\vect{x}_1$};
    \node[draw, circle, minimum size=1cm, fill=red!20] (y1) at (3,2.5) {$\vect{y}_1$};
    \node[draw, circle, minimum size=1.2cm, fill=gray!20] (h1) at (5.5,0) {$\vect{h}_1$};

    \draw[->, thick] (h0) -- (rnn1) node[midway, above, font=\tiny] {};
    \draw[->, thick] (x1) -- (rnn1);
    \draw[->, thick] (rnn1) -- (y1);
    \draw[->, thick] (rnn1) -- (h1);

    \node[above, font=\footnotesize] at (3, 3.2) {$t=1$};

    % Time step 2
    \node[draw, rectangle, minimum width=1.5cm, minimum height=1.2cm, fill=blue!20] (rnn2) at (8,0) {RNN};
    \node[draw, circle, minimum size=1cm, fill=green!20] (x2) at (8,-2.5) {$\vect{x}_2$};
    \node[draw, circle, minimum size=1cm, fill=red!20] (y2) at (8,2.5) {$\vect{y}_2$};
    \node[draw, circle, minimum size=1.2cm, fill=gray!20] (h2) at (10.5,0) {$\vect{h}_2$};

    \draw[->, thick] (h1) -- (rnn2);
    \draw[->, thick] (x2) -- (rnn2);
    \draw[->, thick] (rnn2) -- (y2);
    \draw[->, thick] (rnn2) -- (h2);

    \node[above, font=\footnotesize] at (8, 3.2) {$t=2$};

    % Time step 3
    \node[draw, rectangle, minimum width=1.5cm, minimum height=1.2cm, fill=blue!20] (rnn3) at (13,0) {RNN};
    \node[draw, circle, minimum size=1cm, fill=green!20] (x3) at (13,-2.5) {$\vect{x}_3$};
    \node[draw, circle, minimum size=1cm, fill=red!20] (y3) at (13,2.5) {$\vect{y}_3$};
    \node[draw, circle, minimum size=1.2cm, fill=gray!20] (h3) at (15.5,0) {$\vect{h}_3$};

    \draw[->, thick] (h2) -- (rnn3);
    \draw[->, thick] (x3) -- (rnn3);
    \draw[->, thick] (rnn3) -- (y3);
    \draw[->, thick] (rnn3) -- (h3);

    \node[above, font=\footnotesize] at (13, 3.2) {$t=3$};

    % Continuation
    \node at (17, 0) {$\cdots$};

    % Weight sharing annotation
    \draw[<->, very thick, purple, dashed] (rnn1.south) ++ (0, -1.2) -- ++ (10, 0);
    \node[below, font=\tiny, purple, align=center] at (8, -3.7) {Mêmes poids $\mat{W}_{hh}, \mat{W}_{xh}, \mat{W}_{hy}$ partagés};

    % Equations
    \node[draw, rectangle, fill=yellow!10, font=\tiny, align=left] at (8, -5.5) {
        $\vect{h}_t = \tanh(\mat{W}_{hh} \vect{h}_{t-1} + \mat{W}_{xh} \vect{x}_t + \vect{b}_h)$\\
        $\vect{y}_t = \mat{W}_{hy} \vect{h}_t + \vect{b}_y$
    };

    % Legend
    \node[draw, rectangle, fill=gray!10, minimum width=3cm, minimum height=2.5cm, font=\tiny, align=left] at (17.5, 1.5) {
        \textbf{Légende:}\\[0.1cm]
        • \textcolor{green!60!black}{Vert}: Entrée\\
        • \textcolor{blue!60!black}{Bleu}: RNN cell\\
        • \textcolor{red!60!black}{Rouge}: Sortie\\
        • \textcolor{gray}{Gris}: État caché\\[0.1cm]
        \textbf{Récurrence:}\\
        $\vect{h}_t$ dépend de\\
        $\vect{h}_{t-1}$
    };

\end{tikzpicture}
\caption{RNN déroulé dans le temps (Unfolding). Chaque cellule RNN (en bleu) partage les mêmes poids $\mat{W}_{hh}, \mat{W}_{xh}, \mat{W}_{hy}$. À chaque pas de temps $t$: (1) l'entrée $\vect{x}_t$ et l'état caché précédent $\vect{h}_{t-1}$ sont combinés, (2) l'état caché $\vect{h}_t$ est calculé avec $\tanh$, (3) la sortie $\vect{y}_t$ est produite, (4) $\vect{h}_t$ est propagé au pas suivant. Cette récurrence permet au RNN de capturer les dépendances temporelles dans les séquences.}
\label{fig:rnn_unfolded}
\end{figure}

\clearpage

\subsection{Backpropagation Through Time (BPTT)}

Pour entraîner un RNN, on utilise \textbf{BPTT} : backpropagation appliquée au réseau déroulé.

\textbf{Forward pass} : Calculer $\vect{h}_1, \ldots, \vect{h}_T$ et $\vect{y}_1, \ldots, \vect{y}_T$

\textbf{Backward pass} : Calculer gradients en remontant dans le temps :
\begin{equation}
    \frac{\partial L}{\partial \vect{h}_t} = \frac{\partial L}{\partial \vect{y}_t} \frac{\partial \vect{y}_t}{\partial \vect{h}_t} + \frac{\partial L}{\partial \vect{h}_{t+1}} \frac{\partial \vect{h}_{t+1}}{\partial \vect{h}_t}
\end{equation}

\subsection{Problème du Vanishing/Exploding Gradient}

\begin{attention}
Le gradient se propage à travers tous les pas de temps :
\begin{equation}
    \frac{\partial \vect{h}_t}{\partial \vect{h}_0} = \prod_{k=1}^{t} \frac{\partial \vect{h}_k}{\partial \vect{h}_{k-1}} = \prod_{k=1}^{t} \mat{W}_{hh}^T \cdot \text{diag}(\tanh'(\vect{z}_k))
\end{equation}

\textbf{Vanishing gradient} : Si $\|\mat{W}_{hh}\| < 1$, le gradient $\to 0$ exponentiellement.
\begin{itemize}
    \item Le RNN ne peut pas apprendre de dépendances à long terme
\end{itemize}

\textbf{Exploding gradient} : Si $\|\mat{W}_{hh}\| > 1$, le gradient $\to \infty$.
\begin{itemize}
    \item Solution : \textbf{gradient clipping} (limiter la norme du gradient)
\end{itemize}
\end{attention}

\begin{figure}[h]
\centering
\begin{tikzpicture}[scale=1.0, every node/.style={font=\small}]
    % Timeline axis
    \draw[->, thick] (0,0) -- (13,0) node[right, font=\footnotesize] {Temps $t$};
    \foreach \x in {1,2,...,12} {
        \draw (\x, -0.1) -- (\x, 0.1);
    }
    \node[below] at (1, -0.1) {$t_0$};
    \node[below] at (12, -0.1) {$t_{T}$};

    % Vanishing gradient (exponential decay)
    \begin{scope}[yshift=3cm]
        \draw[->, thick] (0,0) -- (0,3.5) node[above, font=\footnotesize] {$\|\nabla\|$};
        \draw[red!70!black, very thick, smooth, samples=50, domain=1:12]
            plot (\x, {3*exp(-0.35*(\x-1))});

        % Annotations
        \node[red!70!black, font=\footnotesize, align=center] at (6.5, 3.0) {
            \textbf{Vanishing Gradient}\\
            $\|\mat{W}_{hh}\| < 1$
        };
        \draw[->, red!70!black, dashed, thick] (6.5, 2.6) -- (8, 0.8);

        % Gradient values annotations
        \node[red!70!black, font=\scriptsize] at (1, 3.2) {$\nabla_0$};
        \node[red!70!black, font=\scriptsize] at (4, 1.8) {$\nabla_0 \cdot 0.3$};
        \node[red!70!black, font=\scriptsize] at (8, 0.5) {$\nabla_0 \cdot 0.05$};
        \node[red!70!black, font=\scriptsize] at (12, 0.15) {$\approx 0$};

        % Danger zone
        \fill[red!10, opacity=0.5] (8,0) rectangle (13, 1.0);
        \node[red!70!black, font=\scriptsize, align=center] at (10.5, 0.5) {
            Perte de\\mémoire
        };
    \end{scope}

    % Exploding gradient (exponential growth)
    \begin{scope}[yshift=-3cm]
        \draw[->, thick] (0,0) -- (0,3.5) node[above, font=\footnotesize] {$\|\nabla\|$};
        \draw[blue!70!black, very thick, smooth, samples=50, domain=1:10]
            plot (\x, {0.3*exp(0.35*(\x-1))});
        \draw[blue!70!black, very thick, dashed] (10, 3.0) -- (12, 3.0);

        % Annotations
        \node[blue!70!black, font=\footnotesize, align=center] at (6.5, -0.5) {
            \textbf{Exploding Gradient}\\
            $\|\mat{W}_{hh}\| > 1$
        };
        \draw[->, blue!70!black, dashed, thick] (6.5, -0.1) -- (7, 1.5);

        % Gradient values annotations
        \node[blue!70!black, font=\scriptsize] at (1, 0.1) {$\nabla_0$};
        \node[blue!70!black, font=\scriptsize] at (4, 0.8) {$\nabla_0 \cdot 3$};
        \node[blue!70!black, font=\scriptsize] at (7, 2.0) {$\nabla_0 \cdot 15$};
        \node[blue!70!black, font=\scriptsize] at (10, 3.2) {$\to \infty$};

        % Clipping solution
        \fill[green!10, opacity=0.5] (10,0) rectangle (13, 3.0);
        \node[green!50!black, font=\scriptsize, align=center] at (11.5, 1.5) {
            \textbf{Gradient}\\
            \textbf{Clipping}\\
            $\|\nabla\| \leq \tau$
        };
        \draw[green!50!black, very thick] (10, 2.5) -- (13, 2.5);
        \node[green!50!black, font=\scriptsize, right] at (13, 2.5) {$\tau$};
    \end{scope}

    % Stable gradient (LSTM) for comparison
    \begin{scope}[yshift=0cm, xshift=14.5cm]
        \draw[->, thick] (0,0) -- (0,3.5) node[above, font=\footnotesize] {$\|\nabla\|$};
        \draw[->, thick] (0,0) -- (2.5,0) node[right, font=\scriptsize] {$t$};
        \draw[green!60!black, very thick, domain=0:2, samples=30]
            plot (\x, {1.8 + 0.3*sin(\x*180*2) + 0.1*rand});

        \node[green!60!black, font=\footnotesize, align=center] at (1.2, -0.7) {
            \textbf{LSTM}\\
            (stable)
        };
        \draw[green!60!black, dashed] (0, 1.8) -- (2.5, 1.8);
    \end{scope}
\end{tikzpicture}
\caption{Problème du vanishing/exploding gradient dans les RNN. \textbf{(Haut)} Avec $\|\mat{W}_{hh}\| < 1$, le gradient décroît exponentiellement à travers le temps, empêchant l'apprentissage de dépendances à long terme. \textbf{(Bas)} Avec $\|\mat{W}_{hh}\| > 1$, le gradient explose, rendant l'entraînement instable ; le gradient clipping limite la norme maximale. \textbf{(Droite)} Les LSTM maintiennent un gradient stable grâce aux connexions de mémoire.}
\label{fig:rnn_gradient_problem}
\end{figure}

\subsection{Implémentation simple}

\begin{lstlisting}[language=Python, caption=RNN vanilla from scratch]
import numpy as np

class SimpleRNN:
    def __init__(self, input_dim, hidden_dim, output_dim):
        self.hidden_dim = hidden_dim

        # Initialisation Xavier
        self.Wxh = np.random.randn(hidden_dim, input_dim) * 0.01
        self.Whh = np.random.randn(hidden_dim, hidden_dim) * 0.01
        self.Why = np.random.randn(output_dim, hidden_dim) * 0.01
        self.bh = np.zeros((hidden_dim, 1))
        self.by = np.zeros((output_dim, 1))

    def forward(self, inputs):
        """
        inputs: liste de vecteurs (seq_len, input_dim)
        """
        h = np.zeros((self.hidden_dim, 1))  # h0
        self.last_inputs = inputs
        self.last_hs = {0: h}

        # Forward pass
        for t, x in enumerate(inputs):
            x = x.reshape(-1, 1)
            h = np.tanh(self.Wxh @ x + self.Whh @ h + self.bh)
            self.last_hs[t + 1] = h

        # Output au dernier pas de temps (many-to-one)
        y = self.Why @ h + self.by
        return y, h

    def backward(self, dy, learning_rate=0.001):
        """
        BPTT simplifié (output au dernier pas uniquement)
        """
        n = len(self.last_inputs)

        # Gradients accumulés
        dWxh, dWhh, dWhy = np.zeros_like(self.Wxh), \
                           np.zeros_like(self.Whh), \
                           np.zeros_like(self.Why)
        dbh, dby = np.zeros_like(self.bh), np.zeros_like(self.by)

        # Gradient output
        dWhy += dy @ self.last_hs[n].T
        dby += dy

        # Backprop through time
        dh = self.Why.T @ dy

        for t in reversed(range(n)):
            temp = (1 - self.last_hs[t + 1] ** 2) * dh  # tanh'
            dbh += temp
            dWxh += temp @ self.last_inputs[t].reshape(1, -1)
            dWhh += temp @ self.last_hs[t].T
            dh = self.Whh.T @ temp

        # Gradient clipping
        for grad in [dWxh, dWhh, dWhy, dbh, dby]:
            np.clip(grad, -1, 1, out=grad)

        # Update
        self.Wxh -= learning_rate * dWxh
        self.Whh -= learning_rate * dWhh
        self.Why -= learning_rate * dWhy
        self.bh -= learning_rate * dbh
        self.by -= learning_rate * dby
\end{lstlisting}

% ===== SECTION 3: LSTM =====
\section{Long Short-Term Memory (LSTM)}

\subsection{Motivation}

LSTM résout le problème du vanishing gradient en introduisant une \textbf{cellule mémoire} $\vect{c}_t$ et des \textbf{portes (gates)} qui contrôlent le flux d'information.

\subsection{Architecture}

\begin{definition}{LSTM}
Un LSTM a 3 portes et 1 cellule mémoire :
\begin{align}
    \vect{f}_t &= \sigmoid(\mat{W}_f [\vect{h}_{t-1}, \vect{x}_t] + \vect{b}_f) \quad \text{(forget gate)} \\
    \vect{i}_t &= \sigmoid(\mat{W}_i [\vect{h}_{t-1}, \vect{x}_t] + \vect{b}_i) \quad \text{(input gate)} \\
    \tilde{\vect{c}}_t &= \tanh(\mat{W}_c [\vect{h}_{t-1}, \vect{x}_t] + \vect{b}_c) \quad \text{(candidate cell)} \\
    \vect{c}_t &= \vect{f}_t \odot \vect{c}_{t-1} + \vect{i}_t \odot \tilde{\vect{c}}_t \quad \text{(cell state)} \\
    \vect{o}_t &= \sigmoid(\mat{W}_o [\vect{h}_{t-1}, \vect{x}_t] + \vect{b}_o) \quad \text{(output gate)} \\
    \vect{h}_t &= \vect{o}_t \odot \tanh(\vect{c}_t) \quad \text{(hidden state)}
\end{align}
où $\odot$ est le produit élément par élément (Hadamard).
\end{definition}

\subsection{Interprétation des portes}

\begin{itemize}
    \item \textbf{Forget gate} ($\vect{f}_t$) : Décide quelle information de $\vect{c}_{t-1}$ oublier
    \begin{itemize}
        \item $f_t = 0$ : oublier complètement
        \item $f_t = 1$ : conserver complètement
    \end{itemize}

    \item \textbf{Input gate} ($\vect{i}_t$) : Décide quelle nouvelle information stocker dans $\vect{c}_t$
    \begin{itemize}
        \item $i_t = 0$ : ignorer la nouvelle information
        \item $i_t = 1$ : stocker complètement
    \end{itemize}

    \item \textbf{Output gate} ($\vect{o}_t$) : Décide quelle partie de $\vect{c}_t$ exposer dans $\vect{h}_t$
\end{itemize}

\textbf{Flux d'information :}
\begin{equation}
    \vect{c}_t = \underbrace{\vect{f}_t \odot \vect{c}_{t-1}}_{\text{mémoire passée}} + \underbrace{\vect{i}_t \odot \tilde{\vect{c}}_t}_{\text{nouvelle info}}
\end{equation}

\begin{astuce}
La cellule $\vect{c}_t$ agit comme une "autoroute" permettant au gradient de se propager sans atténuation (si $\vect{f}_t \approx 1$). Cela résout le vanishing gradient !
\end{astuce}

\begin{figure}[h]
\centering
\begin{tikzpicture}[
    scale=0.9,
    gate/.style={rectangle, draw, minimum width=1.2cm, minimum height=1cm, fill=orange!30, font=\footnotesize},
    tanh/.style={rectangle, draw, minimum width=1.2cm, minimum height=1cm, fill=green!30, font=\footnotesize},
    multiply/.style={circle, draw, minimum size=0.7cm, fill=red!20, font=\small},
    add/.style={circle, draw, minimum size=0.7cm, fill=blue!20, font=\small}
]

    % Input and previous hidden state
    \node[font=\small] (xt) at (0, 0) {$\vect{x}_t$};
    \node[font=\small] (ht_1) at (0, 4) {$\vect{h}_{t-1}$};

    % Concatenation point
    \node[draw, circle, minimum size=0.5cm, fill=gray!20] (concat) at (1.5, 2) {};
    \draw[->, thick] (xt) -| (concat);
    \draw[->, thick] (ht_1) -| (concat);
    \node[right, font=\tiny] at (concat.east) {$[\vect{h}_{t-1}, \vect{x}_t]$};

    % Forget gate
    \node[gate] (forget) at (3.5, 4.5) {$\sigma$};
    \node[above, font=\tiny] at (forget.north) {Forget Gate};
    \draw[->, thick] (concat) |- (forget);

    % Input gate
    \node[gate] (input) at (3.5, 2.5) {$\sigma$};
    \node[above, font=\tiny] at (input.north) {Input Gate};
    \draw[->, thick] (concat) -- (input);

    % Candidate cell
    \node[tanh] (candidate) at (3.5, 0.5) {$\tanh$};
    \node[above, font=\tiny] at (candidate.north) {Candidate $\tilde{\vect{c}}_t$};
    \draw[->, thick] (concat) |- (candidate);

    % Cell state path (horizontal)
    \node[font=\small] (ct_1) at (1.5, 6) {$\vect{c}_{t-1}$};
    \draw[->, very thick, blue!70!black] (ct_1) -- (5, 6);

    % Forget multiplication
    \node[multiply] (forget_mult) at (5, 6) {$\times$};
    \draw[->, thick] (forget) -| (forget_mult);

    % Input multiplication
    \node[multiply] (input_mult) at (5, 1.5) {$\times$};
    \draw[->, thick] (input) |- (input_mult);
    \draw[->, thick] (candidate) -| (input_mult);

    % Addition (cell state update)
    \node[add] (add) at (7, 6) {$+$};
    \draw[->, thick] (forget_mult) -- (add);
    \draw[->, thick] (input_mult) -| (add);

    % New cell state
    \node[font=\small] (ct) at (9, 6) {$\vect{c}_t$};
    \draw[->, very thick, blue!70!black] (add) -- (ct);
    \draw[->, thick] (ct) -- (11, 6);

    % Output gate
    \node[gate] (output_gate) at (9, 3) {$\sigma$};
    \node[above, font=\tiny] at (output_gate.north) {Output Gate};
    \draw[->, thick] (concat) -| (output_gate);

    % Tanh of cell state
    \node[tanh] (ct_tanh) at (9, 4.5) {$\tanh$};
    \draw[->, thick] (add) |- (ct_tanh);

    % Output multiplication
    \node[multiply] (output_mult) at (10.5, 3.8) {$\times$};
    \draw[->, thick] (output_gate) -| (output_mult);
    \draw[->, thick] (ct_tanh) -| (output_mult);

    % Hidden state output
    \node[font=\small] (ht) at (12.5, 3.8) {$\vect{h}_t$};
    \draw[->, very thick, red!70!black] (output_mult) -- (ht);
    \draw[->, thick] (ht) -- (12.5, 0);
    \node[below, font=\small] at (12.5, 0) {Output};

    % Loop back hidden state
    \draw[->, thick, dashed] (ht) .. controls (13, 5) and (13, 7) .. (11, 7)
        node[midway, above, font=\tiny] {to $\vect{h}_{t+1}$};

    % Gate labels with equations
    \node[below, font=\tiny, align=center] at (forget.south) {$\vect{f}_t$};
    \node[below, font=\tiny, align=center] at (input.south) {$\vect{i}_t$};
    \node[below, font=\tiny, align=center] at (output_gate.south) {$\vect{o}_t$};

    % Legend
    \node[draw, rectangle, minimum width=3.5cm, minimum height=2.5cm, dashed] (legend) at (14, 5.5) {};
    \node[above left, font=\footnotesize\bfseries] at (legend.north west) {Légende:};

    \node[gate, minimum width=0.8cm, minimum height=0.6cm] at (13.2, 5) {};
    \node[right, font=\tiny] at (13.6, 5) {Sigmoid $\sigma$};

    \node[tanh, minimum width=0.8cm, minimum height=0.6cm] at (13.2, 4.3) {};
    \node[right, font=\tiny] at (13.6, 4.3) {Tanh};

    \node[multiply, minimum size=0.5cm] at (12.9, 3.7) {$\times$};
    \node[right, font=\tiny] at (13.3, 3.7) {Mult. élément};

    \node[add, minimum size=0.5cm] at (12.9, 3.2) {$+$};
    \node[right, font=\tiny] at (13.3, 3.2) {Addition};

    % Annotations
    \node[font=\tiny, align=center, fill=yellow!20, draw, dashed] at (7, 7.5) {Cellule mémoire $\vect{c}_t$\\(autoroute du gradient)};
    \draw[->, dashed] (7, 7.3) -- (7, 6.3);

    \node[font=\tiny, align=center, fill=red!10, draw, dashed] at (5.5, -0.5) {3 portes contrôlent\\le flux d'information};

\end{tikzpicture}
\caption{Architecture LSTM (Long Short-Term Memory). \textbf{Portes:} (1) \textbf{Forget gate} $\vect{f}_t$: décide quelle information oublier de $\vect{c}_{t-1}$; (2) \textbf{Input gate} $\vect{i}_t$: décide quelle nouvelle information ajouter via $\tilde{\vect{c}}_t$; (3) \textbf{Output gate} $\vect{o}_t$: décide quelle partie de $\vect{c}_t$ exposer dans $\vect{h}_t$. \textbf{Cellule mémoire} $\vect{c}_t$ (en bleu): "autoroute" permettant au gradient de se propager sans atténuation, résolvant le vanishing gradient. Le flux $\vect{c}_t = \vect{f}_t \odot \vect{c}_{t-1} + \vect{i}_t \odot \tilde{\vect{c}}_t$ permet de conserver/oublier sélectivement les informations sur de longues séquences.}
\label{fig:lstm_architecture}
\end{figure}

\clearpage

\subsection{Nombre de paramètres}

Pour un LSTM avec $h$ unités cachées et $d$ inputs :
\begin{equation}
    \text{Params} = 4 \times (h \times (h + d) + h) = 4h(h + d + 1)
\end{equation}

Le facteur 4 vient des 4 matrices de poids (forget, input, candidate, output).

\begin{exemple}{LSTM(128) avec input(100)}
\begin{equation*}
    \text{Params} = 4 \times 128 \times (128 + 100 + 1) = 117{,}248
\end{equation*}
\end{exemple}

\subsection{Implémentation PyTorch}

\begin{lstlisting}[language=Python, caption=LSTM avec PyTorch]
import torch
import torch.nn as nn

class LSTMModel(nn.Module):
    def __init__(self, input_dim, hidden_dim, output_dim, num_layers=1):
        super(LSTMModel, self).__init__()

        self.hidden_dim = hidden_dim
        self.num_layers = num_layers

        # LSTM layer
        self.lstm = nn.LSTM(input_dim, hidden_dim, num_layers,
                            batch_first=True)

        # Fully-connected output
        self.fc = nn.Linear(hidden_dim, output_dim)

    def forward(self, x):
        """
        x: (batch, seq_len, input_dim)
        """
        # Initialiser h0, c0
        h0 = torch.zeros(self.num_layers, x.size(0),
                         self.hidden_dim).to(x.device)
        c0 = torch.zeros(self.num_layers, x.size(0),
                         self.hidden_dim).to(x.device)

        # LSTM forward
        # out: (batch, seq_len, hidden_dim)
        out, (hn, cn) = self.lstm(x, (h0, c0))

        # Prendre le dernier timestep (many-to-one)
        out = self.fc(out[:, -1, :])
        return out

# Exemple d'utilisation
model = LSTMModel(input_dim=10, hidden_dim=128, output_dim=5)

# Séquence de longueur 20
x = torch.randn(32, 20, 10)  # (batch=32, seq_len=20, input_dim=10)
output = model(x)
print(output.shape)  # (32, 5)
\end{lstlisting}

% ===== SECTION 4: GRU =====
\section{Gated Recurrent Unit (GRU)}

\subsection{Architecture}

GRU est une variante simplifiée du LSTM avec \textbf{2 portes} au lieu de 3.

\begin{definition}{GRU}
\begin{align}
    \vect{z}_t &= \sigmoid(\mat{W}_z [\vect{h}_{t-1}, \vect{x}_t]) \quad \text{(update gate)} \\
    \vect{r}_t &= \sigmoid(\mat{W}_r [\vect{h}_{t-1}, \vect{x}_t]) \quad \text{(reset gate)} \\
    \tilde{\vect{h}}_t &= \tanh(\mat{W} [\vect{r}_t \odot \vect{h}_{t-1}, \vect{x}_t]) \quad \text{(candidate)} \\
    \vect{h}_t &= (1 - \vect{z}_t) \odot \vect{h}_{t-1} + \vect{z}_t \odot \tilde{\vect{h}}_t
\end{align}
\end{definition}

\subsection{Différences avec LSTM}

\begin{table}[h]
\centering
\caption{LSTM vs GRU}
\label{tab:lstm_gru}
\begin{tabular}{lcc}
\toprule
 & \textbf{LSTM} & \textbf{GRU} \\
\midrule
Nombre de portes & 3 & 2 \\
Cellule mémoire séparée & Oui ($\vect{c}_t$) & Non (seulement $\vect{h}_t$) \\
Paramètres & $4h(h + d + 1)$ & $3h(h + d + 1)$ \\
Vitesse & Plus lent & Plus rapide \\
Performance & Légèrement meilleure & Comparable \\
\bottomrule
\end{tabular}
\end{table}

\begin{astuce}
\textbf{Règle pratique :}
\begin{itemize}
    \item LSTM : Séquences très longues, dépendances complexes
    \item GRU : Séquences courtes/moyennes, plus rapide, moins de paramètres
    \item En pratique, essayer les deux et comparer !
\end{itemize}
\end{astuce}

\subsection{Implémentation PyTorch}

\begin{lstlisting}[language=Python, caption=GRU avec PyTorch]
class GRUModel(nn.Module):
    def __init__(self, input_dim, hidden_dim, output_dim, num_layers=1):
        super(GRUModel, self).__init__()

        self.hidden_dim = hidden_dim
        self.num_layers = num_layers

        self.gru = nn.GRU(input_dim, hidden_dim, num_layers,
                          batch_first=True)
        self.fc = nn.Linear(hidden_dim, output_dim)

    def forward(self, x):
        h0 = torch.zeros(self.num_layers, x.size(0),
                         self.hidden_dim).to(x.device)

        out, hn = self.gru(x, h0)
        out = self.fc(out[:, -1, :])
        return out
\end{lstlisting}

% ===== SECTION 5: BIDIRECTIONAL RNN =====
\section{Bidirectional RNN/LSTM/GRU}

\subsection{Motivation}

Un RNN classique ne voit que le \textbf{contexte passé}. Pour certaines tâches (NER, POS tagging), le contexte \textbf{futur} est aussi important.

\begin{exemple}{Prédiction de mots}
Phrase : "Le \_\_\_ mange la souris"

\begin{itemize}
    \item Contexte passé : "Le"
    \item Contexte futur : "mange la souris" → animal carnivore → "chat"
\end{itemize}
\end{exemple}

\subsection{Architecture}

\begin{definition}{Bidirectional RNN}
Un Bi-RNN a deux RNN :
\begin{itemize}
    \item \textbf{Forward RNN} : lit la séquence de gauche à droite → $\overrightarrow{\vect{h}}_t$
    \item \textbf{Backward RNN} : lit la séquence de droite à gauche → $\overleftarrow{\vect{h}}_t$
\end{itemize}

La sortie finale est la concaténation :
\begin{equation}
    \vect{h}_t = [\overrightarrow{\vect{h}}_t ; \overleftarrow{\vect{h}}_t]
\end{equation}
\end{definition}

\textbf{Nombre de paramètres :} Doublé (2 RNN indépendants)

\subsection{Implémentation}

\begin{lstlisting}[language=Python, caption=Bidirectional LSTM]
class BiLSTM(nn.Module):
    def __init__(self, input_dim, hidden_dim, output_dim):
        super(BiLSTM, self).__init__()

        # bidirectional=True
        self.lstm = nn.LSTM(input_dim, hidden_dim,
                            batch_first=True, bidirectional=True)

        # hidden_dim * 2 car bidirectionnel
        self.fc = nn.Linear(hidden_dim * 2, output_dim)

    def forward(self, x):
        # out: (batch, seq_len, hidden_dim * 2)
        out, _ = self.lstm(x)
        out = self.fc(out[:, -1, :])  # Dernier timestep
        return out
\end{lstlisting}

% ===== SECTION 6: ATTENTION MECHANISM =====
\section{Mécanisme d'Attention}

\subsection{Motivation : Problème du goulot d'étranglement}

En seq2seq (traduction), le décodeur doit tout comprendre à partir d'un seul vecteur contexte $\vect{c}$ :
\begin{equation}
    \text{Encoder} : (x_1, \ldots, x_n) \to \vect{c} \to \text{Decoder} : (y_1, \ldots, y_m)
\end{equation}

\textbf{Problème :} $\vect{c}$ est un goulot d'étranglement, surtout pour des séquences longues.

\subsection{Attention de Bahdanau}

\begin{definition}{Attention Mechanism}
Au lieu d'un vecteur contexte fixe, on calcule un vecteur contexte \textbf{dynamique} $\vect{c}_t$ à chaque pas de décodage :
\begin{align}
    e_{t,i} &= \text{score}(\vect{s}_{t-1}, \vect{h}_i) \quad \text{(score d'attention)} \\
    \alpha_{t,i} &= \frac{\exp(e_{t,i})}{\sum_{j=1}^n \exp(e_{t,j})} \quad \text{(poids d'attention)} \\
    \vect{c}_t &= \sum_{i=1}^n \alpha_{t,i} \vect{h}_i \quad \text{(contexte pondéré)}
\end{align}
où :
\begin{itemize}
    \item $\vect{s}_{t-1}$ : état caché du décodeur
    \item $\vect{h}_i$ : états cachés de l'encodeur
    \item $\alpha_{t,i}$ : attention sur le mot $i$ de l'entrée
\end{itemize}
\end{definition}

\subsection{Fonctions de score}

\textbf{Dot product :}
\begin{equation}
    \text{score}(\vect{s}, \vect{h}) = \vect{s}^T \vect{h}
\end{equation}

\textbf{General (bilinear) :}
\begin{equation}
    \text{score}(\vect{s}, \vect{h}) = \vect{s}^T \mat{W}_a \vect{h}
\end{equation}

\textbf{Additive (Bahdanau) :}
\begin{equation}
    \text{score}(\vect{s}, \vect{h}) = \vect{v}_a^T \tanh(\mat{W}_1 \vect{s} + \mat{W}_2 \vect{h})
\end{equation}

\subsection{Scaled Dot-Product Attention}

Version utilisée dans les Transformers :
\begin{equation}
    \text{Attention}(\mat{Q}, \mat{K}, \mat{V}) = \softmax\left(\frac{\mat{Q}\mat{K}^T}{\sqrt{d_k}}\right) \mat{V}
\end{equation}
où :
\begin{itemize}
    \item $\mat{Q}$ (Queries) : ce qu'on cherche
    \item $\mat{K}$ (Keys) : ce qu'on a
    \item $\mat{V}$ (Values) : ce qu'on renvoie
    \item $d_k$ : dimension des clés (facteur de normalisation)
\end{itemize}

\textbf{Interprétation :}
\begin{enumerate}
    \item Calculer similarité entre $\mat{Q}$ et $\mat{K}$ : $\mat{Q}\mat{K}^T$
    \item Normaliser par $\sqrt{d_k}$ pour éviter valeurs trop grandes
    \item Appliquer softmax pour obtenir poids d'attention
    \item Pondérer les valeurs $\mat{V}$
\end{enumerate}

% ===== SECTION 7: TRANSFORMERS =====
\section{Transformers}

\subsection{Motivation}

\textbf{Limites des RNN :}
\begin{itemize}
    \item ❌ Traitement séquentiel (pas de parallélisation)
    \item ❌ Difficulté avec dépendances à très long terme
    \item ❌ Lent à entraîner
\end{itemize}

\textbf{Transformer (Vaswani et al., 2017) :}
\begin{itemize}
    \item ✅ Uniquement basé sur l'attention (pas de récurrence)
    \item ✅ Complètement parallélisable
    \item ✅ Capture dépendances à longue distance facilement
    \item ✅ State-of-the-art en NLP (BERT, GPT, T5, etc.)
\end{itemize}

\subsection{Architecture globale}

\textbf{Encoder-Decoder} avec 6 couches chacun (original paper) :

\begin{itemize}
    \item \textbf{Encoder} : Traite la séquence d'entrée
    \item \textbf{Decoder} : Génère la séquence de sortie (autorégressif)
\end{itemize}

\subsection{Multi-Head Attention}

\begin{definition}{Multi-Head Attention}
Au lieu d'une seule attention, on calcule $h$ attentions en parallèle avec des projections différentes :
\begin{align}
    \text{head}_i &= \text{Attention}(\mat{Q}\mat{W}_i^Q, \mat{K}\mat{W}_i^K, \mat{V}\mat{W}_i^V) \\
    \text{MultiHead}(\mat{Q}, \mat{K}, \mat{V}) &= \text{Concat}(\text{head}_1, \ldots, \text{head}_h) \mat{W}^O
\end{align}
où $\mat{W}_i^Q, \mat{W}_i^K, \mat{W}_i^V$ sont des matrices de projection apprenables.
\end{definition}

\textbf{Avantages :}
\begin{itemize}
    \item Chaque tête peut se concentrer sur des aspects différents (syntaxe, sémantique, etc.)
    \item Plus expressif qu'une seule attention
\end{itemize}

\subsection{Positional Encoding}

\textbf{Problème :} Sans récurrence, le Transformer ne connaît pas l'ordre des mots !

\textbf{Solution :} Ajouter un \textbf{encodage de position} à chaque embedding :
\begin{align}
    PE_{(pos, 2i)} &= \sin\left(\frac{pos}{10000^{2i/d}}\right) \\
    PE_{(pos, 2i+1)} &= \cos\left(\frac{pos}{10000^{2i/d}}\right)
\end{align}
où $pos$ est la position et $i$ la dimension.

\begin{equation}
    \vect{x}_{\text{input}} = \text{Embedding}(\text{token}) + \text{PositionalEncoding}(pos)
\end{equation}

\subsection{Encoder Layer}

Une couche encoder contient :
\begin{enumerate}
    \item \textbf{Multi-Head Self-Attention}
    \item \textbf{Add \& Norm} (Residual connection + Layer Normalization)
    \item \textbf{Feed-Forward Network} (2 couches FC avec ReLU)
    \item \textbf{Add \& Norm}
\end{enumerate}

\begin{align}
    \vect{z} &= \text{LayerNorm}(\vect{x} + \text{MultiHeadAttention}(\vect{x}, \vect{x}, \vect{x})) \\
    \text{output} &= \text{LayerNorm}(\vect{z} + \text{FFN}(\vect{z}))
\end{align}

\subsection{Decoder Layer}

Une couche decoder contient :
\begin{enumerate}
    \item \textbf{Masked Multi-Head Self-Attention} (ne voit que le passé)
    \item \textbf{Add \& Norm}
    \item \textbf{Multi-Head Cross-Attention} (attention sur l'encodeur)
    \item \textbf{Add \& Norm}
    \item \textbf{Feed-Forward Network}
    \item \textbf{Add \& Norm}
\end{enumerate}

\textbf{Masking :} En génération, le décodeur ne doit voir que les tokens précédents (autorégressif).

\subsection{Implémentation simplifiée}

\begin{lstlisting}[language=Python, caption=Scaled Dot-Product Attention]
import torch
import torch.nn.functional as F

def scaled_dot_product_attention(Q, K, V, mask=None):
    """
    Q, K, V: (batch, seq_len, d_k)
    """
    d_k = Q.size(-1)

    # Scores: (batch, seq_len, seq_len)
    scores = torch.matmul(Q, K.transpose(-2, -1)) / torch.sqrt(torch.tensor(d_k, dtype=torch.float32))

    # Masking (optionnel)
    if mask is not None:
        scores = scores.masked_fill(mask == 0, -1e9)

    # Softmax
    attention_weights = F.softmax(scores, dim=-1)

    # Weighted sum
    output = torch.matmul(attention_weights, V)

    return output, attention_weights
\end{lstlisting}

\begin{lstlisting}[language=Python, caption=Multi-Head Attention]
class MultiHeadAttention(nn.Module):
    def __init__(self, d_model, num_heads):
        super(MultiHeadAttention, self).__init__()
        assert d_model % num_heads == 0

        self.d_model = d_model
        self.num_heads = num_heads
        self.d_k = d_model // num_heads

        # Linear projections
        self.W_q = nn.Linear(d_model, d_model)
        self.W_k = nn.Linear(d_model, d_model)
        self.W_v = nn.Linear(d_model, d_model)
        self.W_o = nn.Linear(d_model, d_model)

    def split_heads(self, x):
        """Split into multiple heads"""
        batch_size, seq_len, d_model = x.size()
        return x.view(batch_size, seq_len, self.num_heads, self.d_k).transpose(1, 2)

    def forward(self, Q, K, V, mask=None):
        batch_size = Q.size(0)

        # Linear projections
        Q = self.split_heads(self.W_q(Q))  # (batch, num_heads, seq_len, d_k)
        K = self.split_heads(self.W_k(K))
        V = self.split_heads(self.W_v(V))

        # Scaled dot-product attention
        attn_output, _ = scaled_dot_product_attention(Q, K, V, mask)

        # Concatenate heads
        attn_output = attn_output.transpose(1, 2).contiguous().view(
            batch_size, -1, self.d_model
        )

        # Final linear
        output = self.W_o(attn_output)
        return output
\end{lstlisting}

\begin{lstlisting}[language=Python, caption=Transformer Encoder Layer]
class TransformerEncoderLayer(nn.Module):
    def __init__(self, d_model, num_heads, d_ff, dropout=0.1):
        super(TransformerEncoderLayer, self).__init__()

        self.self_attn = MultiHeadAttention(d_model, num_heads)
        self.feed_forward = nn.Sequential(
            nn.Linear(d_model, d_ff),
            nn.ReLU(),
            nn.Linear(d_ff, d_model)
        )

        self.norm1 = nn.LayerNorm(d_model)
        self.norm2 = nn.LayerNorm(d_model)
        self.dropout = nn.Dropout(dropout)

    def forward(self, x, mask=None):
        # Self-attention + residual + norm
        attn_output = self.self_attn(x, x, x, mask)
        x = self.norm1(x + self.dropout(attn_output))

        # Feed-forward + residual + norm
        ff_output = self.feed_forward(x)
        x = self.norm2(x + self.dropout(ff_output))

        return x
\end{lstlisting}

% ===== SECTION 8: APPLICATIONS NLP =====
\section{Applications NLP}

\subsection{Modèles pré-entraînés}

\subsubsection{BERT (Bidirectional Encoder Representations from Transformers)}

\textbf{Architecture :} Encoder Transformer (12 ou 24 couches)

\textbf{Pré-entraînement :}
\begin{itemize}
    \item \textbf{Masked Language Modeling (MLM)} : Prédire les mots masqués
    \item \textbf{Next Sentence Prediction (NSP)} : Prédire si phrase B suit phrase A
\end{itemize}

\textbf{Fine-tuning :} Classification, NER, Q\&A, etc.

\subsubsection{GPT (Generative Pre-trained Transformer)}

\textbf{Architecture :} Decoder Transformer (autorégressif)

\textbf{Pré-entraînement :} Language modeling (prédire token suivant)

\textbf{Modèles :}
\begin{itemize}
    \item GPT-2 (1.5B params)
    \item GPT-3 (175B params)
    \item GPT-4 (1.76T params estimé)
\end{itemize}

\subsubsection{T5, BART, RoBERTa, etc.}

Nombreuses variantes avec différentes stratégies de pré-entraînement.

\subsection{Utilisation avec HuggingFace Transformers}

\begin{lstlisting}[language=Python, caption=BERT pour classification]
from transformers import BertTokenizer, BertForSequenceClassification
import torch

# Charger modèle pré-entraîné
tokenizer = BertTokenizer.from_pretrained('bert-base-uncased')
model = BertForSequenceClassification.from_pretrained(
    'bert-base-uncased',
    num_labels=2  # Binary classification
)

# Texte à classifier
text = "This movie is fantastic!"
inputs = tokenizer(text, return_tensors='pt', padding=True, truncation=True)

# Forward
outputs = model(**inputs)
logits = outputs.logits
probs = torch.softmax(logits, dim=-1)

print(f"Positive: {probs[0, 1]:.2f}")
print(f"Negative: {probs[0, 0]:.2f}")
\end{lstlisting}

\begin{lstlisting}[language=Python, caption=GPT-2 pour génération de texte]
from transformers import GPT2LMHeadModel, GPT2Tokenizer

tokenizer = GPT2Tokenizer.from_pretrained('gpt2')
model = GPT2LMHeadModel.from_pretrained('gpt2')

# Prompt
prompt = "Once upon a time"
inputs = tokenizer.encode(prompt, return_tensors='pt')

# Génération
outputs = model.generate(
    inputs,
    max_length=50,
    num_return_sequences=1,
    temperature=0.7,
    top_p=0.9,
    do_sample=True
)

generated_text = tokenizer.decode(outputs[0], skip_special_tokens=True)
print(generated_text)
\end{lstlisting}

% ===== SECTION 9: SÉRIES TEMPORELLES =====
\section{Séries Temporelles}

\subsection{Prédiction}

\begin{lstlisting}[language=Python, caption=LSTM pour prédiction de séries temporelles]
import numpy as np
import torch
import torch.nn as nn

# Générer série temporelle synthétique
t = np.linspace(0, 100, 1000)
data = np.sin(t) + 0.1 * np.random.randn(1000)

# Créer séquences (window = 50)
def create_sequences(data, window=50):
    X, y = [], []
    for i in range(len(data) - window):
        X.append(data[i:i+window])
        y.append(data[i+window])
    return np.array(X), np.array(y)

X, y = create_sequences(data, window=50)
X = torch.FloatTensor(X).unsqueeze(-1)  # (N, 50, 1)
y = torch.FloatTensor(y)

# Modèle LSTM
class TimeSeriesLSTM(nn.Module):
    def __init__(self, input_dim=1, hidden_dim=64, num_layers=2):
        super(TimeSeriesLSTM, self).__init__()
        self.lstm = nn.LSTM(input_dim, hidden_dim, num_layers, batch_first=True)
        self.fc = nn.Linear(hidden_dim, 1)

    def forward(self, x):
        out, _ = self.lstm(x)
        out = self.fc(out[:, -1, :])  # Dernier timestep
        return out.squeeze()

model = TimeSeriesLSTM()
criterion = nn.MSELoss()
optimizer = torch.optim.Adam(model.parameters(), lr=0.001)

# Training
num_epochs = 100
for epoch in range(num_epochs):
    model.train()
    optimizer.zero_grad()

    outputs = model(X)
    loss = criterion(outputs, y)

    loss.backward()
    optimizer.step()

    if (epoch + 1) % 10 == 0:
        print(f'Epoch [{epoch+1}/{num_epochs}], Loss: {loss.item():.4f}')
\end{lstlisting}

% ===== SECTION 10: BONNES PRATIQUES =====
\section{Bonnes Pratiques}

\subsection{RNN/LSTM/GRU}

\begin{itemize}
    \item \textbf{Gradient clipping} : Limiter norme du gradient (5-10)
    \item \textbf{Layer Normalization} : Stabilise l'entraînement
    \item \textbf{Dropout} : Sur les connexions non-récurrentes uniquement
    \item \textbf{Teacher Forcing} : En seq2seq, utiliser vraies sorties pendant entraînement
    \item \textbf{Beam Search} : Pour génération (au lieu de greedy)
\end{itemize}

\subsection{Transformers}

\begin{itemize}
    \item \textbf{Learning Rate Warmup} : Augmenter LR progressivement au début
    \item \textbf{Label Smoothing} : Régularisation pour classification
    \item \textbf{Dropout} : Sur attention et FFN
    \item \textbf{Layer Normalization} : Avant ou après chaque sous-couche
    \item \textbf{Gradient Accumulation} : Pour simuler grands batch sizes
\end{itemize}

% ===== SECTION 11: RÉSUMÉ =====
\section{Résumé du Chapitre}

\subsection{Points Clés}

\begin{itemize}
    \item \textbf{RNN} : État caché récurrent, vanishing/exploding gradient
    \item \textbf{LSTM} : 3 portes (forget, input, output) + cellule mémoire
    \item \textbf{GRU} : Variante simplifiée avec 2 portes
    \item \textbf{Bidirectional} : Contexte passé + futur
    \item \textbf{Attention} : Pondération dynamique des inputs
    \item \textbf{Transformer} : Uniquement attention, parallélisable, state-of-the-art NLP
    \item \textbf{BERT} : Encoder pré-entraîné (MLM)
    \item \textbf{GPT} : Decoder autorégressif (LM)
\end{itemize}

\subsection{Formules Essentielles}

\begin{tcolorbox}[colback=blue!5!white, colframe=blue!75!black, title=Formules à retenir]
\textbf{RNN :}
\begin{equation*}
    \vect{h}_t = \tanh(\mat{W}_{hh} \vect{h}_{t-1} + \mat{W}_{xh} \vect{x}_t + \vect{b})
\end{equation*}

\textbf{LSTM (simplifié) :}
\begin{align*}
    \vect{c}_t &= \vect{f}_t \odot \vect{c}_{t-1} + \vect{i}_t \odot \tilde{\vect{c}}_t \\
    \vect{h}_t &= \vect{o}_t \odot \tanh(\vect{c}_t)
\end{align*}

\textbf{Scaled Dot-Product Attention :}
\begin{equation*}
    \text{Attention}(\mat{Q}, \mat{K}, \mat{V}) = \softmax\left(\frac{\mat{Q}\mat{K}^T}{\sqrt{d_k}}\right) \mat{V}
\end{equation*}
\end{tcolorbox}

% ===== SECTION 12: EXERCICES =====
\section{Exercices}

\subsection{Questions de compréhension}

\begin{enumerate}
    \item Pourquoi le RNN vanilla souffre-t-il du vanishing gradient ?
    \item Expliquer le rôle de chaque porte dans un LSTM.
    \item Quelle est la différence principale entre LSTM et GRU ?
    \item Pourquoi utiliser un Bidirectional RNN pour le NER ?
    \item Comment le mécanisme d'attention résout-il le problème du goulot d'étranglement ?
    \item Pourquoi le Transformer a-t-il besoin de positional encoding ?
\end{enumerate}

\subsection{Exercices pratiques}

\begin{enumerate}
    \item \textbf{Prédiction de séries temporelles}
    \begin{itemize}
        \item Implémenter un LSTM pour prédire une série temporelle
        \item Comparer LSTM vs GRU
        \item Visualiser les prédictions
    \end{itemize}

    \item \textbf{Sentiment Analysis}
    \begin{itemize}
        \item Fine-tuner BERT sur un dataset de reviews (IMDB, Yelp)
        \item Comparer avec un LSTM from scratch
    \end{itemize}

    \item \textbf{Génération de texte}
    \begin{itemize}
        \item Entraîner un RNN character-level sur Shakespeare
        \item Générer du texte avec temperature sampling
    \end{itemize}

    \item \textbf{Attention Visualization}
    \begin{itemize}
        \item Implémenter attention de Bahdanau
        \item Visualiser les poids d'attention sur une tâche de traduction
    \end{itemize}
\end{enumerate}

\textit{Solutions disponibles dans} \texttt{08\_exercices.ipynb} \textit{(solutions intégrées dans le notebook)}

% ===== SECTION: NOTEBOOKS PRATIQUES =====
\section{Notebooks Pratiques}

Ce chapitre est accompagné des notebooks suivants :

\begin{itemize}
    \item \texttt{08\_demo\_lstm\_sentiment.ipynb} : Analyse de sentiment avec LSTM
    \begin{itemize}
        \item LSTM bidirectionnel pour classification de texte
        \item Dataset IMDB Reviews (sentiment positif/négatif)
        \item Preprocessing et tokenization
        \item Training avec PyTorch et visualisation des résultats
    \end{itemize}

    \item \texttt{08\_demo\_transformers\_huggingface.ipynb} : Transformers avec Hugging Face
    \begin{itemize}
        \item Fine-tuning de BERT pour classification de texte
        \item Génération de texte avec GPT-2
        \item Utilisation de la bibliothèque Transformers
        \item Tokenization et gestion des modèles pré-entraînés
    \end{itemize}

    \item \texttt{08\_demo\_rag\_llm.ipynb} : RAG et LLMs avancés \textbf{(NOUVEAU)}
    \begin{itemize}
        \item Introduction au Retrieval-Augmented Generation (RAG)
        \item Embeddings avec Sentence-BERT et recherche vectorielle (FAISS)
        \item Pipeline RAG complet : Retrieval + Context + Generation
        \item Chunking et preprocessing de documents longs
        \item Techniques avancées : Reranking avec CrossEncoder
        \item Hybrid Search : combinaison BM25 (keyword) + Dense embeddings
        \item Évaluation du RAG : Precision@K, Recall@K
        \item Visualisation des embeddings avec PCA
        \item Applications pratiques : chatbots, QA systems, documentation
    \end{itemize}

    \item \texttt{08\_exercices.ipynb} : Exercices pratiques avec solutions intégrées
    \begin{itemize}
        \item Implémentation de RNN vanilla et LSTM from scratch
        \item Seq2Seq avec attention pour traduction
        \item Génération de texte avec temperature sampling
        \item Attention visualization
    \end{itemize}
\end{itemize}

% ===== SECTION 13: POUR ALLER PLUS LOIN =====
\section{Pour Aller Plus Loin}

\subsection{Lectures Recommandées}

\begin{itemize}
    \item Hochreiter \& Schmidhuber (1997) - "Long Short-Term Memory"
    \item Bahdanau et al. (2014) - "Neural Machine Translation by Jointly Learning to Align and Translate"
    \item Vaswani et al. (2017) - "Attention Is All You Need" (Transformer)
    \item Devlin et al. (2018) - "BERT: Pre-training of Deep Bidirectional Transformers"
    \item Radford et al. (2019) - "Language Models are Unsupervised Multitask Learners" (GPT-2)
\end{itemize}

\subsection{Ressources}

\begin{itemize}
    \item The Illustrated Transformer : \url{https://jalammar.github.io/illustrated-transformer/}
    \item HuggingFace Transformers : \url{https://huggingface.co/docs/transformers/}
    \item Sequence Models (Coursera) : Andrew Ng
    \item Annotated Transformer : \url{https://nlp.seas.harvard.edu/annotated-transformer/}
\end{itemize}

\subsection{Prochaines Étapes}

Chapitre suivant recommandé : \textbf{Chapitre 09 - Reinforcement Learning}

Le reinforcement learning permet d'entraîner des agents à prendre des décisions séquentielles pour maximiser une récompense.

% ===== BIBLIOGRAPHIE =====
\section*{Références}

\begin{enumerate}
    \item Hochreiter, S., \& Schmidhuber, J. (1997). "Long short-term memory". \textit{Neural computation}, 9(8), 1735-1780.
    \item Cho, K., et al. (2014). "Learning phrase representations using RNN encoder-decoder for statistical machine translation". \textit{EMNLP}.
    \item Bahdanau, D., Cho, K., \& Bengio, Y. (2014). "Neural machine translation by jointly learning to align and translate". \textit{arXiv:1409.0473}.
    \item Vaswani, A., et al. (2017). "Attention is all you need". \textit{NIPS}.
    \item Devlin, J., et al. (2018). "BERT: Pre-training of deep bidirectional transformers for language understanding". \textit{arXiv:1810.04805}.
    \item Radford, A., et al. (2019). "Language models are unsupervised multitask learners". \textit{OpenAI blog}.
\end{enumerate}

\end{document}
