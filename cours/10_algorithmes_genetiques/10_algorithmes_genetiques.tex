\documentclass[11pt,a4paper]{article}

% ===== PACKAGES =====
\usepackage[utf8]{inputenc}
\usepackage[T1]{fontenc}
\usepackage[french]{babel}
\usepackage{lmodern}
\usepackage{amsmath, amssymb, amsthm}
\usepackage{mathtools}
\usepackage[margin=2.5cm]{geometry}
\usepackage{parskip}
\usepackage{setspace}
\setstretch{1.15}
\usepackage{graphicx}
\usepackage{xcolor}

% ===== UNICODE CHARACTERS SUPPORT =====
\usepackage{newunicodechar}

% Emojis et symboles
\newunicodechar{✅}{\textcolor{green!60!black}{$\checkmark$}}
\newunicodechar{❌}{\textcolor{red!60!black}{$\times$}}
\newunicodechar{✓}{\textcolor{green!60!black}{$\checkmark$}}
\newunicodechar{✗}{\textcolor{red!60!black}{$\times$}}
\newunicodechar{⚠}{\textcolor{orange!80!black}{\textbf{/!\textbackslash}}}
\newunicodechar{💡}{\textcolor{blue!70!black}{\textbf{(i)}}}
\newunicodechar{🎯}{\textcolor{purple!70!black}{\textbf{$\star$}}}
\newunicodechar{📊}{\textcolor{blue!70!black}{\textbf{[=]}}}

% Étoiles (pour tableaux)
\newunicodechar{★}{\textcolor{orange!80!black}{$\star$}}
\newunicodechar{☆}{\textcolor{gray!50}{$\star$}}

% Flèches
\newunicodechar{→}{$\rightarrow$}
\newunicodechar{←}{$\leftarrow$}
\newunicodechar{↑}{$\uparrow$}
\newunicodechar{↓}{$\downarrow$}
\usepackage{booktabs}
\usepackage{listings}
\usepackage{algorithm}
\usepackage{algorithmic}
\usepackage{hyperref}
\hypersetup{colorlinks=true, linkcolor=blue, pdftitle={Chapitre 10 - Algorithmes Génétiques}}
\usepackage{tcolorbox}
\tcbuselibrary{skins, breakable}
\usepackage{fancyhdr}
\pagestyle{fancy}
\fancyhf{}
\fancyhead[L]{\small Chapitre 10 - Algorithmes Génétiques}
\fancyhead[R]{\small Cours Machine Learning}
\fancyfoot[C]{\thepage}

\definecolor{codegreen}{rgb}{0,0.6,0}
\definecolor{backcolour}{rgb}{0.95,0.95,0.92}

\lstdefinestyle{pythonstyle}{
    language=Python,
    backgroundcolor=\color{backcolour},
    commentstyle=\color{codegreen},
    keywordstyle=\color{blue},
    basicstyle=\ttfamily\small,
    breaklines=true,
    numbers=left,
    frame=single,
}
\lstset{style=pythonstyle}

\newtcolorbox{definition}[1]{colback=blue!5!white, colframe=blue!75!black, fonttitle=\bfseries, title=Définition: #1, breakable}
\newtcolorbox{exemple}[1]{colback=orange!5!white, colframe=orange!75!black, fonttitle=\bfseries, title=Exemple: #1, breakable}
\newtcolorbox{astuce}{colback=yellow!10!white, colframe=yellow!75!black, fonttitle=\bfseries, title=💡 Astuce, breakable}

\newcommand{\vect}[1]{\mathbf{#1}}
\newcommand{\R}{\mathbb{R}}

\begin{document}

\begin{titlepage}
    \centering
    \vspace*{2cm}
    {\Huge\bfseries Cours Machine Learning}\\[0.5cm]
    \vspace{1cm}
    {\LARGE Chapitre 10}\\[0.3cm]
    {\LARGE\bfseries Algorithmes Génétiques et Optimisation Évolutionnaire}\\[2cm]
    \vfill
    {\large
    \textbf{Objectifs d'apprentissage :}\\[0.5cm]
    \begin{itemize}
        \item Comprendre les algorithmes génétiques (AG)
        \item Maîtriser sélection, crossover, mutation
        \item Appliquer les AG à des problèmes d'optimisation
        \item Découvrir les variantes modernes
    \end{itemize}
    }
    \vfill
    {\large \textbf{Durée estimée :} 4-6 heures}
    \vfill
    {\large Cours ML - Sandbox-ML\\ Version 1.0 - 2026}
\end{titlepage}

\tableofcontents
\newpage

\section{Introduction}

\subsection{Inspiration biologique}

Les \textbf{algorithmes génétiques (AG)} s'inspirent de l'évolution naturelle :
\begin{itemize}
    \item \textbf{Population} : Ensemble de solutions candidates
    \item \textbf{Fitness} : Qualité d'une solution
    \item \textbf{Sélection} : Les meilleurs survivent
    \item \textbf{Crossover} : Reproduction (combinaison de solutions)
    \item \textbf{Mutation} : Variation aléatoire
\end{itemize}

\subsection{Quand utiliser les AG ?}

\begin{itemize}
    \item Espace de recherche discret, combinatoire
    \item Fonction objectif non différentiable
    \item Optimisation multi-objectifs
    \item Pas de méthode analytique disponible
\end{itemize}

\section{Algorithme Génétique Standard}

\subsection{Représentation}

\textbf{Chromosome} : Encodage d'une solution

\begin{exemple}{Encodages courants}
\begin{itemize}
    \item \textbf{Binaire} : $[1, 0, 1, 1, 0]$
    \item \textbf{Réel} : $[2.5, -1.3, 0.8]$
    \item \textbf{Permutation} : $[3, 1, 4, 2]$ (TSP)
\end{itemize}
\end{exemple}

\subsection{Algorithme principal}

\begin{algorithm}[H]
\caption{Algorithme Génétique}
\begin{algorithmic}[1]
\STATE Initialiser population $P$ aléatoirement
\STATE Évaluer fitness de chaque individu
\WHILE{critère d'arrêt non atteint}
    \STATE \textbf{Sélection} : Choisir parents selon fitness
    \STATE \textbf{Crossover} : Créer enfants par recombinaison
    \STATE \textbf{Mutation} : Appliquer mutations aléatoires
    \STATE \textbf{Évaluation} : Calculer fitness des enfants
    \STATE \textbf{Remplacement} : Nouvelle génération
\ENDWHILE
\RETURN Meilleur individu
\end{algorithmic}
\end{algorithm}

\subsection{Opérateurs de sélection}

\textbf{1. Roulette Wheel (Proportionnelle)}
\begin{equation}
    P(\text{sélectionner } i) = \frac{f_i}{\sum_{j=1}^N f_j}
\end{equation}

\textbf{2. Tournament Selection}
\begin{itemize}
    \item Choisir $k$ individus aléatoirement
    \item Sélectionner le meilleur
\end{itemize}

\textbf{3. Rank-Based}
\begin{itemize}
    \item Trier par fitness
    \item Probabilité basée sur le rang
\end{itemize}

\subsection{Crossover (Recombinaison)}

\textbf{One-Point Crossover :}
\begin{verbatim}
Parent 1: [1 1 0 | 1 0 1]
Parent 2: [0 1 1 | 0 1 0]
          -------+-------
Enfant 1: [1 1 0 | 0 1 0]
Enfant 2: [0 1 1 | 1 0 1]
\end{verbatim}

\textbf{Uniform Crossover :}
Chaque gène a 50\% de chance de venir de chaque parent.

\textbf{Arithmetic (réels) :}
\begin{equation}
    \text{enfant} = \alpha \cdot \text{parent}_1 + (1-\alpha) \cdot \text{parent}_2
\end{equation}

\subsection{Mutation}

\textbf{Bit Flip (binaire) :}
\begin{equation}
    [1, 0, 1, 0] \xrightarrow{\text{mutation}} [1, 1, 1, 0]
\end{equation}

\textbf{Gaussian (réel) :}
\begin{equation}
    x' = x + \mathcal{N}(0, \sigma^2)
\end{equation}

\textbf{Taux de mutation :} Typiquement $p_m = 1/L$ où $L$ est la longueur du chromosome.

\section{Implémentation}

\begin{lstlisting}[caption=AG simple en Python]
import numpy as np

class GeneticAlgorithm:
    def __init__(self, fitness_func, pop_size=100, chrom_length=10,
                 crossover_rate=0.8, mutation_rate=0.01):
        self.fitness_func = fitness_func
        self.pop_size = pop_size
        self.chrom_length = chrom_length
        self.crossover_rate = crossover_rate
        self.mutation_rate = mutation_rate

    def initialize_population(self):
        return np.random.randint(0, 2, (self.pop_size, self.chrom_length))

    def evaluate(self, population):
        return np.array([self.fitness_func(ind) for ind in population])

    def selection(self, population, fitness):
        # Tournament selection
        selected = []
        for _ in range(self.pop_size):
            i, j = np.random.choice(self.pop_size, 2, replace=False)
            winner = i if fitness[i] > fitness[j] else j
            selected.append(population[winner].copy())
        return np.array(selected)

    def crossover(self, parent1, parent2):
        if np.random.rand() > self.crossover_rate:
            return parent1.copy(), parent2.copy()

        point = np.random.randint(1, self.chrom_length)
        child1 = np.concatenate([parent1[:point], parent2[point:]])
        child2 = np.concatenate([parent2[:point], parent1[point:]])
        return child1, child2

    def mutate(self, chromosome):
        for i in range(len(chromosome)):
            if np.random.rand() < self.mutation_rate:
                chromosome[i] = 1 - chromosome[i]
        return chromosome

    def run(self, generations=100):
        population = self.initialize_population()
        best_fitness_history = []

        for gen in range(generations):
            fitness = self.evaluate(population)
            best_fitness_history.append(fitness.max())

            if gen % 10 == 0:
                print(f"Gen {gen}: Best fitness = {fitness.max():.4f}")

            # Selection
            parents = self.selection(population, fitness)

            # Crossover + Mutation
            offspring = []
            for i in range(0, self.pop_size, 2):
                child1, child2 = self.crossover(parents[i], parents[i+1])
                offspring.append(self.mutate(child1))
                offspring.append(self.mutate(child2))

            population = np.array(offspring[:self.pop_size])

        # Résultat final
        final_fitness = self.evaluate(population)
        best_idx = final_fitness.argmax()
        return population[best_idx], final_fitness[best_idx], best_fitness_history

# Exemple : Maximiser nombre de 1
def fitness(chromosome):
    return chromosome.sum()

ga = GeneticAlgorithm(fitness, pop_size=50, chrom_length=20)
best, best_fit, history = ga.run(generations=50)
print(f"\nMeilleure solution: {best}")
print(f"Fitness: {best_fit}")
\end{lstlisting}

\section{Applications}

\subsection{Problème du voyageur de commerce (TSP)}

Trouver le chemin le plus court visitant toutes les villes.

\begin{itemize}
    \item \textbf{Chromosome} : Permutation des villes
    \item \textbf{Fitness} : $1 / \text{distance totale}$
    \item \textbf{Crossover} : Order Crossover (OX)
    \item \textbf{Mutation} : Swap, Inversion
\end{itemize}

\subsection{Optimisation de fonctions}

Minimiser $f(x_1, x_2) = x_1^2 + x_2^2 - 10\cos(2\pi x_1) - 10\cos(2\pi x_2)$ (Rastrigin)

\subsection{Hyperparameter Tuning}

Optimiser hyperparamètres d'un ML model (alternative à Grid/Random Search).

\subsection{Neuroévolution}

Optimiser poids et architecture de réseaux de neurones.

\section{Variantes modernes}

\subsection{CMA-ES (Covariance Matrix Adaptation)}

Algorithme évolutionnaire pour optimisation continue, très performant.

\subsection{NSGA-II (Non-dominated Sorting GA)}

Pour optimisation multi-objectifs (Pareto-optimal).

\subsection{Differential Evolution}

Mutation basée sur différences entre individus :
\begin{equation}
    \vect{v} = \vect{x}_r + F(\vect{x}_a - \vect{x}_b)
\end{equation}

\section{Résumé}

\subsection{Points Clés}

\begin{itemize}
    \item AG = Métaheuristique inspirée de l'évolution
    \item Opérateurs : Sélection, Crossover, Mutation
    \item Bon pour optimisation combinatoire, non différentiable
    \item Variantes : CMA-ES, NSGA-II, Differential Evolution
\end{itemize}

\subsection{Avantages et Limites}

\textbf{Avantages :}
\begin{itemize}
    \item Pas besoin de gradient
    \item Exploration globale
    \item Flexible (encodages variés)
\end{itemize}

\textbf{Limites :}
\begin{itemize}
    \item Pas de garantie de convergence
    \item Lent comparé à méthodes gradient
    \item Nombreux hyperparamètres
\end{itemize}

\section{Pour Aller Plus Loin}

\subsection{Bibliothèques Python}
\begin{itemize}
    \item DEAP : Framework AG complet
    \item PyGAD : Simple et pédagogique
    \item Optuna : Hyperparameter tuning
\end{itemize}

\subsection{Lectures}
\begin{itemize}
    \item Goldberg - \textit{Genetic Algorithms in Search, Optimization and Machine Learning}
    \item Eiben \& Smith - \textit{Introduction to Evolutionary Computing}
\end{itemize}

\section{Notebooks Pratiques}

Ce chapitre est accompagné des notebooks suivants :

\begin{itemize}
    \item \texttt{10\_demo\_ag\_base.ipynb} : Introduction aux algorithmes génétiques
    \begin{itemize}
        \item Implémentation AG from scratch
        \item Optimisation de fonctions mathématiques
        \item Résolution du problème du voyageur de commerce (TSP)
        \item Visualisation de l'évolution des populations
    \end{itemize}

    \item \texttt{10\_demo\_applications.ipynb} : Applications pratiques des AG
    \begin{itemize}
        \item Hyperparameter tuning avec AG
        \item Feature selection
        \item Comparaison avec Grid Search et Random Search
        \item Cas d'usage sur datasets réels
    \end{itemize}
\end{itemize}

\end{document}
