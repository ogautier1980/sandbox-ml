\documentclass[11pt,a4paper]{article}

% ===== PACKAGES =====
\usepackage[utf8]{inputenc}
\usepackage[T1]{fontenc}
\usepackage[french]{babel}
\usepackage{lmodern}

% Mathématiques
\usepackage{amsmath, amssymb, amsthm}
\usepackage{mathtools}

% Mise en page
\usepackage[margin=2.5cm]{geometry}
\usepackage{parskip}
\usepackage{setspace}
\setstretch{1.15}

% Graphiques et couleurs
\usepackage{graphicx}
\usepackage{xcolor}

% ===== UNICODE CHARACTERS SUPPORT =====
\usepackage{newunicodechar}

% Emojis et symboles
\newunicodechar{✅}{\textcolor{green!60!black}{$\checkmark$}}
\newunicodechar{❌}{\textcolor{red!60!black}{$\times$}}
\newunicodechar{✓}{\textcolor{green!60!black}{$\checkmark$}}
\newunicodechar{✗}{\textcolor{red!60!black}{$\times$}}
\newunicodechar{⚠}{\textcolor{orange!80!black}{\textbf{/!\textbackslash}}}
\newunicodechar{💡}{\textcolor{blue!70!black}{\textbf{(i)}}}
\newunicodechar{🎯}{\textcolor{purple!70!black}{\textbf{$\star$}}}
\newunicodechar{📊}{\textcolor{blue!70!black}{\textbf{[=]}}}

% Étoiles (pour tableaux)
\newunicodechar{★}{\textcolor{orange!80!black}{$\star$}}
\newunicodechar{☆}{\textcolor{gray!50}{$\star$}}

% Flèches
\newunicodechar{→}{$\rightarrow$}
\newunicodechar{←}{$\leftarrow$}
\newunicodechar{↑}{$\uparrow$}
\newunicodechar{↓}{$\downarrow$}

\usepackage{tikz}
\usetikzlibrary{arrows.meta, positioning, shapes.geometric, calc}

% Tableaux
\usepackage{booktabs}
\usepackage{longtable}
\usepackage{multirow}
\usepackage{tabularx}
\usepackage{colortbl}

% Code et algorithmes
\usepackage{listings}
\usepackage{algorithm}
\usepackage{algorithmic}

% Hyperliens
\usepackage{hyperref}
\hypersetup{
    colorlinks=true,
    linkcolor=blue,
    filecolor=magenta,
    urlcolor=cyan,
    citecolor=green,
    pdftitle={Chapitre 06 - Réseaux de Neurones Fondamentaux},
    pdfauthor={Cours ML},
}

% Boxes colorées
\usepackage{tcolorbox}
\tcbuselibrary{skins, breakable}


% ===== TCOLORBOX AVEC EMOJIS =====
\newtcolorbox{attention}{
    colback=red!5!white,
    colframe=red!75!black,
    fonttitle=\bfseries,
    title=⚠ Attention,
    breakable
}

\newtcolorbox{definition}{
    colback=blue!5!white,
    colframe=blue!75!black,
    fonttitle=\bfseries,
    title=Définition,
    breakable
}

\newtcolorbox{astuce}{
    colback=green!5!white,
    colframe=green!60!black,
    fonttitle=\bfseries,
    title=💡 Astuce,
    breakable
}

\newtcolorbox{remarque}{
    colback=yellow!5!white,
    colframe=orange!75!black,
    fonttitle=\bfseries,
    title=💡 Remarque,
    breakable
}

\newtcolorbox{important}{
    colback=purple!5!white,
    colframe=purple!75!black,
    fonttitle=\bfseries,
    title=⚠ Important,
    breakable
}

\newtcolorbox{exemple}{
    colback=gray!5!white,
    colframe=gray!75!black,
    fonttitle=\bfseries,
    title=Exemple,
    breakable
}

% En-têtes et pieds de page
\usepackage{fancyhdr}
\pagestyle{fancy}
\fancyhf{}
\fancyhead[L]{\small Chapitre 06 - Réseaux de Neurones Fondamentaux}
\fancyhead[R]{\small Cours Machine Learning}
\fancyfoot[C]{\thepage}

% ===== CONFIGURATION LISTINGS (code Python) =====
\definecolor{codegreen}{rgb}{0,0.6,0}
\definecolor{codegray}{rgb}{0.5,0.5,0.5}
\definecolor{codepurple}{rgb}{0.58,0,0.82}
\definecolor{backcolour}{rgb}{0.95,0.95,0.92}

\lstdefinestyle{pythonstyle}{
    language=Python,
    backgroundcolor=\color{backcolour},
    commentstyle=\color{codegreen},
    keywordstyle=\color{blue},
    numberstyle=\tiny\color{codegray},
    stringstyle=\color{codepurple},
    basicstyle=\ttfamily\small,
    breakatwhitespace=false,
    breaklines=true,
    captionpos=b,
    keepspaces=true,
    numbers=left,
    numbersep=5pt,
    showspaces=false,
    showstringspaces=false,
    showtabs=false,
    tabsize=4,
    frame=single,
    rulecolor=\color{black}
}
\lstset{style=pythonstyle}

% ===== CONFIGURATION TCOLORBOX =====


\newtcolorbox{theoreme}[1]{
    colback=green!5!white,
    colframe=green!75!black,
    fonttitle=\bfseries,
    title=Théorème: #1,
    breakable
}







% ===== COMMANDES PERSONNALISÉES =====
\newcommand{\vect}[1]{\mathbf{#1}}
\newcommand{\mat}[1]{\mathbf{#1}}
\newcommand{\R}{\mathbb{R}}
\newcommand{\N}{\mathbb{N}}
\newcommand{\argmin}{\operatorname{argmin}}
\newcommand{\argmax}{\operatorname{argmax}}
\newcommand{\sigmoid}{\operatorname{sigmoid}}
\newcommand{\relu}{\operatorname{ReLU}}
\newcommand{\softmax}{\operatorname{softmax}}

\begin{document}

% ===== PAGE DE TITRE =====
\begin{titlepage}
    \centering
    \vspace*{2cm}

    {\Huge\bfseries Cours Machine Learning}\\[0.5cm]

    \vspace{1cm}

    {\LARGE Chapitre 06}\\[0.3cm]
    {\LARGE\bfseries Réseaux de Neurones Fondamentaux}\\[2cm]

    \vfill

    {\large
    \textbf{Objectifs d'apprentissage :}\\[0.5cm]
    \begin{itemize}
        \item Comprendre le fonctionnement du perceptron et du perceptron multi-couches (MLP)
        \item Maîtriser le forward pass et le backpropagation
        \item Découvrir les fonctions d'activation et leur rôle
        \item Implémenter un réseau de neurones from scratch
        \item Appliquer les techniques de régularisation (dropout, batch norm)
    \end{itemize}
    }

    \vfill

    {\large
    \textbf{Prérequis :} Chapitres 01 (Math), 02 (Métriques), 03 (Régression)\\[0.3cm]
    \textbf{Durée estimée :} 6-8 heures\\[0.3cm]
    \textbf{Notebooks :} \texttt{06\_demo\_*.ipynb}
    }

    \vfill

    {\large Cours ML - Sandbox-ML\\
    Version 1.0 - 2026}
\end{titlepage}

\tableofcontents
\newpage

% ===== SECTION 1: INTRODUCTION =====
\section{Introduction aux Réseaux de Neurones}

\subsection{Motivation}

Les réseaux de neurones artificiels sont inspirés du fonctionnement du cerveau humain. Ils permettent de modéliser des relations non-linéaires complexes entre les données d'entrée et les sorties.

\begin{exemple}{Classification d'images manuscrites}
Reconnaître des chiffres manuscrits (MNIST) : chaque pixel est une entrée, et le réseau doit prédire le chiffre (0-9). Une régression logistique simple obtient ~92\% de précision, tandis qu'un réseau de neurones atteint ~98\%.
\end{exemple}

\subsection{Historique}

\begin{itemize}
    \item \textbf{1943} : McCulloch-Pitts - premier modèle de neurone formel
    \item \textbf{1958} : Rosenblatt - Perceptron (classification binaire linéaire)
    \item \textbf{1969} : Minsky \& Papert - limites du perceptron (XOR)
    \item \textbf{1986} : Rumelhart, Hinton, Williams - Backpropagation
    \item \textbf{2012} : AlexNet - révolution deep learning (ImageNet)
\end{itemize}

\subsection{Analogie biologique}

Un neurone artificiel est une simplification extrême d'un neurone biologique :

\begin{itemize}
    \item \textbf{Dendrites} → Entrées pondérées ($\vect{x} \odot \vect{w}$)
    \item \textbf{Corps cellulaire} → Sommation ($\sum w_i x_i + b$)
    \item \textbf{Axone} → Fonction d'activation ($\sigma(z)$)
    \item \textbf{Synapses} → Poids ajustables ($\vect{w}$)
\end{itemize}

% ===== SECTION 2: LE PERCEPTRON =====
\section{Le Perceptron}

\subsection{Définition}

\begin{definition}{Perceptron}
Le perceptron est un modèle de classification binaire linéaire. Pour une entrée $\vect{x} \in \R^d$, il calcule :
\begin{equation}
    y = \begin{cases}
        1 & \text{si } \vect{w}^T \vect{x} + b > 0 \\
        0 & \text{sinon}
    \end{cases}
\end{equation}
où $\vect{w} \in \R^d$ sont les poids et $b \in \R$ est le biais.
\end{definition}

Le perceptron définit un \textbf{hyperplan de séparation} dans l'espace des features :
\begin{equation}
    \vect{w}^T \vect{x} + b = 0
\end{equation}

\subsection{Algorithme d'apprentissage}

\begin{algorithm}[H]
\caption{Perceptron Learning Algorithm}
\label{alg:perceptron}
\begin{algorithmic}[1]
\REQUIRE Données $\{(\vect{x}_i, y_i)\}_{i=1}^n$, $y_i \in \{0, 1\}$
\REQUIRE Learning rate $\alpha$, nombre d'epochs $T$
\ENSURE Poids $\vect{w}$, biais $b$
\STATE Initialiser $\vect{w} = \vect{0}$, $b = 0$
\FOR{$epoch = 1$ \TO $T$}
    \FOR{$i = 1$ \TO $n$}
        \STATE Prédire : $\hat{y}_i = \mathbb{1}[\vect{w}^T \vect{x}_i + b > 0]$
        \IF{$\hat{y}_i \neq y_i$}
            \STATE $\vect{w} \leftarrow \vect{w} + \alpha (y_i - \hat{y}_i) \vect{x}_i$
            \STATE $b \leftarrow b + \alpha (y_i - \hat{y}_i)$
        \ENDIF
    \ENDFOR
\ENDFOR
\RETURN $\vect{w}, b$
\end{algorithmic}
\end{algorithm}

\subsection{Théorème de convergence}

\begin{theoreme}{Convergence du Perceptron}
Si les données sont linéairement séparables, l'algorithme du perceptron converge en un nombre fini d'itérations.
\end{theoreme}

\begin{attention}
Le perceptron \textbf{ne peut pas} résoudre le problème XOR (non linéairement séparable). C'est une limitation majeure qui a motivé le développement des réseaux multi-couches.
\end{attention}

\subsection{Implémentation}

\begin{lstlisting}[language=Python, caption=Perceptron from scratch]
import numpy as np

class Perceptron:
    def __init__(self, learning_rate=0.01, n_epochs=100):
        self.lr = learning_rate
        self.n_epochs = n_epochs
        self.w = None
        self.b = 0

    def fit(self, X, y):
        n_samples, n_features = X.shape
        self.w = np.zeros(n_features)

        for epoch in range(self.n_epochs):
            for i in range(n_samples):
                # Forward pass
                z = np.dot(X[i], self.w) + self.b
                y_pred = 1 if z > 0 else 0

                # Update weights si erreur
                if y_pred != y[i]:
                    update = self.lr * (y[i] - y_pred)
                    self.w += update * X[i]
                    self.b += update

        return self

    def predict(self, X):
        z = np.dot(X, self.w) + self.b
        return (z > 0).astype(int)
\end{lstlisting}

% ===== SECTION 3: FONCTIONS D'ACTIVATION =====
\section{Fonctions d'Activation}

Les fonctions d'activation introduisent la \textbf{non-linéarité} dans les réseaux de neurones. Sans elles, un réseau multi-couches serait équivalent à une régression linéaire.

\subsection{Sigmoid}

\begin{definition}{Fonction Sigmoid}
\begin{equation}
    \sigma(z) = \frac{1}{1 + e^{-z}}
\end{equation}
Propriétés :
\begin{itemize}
    \item Sortie entre 0 et 1 (interprétable comme probabilité)
    \item Dérivée : $\sigma'(z) = \sigma(z)(1 - \sigma(z))$
    \item Problème : \textbf{vanishing gradient} pour $|z|$ grand
\end{itemize}
\end{definition}

\subsection{Tanh (Tangente Hyperbolique)}

\begin{equation}
    \tanh(z) = \frac{e^z - e^{-z}}{e^z + e^{-z}} = 2\sigma(2z) - 1
\end{equation}

Propriétés :
\begin{itemize}
    \item Sortie entre -1 et 1 (centré à 0, meilleur que sigmoid)
    \item Dérivée : $\tanh'(z) = 1 - \tanh^2(z)$
    \item Aussi sujet au vanishing gradient
\end{itemize}

\subsection{ReLU (Rectified Linear Unit)}

\begin{definition}{ReLU}
\begin{equation}
    \relu(z) = \max(0, z) = \begin{cases}
        z & \text{si } z > 0 \\
        0 & \text{sinon}
    \end{cases}
\end{equation}
Propriétés :
\begin{itemize}
    \item Calcul très rapide
    \item Pas de vanishing gradient pour $z > 0$
    \item Dérivée : $\relu'(z) = \mathbb{1}[z > 0]$
    \item Problème : \textbf{dying ReLU} (neurones inactifs si $z < 0$)
\end{itemize}
\end{definition}

\textbf{ReLU est la fonction d'activation par défaut} dans les réseaux modernes.

\subsection{Variantes de ReLU}

\textbf{Leaky ReLU :}
\begin{equation}
    \text{LeakyReLU}(z) = \begin{cases}
        z & \text{si } z > 0 \\
        \alpha z & \text{sinon}
    \end{cases} \quad (\alpha \approx 0.01)
\end{equation}

\textbf{ELU (Exponential Linear Unit) :}
\begin{equation}
    \text{ELU}(z) = \begin{cases}
        z & \text{si } z > 0 \\
        \alpha(e^z - 1) & \text{sinon}
    \end{cases}
\end{equation}

\textbf{Swish / SiLU :}
\begin{equation}
    \text{Swish}(z) = z \cdot \sigma(z)
\end{equation}

\subsection{Softmax (pour classification multi-classe)}

\begin{definition}{Softmax}
Pour un vecteur $\vect{z} \in \R^K$, la fonction softmax renvoie un vecteur de probabilités :
\begin{equation}
    \softmax(\vect{z})_j = \frac{e^{z_j}}{\sum_{k=1}^K e^{z_k}} \quad \text{pour } j = 1, \ldots, K
\end{equation}
Propriété : $\sum_{j=1}^K \softmax(\vect{z})_j = 1$ (distribution de probabilité)
\end{definition}

\begin{table}[h]
\centering
\caption{Comparaison des fonctions d'activation}
\label{tab:activations}
\begin{tabular}{lccc}
\toprule
\textbf{Fonction} & \textbf{Plage} & \textbf{Vanishing Gradient} & \textbf{Usage} \\
\midrule
Sigmoid & $(0, 1)$ & ✗ Oui & Output binaire \\
Tanh & $(-1, 1)$ & ✗ Oui & RNN (historique) \\
ReLU & $[0, \infty)$ & ✓ Non & Hidden layers (défaut) \\
Leaky ReLU & $(-\infty, \infty)$ & ✓ Non & Alternative ReLU \\
Softmax & $[0, 1]^K$ & - & Output multi-classe \\
\bottomrule
\end{tabular}
\end{table}

\begin{figure}[h]
\centering
\begin{tikzpicture}[scale=1.1]
    % Subplot 1: Sigmoid
    \begin{scope}[xshift=0cm]
        % Axes
        \draw[->] (-3,0) -- (3,0) node[right, font=\footnotesize] {$z$};
        \draw[->] (0,-0.3) -- (0,3.2) node[above, font=\footnotesize] {$\sigma(z)$};

        % Grid
        \draw[gray!30, very thin] (-3,0) grid[step=1] (3,3);

        % Sigmoid curve: σ(z) = 1/(1+e^(-z))
        \draw[blue, very thick, smooth, domain=-3:3, samples=100]
            plot (\x, {3/(1+exp(-2*\x))});

        % Asymptotes
        \draw[dashed, red!60, thin] (-3,0) -- (3,0);
        \draw[dashed, red!60, thin] (-3,3) -- (3,3);

        % Annotations
        \node[font=\footnotesize, left] at (0,3) {1};
        \node[font=\footnotesize, left] at (0,1.5) {0.5};
        \node[font=\footnotesize, below] at (0,0) {0};
        \node[font=\small, blue, above] at (0,3.5) {\textbf{Sigmoid}};
        \node[font=\tiny, align=center] at (0,-1) {$\sigma(z) = \frac{1}{1+e^{-z}}$};

        % Saturation zones
        \node[font=\tiny, red] at (-2.5, 0.3) {Saturation};
        \node[font=\tiny, red] at (2.5, 2.7) {Saturation};
    \end{scope}

    % Subplot 2: Tanh
    \begin{scope}[xshift=7cm]
        % Axes
        \draw[->] (-3,0) -- (3,0) node[right, font=\footnotesize] {$z$};
        \draw[->] (0,-0.3) -- (0,3.2);

        % Grid
        \draw[gray!30, very thin] (-3,0) grid[step=1] (3,3);

        % Tanh curve: tanh(z) = (e^z - e^(-z))/(e^z + e^(-z))
        % Shifted to [0,3] range: (tanh(z)+1)*1.5
        \draw[orange!80!black, very thick, smooth, domain=-3:3, samples=100]
            plot (\x, {1.5*(tanh(\x)+1)});

        % Asymptotes
        \draw[dashed, red!60, thin] (-3,0) -- (3,0);
        \draw[dashed, red!60, thin] (-3,3) -- (3,3);

        % Annotations
        \node[font=\footnotesize, left] at (0,3) {1};
        \node[font=\footnotesize, left] at (0,1.5) {0};
        \node[font=\footnotesize, left] at (0,0) {-1};
        \node[font=\small, orange!80!black, above] at (0,3.5) {\textbf{Tanh}};
        \node[font=\tiny, align=center] at (0,-1) {$\tanh(z) = \frac{e^z - e^{-z}}{e^z + e^{-z}}$};

        % Saturation zones
        \node[font=\tiny, red] at (-2.5, 0.3) {Saturation};
        \node[font=\tiny, red] at (2.5, 2.7) {Saturation};
    \end{scope}

    % Subplot 3: ReLU
    \begin{scope}[xshift=0cm, yshift=-5.5cm]
        % Axes
        \draw[->] (-3,0) -- (3,0) node[right, font=\footnotesize] {$z$};
        \draw[->] (0,-0.3) -- (0,3.2) node[above, font=\footnotesize] {$\text{ReLU}(z)$};

        % Grid
        \draw[gray!30, very thin] (-3,0) grid[step=1] (3,3);

        % ReLU: max(0, z)
        \draw[green!60!black, very thick] (-3,0) -- (0,0);
        \draw[green!60!black, very thick] (0,0) -- (3,3);

        % Dead zone
        \fill[red!20, opacity=0.3] (-3,0) rectangle (0,3);
        \node[font=\tiny, red!70!black] at (-1.5, 2.5) {Dead zone};
        \node[font=\tiny, red!70!black] at (-1.5, 2) {$z \leq 0$};

        % Linear zone
        \node[font=\tiny, green!40!black] at (1.5, 2.5) {Active zone};
        \node[font=\tiny, green!40!black] at (1.5, 2) {$z > 0$};

        % Annotations
        \node[font=\footnotesize, below] at (0,0) {0};
        \node[font=\small, green!60!black, above] at (0,3.5) {\textbf{ReLU}};
        \node[font=\tiny, align=center] at (0,-1) {$\text{ReLU}(z) = \max(0, z)$};
    \end{scope}

    % Subplot 4: Leaky ReLU
    \begin{scope}[xshift=7cm, yshift=-5.5cm]
        % Axes
        \draw[->] (-3,0) -- (3,0) node[right, font=\footnotesize] {$z$};
        \draw[->] (0,-0.3) -- (0,3.2);

        % Grid
        \draw[gray!30, very thin] (-3,0) grid[step=1] (3,3);

        % Leaky ReLU: max(0.1z, z)
        \draw[purple!70!black, very thick] (-3,-0.3) -- (0,0);
        \draw[purple!70!black, very thick] (0,0) -- (3,3);

        % Small gradient zone
        \fill[orange!20, opacity=0.3] (-3,0) rectangle (0,3);
        \node[font=\tiny, orange!70!black] at (-1.5, 2.5) {Small gradient};
        \node[font=\tiny, orange!70!black] at (-1.5, 2) {$\alpha z$ ($\alpha=0.01$)};

        % Full gradient zone
        \node[font=\tiny, purple!40!black] at (1.5, 2.5) {Full gradient};
        \node[font=\tiny, purple!40!black] at (1.5, 2) {$z$};

        % Annotations
        \node[font=\footnotesize, below] at (0,0) {0};
        \node[font=\small, purple!70!black, above] at (0,3.5) {\textbf{Leaky ReLU}};
        \node[font=\tiny, align=center] at (0,-1) {$\text{LReLU}(z) = \begin{cases} \alpha z & z \leq 0 \\ z & z > 0 \end{cases}$};
    \end{scope}

\end{tikzpicture}
\caption{Comparaison des principales fonctions d'activation. \textbf{Sigmoid et Tanh} souffrent de saturation (gradients $\approx 0$ pour $|z|$ grand), causant le \textit{vanishing gradient}. \textbf{ReLU} résout ce problème mais peut mourir (dead neurons) si $z \leq 0$ toujours. \textbf{Leaky ReLU} évite la mort neuronale avec un petit gradient négatif ($\alpha=0.01$). En pratique, ReLU est le choix par défaut pour les couches cachées.}
\label{fig:activation_functions}
\end{figure}

\clearpage

% ===== SECTION 4: MLP (MULTI-LAYER PERCEPTRON) =====
\section{Perceptron Multi-Couches (MLP)}

\subsection{Architecture}

Un MLP est composé de plusieurs couches de neurones :
\begin{itemize}
    \item \textbf{Couche d'entrée (input layer)} : reçoit les features $\vect{x}$
    \item \textbf{Couches cachées (hidden layers)} : transformations non-linéaires
    \item \textbf{Couche de sortie (output layer)} : prédictions $\hat{y}$
\end{itemize}

\begin{figure}[h]
\centering
\begin{tikzpicture}[
    neuron/.style={circle, draw, minimum size=0.8cm, fill=blue!20},
    input/.style={circle, draw, minimum size=0.8cm, fill=green!20},
    output/.style={circle, draw, minimum size=0.8cm, fill=red!20}
]

    % Input layer
    \foreach \y in {1,2,3,4}
        \node[input] (I\y) at (0, -\y*1.2) {};
    \node[above] at (I1.north) {Input Layer};
    \node[left, font=\small] at (I2.west) {$x_1$};
    \node[left, font=\small] at (I3.west) {$x_2$};
    \node[left, font=\small] at (I4.west) {$x_3$};

    % Hidden layer 1
    \foreach \y in {1,2,3,4,5}
        \node[neuron] (H1\y) at (3, -\y*1) {};
    \node[above] at (H11.north) {Hidden Layer 1};
    \node[right, font=\tiny] at (H13.east) {ReLU};

    % Hidden layer 2
    \foreach \y in {1,2,3}
        \node[neuron] (H2\y) at (6, -\y*1.5-0.5) {};
    \node[above] at (H21.north) {Hidden Layer 2};
    \node[right, font=\tiny] at (H22.east) {ReLU};

    % Output layer
    \foreach \y in {1,2}
        \node[output] (O\y) at (9, -\y*2-1) {};
    \node[above] at (O1.north) {Output Layer};
    \node[right, font=\small] at (O1.east) {$\hat{y}_1$};
    \node[right, font=\small] at (O2.east) {$\hat{y}_2$};
    \node[right, font=\tiny] at (O1.south east) {Softmax};

    % Connections Input -> Hidden1
    \foreach \i in {1,2,3,4}
        \foreach \j in {1,2,3,4,5}
            \draw[->] (I\i) -- (H1\j);

    % Connections Hidden1 -> Hidden2
    \foreach \i in {1,2,3,4,5}
        \foreach \j in {1,2,3}
            \draw[->] (H1\i) -- (H2\j);

    % Connections Hidden2 -> Output
    \foreach \i in {1,2,3}
        \foreach \j in {1,2}
            \draw[->] (H2\i) -- (O\j);

    % Annotations dimensions
    \node[below, font=\small, align=center] at (0, -5.5) {$d_0 = 3$\\(features)};
    \node[below, font=\small, align=center] at (3, -5.5) {$d_1 = 5$};
    \node[below, font=\small, align=center] at (6, -5.5) {$d_2 = 3$};
    \node[below, font=\small, align=center] at (9, -5.5) {$d_3 = 2$\\(classes)};

    % Forward pass annotation
    \draw[->, very thick, red] (0, -6.5) -- (9, -6.5);
    \node[red, below] at (4.5, -6.5) {Forward Pass $\rightarrow$};

\end{tikzpicture}
\caption{Architecture MLP (Multi-Layer Perceptron): les neurones d'une couche sont connectés à tous les neurones de la couche suivante (fully-connected). Les activations se propagent de gauche à droite via transformations linéaires ($\mat{W}\vect{a} + \vect{b}$) suivies de fonctions d'activation non-linéaires (ReLU, Softmax).}
\label{fig:mlp_architecture}
\end{figure}

\begin{definition}{MLP à $L$ couches}
Pour une entrée $\vect{x} \in \R^{d_0}$, un MLP calcule :
\begin{align}
    \vect{a}^{[0]} &= \vect{x} \\
    \vect{z}^{[l]} &= \mat{W}^{[l]} \vect{a}^{[l-1]} + \vect{b}^{[l]} \quad \text{pour } l = 1, \ldots, L \\
    \vect{a}^{[l]} &= \sigma^{[l]}(\vect{z}^{[l]}) \\
    \hat{y} &= \vect{a}^{[L]}
\end{align}
où :
\begin{itemize}
    \item $\mat{W}^{[l]} \in \R^{d_l \times d_{l-1}}$ : matrice de poids de la couche $l$
    \item $\vect{b}^{[l]} \in \R^{d_l}$ : vecteur de biais de la couche $l$
    \item $\sigma^{[l]}$ : fonction d'activation de la couche $l$
    \item $\vect{a}^{[l]}$ : activations de la couche $l$
    \item $\vect{z}^{[l]}$ : pré-activations (avant fonction d'activation)
\end{itemize}
\end{definition}

\subsection{Théorème d'approximation universelle}

\begin{theoreme}{Approximation Universelle}
Un MLP avec une seule couche cachée de taille suffisante et une fonction d'activation non-linéaire peut approximer n'importe quelle fonction continue sur un compact de $\R^d$ avec une précision arbitraire.
\end{theoreme}

\begin{attention}
Ce théorème est \textbf{théorique} : en pratique, les réseaux profonds (deep learning) avec plusieurs couches sont plus efficaces et nécessitent moins de neurones.
\end{attention}

\subsection{Exemple : MLP 3 couches}

Architecture : 784 (input) → 128 (hidden) → 64 (hidden) → 10 (output)

\begin{align}
    \vect{x} &\in \R^{784} \quad \text{(image 28×28)} \\
    \vect{z}^{[1]} &= \mat{W}^{[1]} \vect{x} + \vect{b}^{[1]} \quad (\mat{W}^{[1]} \in \R^{128 \times 784}) \\
    \vect{a}^{[1]} &= \relu(\vect{z}^{[1]}) \quad \in \R^{128} \\
    \vect{z}^{[2]} &= \mat{W}^{[2]} \vect{a}^{[1]} + \vect{b}^{[2]} \quad (\mat{W}^{[2]} \in \R^{64 \times 128}) \\
    \vect{a}^{[2]} &= \relu(\vect{z}^{[2]}) \quad \in \R^{64} \\
    \vect{z}^{[3]} &= \mat{W}^{[3]} \vect{a}^{[2]} + \vect{b}^{[3]} \quad (\mat{W}^{[3]} \in \R^{10 \times 64}) \\
    \hat{\vect{y}} &= \softmax(\vect{z}^{[3]}) \quad \in \R^{10}
\end{align}

Nombre total de paramètres :
\begin{equation}
    (784 \times 128 + 128) + (128 \times 64 + 64) + (64 \times 10 + 10) = 109{,}386
\end{equation}

% ===== SECTION 5: FORWARD PASS =====
\section{Forward Pass (Propagation Avant)}

\subsection{Principe}

Le forward pass consiste à calculer la sortie du réseau pour une entrée donnée en propageant les activations couche par couche.

\begin{algorithm}[H]
\caption{Forward Pass}
\label{alg:forward}
\begin{algorithmic}[1]
\REQUIRE Entrée $\vect{x}$, poids $\{\mat{W}^{[l]}, \vect{b}^{[l]}\}_{l=1}^L$
\ENSURE Prédiction $\hat{y}$, cache des activations $\{\vect{a}^{[l]}, \vect{z}^{[l]}\}$
\STATE $\vect{a}^{[0]} = \vect{x}$
\FOR{$l = 1$ \TO $L$}
    \STATE $\vect{z}^{[l]} = \mat{W}^{[l]} \vect{a}^{[l-1]} + \vect{b}^{[l]}$ \quad \COMMENT{Combinaison linéaire}
    \STATE $\vect{a}^{[l]} = \sigma^{[l]}(\vect{z}^{[l]})$ \quad \COMMENT{Activation non-linéaire}
    \STATE Stocker $\vect{z}^{[l]}, \vect{a}^{[l]}$ dans le cache \quad \COMMENT{Pour backprop}
\ENDFOR
\STATE $\hat{y} = \vect{a}^{[L]}$
\RETURN $\hat{y}$, cache
\end{algorithmic}
\end{algorithm}

\subsection{Implémentation vectorisée}

Pour un batch de $m$ exemples $\mat{X} \in \R^{m \times d_0}$ :

\begin{align}
    \mat{A}^{[0]} &= \mat{X} \quad \in \R^{m \times d_0} \\
    \mat{Z}^{[l]} &= \mat{A}^{[l-1]} (\mat{W}^{[l]})^T + \vect{b}^{[l]} \quad \in \R^{m \times d_l} \\
    \mat{A}^{[l]} &= \sigma^{[l]}(\mat{Z}^{[l]})
\end{align}

\begin{lstlisting}[language=Python, caption=Forward pass vectorisé]
def forward_pass(X, parameters):
    """
    X : (m, d0) - batch de m exemples
    parameters : dict avec W[l], b[l] pour chaque couche l
    """
    cache = {}
    A = X
    L = len(parameters) // 2  # Nombre de couches

    for l in range(1, L + 1):
        A_prev = A
        W = parameters[f'W{l}']
        b = parameters[f'b{l}']

        # Linear forward
        Z = np.dot(A_prev, W.T) + b

        # Activation
        if l < L:  # Hidden layers : ReLU
            A = np.maximum(0, Z)
        else:  # Output layer : softmax
            A = softmax(Z)

        # Cache pour backprop
        cache[f'Z{l}'] = Z
        cache[f'A{l}'] = A
        cache[f'A{l-1}'] = A_prev

    return A, cache
\end{lstlisting}

% ===== SECTION 6: FONCTION DE COÛT =====
\section{Fonctions de Coût}

\subsection{Pour la régression : MSE}

\begin{equation}
    L(\vect{\theta}) = \frac{1}{m} \sum_{i=1}^m (\hat{y}_i - y_i)^2
\end{equation}

\subsection{Pour classification binaire : Binary Cross-Entropy}

\begin{equation}
    L(\vect{\theta}) = -\frac{1}{m} \sum_{i=1}^m \left[ y_i \log(\hat{y}_i) + (1 - y_i) \log(1 - \hat{y}_i) \right]
\end{equation}

\subsection{Pour classification multi-classe : Categorical Cross-Entropy}

\begin{definition}{Cross-Entropy Loss}
Pour $K$ classes, avec labels one-hot $\vect{y} \in \{0, 1\}^K$ :
\begin{equation}
    L(\vect{\theta}) = -\frac{1}{m} \sum_{i=1}^m \sum_{k=1}^K y_{i,k} \log(\hat{y}_{i,k})
\end{equation}
où $\hat{\vect{y}}_i = \softmax(\vect{z}_i^{[L]})$.
\end{definition}

En pratique, on utilise souvent la formulation :
\begin{equation}
    L = -\frac{1}{m} \sum_{i=1}^m \log(\hat{y}_{i, c_i})
\end{equation}
où $c_i$ est la classe correcte pour l'exemple $i$.

% ===== SECTION 7: BACKPROPAGATION =====
\section{Backpropagation}

\subsection{Principe}

La \textbf{backpropagation} (rétropropagation) est l'algorithme qui permet de calculer efficacement les gradients de la fonction de coût par rapport à tous les paramètres du réseau.

\textbf{Idée clé :} Utiliser la \textbf{règle de la chaîne (chain rule)} pour propager les gradients de la sortie vers l'entrée.

\begin{figure}[h]
\centering
\begin{tikzpicture}[scale=0.9, every node/.style={font=\small}]
    % Layer boxes
    \node[draw, rectangle, minimum width=1.8cm, minimum height=2.5cm, fill=green!15] (L0) at (0,0) {};
    \node[draw, rectangle, minimum width=1.8cm, minimum height=2.5cm, fill=blue!15] (L1) at (3.5,0) {};
    \node[draw, rectangle, minimum width=1.8cm, minimum height=2.5cm, fill=blue!15] (L2) at (7,0) {};
    \node[draw, rectangle, minimum width=1.8cm, minimum height=2.5cm, fill=red!15] (L3) at (10.5,0) {};

    % Layer titles
    \node[above, font=\footnotesize\bfseries] at (L0.north) {Input};
    \node[above, font=\footnotesize\bfseries] at (L1.north) {Hidden 1};
    \node[above, font=\footnotesize\bfseries] at (L2.north) {Hidden 2};
    \node[above, font=\footnotesize\bfseries] at (L3.north) {Output};

    % Layer labels
    \node at (L0.center) {$\vect{a}^{[0]}$};
    \node[above=0.3cm] at (L0.center) {$\vect{x}$};

    \node at (L1.center) {$\vect{a}^{[1]}$};
    \node[above=0.3cm, font=\tiny] at (L1.center) {$\vect{z}^{[1]}$};

    \node at (L2.center) {$\vect{a}^{[2]}$};
    \node[above=0.3cm, font=\tiny] at (L2.center) {$\vect{z}^{[2]}$};

    \node at (L3.center) {$\vect{a}^{[3]}$};
    \node[above=0.3cm] at (L3.center) {$\hat{\vect{y}}$};

    % Forward pass arrows (top)
    \draw[->, very thick, blue!70!black] (L0.east) ++ (0,0.6) -- ++ (1.5,0)
        node[midway, above, font=\tiny] {$\mat{W}^{[1]}\vect{a}^{[0]} + \vect{b}^{[1]}$}
        node[midway, below, font=\tiny] {$\sigma^{[1]}(\vect{z}^{[1]})$};

    \draw[->, very thick, blue!70!black] (L1.east) ++ (0,0.6) -- ++ (1.5,0)
        node[midway, above, font=\tiny] {$\mat{W}^{[2]}\vect{a}^{[1]} + \vect{b}^{[2]}$}
        node[midway, below, font=\tiny] {$\sigma^{[2]}(\vect{z}^{[2]})$};

    \draw[->, very thick, blue!70!black] (L2.east) ++ (0,0.6) -- ++ (1.5,0)
        node[midway, above, font=\tiny] {$\mat{W}^{[3]}\vect{a}^{[2]} + \vect{b}^{[3]}$}
        node[midway, below, font=\tiny] {$\sigma^{[3]}(\vect{z}^{[3]})$};

    % Backward pass arrows (bottom)
    \draw[<-, very thick, red!70!black] (L0.east) ++ (0,-0.6) -- ++ (1.5,0)
        node[midway, below, font=\tiny] {$\delta^{[1]} (\vect{a}^{[0]})^T$};

    \draw[<-, very thick, red!70!black] (L1.east) ++ (0,-0.6) -- ++ (1.5,0)
        node[midway, above, font=\tiny] {$(\mat{W}^{[2]})^T \delta^{[2]} \odot \sigma'^{[1]}$}
        node[midway, below, font=\tiny] {$\delta^{[2]} (\vect{a}^{[1]})^T$};

    \draw[<-, very thick, red!70!black] (L2.east) ++ (0,-0.6) -- ++ (1.5,0)
        node[midway, above, font=\tiny] {$(\mat{W}^{[3]})^T \delta^{[3]} \odot \sigma'^{[2]}$}
        node[midway, below, font=\tiny] {$\delta^{[3]} (\vect{a}^{[2]})^T$};

    % Loss function
    \node[draw, circle, minimum size=1.2cm, fill=orange!20] (Loss) at (13,0) {$L$};
    \draw[->, very thick, blue!70!black] (L3.east) ++ (0,0.4) -- (Loss.west |- L3.east) ++ (0,0.4)
        node[midway, above, font=\tiny] {$L(\hat{\vect{y}}, \vect{y})$};

    % Gradient from loss
    \draw[<-, very thick, red!70!black] (L3.east) ++ (0,-0.4) -- (Loss.west |- L3.east) ++ (0,-0.4)
        node[midway, below, font=\tiny] {$\delta^{[3]} = \hat{\vect{y}} - \vect{y}$};

    % Legend
    \node[draw, rectangle, minimum width=3.5cm, minimum height=1.8cm, dashed] (legend) at (13,-3.5) {};
    \node[above left, font=\footnotesize\bfseries] at (legend.north west) {Légende:};

    \draw[->, very thick, blue!70!black] (11.5,-3) -- ++ (1,0);
    \node[right, font=\footnotesize] at (12.5,-3) {Forward Pass};

    \draw[<-, very thick, red!70!black] (11.5,-4) -- ++ (1,0);
    \node[right, font=\footnotesize] at (12.5,-4) {Backward Pass};

    % Annotations
    \node[blue!70!black, font=\footnotesize, align=center] at (5.25, 2.2) {FORWARD PROPAGATION};
    \node[blue!70!black, font=\tiny, align=center] at (5.25, 1.7) {Calcul des prédictions $\hat{\vect{y}}$};

    \node[red!70!black, font=\footnotesize, align=center] at (5.25, -2.2) {BACKWARD PROPAGATION};
    \node[red!70!black, font=\tiny, align=center] at (5.25, -2.7) {Calcul des gradients $\frac{\partial L}{\partial \mat{W}^{[l]}}, \frac{\partial L}{\partial \vect{b}^{[l]}}$};

    % Cache annotation
    \node[font=\tiny, align=center, fill=yellow!20, draw, dashed] at (5.25, 0) {Cache: $\vect{z}^{[l]}, \vect{a}^{[l]}$};

\end{tikzpicture}
\caption{Flux bidirectionnel Forward et Backward Propagation dans un MLP à 2 couches cachées. \textbf{Forward Pass (bleu):} les activations se propagent de gauche à droite pour calculer $\hat{\vect{y}}$, puis la loss $L(\hat{\vect{y}}, \vect{y})$. Les valeurs intermédiaires ($\vect{z}^{[l]}, \vect{a}^{[l]}$) sont stockées dans un cache. \textbf{Backward Pass (rouge):} les gradients se propagent de droite à gauche via la règle de la chaîne, calculant $\frac{\partial L}{\partial \mat{W}^{[l]}}$ et $\frac{\partial L}{\partial \vect{b}^{[l]}}$ pour chaque couche. Le gradient initial $\delta^{[L]} = \hat{\vect{y}} - \vect{y}$ (pour cross-entropy + softmax).}
\label{fig:forward_backward_propagation}
\end{figure}

\clearpage

\subsection{Notations}

Définissons :
\begin{align}
    \delta^{[l]} &= \frac{\partial L}{\partial \vect{z}^{[l]}} \quad \text{(gradient par rapport aux pré-activations)} \\
    \frac{\partial L}{\partial \mat{W}^{[l]}} &\quad \text{(gradient par rapport aux poids)} \\
    \frac{\partial L}{\partial \vect{b}^{[l]}} &\quad \text{(gradient par rapport aux biais)}
\end{align}

\subsection{Dérivation des équations}

\textbf{Couche de sortie ($l = L$) :}

Pour cross-entropy avec softmax :
\begin{equation}
    \delta^{[L]} = \hat{\vect{y}} - \vect{y} \quad \text{(formule simplifiée !)}
\end{equation}

\textbf{Couches cachées ($l < L$) :}

Par la règle de la chaîne :
\begin{align}
    \delta^{[l]} &= \frac{\partial L}{\partial \vect{z}^{[l]}} \\
    &= \frac{\partial L}{\partial \vect{a}^{[l]}} \odot \frac{\partial \vect{a}^{[l]}}{\partial \vect{z}^{[l]}} \\
    &= \left[ (\mat{W}^{[l+1]})^T \delta^{[l+1]} \right] \odot \sigma'^{[l]}(\vect{z}^{[l]})
\end{align}

où $\odot$ est le produit élément par élément (Hadamard).

\textbf{Gradients des paramètres :}
\begin{align}
    \frac{\partial L}{\partial \mat{W}^{[l]}} &= \delta^{[l]} (\vect{a}^{[l-1]})^T \\
    \frac{\partial L}{\partial \vect{b}^{[l]}} &= \delta^{[l]}
\end{align}

Pour un batch de $m$ exemples :
\begin{align}
    \frac{\partial L}{\partial \mat{W}^{[l]}} &= \frac{1}{m} \mat{\Delta}^{[l]} (\mat{A}^{[l-1]})^T \\
    \frac{\partial L}{\partial \vect{b}^{[l]}} &= \frac{1}{m} \sum_{i=1}^m \delta_i^{[l]}
\end{align}

\subsection{Algorithme complet}

\begin{algorithm}[H]
\caption{Backpropagation}
\label{alg:backprop}
\begin{algorithmic}[1]
\REQUIRE Cache du forward pass $\{\vect{a}^{[l]}, \vect{z}^{[l]}\}$
\REQUIRE Gradient de la loss $\frac{\partial L}{\partial \vect{a}^{[L]}}$
\ENSURE Gradients $\{\frac{\partial L}{\partial \mat{W}^{[l]}}, \frac{\partial L}{\partial \vect{b}^{[l]}}\}$
\STATE Calculer $\delta^{[L]} = \hat{\vect{y}} - \vect{y}$ \quad \COMMENT{Output layer}
\FOR{$l = L$ \TO $1$} \COMMENT{Parcourir en sens inverse}
    \STATE $\frac{\partial L}{\partial \mat{W}^{[l]}} = \frac{1}{m} \delta^{[l]} (\vect{a}^{[l-1]})^T$
    \STATE $\frac{\partial L}{\partial \vect{b}^{[l]}} = \frac{1}{m} \sum_i \delta_i^{[l]}$
    \IF{$l > 1$}
        \STATE $\delta^{[l-1]} = [(\mat{W}^{[l]})^T \delta^{[l]}] \odot \sigma'^{[l-1]}(\vect{z}^{[l-1]})$
    \ENDIF
\ENDFOR
\RETURN Gradients
\end{algorithmic}
\end{algorithm}

\subsection{Dérivées des fonctions d'activation}

\begin{itemize}
    \item \textbf{Sigmoid :} $\sigma'(z) = \sigma(z)(1 - \sigma(z))$
    \item \textbf{Tanh :} $\tanh'(z) = 1 - \tanh^2(z)$
    \item \textbf{ReLU :} $\relu'(z) = \mathbb{1}[z > 0]$
    \item \textbf{Leaky ReLU :} $\text{LReLU}'(z) = \begin{cases} 1 & z > 0 \\ \alpha & z \leq 0 \end{cases}$
\end{itemize}

\subsection{Implémentation}

\begin{lstlisting}[language=Python, caption=Backpropagation]
def backward_pass(y_true, cache, parameters):
    """
    y_true : (m, K) - labels one-hot
    cache : dict des activations du forward pass
    parameters : dict des poids W[l], b[l]
    """
    m = y_true.shape[0]
    grads = {}
    L = len(parameters) // 2

    # Output layer gradient (cross-entropy + softmax)
    y_pred = cache[f'A{L}']
    dZ = y_pred - y_true  # Shape: (m, K)

    # Backprop à travers les couches
    for l in reversed(range(1, L + 1)):
        A_prev = cache[f'A{l-1}']

        # Gradients des paramètres
        grads[f'dW{l}'] = (1/m) * np.dot(dZ.T, A_prev)
        grads[f'db{l}'] = (1/m) * np.sum(dZ, axis=0, keepdims=True)

        if l > 1:
            # Gradient pour la couche précédente
            W = parameters[f'W{l}']
            dA_prev = np.dot(dZ, W)

            # Gradient à travers l'activation (ReLU)
            Z_prev = cache[f'Z{l-1}']
            dZ = dA_prev * (Z_prev > 0)  # ReLU derivative

    return grads
\end{lstlisting}

% ===== SECTION 8: ENTRAÎNEMENT =====
\section{Entraînement d'un MLP}

\subsection{Optimiseurs}

\subsubsection{Gradient Descent}

\begin{equation}
    \vect{\theta}_{t+1} = \vect{\theta}_t - \alpha \nabla L(\vect{\theta}_t)
\end{equation}

\subsubsection{Stochastic Gradient Descent (SGD)}

Mise à jour sur un seul exemple (ou un mini-batch) :
\begin{equation}
    \vect{\theta}_{t+1} = \vect{\theta}_t - \alpha \nabla L_i(\vect{\theta}_t)
\end{equation}

\textbf{Avec momentum :}
\begin{align}
    \vect{v}_t &= \beta \vect{v}_{t-1} + (1 - \beta) \nabla L(\vect{\theta}_t) \\
    \vect{\theta}_{t+1} &= \vect{\theta}_t - \alpha \vect{v}_t
\end{align}

Typiquement $\beta = 0.9$.

\subsubsection{Adam (Adaptive Moment Estimation)}

Combine momentum et RMSprop :
\begin{align}
    \vect{m}_t &= \beta_1 \vect{m}_{t-1} + (1 - \beta_1) \nabla L(\vect{\theta}_t) \quad \text{(1st moment)} \\
    \vect{v}_t &= \beta_2 \vect{v}_{t-1} + (1 - \beta_2) (\nabla L(\vect{\theta}_t))^2 \quad \text{(2nd moment)} \\
    \hat{\vect{m}}_t &= \frac{\vect{m}_t}{1 - \beta_1^t} \quad \text{(bias correction)} \\
    \hat{\vect{v}}_t &= \frac{\vect{v}_t}{1 - \beta_2^t} \\
    \vect{\theta}_{t+1} &= \vect{\theta}_t - \alpha \frac{\hat{\vect{m}}_t}{\sqrt{\hat{\vect{v}}_t} + \epsilon}
\end{align}

Hyperparamètres par défaut : $\beta_1 = 0.9$, $\beta_2 = 0.999$, $\epsilon = 10^{-8}$.

\begin{astuce}
\textbf{Adam est l'optimiseur par défaut} pour la plupart des problèmes deep learning. Il est robuste et nécessite peu de tuning.
\end{astuce}

\subsection{Algorithme d'entraînement complet}

\begin{algorithm}[H]
\caption{Entraînement MLP}
\label{alg:mlp_training}
\begin{algorithmic}[1]
\REQUIRE Dataset $(X, y)$, architecture $\{d_0, d_1, \ldots, d_L\}$
\REQUIRE Learning rate $\alpha$, batch size $B$, nombre d'epochs $T$
\ENSURE Paramètres entraînés $\vect{\theta}^*$
\STATE Initialiser aléatoirement $\mat{W}^{[l]}, \vect{b}^{[l]}$ pour $l = 1, \ldots, L$
\FOR{$epoch = 1$ \TO $T$}
    \STATE Mélanger les données
    \FOR{chaque mini-batch $(X_{batch}, y_{batch})$}
        \STATE \COMMENT{Forward pass}
        \STATE $\hat{y}_{batch}, cache = \text{forward\_pass}(X_{batch}, \vect{\theta})$
        \STATE $L_{batch} = \text{compute\_loss}(\hat{y}_{batch}, y_{batch})$
        \STATE \COMMENT{Backward pass}
        \STATE $grads = \text{backward\_pass}(y_{batch}, cache, \vect{\theta})$
        \STATE \COMMENT{Update parameters}
        \FOR{chaque paramètre $\theta$}
            \STATE $\theta \leftarrow \theta - \alpha \cdot \frac{\partial L}{\partial \theta}$
        \ENDFOR
    \ENDFOR
    \STATE Évaluer sur validation set
\ENDFOR
\RETURN $\vect{\theta}^*$
\end{algorithmic}
\end{algorithm}

\subsection{Initialisation des poids}

\begin{attention}
Ne JAMAIS initialiser tous les poids à 0 ! Les neurones seraient tous identiques (symétrie).
\end{attention}

\textbf{Xavier/Glorot initialization (pour Sigmoid/Tanh) :}
\begin{equation}
    W^{[l]}_{ij} \sim \mathcal{N}\left(0, \frac{2}{d_{l-1} + d_l}\right)
\end{equation}

\textbf{He initialization (pour ReLU) :}
\begin{equation}
    W^{[l]}_{ij} \sim \mathcal{N}\left(0, \frac{2}{d_{l-1}}\right)
\end{equation}

\begin{lstlisting}[language=Python, caption=Initialisation He]
def initialize_parameters(layer_dims):
    """
    layer_dims : [d0, d1, d2, ..., dL]
    """
    parameters = {}
    L = len(layer_dims) - 1

    for l in range(1, L + 1):
        # He initialization
        parameters[f'W{l}'] = np.random.randn(
            layer_dims[l], layer_dims[l-1]
        ) * np.sqrt(2 / layer_dims[l-1])

        parameters[f'b{l}'] = np.zeros((1, layer_dims[l]))

    return parameters
\end{lstlisting}

% ===== SECTION 9: RÉGULARISATION =====
\section{Régularisation}

\subsection{L2 Regularization (Weight Decay)}

Ajouter un terme de pénalité sur les poids :
\begin{equation}
    L_{reg}(\vect{\theta}) = L(\vect{\theta}) + \frac{\lambda}{2m} \sum_{l=1}^L \|\mat{W}^{[l]}\|_F^2
\end{equation}

où $\|\mat{W}\|_F^2 = \sum_{i,j} W_{ij}^2$ est la norme de Frobenius.

Le gradient devient :
\begin{equation}
    \frac{\partial L_{reg}}{\partial \mat{W}^{[l]}} = \frac{\partial L}{\partial \mat{W}^{[l]}} + \frac{\lambda}{m} \mat{W}^{[l]}
\end{equation}

\subsection{Dropout}

\begin{definition}{Dropout}
Pendant l'entraînement, chaque neurone est désactivé avec probabilité $p$ (typiquement $p = 0.5$). Au test, on utilise tous les neurones mais on multiplie les sorties par $(1-p)$.
\end{definition}

\textbf{Implémentation (inverted dropout) :}
\begin{lstlisting}[language=Python]
def dropout_forward(A, keep_prob=0.5, training=True):
    if training:
        mask = np.random.rand(*A.shape) < keep_prob
        A = A * mask / keep_prob  # Inverted dropout
        return A, mask
    else:
        return A, None
\end{lstlisting}

\begin{astuce}
Le dropout agit comme un \textbf{ensemble de réseaux} : à chaque itération, on entraîne un sous-réseau différent.
\end{astuce}

\subsection{Batch Normalization}

Normaliser les activations à chaque couche :
\begin{align}
    \mu_B &= \frac{1}{m} \sum_{i=1}^m z_i^{[l]} \\
    \sigma_B^2 &= \frac{1}{m} \sum_{i=1}^m (z_i^{[l]} - \mu_B)^2 \\
    \hat{z}_i^{[l]} &= \frac{z_i^{[l]} - \mu_B}{\sqrt{\sigma_B^2 + \epsilon}} \\
    \tilde{z}_i^{[l]} &= \gamma \hat{z}_i^{[l]} + \beta
\end{align}

où $\gamma, \beta$ sont des paramètres apprenables.

\textbf{Avantages :}
\begin{itemize}
    \item Accélère l'entraînement (learning rates plus élevés)
    \item Réduit la sensibilité à l'initialisation
    \item Effet régularisant (similaire au dropout)
\end{itemize}

\subsection{Early Stopping}

Surveiller la loss sur le validation set et arrêter l'entraînement quand elle commence à augmenter.

\begin{lstlisting}[language=Python]
best_val_loss = float('inf')
patience = 10
counter = 0

for epoch in range(n_epochs):
    train_model()
    val_loss = evaluate_validation()

    if val_loss < best_val_loss:
        best_val_loss = val_loss
        save_checkpoint()
        counter = 0
    else:
        counter += 1
        if counter >= patience:
            print("Early stopping")
            break
\end{lstlisting}

% ===== SECTION 10: IMPLÉMENTATION COMPLÈTE =====
\section{Implémentation Complète avec NumPy}

\begin{lstlisting}[language=Python, caption=MLP complet from scratch]
import numpy as np

class MLP:
    def __init__(self, layer_dims, learning_rate=0.01):
        self.layer_dims = layer_dims
        self.lr = learning_rate
        self.parameters = self._initialize_parameters()
        self.L = len(layer_dims) - 1

    def _initialize_parameters(self):
        """He initialization"""
        params = {}
        for l in range(1, len(self.layer_dims)):
            params[f'W{l}'] = np.random.randn(
                self.layer_dims[l], self.layer_dims[l-1]
            ) * np.sqrt(2 / self.layer_dims[l-1])
            params[f'b{l}'] = np.zeros((1, self.layer_dims[l]))
        return params

    def _relu(self, Z):
        return np.maximum(0, Z)

    def _softmax(self, Z):
        expZ = np.exp(Z - np.max(Z, axis=1, keepdims=True))
        return expZ / np.sum(expZ, axis=1, keepdims=True)

    def forward(self, X):
        """Forward pass"""
        cache = {'A0': X}
        A = X

        for l in range(1, self.L + 1):
            Z = np.dot(A, self.parameters[f'W{l}'].T) + \
                self.parameters[f'b{l}']

            if l < self.L:
                A = self._relu(Z)
            else:
                A = self._softmax(Z)

            cache[f'Z{l}'] = Z
            cache[f'A{l}'] = A

        return A, cache

    def backward(self, y_true, cache):
        """Backpropagation"""
        m = y_true.shape[0]
        grads = {}

        # Output layer
        dZ = cache[f'A{self.L}'] - y_true

        for l in reversed(range(1, self.L + 1)):
            A_prev = cache[f'A{l-1}']
            grads[f'dW{l}'] = (1/m) * np.dot(dZ.T, A_prev)
            grads[f'db{l}'] = (1/m) * np.sum(dZ, axis=0, keepdims=True)

            if l > 1:
                W = self.parameters[f'W{l}']
                dA_prev = np.dot(dZ, W)
                dZ = dA_prev * (cache[f'Z{l-1}'] > 0)

        return grads

    def update_parameters(self, grads):
        """Gradient descent update"""
        for l in range(1, self.L + 1):
            self.parameters[f'W{l}'] -= self.lr * grads[f'dW{l}']
            self.parameters[f'b{l}'] -= self.lr * grads[f'db{l}']

    def compute_loss(self, y_pred, y_true):
        """Cross-entropy loss"""
        m = y_true.shape[0]
        loss = -np.sum(y_true * np.log(y_pred + 1e-8)) / m
        return loss

    def fit(self, X, y, epochs=100, batch_size=32, verbose=True):
        """Training loop"""
        m = X.shape[0]
        history = {'loss': []}

        for epoch in range(epochs):
            # Shuffle
            indices = np.random.permutation(m)
            X_shuffled = X[indices]
            y_shuffled = y[indices]

            epoch_loss = 0
            n_batches = m // batch_size

            for i in range(n_batches):
                start = i * batch_size
                end = start + batch_size
                X_batch = X_shuffled[start:end]
                y_batch = y_shuffled[start:end]

                # Forward
                y_pred, cache = self.forward(X_batch)
                loss = self.compute_loss(y_pred, y_batch)
                epoch_loss += loss

                # Backward
                grads = self.backward(y_batch, cache)

                # Update
                self.update_parameters(grads)

            avg_loss = epoch_loss / n_batches
            history['loss'].append(avg_loss)

            if verbose and (epoch + 1) % 10 == 0:
                print(f"Epoch {epoch+1}/{epochs}, Loss: {avg_loss:.4f}")

        return history

    def predict(self, X):
        """Predictions"""
        y_pred, _ = self.forward(X)
        return np.argmax(y_pred, axis=1)

# Exemple d'utilisation
if __name__ == "__main__":
    from sklearn.datasets import load_digits
    from sklearn.model_selection import train_test_split
    from sklearn.preprocessing import OneHotEncoder

    # Charger données
    digits = load_digits()
    X, y = digits.data, digits.target

    # Normaliser
    X = X / 16.0

    # One-hot encoding
    y_onehot = np.zeros((len(y), 10))
    y_onehot[np.arange(len(y)), y] = 1

    # Split
    X_train, X_test, y_train, y_test = train_test_split(
        X, y_onehot, test_size=0.2, random_state=42
    )

    # Entraîner
    mlp = MLP(layer_dims=[64, 128, 64, 10], learning_rate=0.1)
    history = mlp.fit(X_train, y_train, epochs=100, batch_size=32)

    # Évaluer
    y_pred = mlp.predict(X_test)
    y_test_labels = np.argmax(y_test, axis=1)
    accuracy = np.mean(y_pred == y_test_labels)
    print(f"\nTest Accuracy: {accuracy:.4f}")
\end{lstlisting}

% ===== SECTION 11: AVEC PYTORCH =====
\section{Implémentation avec PyTorch}

\begin{lstlisting}[language=Python, caption=MLP avec PyTorch]
import torch
import torch.nn as nn
import torch.optim as optim
from torch.utils.data import DataLoader, TensorDataset

class MLP_PyTorch(nn.Module):
    def __init__(self, input_dim, hidden_dims, output_dim):
        super(MLP_PyTorch, self).__init__()

        layers = []
        prev_dim = input_dim

        # Hidden layers
        for hidden_dim in hidden_dims:
            layers.append(nn.Linear(prev_dim, hidden_dim))
            layers.append(nn.ReLU())
            layers.append(nn.Dropout(0.2))
            prev_dim = hidden_dim

        # Output layer
        layers.append(nn.Linear(prev_dim, output_dim))

        self.network = nn.Sequential(*layers)

    def forward(self, x):
        return self.network(x)

# Exemple d'utilisation
model = MLP_PyTorch(input_dim=64, hidden_dims=[128, 64], output_dim=10)

# Loss et optimizer
criterion = nn.CrossEntropyLoss()
optimizer = optim.Adam(model.parameters(), lr=0.001)

# Training loop
def train_pytorch(model, train_loader, epochs=50):
    model.train()
    for epoch in range(epochs):
        total_loss = 0
        for X_batch, y_batch in train_loader:
            # Forward
            outputs = model(X_batch)
            loss = criterion(outputs, y_batch)

            # Backward
            optimizer.zero_grad()
            loss.backward()
            optimizer.step()

            total_loss += loss.item()

        if (epoch + 1) % 10 == 0:
            print(f"Epoch {epoch+1}, Loss: {total_loss/len(train_loader):.4f}")

# Utilisation
X_train_tensor = torch.FloatTensor(X_train)
y_train_tensor = torch.LongTensor(np.argmax(y_train, axis=1))

train_dataset = TensorDataset(X_train_tensor, y_train_tensor)
train_loader = DataLoader(train_dataset, batch_size=32, shuffle=True)

train_pytorch(model, train_loader, epochs=50)
\end{lstlisting}

% ===== SECTION 12: DIAGNOSTIC =====
\section{Diagnostic et Debugging}

\subsection{Problèmes courants}

\begin{table}[h]
\centering
\caption{Diagnostic des problèmes d'entraînement}
\label{tab:debugging}
\begin{tabular}{lll}
\toprule
\textbf{Symptôme} & \textbf{Cause probable} & \textbf{Solution} \\
\midrule
Loss = NaN & Explosion gradients & Réduire learning rate \\
 & & Gradient clipping \\
 & & Vérifier normalisation \\
\midrule
Loss ne diminue pas & Learning rate trop faible & Augmenter LR \\
 & Mauvaise init & He/Xavier init \\
 & Architecture inadaptée & Changer nb couches \\
\midrule
Overfitting & Trop de paramètres & Dropout, L2 reg \\
 & Pas assez de données & Data augmentation \\
 & & Early stopping \\
\midrule
Underfitting & Modèle trop simple & Ajouter couches/neurones \\
 & Régularisation trop forte & Réduire $\lambda$, dropout \\
\midrule
Vanishing gradients & Sigmoid/Tanh profond & Utiliser ReLU \\
 & & Batch normalization \\
 & & Residual connections \\
\bottomrule
\end{tabular}
\end{table}

\subsection{Gradient checking}

Vérifier l'implémentation de backprop en comparant avec gradient numérique :
\begin{equation}
    \frac{\partial L}{\partial \theta} \approx \frac{L(\theta + \epsilon) - L(\theta - \epsilon)}{2\epsilon}
\end{equation}

\begin{lstlisting}[language=Python, caption=Gradient checking]
def gradient_check(model, X, y, epsilon=1e-7):
    """Vérifie backprop avec gradients numériques"""
    # Calculer gradients analytiques (backprop)
    y_pred, cache = model.forward(X)
    grads_analytic = model.backward(y, cache)

    # Calculer gradients numériques
    for param_name in model.parameters:
        param = model.parameters[param_name]
        grad_numeric = np.zeros_like(param)

        it = np.nditer(param, flags=['multi_index'])
        while not it.finished:
            idx = it.multi_index
            old_value = param[idx]

            # f(theta + epsilon)
            param[idx] = old_value + epsilon
            y_pred_plus, _ = model.forward(X)
            loss_plus = model.compute_loss(y_pred_plus, y)

            # f(theta - epsilon)
            param[idx] = old_value - epsilon
            y_pred_minus, _ = model.forward(X)
            loss_minus = model.compute_loss(y_pred_minus, y)

            # Gradient numérique
            grad_numeric[idx] = (loss_plus - loss_minus) / (2 * epsilon)

            param[idx] = old_value
            it.iternext()

        # Comparer
        grad_analytic = grads_analytic[f'd{param_name}']
        diff = np.linalg.norm(grad_numeric - grad_analytic) / \
               (np.linalg.norm(grad_numeric) + np.linalg.norm(grad_analytic))

        print(f"{param_name}: diff = {diff:.2e}")
        if diff < 1e-7:
            print("  ✓ Gradient correct")
        else:
            print("  ✗ Erreur dans backprop!")
\end{lstlisting}

% ===== SECTION 13: AVANTAGES ET LIMITES =====
\section{Avantages et Limites}

\subsection{Avantages}
\begin{itemize}
    \item ✅ Modèle les relations non-linéaires complexes
    \item ✅ Approximation universelle (théoriquement)
    \item ✅ Flexible : régression, classification, séries temporelles
    \item ✅ Apprentissage de représentations (features automatiques)
    \item ✅ Parallélisation sur GPU
\end{itemize}

\subsection{Limites}
\begin{itemize}
    \item ❌ Boîte noire (difficile à interpréter)
    \item ❌ Nécessite beaucoup de données
    \item ❌ Sensible aux hyperparamètres
    \item ❌ Risque d'overfitting
    \item ❌ Temps d'entraînement long
    \item ❌ Optima locaux (pas de garantie de convergence globale)
\end{itemize}

\subsection{Quand utiliser un MLP ?}

\begin{astuce}
Un MLP est particulièrement adapté quand :
\begin{itemize}
    \item Les données sont tabulaires (non structurées spatialement)
    \item Relations non-linéaires complexes
    \item Beaucoup de données disponibles
    \item Performance > interprétabilité
\end{itemize}
\end{astuce}

\begin{attention}
Préférer d'autres modèles dans les cas suivants :
\begin{itemize}
    \item Petites données ($< 1000$ exemples) → Random Forest, SVM
    \item Images → CNN (Chapitre 07)
    \item Séquences/texte → RNN, Transformers (Chapitre 08)
    \item Besoin d'interprétabilité → Arbres de décision, régression linéaire
\end{itemize}
\end{attention}

% ===== SECTION 14: HYPERPARAMÈTRES =====
\section{Hyperparamètres et Tuning}

\begin{table}[h]
\centering
\caption{Hyperparamètres principaux d'un MLP}
\label{tab:hyperparams_mlp}
\begin{tabular}{lll}
\toprule
\textbf{Paramètre} & \textbf{Valeurs typiques} & \textbf{Impact} \\
\midrule
\textbf{Architecture} & & \\
Nombre de couches & 2-5 & Capacité du modèle \\
Neurones par couche & 64, 128, 256, 512 & Capacité, overfitting \\
\midrule
\textbf{Optimisation} & & \\
Learning rate & $10^{-4}$ à $10^{-1}$ & Vitesse/stabilité \\
Batch size & 32, 64, 128, 256 & Vitesse, généralisation \\
Optimizer & SGD, Adam & Convergence \\
\midrule
\textbf{Régularisation} & & \\
Dropout & 0.2, 0.5 & Overfitting \\
L2 weight decay & $10^{-5}$ à $10^{-3}$ & Overfitting \\
\midrule
\textbf{Autres} & & \\
Activation & ReLU, Leaky ReLU & Gradient flow \\
Initialisation & He, Xavier & Convergence initiale \\
\bottomrule
\end{tabular}
\end{table}

\subsection{Stratégies de tuning}

\begin{enumerate}
    \item \textbf{Grid Search} : Tester toutes les combinaisons (coûteux)
    \item \textbf{Random Search} : Échantillonner aléatoirement (souvent meilleur)
    \item \textbf{Bayesian Optimization} : Optimisation intelligente (Optuna, Hyperopt)
\end{enumerate}

\begin{lstlisting}[language=Python, caption=Random search avec scikit-learn]
from sklearn.model_selection import RandomizedSearchCV
from sklearn.neural_network import MLPClassifier

param_distributions = {
    'hidden_layer_sizes': [(64,), (128,), (64, 32), (128, 64)],
    'activation': ['relu', 'tanh'],
    'alpha': [0.0001, 0.001, 0.01],  # L2 regularization
    'learning_rate_init': [0.001, 0.01, 0.1],
    'batch_size': [32, 64, 128]
}

mlp = MLPClassifier(max_iter=100)
random_search = RandomizedSearchCV(
    mlp, param_distributions, n_iter=20, cv=3, n_jobs=-1
)

random_search.fit(X_train, y_train)
print("Best params:", random_search.best_params_)
\end{lstlisting}

% ===== SECTION 15: APPLICATIONS =====
\section{Applications Pratiques}

\begin{enumerate}
    \item \textbf{Classification d'images} : MNIST, CIFAR-10 (avant CNN)
    \item \textbf{Reconnaissance vocale} : Phonèmes, commandes vocales
    \item \textbf{Prédiction de séries temporelles} : Finance, météo
    \item \textbf{Systèmes de recommandation} : Collaborative filtering
    \item \textbf{Détection de fraude} : Transactions bancaires
    \item \textbf{Diagnostic médical} : Prédiction de maladies à partir de biomarqueurs
    \item \textbf{NLP} : Sentiment analysis, classification de textes
\end{enumerate}

% ===== SECTION 16: RÉSUMÉ =====
\section{Résumé du Chapitre}

\subsection{Points Clés}
\begin{itemize}
    \item \textbf{Perceptron} : Classification binaire linéaire (limité au linéairement séparable)
    \item \textbf{MLP} : Empilage de couches avec activations non-linéaires (approximation universelle)
    \item \textbf{Forward pass} : Propagation des activations de l'entrée à la sortie
    \item \textbf{Backpropagation} : Calcul efficace des gradients via la chaîne rule
    \item \textbf{Fonctions d'activation} : ReLU (défaut), sigmoid (output binaire), softmax (multi-classe)
    \item \textbf{Optimiseurs} : SGD, Adam (défaut), momentum
    \item \textbf{Régularisation} : Dropout, L2, batch norm, early stopping
    \item \textbf{Initialisation} : He (ReLU), Xavier (sigmoid/tanh)
\end{itemize}

\subsection{Formules Essentielles}

\begin{tcolorbox}[colback=blue!5!white, colframe=blue!75!black, title=Formules à retenir]
\textbf{Forward pass (couche $l$) :}
\begin{align*}
    \vect{z}^{[l]} &= \mat{W}^{[l]} \vect{a}^{[l-1]} + \vect{b}^{[l]} \\
    \vect{a}^{[l]} &= \sigma^{[l]}(\vect{z}^{[l]})
\end{align*}

\textbf{Backpropagation :}
\begin{align*}
    \delta^{[l]} &= [(\mat{W}^{[l+1]})^T \delta^{[l+1]}] \odot \sigma'^{[l]}(\vect{z}^{[l]}) \\
    \frac{\partial L}{\partial \mat{W}^{[l]}} &= \delta^{[l]} (\vect{a}^{[l-1]})^T \\
    \frac{\partial L}{\partial \vect{b}^{[l]}} &= \delta^{[l]}
\end{align*}

\textbf{Cross-Entropy + Softmax :}
\begin{align*}
    L &= -\frac{1}{m} \sum_{i=1}^m \sum_{k=1}^K y_{i,k} \log(\hat{y}_{i,k}) \\
    \delta^{[L]} &= \hat{\vect{y}} - \vect{y}
\end{align*}

\textbf{Gradient Descent :}
\begin{equation*}
    \vect{\theta} \leftarrow \vect{\theta} - \alpha \nabla L(\vect{\theta})
\end{equation*}
\end{tcolorbox}

% ===== SECTION 17: EXERCICES =====
\section{Exercices}

\subsection{Questions de compréhension}
\begin{enumerate}
    \item Pourquoi le perceptron ne peut-il pas résoudre le problème XOR ?
    \item Que se passe-t-il si on initialise tous les poids à 0 ?
    \item Expliquer le problème du vanishing gradient avec la fonction sigmoid.
    \item Pourquoi ReLU est-il préféré à sigmoid dans les couches cachées ?
    \item Quelle est la différence entre batch, epoch et iteration ?
    \item Pourquoi utilise-t-on softmax + cross-entropy pour la classification multi-classe ?
\end{enumerate}

\subsection{Exercices pratiques}

\begin{enumerate}
    \item \textbf{Perceptron from scratch}
    \begin{itemize}
        \item Implémenter le perceptron en NumPy
        \item Tester sur un dataset linéairement séparable (make\_classification)
        \item Visualiser la frontière de décision
    \end{itemize}

    \item \textbf{MLP pour MNIST}
    \begin{itemize}
        \item Charger le dataset MNIST (sklearn.datasets.load\_digits)
        \item Entraîner un MLP from scratch (architecture libre)
        \item Comparer avec MLPClassifier de scikit-learn
        \item Objectif : > 95\% accuracy
    \end{itemize}

    \item \textbf{Gradient checking}
    \begin{itemize}
        \item Implémenter le gradient checking
        \item Vérifier votre implémentation de backprop
    \end{itemize}

    \item \textbf{Régularisation}
    \begin{itemize}
        \item Comparer MLP avec/sans dropout
        \item Tester différentes valeurs de L2 regularization
        \item Tracer les courbes d'apprentissage
    \end{itemize}

    \item \textbf{Hyperparameter tuning}
    \begin{itemize}
        \item Utiliser RandomizedSearchCV pour trouver les meilleurs hyperparamètres
        \item Comparer GridSearch vs RandomSearch
    \end{itemize}
\end{enumerate}

\textit{Solutions disponibles dans} \texttt{06\_exercices.ipynb} \textit{(solutions intégrées dans le notebook)}

% ===== SECTION 18: POUR ALLER PLUS LOIN =====
\section{Pour Aller Plus Loin}

\subsection{Lectures Recommandées}
\begin{itemize}
    \item \textbf{Deep Learning Book} (Goodfellow et al., 2016) - Chapitres 6-8
    \item \textbf{Neural Networks and Deep Learning} (Michael Nielsen) - En ligne gratuit
    \item Rumelhart et al. (1986) - "Learning representations by back-propagating errors"
    \item Glorot \& Bengio (2010) - "Understanding the difficulty of training deep feedforward neural networks"
\end{itemize}

\subsection{Ressources en Ligne}
\begin{itemize}
    \item Playground TensorFlow : \url{https://playground.tensorflow.org/}
    \item 3Blue1Brown - Neural Networks : \url{https://www.youtube.com/watch?v=aircAruvnKk}
    \item Documentation scikit-learn MLP : \url{https://scikit-learn.org/stable/modules/neural_networks_supervised.html}
    \item PyTorch Tutorials : \url{https://pytorch.org/tutorials/beginner/basics/buildmodel_tutorial.html}
\end{itemize}

\subsection{Extensions}
\begin{itemize}
    \item \textbf{Residual Networks (ResNet)} : Connexions résiduelles pour réseaux très profonds
    \item \textbf{Batch Normalization} : Normalisation des activations
    \item \textbf{Autoencoders} : Apprentissage de représentations non supervisé
    \item \textbf{Transfer Learning} : Réutilisation de réseaux pré-entraînés
\end{itemize}

\subsection{Prochaines Étapes}
Chapitre suivant recommandé : \textbf{Chapitre 07 - Deep Learning : Réseaux de Neurones Convolutifs (CNN)}

Les CNN sont une architecture spécialisée pour les données spatiales (images) qui exploite la localité et l'invariance par translation.

% ===== BIBLIOGRAPHIE =====
\section*{Références}
\begin{enumerate}
    \item Goodfellow, I., Bengio, Y., \& Courville, A. (2016). \textit{Deep Learning}. MIT Press.
    \item Rumelhart, D. E., Hinton, G. E., \& Williams, R. J. (1986). "Learning representations by back-propagating errors". \textit{Nature}, 323(6088), 533-536.
    \item LeCun, Y., Bengio, Y., \& Hinton, G. (2015). "Deep learning". \textit{Nature}, 521(7553), 436-444.
    \item Kingma, D. P., \& Ba, J. (2014). "Adam: A method for stochastic optimization". \textit{arXiv preprint arXiv:1412.6980}.
    \item Srivastava, N., et al. (2014). "Dropout: A simple way to prevent neural networks from overfitting". \textit{JMLR}, 15(1), 1929-1958.
    \item He, K., et al. (2015). "Delving deep into rectifiers: Surpassing human-level performance on ImageNet classification". \textit{ICCV}.
\end{enumerate}

\end{document}
