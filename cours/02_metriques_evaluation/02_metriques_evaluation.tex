% Chapitre 02 - Métriques d'Évaluation
% Cours Machine Learning - Sandbox-ML

\documentclass[11pt,a4paper]{article}

% ===== PACKAGES =====
\usepackage[utf8]{inputenc}
\usepackage[T1]{fontenc}
\usepackage[french]{babel}
\usepackage{lmodern}

% Mathématiques
\usepackage{amsmath, amssymb, amsthm}
\usepackage{mathtools}

% Mise en page
\usepackage[margin=2.5cm]{geometry}
\usepackage{parskip}
\usepackage{setspace}
\setstretch{1.15}

% Graphiques et couleurs
\usepackage{graphicx}
\usepackage{xcolor}
\usepackage{tikz}
\usetikzlibrary{arrows.meta, positioning, shapes.geometric, matrix, fit}

% Tableaux
\usepackage{booktabs}
\usepackage{longtable}
\usepackage{multirow}
\usepackage{tabularx}
\usepackage{colortbl}

% Code et algorithmes
\usepackage{listings}
\usepackage{algorithm}
\usepackage{algorithmic}

% Hyperliens
\usepackage{hyperref}
\hypersetup{
    colorlinks=true,
    linkcolor=blue,
    filecolor=magenta,
    urlcolor=cyan,
    citecolor=green,
    pdftitle={Chapitre 02 - Métriques d'Évaluation},
    pdfauthor={Cours ML},
}

% Boxes colorées
\usepackage{tcolorbox}
\tcbuselibrary{skins, breakable}

% En-têtes et pieds de page
\usepackage{fancyhdr}
\pagestyle{fancy}
\fancyhf{}
\fancyhead[L]{\small Chapitre 02 - Métriques d'Évaluation}
\fancyhead[R]{\small Cours Machine Learning}
\fancyfoot[C]{\thepage}

% ===== CONFIGURATION LISTINGS (code Python) =====
\definecolor{codegreen}{rgb}{0,0.6,0}
\definecolor{codegray}{rgb}{0.5,0.5,0.5}
\definecolor{codepurple}{rgb}{0.58,0,0.82}
\definecolor{backcolour}{rgb}{0.95,0.95,0.92}

\lstdefinestyle{pythonstyle}{
    language=Python,
    backgroundcolor=\color{backcolour},
    commentstyle=\color{codegreen},
    keywordstyle=\color{blue},
    numberstyle=\tiny\color{codegray},
    stringstyle=\color{codepurple},
    basicstyle=\ttfamily\small,
    breakatwhitespace=false,
    breaklines=true,
    captionpos=b,
    keepspaces=true,
    numbers=left,
    numbersep=5pt,
    showspaces=false,
    showstringspaces=false,
    showtabs=false,
    tabsize=4,
    frame=single,
    rulecolor=\color{black}
}
\lstset{style=pythonstyle}

% ===== CONFIGURATION TCOLORBOX =====
% Box pour définitions
\newtcolorbox{definition}[1]{
    colback=blue!5!white,
    colframe=blue!75!black,
    fonttitle=\bfseries,
    title=Définition: #1,
    breakable
}

% Box pour théorèmes
\newtcolorbox{theoreme}[1]{
    colback=green!5!white,
    colframe=green!75!black,
    fonttitle=\bfseries,
    title=Théorème: #1,
    breakable
}

% Box pour exemples
\newtcolorbox{exemple}[1]{
    colback=orange!5!white,
    colframe=orange!75!black,
    fonttitle=\bfseries,
    title=Exemple: #1,
    breakable
}

% Box pour attention/warning
\newtcolorbox{attention}{
    colback=red!5!white,
    colframe=red!75!black,
    fonttitle=\bfseries,
    title=⚠️ Attention,
    breakable
}

% Box pour astuce/tips
\newtcolorbox{astuce}{
    colback=yellow!10!white,
    colframe=yellow!75!black,
    fonttitle=\bfseries,
    title=💡 Astuce,
    breakable
}

% ===== COMMANDES PERSONNALISÉES =====
\newcommand{\vect}[1]{\mathbf{#1}}  % Vecteur
\newcommand{\mat}[1]{\mathbf{#1}}   % Matrice
\newcommand{\R}{\mathbb{R}}         % Réels
\newcommand{\N}{\mathbb{N}}         % Naturels
\newcommand{\argmin}{\operatorname{argmin}}
\newcommand{\argmax}{\operatorname{argmax}}

% ===== DÉBUT DU DOCUMENT =====
\begin{document}

% ===== PAGE DE TITRE =====
\begin{titlepage}
    \centering
    \vspace*{2cm}

    {\Huge\bfseries Cours Machine Learning}\\[0.5cm]

    \vspace{1cm}

    {\LARGE Chapitre 02}\\[0.3cm]
    {\LARGE\bfseries Métriques d'Évaluation}\\[2cm]

    \vfill

    {\large
    \textbf{Objectifs d'apprentissage :}\\[0.5cm]
    \begin{itemize}
        \item Maîtriser les métriques de classification (matrice de confusion, accuracy, precision, recall, F1-score, ROC, AUC)
        \item Comprendre les métriques de régression (MSE, RMSE, MAE, R², MAPE)
        \item Savoir choisir la bonne métrique selon le problème
        \item Maîtriser les techniques de validation (train/test split, K-fold, stratified K-fold)
        \item Identifier les pièges courants et éviter l'overfitting
    \end{itemize}
    }

    \vfill

    {\large
    \textbf{Prérequis :} Chapitre 01 - Fondamentaux Mathématiques\\[0.3cm]
    \textbf{Durée estimée :} 6-8 heures\\[0.3cm]
    \textbf{Notebooks :} \texttt{02_demo_*.ipynb}
    }

    \vfill

    {\large Cours ML - Sandbox-ML\\
    Version 1.0 - 2026}
\end{titlepage}

% ===== TABLE DES MATIÈRES =====
\tableofcontents
\newpage

% ===== SECTION 1: MOTIVATION =====
\section{Motivation}

L'évaluation des modèles de Machine Learning est une étape cruciale qui détermine si un modèle est prêt pour la production. Contrairement à la programmation traditionnelle où on peut vérifier la correction d'un programme par des tests unitaires, en ML, un modèle n'est jamais parfait à 100\%.

\begin{exemple}{Le dilemme du diagnostic médical}
Imaginez un système de détection automatique du cancer à partir de radiographies :

\textbf{Scénario A :} Le modèle détecte 95\% des cancers (excellent !) mais donne 40\% de faux positifs (mauvais !).
\begin{itemize}
    \item \textbf{Conséquence :} 40\% des patients sains passent des examens invasifs inutiles
    \item \textbf{Coût :} Anxiété des patients + surcharge du système de santé
\end{itemize}

\textbf{Scénario B :} Le modèle ne donne que 2\% de faux positifs (excellent !) mais ne détecte que 60\% des cancers (catastrophique !).
\begin{itemize}
    \item \textbf{Conséquence :} 40\% des malades ne sont pas diagnostiqués
    \item \textbf{Coût :} Vies humaines
\end{itemize}

\textbf{Question clé :} Quelle métrique unique utiliser pour comparer ces deux modèles ? L'accuracy ? La précision ? Le recall ? Le F1-score ?

\textbf{Réponse :} Cela dépend du contexte médical et des coûts associés. Ce chapitre vous apprendra à faire ce choix de manière éclairée.
\end{exemple}

\begin{attention}
Une métrique unique ne suffit jamais ! Un modèle avec 99\% d'accuracy peut être complètement inutile si :
\begin{itemize}
    \item Les classes sont déséquilibrées (ex: 1\% de fraudes bancaires)
    \item Les coûts d'erreur sont asymétriques (faux négatif en médecine >> faux positif)
    \item Le modèle ne généralise pas (overfitting)
\end{itemize}
\end{attention}

% ===== SECTION 2: MÉTRIQUES DE CLASSIFICATION =====
\section{Métriques de Classification}

\subsection{Matrice de Confusion}

\begin{definition}{Matrice de Confusion}
La \textbf{matrice de confusion} est un tableau qui décrit les performances d'un modèle de classification en comparant les prédictions aux vraies valeurs. Pour un problème de classification binaire, elle contient 4 valeurs :
\end{definition}

\begin{table}[h]
\centering
\caption{Matrice de Confusion (Classification Binaire)}
\label{tab:confusion_matrix}
\begin{tabular}{cc|cc}
\toprule
\multicolumn{2}{c|}{} & \multicolumn{2}{c}{\textbf{Prédiction}} \\
\multicolumn{2}{c|}{} & Positive & Negative \\
\midrule
\multirow{2}{*}{\rotatebox[origin=c]{90}{\textbf{Réalité}}}
& Positive & \cellcolor{green!20}TP (Vrai Positif) & \cellcolor{red!20}FN (Faux Négatif) \\
& Negative & \cellcolor{red!20}FP (Faux Positif) & \cellcolor{green!20}TN (Vrai Négatif) \\
\bottomrule
\end{tabular}
\end{table}

\begin{itemize}
    \item \textbf{TP (True Positive)} : Cas positifs correctement classés comme positifs
    \item \textbf{TN (True Negative)} : Cas négatifs correctement classés comme négatifs
    \item \textbf{FP (False Positive)} : Cas négatifs incorrectement classés comme positifs (Erreur de Type I)
    \item \textbf{FN (False Negative)} : Cas positifs incorrectement classés comme négatifs (Erreur de Type II)
\end{itemize}

\begin{exemple}{Détection de spam}
Sur 1000 emails analysés :
\begin{itemize}
    \item 150 vrais spams, 850 emails légitimes
    \item TP = 130 : spams détectés correctement
    \item FN = 20 : spams non détectés (passent en boîte de réception)
    \item FP = 40 : emails légitimes classés comme spam (perdus !)
    \item TN = 810 : emails légitimes correctement classés
\end{itemize}

\begin{table}[h]
\centering
\begin{tabular}{cc|cc|c}
\toprule
\multicolumn{2}{c|}{} & \multicolumn{2}{c|}{\textbf{Prédiction}} & \\
\multicolumn{2}{c|}{} & Spam & Légitime & Total \\
\midrule
\multirow{2}{*}{\rotatebox[origin=c]{90}{\textbf{Réalité}}}
& Spam & \cellcolor{green!20}130 & \cellcolor{red!20}20 & 150 \\
& Légitime & \cellcolor{red!20}40 & \cellcolor{green!20}810 & 850 \\
\midrule
& Total & 170 & 830 & 1000 \\
\bottomrule
\end{tabular}
\end{table}
\end{exemple}

\subsection{Accuracy (Exactitude)}

\begin{definition}{Accuracy}
L'\textbf{accuracy} (exactitude) est la proportion de prédictions correctes sur l'ensemble total :
\begin{equation}
    \text{Accuracy} = \frac{\text{TP} + \text{TN}}{\text{TP} + \text{TN} + \text{FP} + \text{FN}} = \frac{\text{Correct}}{\text{Total}}
\end{equation}
\end{definition}

Valeur : entre 0 et 1 (ou 0\% et 100\%).

\textbf{Pour l'exemple spam :}
\begin{equation}
    \text{Accuracy} = \frac{130 + 810}{1000} = \frac{940}{1000} = 0.94 = 94\%
\end{equation}

\begin{attention}
L'accuracy peut être trompeuse avec des classes déséquilibrées !

\textbf{Exemple :} Détection de fraude bancaire (0.1\% de fraudes)
\begin{itemize}
    \item Sur 10 000 transactions : 10 fraudes, 9 990 légitimes
    \item Modèle stupide : prédire toujours "légitime"
    \item Accuracy = $\frac{9990}{10000} = 99.9\%$ (excellent en apparence !)
    \item Mais 0\% de fraudes détectées (catastrophique !)
\end{itemize}

\textbf{Conclusion :} L'accuracy seule est insuffisante pour des classes déséquilibrées.
\end{attention}

\subsection{Precision (Précision)}

\begin{definition}{Precision}
La \textbf{precision} (précision) mesure la proportion de prédictions positives qui sont correctes :
\begin{equation}
    \text{Precision} = \frac{\text{TP}}{\text{TP} + \text{FP}} = \frac{\text{TP}}{\text{Total Prédictions Positives}}
\end{equation}
\end{definition}

\textbf{Interprétation :} "Quand le modèle prédit 'positif', à quelle fréquence a-t-il raison ?"

\textbf{Pour l'exemple spam :}
\begin{equation}
    \text{Precision} = \frac{130}{130 + 40} = \frac{130}{170} \approx 0.765 = 76.5\%
\end{equation}

Cela signifie que 76.5\% des emails classés comme spam sont réellement des spams. Les 23.5\% restants sont des faux positifs (emails légitimes perdus).

\begin{astuce}
Utilisez la \textbf{precision} quand les \textbf{faux positifs sont coûteux} :
\begin{itemize}
    \item Filtrage spam : éviter de perdre des emails importants
    \item Recommandation de produits : éviter de recommander des produits non pertinents
    \item Publicité ciblée : éviter de dépenser du budget sur de mauvaises cibles
\end{itemize}
\end{astuce}

\subsection{Recall (Rappel / Sensibilité)}

\begin{definition}{Recall (Sensibilité)}
Le \textbf{recall} (rappel ou sensibilité) mesure la proportion de cas positifs réels qui sont détectés :
\begin{equation}
    \text{Recall} = \frac{\text{TP}}{\text{TP} + \text{FN}} = \frac{\text{TP}}{\text{Total Cas Positifs Réels}}
\end{equation}
\end{definition}

\textbf{Interprétation :} "Parmi tous les cas positifs réels, combien le modèle en détecte-t-il ?"

\textbf{Pour l'exemple spam :}
\begin{equation}
    \text{Recall} = \frac{130}{130 + 20} = \frac{130}{150} \approx 0.867 = 86.7\%
\end{equation}

Cela signifie que 86.7\% des spams sont détectés. Les 13.3\% restants passent en boîte de réception (faux négatifs).

\begin{astuce}
Utilisez le \textbf{recall} quand les \textbf{faux négatifs sont coûteux} :
\begin{itemize}
    \item Détection de cancer : manquer un cancer peut être fatal
    \item Détection de fraude : manquer une fraude peut coûter cher
    \item Détection d'intrusion réseau : manquer une attaque peut compromettre le système
\end{itemize}
\end{astuce}

\subsection{Spécificité}

\begin{definition}{Spécificité}
La \textbf{spécificité} mesure la proportion de cas négatifs réels qui sont correctement classés :
\begin{equation}
    \text{Spécificité} = \frac{\text{TN}}{\text{TN} + \text{FP}} = \frac{\text{TN}}{\text{Total Cas Négatifs Réels}}
\end{equation}
\end{definition}

\textbf{Pour l'exemple spam :}
\begin{equation}
    \text{Spécificité} = \frac{810}{810 + 40} = \frac{810}{850} \approx 0.953 = 95.3\%
\end{equation}

La spécificité est le complément du taux de faux positifs (FPR) :
\begin{equation}
    \text{FPR} = 1 - \text{Spécificité} = \frac{\text{FP}}{\text{TN} + \text{FP}}
\end{equation}

\subsection{F1-Score (Moyenne Harmonique)}

\begin{definition}{F1-Score}
Le \textbf{F1-score} est la moyenne harmonique entre la precision et le recall :
\begin{equation}
    F_1 = 2 \times \frac{\text{Precision} \times \text{Recall}}{\text{Precision} + \text{Recall}} = \frac{2 \times \text{TP}}{2 \times \text{TP} + \text{FP} + \text{FN}}
\end{equation}
\end{definition}

\textbf{Pour l'exemple spam :}
\begin{equation}
    F_1 = 2 \times \frac{0.765 \times 0.867}{0.765 + 0.867} = 2 \times \frac{0.663}{1.632} \approx 0.813 = 81.3\%
\end{equation}

\textbf{Pourquoi la moyenne harmonique ?}

La moyenne harmonique pénalise les déséquilibres :
\begin{itemize}
    \item Si Precision = 100\% et Recall = 10\% : $F_1 \approx 18\%$ (faible !)
    \item Si Precision = 90\% et Recall = 90\% : $F_1 = 90\%$ (bon équilibre)
\end{itemize}

En comparaison, la moyenne arithmétique donnerait :
\begin{itemize}
    \item $(100\% + 10\%) / 2 = 55\%$ (trop optimiste !)
    \item $(90\% + 90\%) / 2 = 90\%$ (identique)
\end{itemize}

\begin{astuce}
Le F1-score est la métrique par défaut quand :
\begin{itemize}
    \item Les classes sont déséquilibrées
    \item Precision et Recall sont tous deux importants
    \item Vous cherchez un bon compromis entre FP et FN
\end{itemize}
\end{astuce}

\subsubsection{Variante : F-Beta Score}

Pour donner plus de poids au recall ou à la precision, on utilise le \textbf{F-beta score} :
\begin{equation}
    F_\beta = (1 + \beta^2) \times \frac{\text{Precision} \times \text{Recall}}{\beta^2 \times \text{Precision} + \text{Recall}}
\end{equation}

\begin{itemize}
    \item $\beta = 1$ : F1-score (poids égal)
    \item $\beta = 0.5$ : F0.5-score (favorise la precision)
    \item $\beta = 2$ : F2-score (favorise le recall)
\end{itemize}

\subsection{Courbe ROC et AUC}

\subsubsection{Courbe ROC}

\begin{definition}{Courbe ROC}
La \textbf{courbe ROC} (Receiver Operating Characteristic) représente le compromis entre le taux de vrais positifs (TPR = Recall) et le taux de faux positifs (FPR) pour différents seuils de classification.

\textbf{Axes :}
\begin{itemize}
    \item Axe X : FPR (False Positive Rate) = $\frac{\text{FP}}{\text{FP} + \text{TN}} = 1 - \text{Spécificité}$
    \item Axe Y : TPR (True Positive Rate) = $\frac{\text{TP}}{\text{TP} + \text{FN}} = \text{Recall}$
\end{itemize}
\end{definition}

La plupart des classifieurs produisent des probabilités (ex: \texttt{predict\_proba()} en scikit-learn). La classification finale dépend d'un seuil (généralement 0.5) :
\begin{itemize}
    \item Si $P(\text{classe positive}) \geq 0.5$ : prédire "positif"
    \item Si $P(\text{classe positive}) < 0.5$ : prédire "négatif"
\end{itemize}

La courbe ROC trace TPR vs FPR pour tous les seuils possibles (0 à 1).

\textbf{Interprétation visuelle :}
\begin{itemize}
    \item \textbf{Classifieur aléatoire} : ligne diagonale (TPR = FPR)
    \item \textbf{Classifieur parfait} : passe par le point (0, 1) (TPR = 100\%, FPR = 0\%)
    \item \textbf{Meilleur modèle} : courbe la plus proche du coin supérieur gauche
\end{itemize}

\subsubsection{AUC (Area Under Curve)}

\begin{definition}{AUC}
L'\textbf{AUC} (Area Under the ROC Curve) mesure l'aire sous la courbe ROC. Elle représente la probabilité qu'un classifieur classe un exemple positif aléatoire plus haut qu'un exemple négatif aléatoire.
\begin{equation}
    \text{AUC} \in [0, 1]
\end{equation}
\end{definition}

\textbf{Interprétation :}
\begin{itemize}
    \item AUC = 0.5 : Classifieur aléatoire (inutile)
    \item AUC = 0.7 : Acceptable
    \item AUC = 0.8 : Bon
    \item AUC = 0.9 : Excellent
    \item AUC = 1.0 : Parfait (rare en pratique, souvent signe d'overfitting)
\end{itemize}

\begin{astuce}
L'AUC est une excellente métrique car :
\begin{itemize}
    \item Insensible au seuil de classification
    \item Robuste aux classes déséquilibrées (dans une certaine mesure)
    \item Permet de comparer différents modèles indépendamment du seuil choisi
\end{itemize}
\end{astuce}

\subsection{Courbe Precision-Recall}

\begin{definition}{Courbe Precision-Recall}
La \textbf{courbe Precision-Recall} trace la precision en fonction du recall pour différents seuils.

\textbf{Axes :}
\begin{itemize}
    \item Axe X : Recall (TPR)
    \item Axe Y : Precision
\end{itemize}
\end{definition}

\textbf{Différence avec ROC :}
\begin{itemize}
    \item La courbe ROC utilise FPR (sensible aux TN)
    \item La courbe PR utilise Precision (ignore les TN)
\end{itemize}

\begin{astuce}
Préférez la courbe \textbf{Precision-Recall} plutôt que ROC quand :
\begin{itemize}
    \item Les classes sont très déséquilibrées (ex: 1\% de positifs)
    \item Vous vous souciez plus des positifs que des négatifs
    \item Exemples : détection de fraude, détection d'anomalies, recherche d'information
\end{itemize}

\textbf{Raison :} Avec 99\% de négatifs, FPR reste proche de 0 même avec beaucoup de FP, ce qui rend la courbe ROC trop optimiste.
\end{astuce}

\subsection{Métriques Multi-Classes}

Pour les problèmes de classification à plus de 2 classes, on peut calculer les métriques de trois façons :

\begin{definition}{Moyennes pour Multi-Classes}
\begin{itemize}
    \item \textbf{Macro Average} : moyenne simple des métriques de chaque classe
    \begin{equation}
        \text{Precision}_{\text{macro}} = \frac{1}{C} \sum_{i=1}^{C} \text{Precision}_i
    \end{equation}
    Traite toutes les classes de manière égale (utile si toutes les classes sont aussi importantes).

    \item \textbf{Micro Average} : calcule les métriques globalement (agrège TP, FP, FN)
    \begin{equation}
        \text{Precision}_{\text{micro}} = \frac{\sum_{i=1}^{C} \text{TP}_i}{\sum_{i=1}^{C} (\text{TP}_i + \text{FP}_i)}
    \end{equation}
    Favorise les classes majoritaires (utile si les classes déséquilibrées reflètent la réalité).

    \item \textbf{Weighted Average} : moyenne pondérée par le nombre d'exemples de chaque classe
    \begin{equation}
        \text{Precision}_{\text{weighted}} = \frac{1}{N} \sum_{i=1}^{C} n_i \times \text{Precision}_i
    \end{equation}
    où $n_i$ est le nombre d'exemples de la classe $i$ et $N = \sum n_i$.
\end{itemize}
\end{definition}

\begin{exemple}{Classification d'images (3 classes)}
\begin{table}[h]
\centering
\caption{Résultats par classe}
\begin{tabular}{lcccc}
\toprule
\textbf{Classe} & \textbf{Support} & \textbf{Precision} & \textbf{Recall} & \textbf{F1} \\
\midrule
Chat & 100 & 0.90 & 0.85 & 0.87 \\
Chien & 100 & 0.88 & 0.90 & 0.89 \\
Oiseau & 20 & 0.70 & 0.60 & 0.65 \\
\midrule
\textbf{Macro avg} & 220 & 0.83 & 0.78 & 0.80 \\
\textbf{Weighted avg} & 220 & 0.87 & 0.85 & 0.86 \\
\bottomrule
\end{tabular}
\end{table}

\textbf{Analyse :}
\begin{itemize}
    \item Macro average = $(0.90 + 0.88 + 0.70) / 3 = 0.83$ (classe "Oiseau" pèse autant que les autres)
    \item Weighted average plus élevée car elle donne plus de poids aux classes majoritaires (Chat, Chien)
\end{itemize}
\end{exemple}

% ===== SECTION 3: MÉTRIQUES DE RÉGRESSION =====
\section{Métriques de Régression}

Pour les problèmes de régression, on cherche à mesurer l'écart entre les prédictions continues et les vraies valeurs.

\textbf{Notations :}
\begin{itemize}
    \item $y_i$ : vraie valeur pour l'exemple $i$
    \item $\hat{y}_i$ : prédiction pour l'exemple $i$
    \item $n$ : nombre d'exemples
    \item $\bar{y} = \frac{1}{n} \sum_{i=1}^{n} y_i$ : moyenne des vraies valeurs
\end{itemize}

\subsection{MSE (Mean Squared Error)}

\begin{definition}{MSE}
Le \textbf{MSE} (Mean Squared Error) est la moyenne des carrés des erreurs :
\begin{equation}
    \text{MSE} = \frac{1}{n} \sum_{i=1}^{n} (y_i - \hat{y}_i)^2
\end{equation}
\end{definition}

\textbf{Propriétés :}
\begin{itemize}
    \item Toujours $\geq 0$ (0 = prédictions parfaites)
    \item Pénalise fortement les grandes erreurs (à cause du carré)
    \item Unité : carré de l'unité de $y$ (difficile à interpréter)
    \item Très sensible aux outliers
\end{itemize}

\begin{exemple}{Prédiction de prix immobiliers}
Prix réels : $[200, 250, 300, 350, 400]$ (en k€)\\
Prédictions : $[210, 240, 310, 340, 500]$ (en k€)\\
Erreurs : $[10, -10, 10, -10, 100]$

\begin{equation}
    \text{MSE} = \frac{10^2 + 10^2 + 10^2 + 10^2 + 100^2}{5} = \frac{10300}{5} = 2060 \text{ (k€)}^2
\end{equation}

L'erreur de 100 k€ domine complètement le MSE.
\end{exemple}

\subsection{RMSE (Root Mean Squared Error)}

\begin{definition}{RMSE}
Le \textbf{RMSE} est la racine carrée du MSE :
\begin{equation}
    \text{RMSE} = \sqrt{\text{MSE}} = \sqrt{\frac{1}{n} \sum_{i=1}^{n} (y_i - \hat{y}_i)^2}
\end{equation}
\end{definition}

\textbf{Avantage :} Même unité que $y$ (plus facile à interpréter).

\textbf{Pour l'exemple précédent :}
\begin{equation}
    \text{RMSE} = \sqrt{2060} \approx 45.4 \text{ k€}
\end{equation}

Interprétation : en moyenne, les prédictions s'écartent de 45.4 k€ de la réalité.

\subsection{MAE (Mean Absolute Error)}

\begin{definition}{MAE}
Le \textbf{MAE} (Mean Absolute Error) est la moyenne des valeurs absolues des erreurs :
\begin{equation}
    \text{MAE} = \frac{1}{n} \sum_{i=1}^{n} |y_i - \hat{y}_i|
\end{equation}
\end{definition}

\textbf{Propriétés :}
\begin{itemize}
    \item Toujours $\geq 0$
    \item Moins sensible aux outliers que MSE/RMSE
    \item Même unité que $y$
    \item Pénalise toutes les erreurs de manière proportionnelle
\end{itemize}

\textbf{Pour l'exemple précédent :}
\begin{equation}
    \text{MAE} = \frac{|10| + |10| + |10| + |10| + |100|}{5} = \frac{140}{5} = 28 \text{ k€}
\end{equation}

\textbf{Comparaison MSE vs MAE :}
\begin{itemize}
    \item RMSE = 45.4 k€ (fortement influencé par l'outlier de 100 k€)
    \item MAE = 28 k€ (moins influencé par l'outlier)
\end{itemize}

\begin{astuce}
Choisir entre RMSE et MAE :
\begin{itemize}
    \item Utilisez \textbf{RMSE} si les grandes erreurs sont très pénalisantes (finance, sécurité)
    \item Utilisez \textbf{MAE} si toutes les erreurs ont un coût similaire (météo, estimation générale)
    \item En cas de doute, reportez les deux !
\end{itemize}
\end{astuce}

\subsection{R² (Coefficient de Détermination)}

\begin{definition}{R²}
Le \textbf{R²} (coefficient de détermination) mesure la proportion de variance expliquée par le modèle :
\begin{equation}
    R^2 = 1 - \frac{\text{SS}_{\text{res}}}{\text{SS}_{\text{tot}}} = 1 - \frac{\sum_{i=1}^{n} (y_i - \hat{y}_i)^2}{\sum_{i=1}^{n} (y_i - \bar{y})^2}
\end{equation}

où :
\begin{itemize}
    \item $\text{SS}_{\text{res}} = \sum (y_i - \hat{y}_i)^2$ : somme des carrés des résidus (variance non expliquée)
    \item $\text{SS}_{\text{tot}} = \sum (y_i - \bar{y})^2$ : variance totale
\end{itemize}
\end{definition}

\textbf{Interprétation :}
\begin{itemize}
    \item $R^2 = 1$ : modèle parfait (prédictions exactes)
    \item $R^2 = 0$ : modèle équivalent à prédire la moyenne $\bar{y}$
    \item $R^2 < 0$ : modèle pire que prédire la moyenne (très mauvais !)
\end{itemize}

\textbf{Exemple :}
\begin{itemize}
    \item Si $R^2 = 0.85$ : le modèle explique 85\% de la variance des données
    \item Les 15\% restants sont dus au bruit ou à des variables non prises en compte
\end{itemize}

\begin{attention}
Limites du R² :
\begin{itemize}
    \item Augmente automatiquement quand on ajoute des features (même inutiles)
    \item Ne détecte pas l'overfitting
    \item Solution : utiliser le \textbf{R² ajusté} qui pénalise le nombre de features
\end{itemize}

\begin{equation}
    R^2_{\text{ajusté}} = 1 - \frac{(1 - R^2)(n - 1)}{n - p - 1}
\end{equation}
où $p$ est le nombre de features.
\end{attention}

\subsection{MAPE (Mean Absolute Percentage Error)}

\begin{definition}{MAPE}
Le \textbf{MAPE} mesure l'erreur moyenne en pourcentage :
\begin{equation}
    \text{MAPE} = \frac{100\%}{n} \sum_{i=1}^{n} \left| \frac{y_i - \hat{y}_i}{y_i} \right|
\end{equation}
\end{definition}

\textbf{Avantage :} Indépendant de l'échelle (permet de comparer des modèles sur différents datasets).

\textbf{Interprétation :} MAPE = 10\% signifie que les prédictions s'écartent en moyenne de 10\% des vraies valeurs.

\begin{attention}
Problème du MAPE :
\begin{itemize}
    \item Non défini si $y_i = 0$
    \item Asymétrique : pénalise plus les surestimations que les sous-estimations
    \item Exemple : erreur de +50\% vs -50\% ne donnent pas le même MAPE
\end{itemize}
\end{attention}

\subsection{Comparaison des Métriques de Régression}

\begin{table}[h]
\centering
\caption{Comparaison des métriques de régression}
\label{tab:regression_metrics}
\begin{tabular}{lccc}
\toprule
\textbf{Métrique} & \textbf{Sensibilité outliers} & \textbf{Interprétation} & \textbf{Unité} \\
\midrule
MSE & Très élevée & Difficile & $(y)^2$ \\
RMSE & Élevée & Facile & $y$ \\
MAE & Modérée & Facile & $y$ \\
R² & N/A & Facile & Sans unité \\
MAPE & Modérée & Facile & \% \\
\bottomrule
\end{tabular}
\end{table}

% ===== SECTION 4: VALIDATION =====
\section{Techniques de Validation}

L'évaluation d'un modèle ne peut pas se faire sur les données d'entraînement ! Un modèle peut mémoriser les données (overfitting) et obtenir 100\% d'accuracy en entraînement mais échouer en production.

\begin{attention}
\textbf{Règle d'or :} Ne jamais évaluer un modèle sur les données utilisées pour l'entraîner !
\end{attention}

\subsection{Train/Test Split}

\begin{definition}{Train/Test Split}
La technique la plus simple : diviser aléatoirement le dataset en deux ensembles :
\begin{itemize}
    \item \textbf{Training set} (70-80\%) : pour entraîner le modèle
    \item \textbf{Test set} (20-30\%) : pour évaluer le modèle
\end{itemize}
\end{definition}

\begin{lstlisting}[language=Python, caption=Train/Test Split avec scikit-learn]
from sklearn.model_selection import train_test_split

# Split 80/20
X_train, X_test, y_train, y_test = train_test_split(
    X, y,
    test_size=0.2,      # 20% pour le test
    random_state=42,    # reproductibilité
    stratify=y          # préserve distribution des classes
)

# Entraînement
model.fit(X_train, y_train)

# Évaluation sur données JAMAIS vues
y_pred = model.predict(X_test)
accuracy = accuracy_score(y_test, y_pred)
\end{lstlisting}

\textbf{Avantages :}
\begin{itemize}
    \item Simple et rapide
    \item Suffisant pour les grands datasets
\end{itemize}

\textbf{Limites :}
\begin{itemize}
    \item Résultat dépend du split aléatoire (variance élevée)
    \item Perte de données (20-30\% non utilisés pour l'entraînement)
    \item Risque de split non représentatif sur petits datasets
\end{itemize}

\subsection{Validation Set (Train/Validation/Test)}

Pour optimiser les hyperparamètres, il faut un 3ème ensemble :

\begin{definition}{Train/Validation/Test Split}
\begin{itemize}
    \item \textbf{Training set} (60\%) : entraînement du modèle
    \item \textbf{Validation set} (20\%) : tuning des hyperparamètres
    \item \textbf{Test set} (20\%) : évaluation finale (jamais touché avant)
\end{itemize}
\end{definition}

\textbf{Workflow :}
\begin{enumerate}
    \item Entraîner plusieurs modèles avec différents hyperparamètres sur le training set
    \item Évaluer chaque modèle sur le validation set
    \item Choisir le meilleur modèle
    \item Évaluation finale sur le test set (une seule fois !)
\end{enumerate}

\begin{attention}
Si vous évaluez plusieurs fois sur le test set et ajustez le modèle en conséquence, le test set devient un validation set déguisé ! Vous risquez l'overfitting sur le test set.
\end{attention}

\subsection{K-Fold Cross-Validation}

\begin{definition}{K-Fold Cross-Validation}
La \textbf{validation croisée K-fold} divise le dataset en $K$ sous-ensembles (folds) de taille égale. Le modèle est entraîné et évalué $K$ fois, chaque fold servant une fois de test set.
\end{definition}

\textbf{Algorithme :}
\begin{algorithm}[H]
\caption{K-Fold Cross-Validation}
\label{alg:kfold}
\begin{algorithmic}[1]
\REQUIRE Dataset $D$, Modèle $M$, Nombre de folds $K$
\ENSURE Score moyen et variance
\STATE Diviser $D$ en $K$ folds : $D_1, D_2, \ldots, D_K$
\STATE Initialiser liste de scores : $\text{scores} = []$
\FOR{$i = 1$ \TO $K$}
    \STATE $\text{test\_set} = D_i$
    \STATE $\text{train\_set} = D \setminus D_i$ (tous les folds sauf $D_i$)
    \STATE Entraîner $M$ sur $\text{train\_set}$
    \STATE $\text{score}_i = $ évaluer $M$ sur $\text{test\_set}$
    \STATE Ajouter $\text{score}_i$ à $\text{scores}$
\ENDFOR
\RETURN $\text{mean}(\text{scores})$, $\text{std}(\text{scores})$
\end{algorithmic}
\end{algorithm}

\begin{lstlisting}[language=Python, caption=K-Fold Cross-Validation avec scikit-learn]
from sklearn.model_selection import cross_val_score
from sklearn.linear_model import LogisticRegression

model = LogisticRegression()

# 5-fold cross-validation
scores = cross_val_score(
    model, X, y,
    cv=5,                    # 5 folds
    scoring='accuracy'       # métrique
)

print(f"Scores: {scores}")
print(f"Moyenne: {scores.mean():.3f} (+/- {scores.std():.3f})")

# Exemple de sortie :
# Scores: [0.82, 0.85, 0.81, 0.84, 0.83]
# Moyenne: 0.830 (+/- 0.015)
\end{lstlisting}

\textbf{Avantages :}
\begin{itemize}
    \item Utilise toutes les données pour l'entraînement et le test
    \item Réduit la variance de l'estimation (score moyen sur $K$ runs)
    \item Fournit un intervalle de confiance (écart-type)
\end{itemize}

\textbf{Limites :}
\begin{itemize}
    \item Coût computationnel : $K$ fois plus long qu'un simple train/test
    \item Valeur typique : $K = 5$ ou $K = 10$
\end{itemize}

\subsection{Stratified K-Fold}

\begin{definition}{Stratified K-Fold}
Le \textbf{Stratified K-Fold} est une variante de K-Fold qui préserve la proportion de chaque classe dans chaque fold.
\end{definition}

\textbf{Pourquoi c'est important :}

Exemple : Dataset de 100 exemples (90 classe A, 10 classe B)

\textbf{K-Fold classique :}
\begin{itemize}
    \item Un fold pourrait contenir 0 exemple de classe B (problème !)
    \item Entraînement sur données non représentatives
\end{itemize}

\textbf{Stratified K-Fold :}
\begin{itemize}
    \item Chaque fold contient environ 90\% classe A et 10\% classe B
    \item Distribution préservée
\end{itemize}

\begin{lstlisting}[language=Python, caption=Stratified K-Fold]
from sklearn.model_selection import StratifiedKFold, cross_val_score

skf = StratifiedKFold(n_splits=5, shuffle=True, random_state=42)

scores = cross_val_score(
    model, X, y,
    cv=skf,                  # utiliser stratified K-fold
    scoring='f1_weighted'
)
\end{lstlisting}

\begin{astuce}
Utilisez \textbf{Stratified K-Fold} par défaut pour la classification, surtout si :
\begin{itemize}
    \item Les classes sont déséquilibrées
    \item Le dataset est petit
\end{itemize}

Pour la régression, utilisez K-Fold classique.
\end{astuce}

\subsection{Leave-One-Out Cross-Validation (LOOCV)}

\begin{definition}{LOOCV}
Le \textbf{LOOCV} est un cas extrême de K-Fold où $K = n$ (nombre d'exemples). Chaque exemple sert une fois de test set.
\end{definition}

\textbf{Avantages :}
\begin{itemize}
    \item Estimation quasi non biaisée
    \item Utile pour très petits datasets (< 100 exemples)
\end{itemize}

\textbf{Limites :}
\begin{itemize}
    \item Extrêmement coûteux : $n$ entraînements complets
    \item Variance élevée de l'estimation
    \item Rarement utilisé en pratique (préférer 5 ou 10-fold)
\end{itemize}

\subsection{Time Series Split}

\begin{definition}{Time Series Split}
Pour les séries temporelles, il faut respecter l'ordre chronologique. Le \textbf{Time Series Split} crée des folds où le training set contient toujours des données antérieures au test set.
\end{definition}

\textbf{Exemple avec 5 splits :}
\begin{verbatim}
Fold 1: train [1:100], test [101:120]
Fold 2: train [1:120], test [121:140]
Fold 3: train [1:140], test [141:160]
Fold 4: train [1:160], test [161:180]
Fold 5: train [1:180], test [181:200]
\end{verbatim}

\begin{lstlisting}[language=Python, caption=Time Series Split]
from sklearn.model_selection import TimeSeriesSplit

tscv = TimeSeriesSplit(n_splits=5)

for train_index, test_index in tscv.split(X):
    X_train, X_test = X[train_index], X[test_index]
    y_train, y_test = y[train_index], y[test_index]

    model.fit(X_train, y_train)
    score = model.score(X_test, y_test)
    print(f"Score: {score:.3f}")
\end{lstlisting}

\begin{attention}
Pour les séries temporelles, \textbf{JAMAIS de shuffle} ni de K-Fold classique !
Utiliser uniquement Time Series Split ou un simple train/test split chronologique.
\end{attention}

\subsection{Comparaison des Techniques de Validation}

\begin{table}[h]
\centering
\caption{Comparaison des techniques de validation}
\label{tab:validation_comparison}
\resizebox{\textwidth}{!}{
\begin{tabular}{lcccc}
\toprule
\textbf{Technique} & \textbf{Données utilisées} & \textbf{Variance} & \textbf{Coût} & \textbf{Cas d'usage} \\
\midrule
Train/Test & 80\% & Élevée & Faible & Grands datasets, première itération \\
Train/Val/Test & 60\% & Élevée & Faible & Tuning hyperparamètres \\
K-Fold (K=5) & 100\% & Modérée & Moyen & Usage général, petits/moyens datasets \\
K-Fold (K=10) & 100\% & Faible & Élevé & Datasets moyens, besoin précision \\
LOOCV & 100\% & Très faible & Très élevé & Très petits datasets (< 100) \\
Stratified K-Fold & 100\% & Modérée & Moyen & Classes déséquilibrées \\
Time Series Split & 100\% & Modérée & Moyen & Séries temporelles uniquement \\
\bottomrule
\end{tabular}
}
\end{table}

% ===== SECTION 5: PIÈGES COURANTS =====
\section{Pièges Courants et Bonnes Pratiques}

\subsection{Data Leakage (Fuite de données)}

\begin{attention}
Le \textbf{data leakage} est la principale cause de modèles qui performent bien en développement mais échouent en production.
\end{attention}

\begin{definition}{Data Leakage}
Il y a \textbf{data leakage} quand des informations du test set influencent l'entraînement du modèle. Cela conduit à des performances surestimées.
\end{definition}

\textbf{Types de leakage :}

\subsubsection{Leakage via le preprocessing}

\textbf{MAUVAIS :}
\begin{lstlisting}[language=Python, caption=Leakage via normalisation (INCORRECT)]
from sklearn.preprocessing import StandardScaler

# ERREUR: normalisation AVANT le split
scaler = StandardScaler()
X_scaled = scaler.fit_transform(X)  # utilise mean/std de TOUT le dataset

# Split
X_train, X_test, y_train, y_test = train_test_split(X_scaled, y)

# Le test set a influencé la normalisation !
\end{lstlisting}

\textbf{BON :}
\begin{lstlisting}[language=Python, caption=Normalisation correcte (CORRECT)]
# Split AVANT preprocessing
X_train, X_test, y_train, y_test = train_test_split(X, y)

# Normalisation basée UNIQUEMENT sur train set
scaler = StandardScaler()
X_train_scaled = scaler.fit_transform(X_train)  # fit sur train
X_test_scaled = scaler.transform(X_test)        # transform sur test (sans fit!)
\end{lstlisting}

\textbf{Pourquoi c'est important :}
\begin{itemize}
    \item Le scaler calculé sur tout le dataset "voit" les statistiques du test set
    \item En production, vous n'aurez pas accès aux données futures
    \item Impact : performances surestimées de 5-10\% ou plus
\end{itemize}

\subsubsection{Leakage via les features}

\begin{exemple}{Prédiction de fraude bancaire (leakage)}
\textbf{Feature problématique :} \texttt{fraud\_score} (score de fraude calculé manuellement après investigation)

\textbf{Problème :} Ce score n'existe qu'après l'enquête, donc inutilisable en production (au moment de la transaction).

\textbf{Symptôme :} Modèle avec 99\% d'accuracy en validation, mais 0\% en production.
\end{exemple}

\subsubsection{Leakage temporel}

\begin{exemple}{Prédiction de prix d'actions}
\textbf{Erreur :} Utiliser K-Fold classique sur des données temporelles.

\textbf{Problème :} Le modèle s'entraîne sur des données futures pour prédire le passé (impossible en production).

\textbf{Solution :} Utiliser Time Series Split.
\end{exemple}

\subsection{Classes Déséquilibrées}

\begin{attention}
Avec des classes très déséquilibrées (99\% / 1\%), l'accuracy devient inutile.
\end{attention}

\textbf{Stratégies :}

\begin{enumerate}
    \item \textbf{Changer la métrique :}
    \begin{itemize}
        \item Utiliser Precision, Recall, F1-score, AUC
        \item Ignorer l'accuracy
    \end{itemize}

    \item \textbf{Rééchantillonnage :}
    \begin{itemize}
        \item \textbf{Oversampling :} dupliquer la classe minoritaire (risque d'overfitting)
        \item \textbf{Undersampling :} réduire la classe majoritaire (perte de données)
        \item \textbf{SMOTE :} générer des exemples synthétiques (meilleure approche)
    \end{itemize}

    \item \textbf{Class weights :}
    \begin{lstlisting}[language=Python]
from sklearn.linear_model import LogisticRegression

# Pénaliser plus les erreurs sur la classe minoritaire
model = LogisticRegression(class_weight='balanced')
    \end{lstlisting}

    \item \textbf{Algorithmes robustes :}
    \begin{itemize}
        \item Random Forest, XGBoost (gèrent bien le déséquilibre)
    \end{itemize}
\end{enumerate}

\subsection{Overfitting et Underfitting}

\begin{definition}{Overfitting et Underfitting}
\begin{itemize}
    \item \textbf{Overfitting :} Le modèle mémorise les données d'entraînement (bruit inclus). Performance excellente en train, mauvaise en test.
    \item \textbf{Underfitting :} Le modèle est trop simple pour capturer les patterns. Performance médiocre en train ET en test.
\end{itemize}
\end{definition}

\textbf{Détection :}
\begin{table}[h]
\centering
\begin{tabular}{lccc}
\toprule
\textbf{Situation} & \textbf{Train Score} & \textbf{Test Score} & \textbf{Diagnostic} \\
\midrule
Bon modèle & 85\% & 83\% & Équilibre \\
Overfitting & 99\% & 70\% & Trop complexe \\
Underfitting & 65\% & 63\% & Trop simple \\
\bottomrule
\end{tabular}
\end{table}

\textbf{Solutions :}

\textbf{Contre l'overfitting :}
\begin{itemize}
    \item Augmenter les données d'entraînement
    \item Régularisation (L1, L2, Dropout)
    \item Réduire la complexité du modèle
    \item Early stopping
    \item Data augmentation
\end{itemize}

\textbf{Contre l'underfitting :}
\begin{itemize}
    \item Augmenter la complexité du modèle
    \item Ajouter des features
    \item Réduire la régularisation
    \item Entraîner plus longtemps
\end{itemize}

\subsection{Sélection de Métriques : Guide Pratique}

\begin{table}[h]
\centering
\caption{Quelle métrique utiliser ?}
\label{tab:metric_selection}
\resizebox{\textwidth}{!}{
\begin{tabular}{lll}
\toprule
\textbf{Contexte} & \textbf{Métrique recommandée} & \textbf{Raison} \\
\midrule
Classification équilibrée & Accuracy, F1-score & Classes également importantes \\
Classification déséquilibrée & Precision, Recall, F1, AUC & Accuracy trompeuse \\
Spam detection & Precision (+ Recall) & Éviter de perdre emails légitimes \\
Détection cancer & Recall (+ Precision) & Ne pas manquer de malades \\
Détection fraude & AUC, F1-score & Compromis FP/FN + robustesse seuil \\
Multi-classes équilibrées & Macro average & Toutes classes égales \\
Multi-classes déséquilibrées & Weighted average & Classes réalistes \\
Régression générale & RMSE, MAE, R² & Standard, facile à interpréter \\
Régression avec outliers & MAE & Moins sensible aux outliers \\
Régression sans outliers & RMSE & Pénalise les grandes erreurs \\
Comparaison multi-datasets & MAPE, R² & Indépendant de l'échelle \\
Séries temporelles & Time Series Split + RMSE/MAE & Respect de la temporalité \\
\bottomrule
\end{tabular}
}
\end{table}

% ===== SECTION 6: RÉSUMÉ =====
\section{Résumé du Chapitre}

\subsection{Points Clés}

\begin{itemize}
    \item \textbf{Métriques de classification :}
    \begin{itemize}
        \item Matrice de confusion : TP, TN, FP, FN
        \item Accuracy : proportion de prédictions correctes (attention aux classes déséquilibrées)
        \item Precision : "Quand je prédis positif, ai-je raison ?" (important si FP coûteux)
        \item Recall : "Parmi les vrais positifs, combien sont détectés ?" (important si FN coûteux)
        \item F1-score : compromis entre Precision et Recall
        \item AUC-ROC : performance indépendante du seuil (bon pour comparer des modèles)
        \item Precision-Recall curve : meilleure que ROC pour classes déséquilibrées
    \end{itemize}

    \item \textbf{Métriques de régression :}
    \begin{itemize}
        \item MSE : pénalise fortement les grandes erreurs (sensible aux outliers)
        \item RMSE : version interprétable du MSE (même unité que $y$)
        \item MAE : robuste aux outliers, pénalise toutes les erreurs également
        \item R² : proportion de variance expliquée (0 = modèle inutile, 1 = parfait)
        \item MAPE : erreur en pourcentage (indépendant de l'échelle, mais problèmes si $y=0$)
    \end{itemize}

    \item \textbf{Validation :}
    \begin{itemize}
        \item Ne JAMAIS évaluer sur les données d'entraînement
        \item Train/Test split : simple, mais variance élevée
        \item K-Fold CV : utilise toutes les données, réduit la variance
        \item Stratified K-Fold : préserve la distribution des classes (recommandé)
        \item Time Series Split : obligatoire pour les séries temporelles
    \end{itemize}

    \item \textbf{Pièges à éviter :}
    \begin{itemize}
        \item Data leakage (preprocessing APRÈS split)
        \item Choisir la mauvaise métrique (accuracy avec classes déséquilibrées)
        \item Overfitting (train score >> test score)
        \item Évaluer plusieurs fois sur le test set
    \end{itemize}
\end{itemize}

\subsection{Formules Essentielles}

\begin{tcolorbox}[colback=blue!5!white, colframe=blue!75!black, title=Formules à retenir]

\textbf{Classification :}
\begin{align}
    \text{Accuracy} &= \frac{\text{TP} + \text{TN}}{\text{TP} + \text{TN} + \text{FP} + \text{FN}} \\
    \text{Precision} &= \frac{\text{TP}}{\text{TP} + \text{FP}} \\
    \text{Recall} &= \frac{\text{TP}}{\text{TP} + \text{FN}} \\
    F_1 &= 2 \times \frac{\text{Precision} \times \text{Recall}}{\text{Precision} + \text{Recall}}
\end{align}

\textbf{Régression :}
\begin{align}
    \text{MSE} &= \frac{1}{n} \sum_{i=1}^{n} (y_i - \hat{y}_i)^2 \\
    \text{RMSE} &= \sqrt{\text{MSE}} \\
    \text{MAE} &= \frac{1}{n} \sum_{i=1}^{n} |y_i - \hat{y}_i| \\
    R^2 &= 1 - \frac{\sum (y_i - \hat{y}_i)^2}{\sum (y_i - \bar{y})^2}
\end{align}
\end{tcolorbox}

% ===== SECTION 7: EXERCICES =====
\section{Exercices}

\subsection{Questions de compréhension}

\begin{enumerate}
    \item Expliquez pourquoi l'accuracy peut être trompeuse avec des classes déséquilibrées (99\% / 1\%). Donnez un exemple concret.

    \item Un modèle de détection de cancer a Precision = 80\% et Recall = 95\%. Que signifient ces chiffres concrètement ? Quelle métrique est la plus importante dans ce contexte ?

    \item Pourquoi utilise-t-on la moyenne harmonique (F1) plutôt que la moyenne arithmétique entre Precision et Recall ?

    \item Vous entraînez un modèle et obtenez : Train accuracy = 98\%, Test accuracy = 72\%. Quel est le problème ? Quelles solutions proposez-vous ?

    \item Expliquez la différence entre RMSE et MAE. Dans quel contexte préférer l'un à l'autre ?

    \item Pourquoi ne peut-on pas utiliser K-Fold classique pour des séries temporelles ? Quelle technique utiliser ?

    \item Qu'est-ce que le data leakage ? Donnez un exemple de leakage via le preprocessing.

    \item Un dataset contient 1\% de fraudes. Vous voulez optimiser un seuil de classification. Quelle courbe utiliser : ROC ou Precision-Recall ? Pourquoi ?
\end{enumerate}

\subsection{Exercices pratiques}

\begin{enumerate}
    \item \textbf{Calcul manuel de métriques :}
    \begin{itemize}
        \item Soit une matrice de confusion : TP = 45, FP = 10, FN = 5, TN = 140
        \item Calculez : Accuracy, Precision, Recall, F1-score, Spécificité
        \item Interprétez les résultats
    \end{itemize}

    \item \textbf{Comparaison de modèles (régression) :}
    \begin{itemize}
        \item Modèle A : MSE = 25, MAE = 4, R² = 0.85
        \item Modèle B : MSE = 30, MAE = 3.5, R² = 0.82
        \item Quel modèle choisir ? Justifiez selon le contexte.
    \end{itemize}

    \item \textbf{Implémentation from scratch :}
    \begin{itemize}
        \item Implémenter Precision, Recall, F1-score en NumPy pur (sans scikit-learn)
        \item Tester sur des données synthétiques
        \item Comparer avec \texttt{sklearn.metrics}
    \end{itemize}

    \item \textbf{Validation croisée :}
    \begin{itemize}
        \item Dataset : Breast Cancer (sklearn)
        \item Comparer : simple train/test, 5-fold CV, 10-fold CV, Stratified 5-fold
        \item Analyser moyenne et écart-type des scores
    \end{itemize}

    \item \textbf{Classes déséquilibrées :}
    \begin{itemize}
        \item Créer un dataset déséquilibré (95\% / 5\%)
        \item Entraîner un modèle baseline
        \item Tester : SMOTE, class weights, undersampling
        \item Comparer les métriques (Accuracy, F1, AUC)
    \end{itemize}

    \item \textbf{Visualisations :}
    \begin{itemize}
        \item Tracer une matrice de confusion avec seaborn
        \item Tracer courbes ROC et Precision-Recall
        \item Comparer 3 modèles sur le même graphique
    \end{itemize}
\end{enumerate}


% ===== SECTION 8: POUR ALLER PLUS LOIN =====
\section{Pour Aller Plus Loin}

\subsection{Lectures Recommandées}

\begin{itemize}
    \item \textbf{Articles :}
    \begin{itemize}
        \item "The Precision-Recall Plot Is More Informative than the ROC Plot When Evaluating Binary Classifiers on Imbalanced Datasets" (Saito \& Rehmsmeier, 2015)
        \item "A Survey on Deep Learning for Imbalanced Classification" (Zhang et al., 2023)
    \end{itemize}

    \item \textbf{Livres :}
    \begin{itemize}
        \item "Hands-On Machine Learning" (Aurélien Géron) - Chapitre 3
        \item "The Elements of Statistical Learning" (Hastie et al.) - Chapitre 7
    \end{itemize}

    \item \textbf{Blogs :}
    \begin{itemize}
        \item \url{https://machinelearningmastery.com/metrics-evaluate-machine-learning-algorithms-python/}
        \item \url{https://neptune.ai/blog/f1-score-accuracy-roc-auc-pr-auc}
    \end{itemize}
\end{itemize}

\subsection{Ressources en Ligne}

\begin{itemize}
    \item Documentation scikit-learn : \url{https://scikit-learn.org/stable/modules/model_evaluation.html}
    \item Imbalanced-learn (SMOTE) : \url{https://imbalanced-learn.org/}
    \item Interactive ROC/PR curves : \url{https://roc.mlvis.com/}
\end{itemize}

\subsection{Notebooks Associés}

\begin{enumerate}
    \item \texttt{02_demo_metriques\_classification.ipynb} : Calcul manuel, visualisations ROC/PR
    \item \texttt{02_demo_metriques\_regression.ipynb} : Comparaison MSE/MAE/R²
\end{enumerate}

\subsection{Prochaines Étapes}

Chapitre suivant recommandé : \textbf{Chapitre 03 - Régression Linéaire}

Vous savez maintenant évaluer un modèle. Le chapitre suivant vous apprendra à construire votre premier modèle de régression, en appliquant toutes les métriques et techniques de validation vues ici.

% ===== BIBLIOGRAPHIE =====
\section*{Références}

\begin{enumerate}
    \item Géron, A. (2022). \textit{Hands-On Machine Learning with Scikit-Learn, Keras, and TensorFlow} (3rd ed.). O'Reilly Media.
    \item Hastie, T., Tibshirani, R., \& Friedman, J. (2009). \textit{The Elements of Statistical Learning} (2nd ed.). Springer.
    \item Saito, T., \& Rehmsmeier, M. (2015). "The Precision-Recall Plot Is More Informative than the ROC Plot When Evaluating Binary Classifiers on Imbalanced Datasets". \textit{PLOS ONE}, 10(3).
    \item Pedregosa, F. et al. (2011). "Scikit-learn: Machine Learning in Python". \textit{Journal of Machine Learning Research}, 12, 2825-2830.
    \item Provost, F., \& Fawcett, T. (2013). \textit{Data Science for Business}. O'Reilly Media.
\end{enumerate}

\end{document}
