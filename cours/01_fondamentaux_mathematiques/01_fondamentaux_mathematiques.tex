% Chapitre 01 - Fondamentaux Mathématiques pour le Machine Learning

\documentclass[11pt,a4paper]{article}

% ===== PACKAGES =====
\usepackage[utf8]{inputenc}
\usepackage[T1]{fontenc}
\usepackage[french]{babel}
\usepackage{lmodern}

% Mathématiques
\usepackage{amsmath, amssymb, amsthm}
\usepackage{mathtools}

% Mise en page
\usepackage[margin=2.5cm]{geometry}
\usepackage{parskip}
\usepackage{setspace}
\setstretch{1.15}

% Graphiques et couleurs
\usepackage{graphicx}
\usepackage{xcolor}

% ===== UNICODE CHARACTERS SUPPORT =====
\usepackage{newunicodechar}

% Emojis et symboles
\newunicodechar{✅}{\textcolor{green!60!black}{$\checkmark$}}
\newunicodechar{❌}{\textcolor{red!60!black}{$\times$}}
\newunicodechar{✓}{\textcolor{green!60!black}{$\checkmark$}}
\newunicodechar{✗}{\textcolor{red!60!black}{$\times$}}
\newunicodechar{⚠}{\textcolor{orange!80!black}{\textbf{/!\textbackslash}}}
\newunicodechar{💡}{\textcolor{blue!70!black}{\textbf{(i)}}}
\newunicodechar{🎯}{\textcolor{purple!70!black}{\textbf{$\star$}}}
\newunicodechar{📊}{\textcolor{blue!70!black}{\textbf{[=]}}}

% Étoiles (pour tableaux)
\newunicodechar{★}{\textcolor{orange!80!black}{$\star$}}
\newunicodechar{☆}{\textcolor{gray!50}{$\star$}}

% Flèches
\newunicodechar{→}{$\rightarrow$}
\newunicodechar{←}{$\leftarrow$}
\newunicodechar{↑}{$\uparrow$}
\newunicodechar{↓}{$\downarrow$}

\usepackage{tikz}
\usetikzlibrary{arrows.meta, positioning, shapes.geometric, matrix}

% Tableaux
\usepackage{booktabs}
\usepackage{longtable}
\usepackage{multirow}
\usepackage{tabularx}
\usepackage{colortbl}

% Code et algorithmes
\usepackage{listings}
\usepackage{algorithm}
\usepackage{algorithmic}

% Hyperliens
\usepackage{hyperref}
\hypersetup{
    colorlinks=true,
    linkcolor=blue,
    filecolor=magenta,
    urlcolor=cyan,
    citecolor=green,
    pdftitle={Chapitre 01 - Fondamentaux Mathématiques},
    pdfauthor={Cours ML},
}

% Boxes colorées
\usepackage{tcolorbox}
\tcbuselibrary{skins, breakable}


% ===== TCOLORBOX AVEC EMOJIS =====
\newtcolorbox{attention}{
    colback=red!5!white,
    colframe=red!75!black,
    fonttitle=\bfseries,
    title=⚠ Attention,
    breakable
}

\newtcolorbox{definition}{
    colback=blue!5!white,
    colframe=blue!75!black,
    fonttitle=\bfseries,
    title=Définition,
    breakable
}

\newtcolorbox{astuce}{
    colback=green!5!white,
    colframe=green!60!black,
    fonttitle=\bfseries,
    title=💡 Astuce,
    breakable
}

\newtcolorbox{remarque}{
    colback=yellow!5!white,
    colframe=orange!75!black,
    fonttitle=\bfseries,
    title=💡 Remarque,
    breakable
}

\newtcolorbox{important}{
    colback=purple!5!white,
    colframe=purple!75!black,
    fonttitle=\bfseries,
    title=⚠ Important,
    breakable
}

\newtcolorbox{exemple}{
    colback=gray!5!white,
    colframe=gray!75!black,
    fonttitle=\bfseries,
    title=Exemple,
    breakable
}

% En-têtes et pieds de page
\usepackage{fancyhdr}
\pagestyle{fancy}
\fancyhf{}
\fancyhead[L]{\small Chapitre 01 - Fondamentaux Mathématiques}
\fancyhead[R]{\small Cours Machine Learning}
\fancyfoot[C]{\thepage}

% ===== CONFIGURATION LISTINGS (code Python) =====
\definecolor{codegreen}{rgb}{0,0.6,0}
\definecolor{codegray}{rgb}{0.5,0.5,0.5}
\definecolor{codepurple}{rgb}{0.58,0,0.82}
\definecolor{backcolour}{rgb}{0.95,0.95,0.92}

\lstdefinestyle{pythonstyle}{
    language=Python,
    backgroundcolor=\color{backcolour},
    commentstyle=\color{codegreen},
    keywordstyle=\color{blue},
    numberstyle=\tiny\color{codegray},
    stringstyle=\color{codepurple},
    basicstyle=\ttfamily\small,
    breakatwhitespace=false,
    breaklines=true,
    captionpos=b,
    keepspaces=true,
    numbers=left,
    numbersep=5pt,
    showspaces=false,
    showstringspaces=false,
    showtabs=false,
    tabsize=4,
    frame=single,
    rulecolor=\color{black},
    inputencoding=utf8,
    extendedchars=true,
    literate=
        {é}{{\'e}}{1}
        {è}{{\`e}}{1}
        {ê}{{\^e}}{1}
        {à}{{\`a}}{1}
        {â}{{\^a}}{1}
        {ù}{{\`u}}{1}
        {û}{{\^u}}{1}
        {ô}{{\^o}}{1}
        {î}{{\^i}}{1}
        {ï}{{\"i}}{1}
        {ë}{{\"e}}{1}
        {ç}{{\c{c}}}{1}
}
\lstset{style=pythonstyle}

% ===== CONFIGURATION TCOLORBOX =====
% Box pour définitions


% Box pour théorèmes
\newtcolorbox{theoreme}[1]{
    colback=green!5!white,
    colframe=green!75!black,
    fonttitle=\bfseries,
    title=Théorème: #1,
    breakable
}

% Box pour exemples


% Box pour attention/warning


% Box pour astuce/tips


% Box pour intuition
\newtcolorbox{intuition}{
    colback=purple!5!white,
    colframe=purple!75!black,
    fonttitle=\bfseries,
    title=Intuition,
    breakable
}

% Box pour "Pourquoi c'est important"
\newtcolorbox{pourquoi}{
    colback=cyan!5!white,
    colframe=cyan!75!black,
    fonttitle=\bfseries,
    title=Pourquoi c'est important en ML,
    breakable
}

% Box pour mini-projet
\newtcolorbox{miniprojet}[1]{
    colback=green!5!white,
    colframe=green!75!black,
    fonttitle=\bfseries,
    title=Mini-Projet: #1,
    breakable
}

% ===== COMMANDES PERSONNALISÉES =====
\newcommand{\vect}[1]{\mathbf{#1}}  % Vecteur
\newcommand{\mat}[1]{\mathbf{#1}}   % Matrice
\newcommand{\R}{\mathbb{R}}         % Réels
\newcommand{\N}{\mathbb{N}}         % Naturels
\newcommand{\Z}{\mathbb{Z}}         % Entiers
\newcommand{\C}{\mathbb{C}}         % Complexes
\newcommand{\E}{\mathbb{E}}         % Espérance
\newcommand{\Var}{\operatorname{Var}}
\newcommand{\Cov}{\operatorname{Cov}}
\newcommand{\argmin}{\operatorname{argmin}}
\newcommand{\argmax}{\operatorname{argmax}}
\newcommand{\tr}{\operatorname{tr}}  % Trace
\newcommand{\rank}{\operatorname{rank}}  % Rang

% ===== DÉBUT DU DOCUMENT =====
\begin{document}

% ===== PAGE DE TITRE =====
\begin{titlepage}
    \centering
    \vspace*{2cm}

    {\Huge\bfseries Cours Machine Learning}\\[0.5cm]

    \vspace{1cm}

    {\LARGE Chapitre 01}\\[0.3cm]
    {\LARGE\bfseries Fondamentaux Mathématiques}\\[2cm]

    \vfill

    {\large
    \textbf{Objectifs d'apprentissage :}\\[0.5cm]
    \begin{itemize}
        \item Maîtriser les concepts d'algèbre linéaire essentiels (vecteurs, matrices, décompositions)
        \item Comprendre les probabilités et statistiques nécessaires au ML
        \item Appréhender le calcul différentiel et les techniques d'optimisation
        \item Savoir appliquer ces outils mathématiques aux algorithmes de ML
    \end{itemize}
    }

    \vfill

    {\large
    \textbf{Prérequis :} Mathématiques niveau Licence (algèbre, analyse)\\[0.3cm]
    \textbf{Durée estimée :} 5-7 heures\\[0.3cm]
    \textbf{Notebooks :} \texttt{01\_demo\_*.ipynb}
    }

    \vfill

    {\large Cours ML - Sandbox-ML\\
    Version 1.0 - 2026}
\end{titlepage}

% ===== TABLE DES MATIÈRES =====
\tableofcontents
\newpage

% ===== PARTIE I: ALGÈBRE LINÉAIRE =====
\part{Algèbre Linéaire}

\begin{pourquoi}
L'algèbre linéaire est \textbf{partout} en Machine Learning :
\begin{itemize}
    \item \textbf{Vos données} : Chaque image 28×28 pixels = un vecteur de 784 dimensions
    \item \textbf{Les modèles} : Un réseau de neurones = une série de multiplications matricielles
    \item \textbf{La réduction de dimension} : PCA pour passer de 10,000 features à 50 = décomposition SVD
    \item \textbf{L'entraînement} : Trouver les meilleurs poids $\vect{w}$ = résoudre un système linéaire
\end{itemize}

Sans algèbre linéaire, impossible de faire du ML moderne !
\end{pourquoi}

\section{Vecteurs et Espaces Vectoriels}

\subsection{Définitions Fondamentales}

\begin{definition}{Vecteur}
Un vecteur $\vect{v} \in \R^n$ est un tuple ordonné de $n$ nombres réels :
\begin{equation}
    \vect{v} = \begin{pmatrix} v_1 \\ v_2 \\ \vdots \\ v_n \end{pmatrix}
\end{equation}
On le représente comme un vecteur colonne par défaut. Un vecteur ligne s'écrit $\vect{v}^T$.
\end{definition}

\textbf{Interprétations :}
\begin{itemize}
    \item \textbf{Géométrique :} Point ou flèche dans l'espace $\R^n$
    \item \textbf{ML :} Une instance de données (sample), un vecteur de features
    \item \textbf{Algébrique :} Élément d'un espace vectoriel
\end{itemize}

\begin{exemple}{Vecteur de features}
Un appartement décrit par 3 features :
\begin{equation}
    \vect{x} = \begin{pmatrix} 75 \\ 3 \\ 2010 \end{pmatrix} \quad
    \begin{matrix}
        \text{(surface en m}^2\text{)} \\
        \text{(nombre de pièces)} \\
        \text{(année de construction)}
    \end{matrix}
\end{equation}
\end{exemple}

\subsection{Opérations sur les Vecteurs}

\textbf{Addition vectorielle :}
\begin{equation}
    \vect{u} + \vect{v} = \begin{pmatrix} u_1 + v_1 \\ u_2 + v_2 \\ \vdots \\ u_n + v_n \end{pmatrix}
\end{equation}

\textbf{Multiplication par un scalaire :}
\begin{equation}
    \alpha \vect{v} = \begin{pmatrix} \alpha v_1 \\ \alpha v_2 \\ \vdots \\ \alpha v_n \end{pmatrix}
\end{equation}

\subsection{Produit Scalaire (Dot Product)}

\begin{intuition}
Imaginez deux films décrits par leurs genres :
\begin{itemize}
    \item Film A : $\vect{u} = \begin{pmatrix} 0.9 \\ 0.1 \\ 0.2 \end{pmatrix}$ (Action, Romance, Comédie)
    \item Film B : $\vect{v} = \begin{pmatrix} 0.8 \\ 0.3 \\ 0.1 \end{pmatrix}$
\end{itemize}

Le produit scalaire $\vect{u} \cdot \vect{v} = 0.9 \times 0.8 + 0.1 \times 0.3 + 0.2 \times 0.1 = 0.77$ mesure à quel point ces films sont similaires ! Plus le résultat est grand, plus ils se ressemblent.

\textbf{Application concrète :} Systèmes de recommandation Netflix, Spotify...
\end{intuition}

\begin{definition}{Produit scalaire}
Le produit scalaire de deux vecteurs $\vect{u}, \vect{v} \in \R^n$ est :
\begin{equation}
    \vect{u} \cdot \vect{v} = \vect{u}^T \vect{v} = \sum_{i=1}^n u_i v_i = u_1 v_1 + u_2 v_2 + \cdots + u_n v_n
\end{equation}
\end{definition}

\textbf{Propriétés :}
\begin{itemize}
    \item Commutativité : $\vect{u} \cdot \vect{v} = \vect{v} \cdot \vect{u}$
    \item Linéarité : $(\alpha\vect{u} + \beta\vect{w}) \cdot \vect{v} = \alpha(\vect{u} \cdot \vect{v}) + \beta(\vect{w} \cdot \vect{v})$
    \item $\vect{v} \cdot \vect{v} \geq 0$, égalité ssi $\vect{v} = \vect{0}$
\end{itemize}

\textbf{Interprétation géométrique :}
\begin{equation}
    \vect{u} \cdot \vect{v} = \|\vect{u}\| \|\vect{v}\| \cos(\theta)
\end{equation}
où $\theta$ est l'angle entre $\vect{u}$ et $\vect{v}$.

\begin{figure}[h]
\centering
\begin{tikzpicture}[scale=1.2]
    % Axes
    \draw[->] (-0.5,0) -- (5,0) node[right] {$x$};
    \draw[->] (0,-0.5) -- (0,4) node[above] {$y$};

    % Vecteur u
    \draw[->, very thick, blue] (0,0) -- (4,1);
    \node[blue, font=\small, below right] at (2,0.5) {$\vect{u}$};
    \node[blue, font=\small, right] at (4.2,1) {$(4, 1)$};

    % Vecteur v
    \draw[->, very thick, red] (0,0) -- (2,3);
    \node[red, font=\small, left] at (1,1.5) {$\vect{v}$};
    \node[red, font=\small, above] at (2,3.2) {$(2, 3)$};

    % Angle theta
    \draw[green!60!black, thick] (0.8,0) arc (0:56:0.8);
    \node[green!60!black, font=\small] at (1.3,0.4) {$\theta$};

    % Projection
    \draw[dashed, purple] (2,3) -- (2.8,0.7);
    \filldraw[purple] (2.8,0.7) circle (1.5pt);

    % Annotations
    \node[purple, font=\small, right, align=left] at (3.0,1.85) {Projection\\[-2pt]de $\vect{v}$ sur $\vect{u}$};

    % Calculs
    \node[align=left, draw, fill=yellow!10, rounded corners] at (7,2) {
        $\vect{u} \cdot \vect{v} = 4 \times 2 + 1 \times 3 = 11$\\[5pt]
        $\|\vect{u}\| = \sqrt{16+1} = \sqrt{17}$\\
        $\|\vect{v}\| = \sqrt{4+9} = \sqrt{13}$\\[5pt]
        $\cos(\theta) = \frac{11}{\sqrt{17}\sqrt{13}} \approx 0.74$\\[3pt]
        $\theta \approx 42°$
    };

\end{tikzpicture}
\caption{Interprétation géométrique du produit scalaire: $\vect{u} \cdot \vect{v}$ mesure à quel point les vecteurs pointent dans la même direction. L'angle $\theta$ entre eux détermine le cosinus, donc la similarité.}
\label{fig:dot_product}
\end{figure}

\begin{astuce}
\textbf{En Machine Learning :}
\begin{itemize}
    \item Le produit scalaire mesure la \textbf{similarité} entre vecteurs
    \item Deux vecteurs orthogonaux ($\vect{u} \cdot \vect{v} = 0$) sont \textbf{non corrélés}
    \item Utilisé partout : régression linéaire ($\vect{w}^T\vect{x}$), réseaux de neurones, kernels...
\end{itemize}
\end{astuce}

\subsection{Norme Vectorielle}

\begin{definition}{Norme $L^p$}
La norme $L^p$ d'un vecteur $\vect{v} \in \R^n$ est :
\begin{equation}
    \|\vect{v}\|_p = \left(\sum_{i=1}^n |v_i|^p\right)^{1/p}
\end{equation}
\end{definition}

\textbf{Normes principales :}

\begin{itemize}
    \item \textbf{Norme $L^1$ (Manhattan)} :
    \begin{equation}
        \|\vect{v}\|_1 = \sum_{i=1}^n |v_i|
    \end{equation}
    Utilisée pour la régularisation Lasso, robuste aux outliers.

    \item \textbf{Norme $L^2$ (Euclidienne)} :
    \begin{equation}
        \|\vect{v}\|_2 = \sqrt{\sum_{i=1}^n v_i^2} = \sqrt{\vect{v}^T \vect{v}}
    \end{equation}
    La plus courante en ML. Utilisée pour mesurer des distances.

    \item \textbf{Norme $L^\infty$ (Maximum)} :
    \begin{equation}
        \|\vect{v}\|_\infty = \max_i |v_i|
    \end{equation}
\end{itemize}

\begin{exemple}{Calcul de normes}
Pour $\vect{v} = \begin{pmatrix} 3 \\ -4 \end{pmatrix}$ :
\begin{align}
    \|\vect{v}\|_1 &= |3| + |-4| = 7 \\
    \|\vect{v}\|_2 &= \sqrt{3^2 + (-4)^2} = \sqrt{25} = 5 \\
    \|\vect{v}\|_\infty &= \max(|3|, |-4|) = 4
\end{align}
\end{exemple}

\subsection{Distance entre Vecteurs}

\begin{definition}{Distance euclidienne}
La distance entre deux points $\vect{u}, \vect{v} \in \R^n$ est :
\begin{equation}
    d(\vect{u}, \vect{v}) = \|\vect{u} - \vect{v}\|_2 = \sqrt{\sum_{i=1}^n (u_i - v_i)^2}
\end{equation}
\end{definition}

\textbf{Applications en ML :}
\begin{itemize}
    \item K-Nearest Neighbors (KNN)
    \item Clustering (K-Means)
    \item Embeddings (mesurer similarité sémantique)
\end{itemize}

% ===== SECTION 2: MATRICES =====
\section{Matrices}

\subsection{Définitions et Notation}

\begin{definition}{Matrice}
Une matrice $\mat{A} \in \R^{m \times n}$ est un tableau rectangulaire de $m$ lignes et $n$ colonnes :
\begin{equation}
    \mat{A} = \begin{pmatrix}
        a_{11} & a_{12} & \cdots & a_{1n} \\
        a_{21} & a_{22} & \cdots & a_{2n} \\
        \vdots & \vdots & \ddots & \vdots \\
        a_{m1} & a_{m2} & \cdots & a_{mn}
    \end{pmatrix}
\end{equation}
\end{definition}

\textbf{Notation :}
\begin{itemize}
    \item $a_{ij}$ : élément ligne $i$, colonne $j$
    \item $\mat{A}_{i:}$ : $i$-ème ligne (vecteur ligne)
    \item $\mat{A}_{:j}$ : $j$-ème colonne (vecteur colonne)
\end{itemize}

\textbf{Interprétation ML :}
\begin{itemize}
    \item \textbf{Dataset} : $\mat{X} \in \R^{n \times d}$ où $n$ = nombre d'instances, $d$ = nombre de features
    \item Chaque ligne $\mat{X}_{i:} = \vect{x}_i^T$ est une instance
    \item Chaque colonne $\mat{X}_{:j}$ est une feature
\end{itemize}

\subsection{Opérations Matricielles}

\textbf{Addition :} $(\mat{A} + \mat{B})_{ij} = a_{ij} + b_{ij}$ (même dimension)

\textbf{Multiplication par scalaire :} $(\alpha \mat{A})_{ij} = \alpha a_{ij}$

\textbf{Transposée :}
\begin{equation}
    (\mat{A}^T)_{ij} = a_{ji}
\end{equation}

\textbf{Propriétés de la transposée :}
\begin{itemize}
    \item $(\mat{A}^T)^T = \mat{A}$
    \item $(\mat{A} + \mat{B})^T = \mat{A}^T + \mat{B}^T$
    \item $(\mat{A}\mat{B})^T = \mat{B}^T\mat{A}^T$
\end{itemize}

\subsection{Multiplication Matricielle}

\begin{definition}{Produit matriciel}
Si $\mat{A} \in \R^{m \times n}$ et $\mat{B} \in \R^{n \times p}$, alors $\mat{C} = \mat{A}\mat{B} \in \R^{m \times p}$ avec :
\begin{equation}
    c_{ij} = \sum_{k=1}^n a_{ik} b_{kj} = \mat{A}_{i:} \cdot \mat{B}_{:j}
\end{equation}
\end{definition}

\begin{figure}[h]
\centering
\begin{tikzpicture}[scale=0.8]
    % Matrice A (2x3)
    \matrix[matrix of math nodes, left delimiter=(, right delimiter=),
            nodes={minimum width=1cm, minimum height=0.8cm}] (A) at (0,0) {
        2 & 1 & 3 \\
        4 & 0 & 1 \\
    };
    \node[above] at (A.north) {$\mat{A}_{2 \times 3}$};

    % Symbole multiplication
    \node at (3,0) {$\times$};

    % Matrice B (3x2)
    \matrix[matrix of math nodes, left delimiter=(, right delimiter=),
            nodes={minimum width=1cm, minimum height=0.8cm}] (B) at (5,0) {
        1 & 2 \\
        0 & 3 \\
        2 & 1 \\
    };
    \node[above] at (B.north) {$\mat{B}_{3 \times 2}$};

    % Symbole égal
    \node at (8,0) {$=$};

    % Matrice C (2x2)
    \matrix[matrix of math nodes, left delimiter=(, right delimiter=),
            nodes={minimum width=1.2cm, minimum height=0.8cm}] (C) at (10.5,0) {
        |[fill=green!20]| 8 & 10 \\
        6 & 9 \\
    };
    \node[above] at (C.north) {$\mat{C}_{2 \times 2}$};

    % Mise en évidence pour c_11
    \draw[blue, very thick] (A-1-1.north west) rectangle (A-1-3.south east);
    \draw[red, very thick] (B-1-2.north west) rectangle (B-3-2.south east);

    % Flèches et calcul
    \draw[->, blue, thick] (A-1-2.south) -- ++(0,-0.8) -- ++(8,0) -- (C-1-1.west);
    \draw[->, red, thick] (B-2-2.east) -- ++(0.8,0) -- ++(0,-0.5) -- (C-1-1.north);

    % Explication du calcul
    \node[align=left, draw, fill=yellow!10, rounded corners, text width=6cm] at (5,-3.5) {
        \textbf{Calcul de } $c_{11}$ (élément vert):\\[5pt]
        Ligne 1 de $\mat{A}$ $\times$ Colonne 1 de $\mat{B}$:\\[3pt]
        $c_{11} = 2 \times 1 + 1 \times 0 + 3 \times 2$\\
        $c_{11} = 2 + 0 + 6 = 8$
    };

\end{tikzpicture}
\caption{Multiplication matricielle: chaque élément $c_{ij}$ est le produit scalaire de la ligne $i$ de $\mat{A}$ et de la colonne $j$ de $\mat{B}$. Les dimensions doivent être compatibles: $(m \times n) \times (n \times p) = (m \times p)$.}
\label{fig:matrix_multiplication}
\end{figure}

\begin{attention}
\textbf{Attention :}
\begin{itemize}
    \item La multiplication matricielle n'est \textbf{pas commutative} : $\mat{A}\mat{B} \neq \mat{B}\mat{A}$ en général
    \item Les dimensions doivent être compatibles : $(m \times n) \times (n \times p) \to (m \times p)$
\end{itemize}
\end{attention}

\textbf{Complexité :} Multiplication naïve : $O(mnp)$ opérations

\begin{exemple}{Prédiction en régression linéaire}
Pour $n$ instances avec $d$ features, la prédiction vectorielle s'écrit :
\begin{equation}
    \vect{\hat{y}} = \mat{X} \vect{w} + b\mathbf{1}
\end{equation}
où $\mat{X} \in \R^{n \times d}$, $\vect{w} \in \R^d$, $\vect{\hat{y}} \in \R^n$. Une seule opération matricielle remplace $n$ produits scalaires !
\end{exemple}

\subsection{Matrices Spéciales}

\textbf{Matrice identité :}
\begin{equation}
    \mat{I}_n = \begin{pmatrix}
        1 & 0 & \cdots & 0 \\
        0 & 1 & \cdots & 0 \\
        \vdots & \vdots & \ddots & \vdots \\
        0 & 0 & \cdots & 1
    \end{pmatrix}, \quad \mat{A}\mat{I} = \mat{I}\mat{A} = \mat{A}
\end{equation}

\textbf{Matrice diagonale :}
\begin{equation}
    \mat{D} = \begin{pmatrix}
        d_1 & 0 & \cdots & 0 \\
        0 & d_2 & \cdots & 0 \\
        \vdots & \vdots & \ddots & \vdots \\
        0 & 0 & \cdots & d_n
    \end{pmatrix}
\end{equation}

\textbf{Matrice symétrique :} $\mat{A} = \mat{A}^T$ (ex: matrices de covariance)

\textbf{Matrice orthogonale :} $\mat{Q}^T\mat{Q} = \mat{Q}\mat{Q}^T = \mat{I}$

\subsection{Déterminant}

\begin{definition}{Déterminant}
Pour une matrice carrée $\mat{A} \in \R^{n \times n}$, le déterminant $\det(\mat{A})$ mesure le « volume orienté » de la transformation linéaire associée.
\end{definition}

\textbf{Cas $2 \times 2$ :}
\begin{equation}
    \det\begin{pmatrix} a & b \\ c & d \end{pmatrix} = ad - bc
\end{equation}

\textbf{Propriétés :}
\begin{itemize}
    \item $\det(\mat{A}\mat{B}) = \det(\mat{A}) \det(\mat{B})$
    \item $\det(\mat{A}^T) = \det(\mat{A})$
    \item $\det(\mat{A}^{-1}) = 1/\det(\mat{A})$
    \item Si $\det(\mat{A}) = 0$, alors $\mat{A}$ n'est pas inversible (singulière)
\end{itemize}

\subsection{Inverse de Matrice}

\begin{definition}{Matrice inverse}
Pour une matrice carrée $\mat{A} \in \R^{n \times n}$, l'inverse $\mat{A}^{-1}$ (si elle existe) vérifie :
\begin{equation}
    \mat{A}\mat{A}^{-1} = \mat{A}^{-1}\mat{A} = \mat{I}
\end{equation}
\end{definition}

\textbf{Conditions d'existence :}
\begin{itemize}
    \item $\mat{A}$ doit être carrée
    \item $\det(\mat{A}) \neq 0$ (matrice non-singulière)
\end{itemize}

\textbf{Cas $2 \times 2$ :}
\begin{equation}
    \begin{pmatrix} a & b \\ c & d \end{pmatrix}^{-1} = \frac{1}{ad-bc} \begin{pmatrix} d & -b \\ -c & a \end{pmatrix}
\end{equation}

\begin{attention}
En pratique, on \textbf{n'inverse jamais explicitement} une matrice ! On résout plutôt le système linéaire $\mat{A}\vect{x} = \vect{b}$ via des méthodes numériques (décomposition LU, Cholesky, etc.).
\end{attention}

\subsection{Trace}

\begin{definition}{Trace}
La trace d'une matrice carrée $\mat{A} \in \R^{n \times n}$ est la somme de ses éléments diagonaux :
\begin{equation}
    \tr(\mat{A}) = \sum_{i=1}^n a_{ii}
\end{equation}
\end{definition}

\textbf{Propriétés :}
\begin{itemize}
    \item $\tr(\mat{A} + \mat{B}) = \tr(\mat{A}) + \tr(\mat{B})$
    \item $\tr(\mat{A}\mat{B}) = \tr(\mat{B}\mat{A})$ (propriété cyclique)
    \item $\tr(\mat{A}) = \sum_i \lambda_i$ où $\lambda_i$ sont les valeurs propres
\end{itemize}

\subsection{Rang}

\begin{definition}{Rang}
Le rang d'une matrice $\mat{A} \in \R^{m \times n}$ est le nombre maximal de colonnes (ou lignes) linéairement indépendantes.
\end{definition}

\textbf{Propriétés :}
\begin{itemize}
    \item $\rank(\mat{A}) \leq \min(m, n)$
    \item $\rank(\mat{A}) = \rank(\mat{A}^T)$
    \item Si $\rank(\mat{A}) = n$ (rang plein en colonnes), alors $\mat{A}^T\mat{A}$ est inversible
\end{itemize}

\textbf{En ML :}
\begin{itemize}
    \item Rang faible $\Rightarrow$ redondance dans les features
    \item PCA, SVD exploitent les structures de rang faible
\end{itemize}

% ===== SECTION 3: SYSTÈMES LINÉAIRES =====
\section{Systèmes Linéaires et Résolution}

\subsection{Formulation}

Un système linéaire de $m$ équations à $n$ inconnues s'écrit :
\begin{equation}
    \mat{A}\vect{x} = \vect{b}
\end{equation}
où $\mat{A} \in \R^{m \times n}$, $\vect{x} \in \R^n$, $\vect{b} \in \R^m$.

\textbf{Cas en ML :}
\begin{itemize}
    \item \textbf{Régression linéaire} : Trouver $\vect{w}$ tel que $\mat{X}\vect{w} \approx \vect{y}$
    \item \textbf{Optimisation} : Résoudre $\nabla f(\vect{x}) = \vect{0}$ (gradient nul)
\end{itemize}

\subsection{Solutions}

\textbf{Cas carré ($m = n$) :}
\begin{itemize}
    \item Si $\det(\mat{A}) \neq 0$ : solution unique $\vect{x} = \mat{A}^{-1}\vect{b}$
    \item Si $\det(\mat{A}) = 0$ : aucune solution ou infinité de solutions
\end{itemize}

\textbf{Cas rectangulaire ($m \neq n$) :}
\begin{itemize}
    \item $m > n$ (surdéterminé) : en général pas de solution exacte $\Rightarrow$ solution des moindres carrés
    \item $m < n$ (sous-déterminé) : infinité de solutions $\Rightarrow$ choisir celle de norme minimale
\end{itemize}

\subsection{Solution des Moindres Carrés}

\begin{theoreme}{Solution des moindres carrés}
Pour le système surdéterminé $\mat{A}\vect{x} \approx \vect{b}$, la solution qui minimise $\|\mat{A}\vect{x} - \vect{b}\|_2^2$ est :
\begin{equation}
    \vect{x}^* = (\mat{A}^T\mat{A})^{-1}\mat{A}^T\vect{b}
\end{equation}
sous réserve que $\mat{A}^T\mat{A}$ soit inversible.
\end{theoreme}

\textbf{Application directe :} Régression linéaire ! Si $\mat{X} \in \R^{n \times d}$ et $\vect{y} \in \R^n$, alors :
\begin{equation}
    \vect{w}^* = (\mat{X}^T\mat{X})^{-1}\mat{X}^T\vect{y}
\end{equation}

\begin{astuce}
La matrice $\mat{A}^\dagger = (\mat{A}^T\mat{A})^{-1}\mat{A}^T$ s'appelle la \textbf{pseudo-inverse de Moore-Penrose}. En Python : \texttt{numpy.linalg.pinv(A)}.
\end{astuce}

% ===== SECTION 4: VALEURS PROPRES =====
\section{Valeurs et Vecteurs Propres}

\subsection{Définitions}

\begin{definition}{Valeur propre et vecteur propre}
Pour une matrice carrée $\mat{A} \in \R^{n \times n}$, $\lambda \in \C$ est une \textbf{valeur propre} et $\vect{v} \neq \vect{0}$ est un \textbf{vecteur propre} associé si :
\begin{equation}
    \mat{A}\vect{v} = \lambda \vect{v}
\end{equation}
\end{definition}

\textbf{Interprétation :} La transformation $\mat{A}$ étire le vecteur $\vect{v}$ par un facteur $\lambda$, sans changer sa direction.

\subsection{Calcul des Valeurs Propres}

Les valeurs propres sont les racines du \textbf{polynôme caractéristique} :
\begin{equation}
    \det(\mat{A} - \lambda \mat{I}) = 0
\end{equation}

\begin{exemple}{Matrice $2 \times 2$}
Pour $\mat{A} = \begin{pmatrix} 4 & 2 \\ 1 & 3 \end{pmatrix}$ :
\begin{align}
    \det(\mat{A} - \lambda\mat{I}) &= \det\begin{pmatrix} 4-\lambda & 2 \\ 1 & 3-\lambda \end{pmatrix} \\
    &= (4-\lambda)(3-\lambda) - 2 \times 1 \\
    &= \lambda^2 - 7\lambda + 10 = 0
\end{align}
Solutions : $\lambda_1 = 5$, $\lambda_2 = 2$.
\end{exemple}

\subsection{Diagonalisation}

\begin{theoreme}{Diagonalisation}
Si $\mat{A} \in \R^{n \times n}$ possède $n$ vecteurs propres linéairement indépendants $\vect{v}_1, \ldots, \vect{v}_n$ avec valeurs propres $\lambda_1, \ldots, \lambda_n$, alors :
\begin{equation}
    \mat{A} = \mat{P}\mat{D}\mat{P}^{-1}
\end{equation}
où $\mat{P} = [\vect{v}_1 \, \vect{v}_2 \, \cdots \, \vect{v}_n]$ et $\mat{D} = \text{diag}(\lambda_1, \ldots, \lambda_n)$.
\end{theoreme}

\textbf{Avantages de la diagonalisation :}
\begin{itemize}
    \item Calcul de puissances : $\mat{A}^k = \mat{P}\mat{D}^k\mat{P}^{-1}$
    \item Exponentielle : $e^{\mat{A}} = \mat{P}e^{\mat{D}}\mat{P}^{-1}$
    \item Comprendre la dynamique des systèmes linéaires
\end{itemize}

\subsection{Matrices Symétriques Réelles}

\begin{theoreme}{Théorème spectral}
Toute matrice symétrique réelle $\mat{A} \in \R^{n \times n}$ est diagonalisable avec :
\begin{itemize}
    \item Toutes les valeurs propres sont \textbf{réelles}
    \item Les vecteurs propres peuvent être choisis \textbf{orthonormaux}
    \item Décomposition : $\mat{A} = \mat{Q}\mat{\Lambda}\mat{Q}^T$ où $\mat{Q}$ est orthogonale
\end{itemize}
\end{theoreme}

\textbf{Applications en ML :}
\begin{itemize}
    \item PCA (Principal Component Analysis)
    \item Matrices de covariance
    \item Kernel PCA
\end{itemize}

% ===== SECTION 5: DÉCOMPOSITIONS =====
\section{Décompositions Matricielles}

\subsection{Décomposition en Valeurs Singulières (SVD)}

\begin{intuition}
Imaginez une image comme une recette de cuisine. La SVD dit : "cette image de 1000×1000 pixels est en fait un mélange de" :
\begin{itemize}
    \item 40\% de "pattern horizontal" ($\sigma_1 = 400$)
    \item 30\% de "pattern vertical" ($\sigma_2 = 300$)
    \item 20\% de "pattern diagonal" ($\sigma_3 = 200$)
    \item ...
    \item 0.1\% de bruit ($\sigma_{999}, \sigma_{1000}$ très petits)
\end{itemize}

\textbf{L'astuce :} On peut jeter le bruit (les petites valeurs) et garder seulement les 50 premiers "patterns" essentiels. Résultat : on passe de 1,000,000 de nombres à stocker à seulement 50,000 → \textbf{compression de 95\% !}

\textbf{C'est exactement ce que fait} : Netflix (recommandations), JPEG (compression d'images), PCA (réduction de dimension)
\end{intuition}

\begin{theoreme}{SVD (Singular Value Decomposition)}
Toute matrice $\mat{A} \in \R^{m \times n}$ peut être décomposée en :
\begin{equation}
    \mat{A} = \mat{U}\mat{\Sigma}\mat{V}^T
\end{equation}
où :
\begin{itemize}
    \item $\mat{U} \in \R^{m \times m}$ : matrice orthogonale (vecteurs singuliers gauches)
    \item $\mat{\Sigma} \in \R^{m \times n}$ : matrice diagonale avec valeurs singulières $\sigma_1 \geq \sigma_2 \geq \cdots \geq 0$
    \item $\mat{V} \in \R^{n \times n}$ : matrice orthogonale (vecteurs singuliers droits)
\end{itemize}
\end{theoreme}

\textbf{Propriétés :}
\begin{itemize}
    \item Les valeurs singulières sont les racines carrées des valeurs propres de $\mat{A}^T\mat{A}$
    \item $\rank(\mat{A}) = $ nombre de valeurs singulières non nulles
    \item Fonctionne pour \textbf{toute} matrice (pas besoin qu'elle soit carrée ou symétrique)
\end{itemize}

\begin{astuce}
\textbf{SVD en Python :}
\begin{lstlisting}[language=Python]
import numpy as np
U, Sigma, VT = np.linalg.svd(A, full_matrices=False)
# Reconstruction: A_approx = U @ np.diag(Sigma) @ VT
\end{lstlisting}
\end{astuce}

\subsection{Applications de la SVD}

\textbf{1. Approximation de rang faible :}

On peut approximer $\mat{A}$ en gardant seulement les $k$ plus grandes valeurs singulières :
\begin{equation}
    \mat{A}_k = \sum_{i=1}^k \sigma_i \vect{u}_i \vect{v}_i^T
\end{equation}

C'est l'approximation de rang $k$ optimale au sens de la norme de Frobenius.

\textbf{2. PCA (Principal Component Analysis) :}

La PCA est équivalente à une SVD des données centrées :
\begin{itemize}
    \item Centrer : $\mat{X}_c = \mat{X} - \bar{\vect{x}}$
    \item SVD : $\mat{X}_c = \mat{U}\mat{\Sigma}\mat{V}^T$
    \item Composantes principales = colonnes de $\mat{V}$
\end{itemize}

\textbf{3. Compression d'images :}

Une image est une matrice. La SVD permet de compresser en ne gardant que les $k$ premières valeurs singulières.

\textbf{4. Recommandation (Matrix Factorization) :}

Netflix Prize : décomposer la matrice utilisateurs $\times$ films en produit de matrices de rang faible.

\subsection{Autres Décompositions}

\textbf{Décomposition LU :} $\mat{A} = \mat{L}\mat{U}$ (Lower-Upper), pour résoudre $\mat{A}\vect{x} = \vect{b}$ efficacement.

\textbf{Décomposition de Cholesky :} Si $\mat{A}$ est symétrique définie positive, $\mat{A} = \mat{L}\mat{L}^T$.

\textbf{Décomposition QR :} $\mat{A} = \mat{Q}\mat{R}$ où $\mat{Q}$ est orthogonale et $\mat{R}$ est triangulaire supérieure.

\begin{miniprojet}{Prédire le prix des maisons}
\textbf{Objectif :} Appliquer l'algèbre linéaire pour prédire le prix d'une maison.

\textbf{Données :} Vous avez 3 maisons avec [surface m², chambres, année] et leur prix :
\begin{equation*}
    \mat{X} = \begin{pmatrix}
        75 & 3 & 2010 \\
        50 & 2 & 2005 \\
        100 & 4 & 2015
    \end{pmatrix}, \quad
    \vect{y} = \begin{pmatrix} 300k\text{€} \\ 200k\text{€} \\ 400k\text{€} \end{pmatrix}
\end{equation*}

\textbf{À faire :}
\begin{enumerate}
    \item Normaliser les features avec $\vect{x}_{\text{norm}} = \frac{\vect{x} - \bar{\vect{x}}}{\|\vect{x}\|_2}$
    \item Calculer les poids optimaux : $\vect{w}^* = (\mat{X}^T\mat{X})^{-1}\mat{X}^T\vect{y}$
    \item Prédire le prix d'une nouvelle maison [80 m², 3 chambres, 2012]
    \item Calculer l'erreur : $\|\mat{X}\vect{w}^* - \vect{y}\|_2$
\end{enumerate}

\textbf{Code complet dans :} \texttt{01\_demo\_algebre\_lineaire.ipynb}
\end{miniprojet}

% ===== PARTIE II: PROBABILITÉS ET STATISTIQUES =====
\part{Probabilités et Statistiques}

\begin{pourquoi}
Les probabilités sont le langage de l'incertitude en ML :
\begin{itemize}
    \item \textbf{Classification} : "Il y a 87\% de chances que cet email soit un spam" → Probabilité conditionnelle
    \item \textbf{Modèles génératifs} : ChatGPT prédit le prochain mot en calculant $P(\text{mot} \mid \text{contexte})$ → Loi normale multivariée
    \item \textbf{Diagnostic médical} : Théorème de Bayes pour calculer $P(\text{malade} \mid \text{test positif})$
    \item \textbf{A/B Testing} : Tester si une nouvelle feature améliore réellement votre app → Tests statistiques
\end{itemize}

\textbf{Le Machine Learning = apprendre des distributions de probabilité à partir des données !}
\end{pourquoi}

\section{Probabilités Fondamentales}

\subsection{Concepts de Base}

\begin{definition}{Probabilité}
Une probabilité $P$ est une mesure sur un espace d'événements $\Omega$ telle que :
\begin{itemize}
    \item $0 \leq P(A) \leq 1$ pour tout événement $A$
    \item $P(\Omega) = 1$
    \item Si $A_1, A_2, \ldots$ sont disjoints, $P(\cup_i A_i) = \sum_i P(A_i)$
\end{itemize}
\end{definition}

\textbf{Règles de base :}
\begin{align}
    P(A \cup B) &= P(A) + P(B) - P(A \cap B) \\
    P(A^c) &= 1 - P(A) \\
    P(A \mid B) &= \frac{P(A \cap B)}{P(B)} \quad \text{(probabilité conditionnelle)}
\end{align}

\subsection{Indépendance}

\begin{definition}{Indépendance}
Deux événements $A$ et $B$ sont indépendants si :
\begin{equation}
    P(A \cap B) = P(A) \cdot P(B)
\end{equation}
Équivalent à : $P(A \mid B) = P(A)$.
\end{definition}

\subsection{Théorème de Bayes}

\begin{intuition}
\textbf{Le paradoxe du test médical :}

Vous faites un test pour une maladie rare (0.1\% de la population). Le test est précis à 99\%.

\textbf{Question :} Si votre test est positif, quelle est la probabilité que vous soyez vraiment malade ?

\textbf{Intuition naïve :} "Le test est précis à 99\%, donc j'ai 99\% de chances d'être malade !" [FAUX]

\textbf{Réalité avec Bayes :} Seulement \textbf{9\%} de chances d'être malade !

\textbf{Pourquoi ?} Parmi 10,000 personnes :
\begin{itemize}
    \item 10 sont vraiment malades → 9.9 tests positifs (99\% de précision)
    \item 9,990 sont saines → 99.9 faux positifs (1\% d'erreur)
    \item Total tests positifs : $\approx$ 110
    \item Vraiment malades : seulement 9.9 sur 110 = 9\%
\end{itemize}

\textbf{Morale :} Les tests sur maladies rares donnent beaucoup de faux positifs, même s'ils sont précis !
\end{intuition}

\begin{theoreme}{Théorème de Bayes}
\begin{equation}
    P(A \mid B) = \frac{P(B \mid A) \cdot P(A)}{P(B)}
\end{equation}
\end{theoreme}

\textbf{Terminologie :}
\begin{itemize}
    \item $P(A)$ : probabilité a priori
    \item $P(A \mid B)$ : probabilité a posteriori
    \item $P(B \mid A)$ : vraisemblance (likelihood)
    \item $P(B)$ : évidence
\end{itemize}

\textbf{Forme étendue :}
\begin{equation}
    P(A \mid B) = \frac{P(B \mid A) \cdot P(A)}{\sum_{i} P(B \mid A_i) \cdot P(A_i)}
\end{equation}

\begin{astuce}
\textbf{Applications en ML :}
\begin{itemize}
    \item Naive Bayes Classifier
    \item Inférence bayésienne
    \item Filtres bayésiens (spam, Kalman)
\end{itemize}
\end{astuce}

\section{Variables Aléatoires}

\subsection{Variable Aléatoire Discrète}

\begin{definition}{Variable aléatoire discrète}
Une variable aléatoire $X$ prend des valeurs dans un ensemble discret (fini ou dénombrable). Sa distribution est caractérisée par la \textbf{fonction de masse} :
\begin{equation}
    p_X(x) = P(X = x)
\end{equation}
avec $\sum_x p_X(x) = 1$.
\end{definition}

\textbf{Distributions classiques :}

\begin{itemize}
    \item \textbf{Bernoulli} : $X \in \{0, 1\}$, $P(X=1) = p$
    \begin{equation}
        p_X(x) = p^x (1-p)^{1-x}
    \end{equation}

    \item \textbf{Binomiale} : $X \sim \text{Bin}(n, p)$ (nombre de succès en $n$ essais)
    \begin{equation}
        p_X(k) = \binom{n}{k} p^k (1-p)^{n-k}
    \end{equation}

    \item \textbf{Poisson} : $X \sim \text{Poisson}(\lambda)$ (événements rares)
    \begin{equation}
        p_X(k) = \frac{\lambda^k e^{-\lambda}}{k!}
    \end{equation}
\end{itemize}

\subsection{Variable Aléatoire Continue}

\begin{definition}{Variable aléatoire continue}
Une variable continue $X$ est caractérisée par une \textbf{fonction de densité} $f_X(x)$ telle que :
\begin{equation}
    P(a \leq X \leq b) = \int_a^b f_X(x) \, dx
\end{equation}
avec $\int_{-\infty}^{\infty} f_X(x) \, dx = 1$.
\end{definition}

\textbf{Distributions classiques :}

\begin{itemize}
    \item \textbf{Uniforme} : $X \sim \text{Uniform}(a, b)$
    \begin{equation}
        f_X(x) = \begin{cases}
            \frac{1}{b-a} & \text{si } a \leq x \leq b \\
            0 & \text{sinon}
        \end{cases}
    \end{equation}

    \item \textbf{Normale (Gaussienne)} : $X \sim \mathcal{N}(\mu, \sigma^2)$
    \begin{equation}
        f_X(x) = \frac{1}{\sqrt{2\pi\sigma^2}} \exp\left(-\frac{(x-\mu)^2}{2\sigma^2}\right)
    \end{equation}

    \item \textbf{Exponentielle} : $X \sim \text{Exp}(\lambda)$ (temps entre événements)
    \begin{equation}
        f_X(x) = \lambda e^{-\lambda x} \quad (x \geq 0)
    \end{equation}
\end{itemize}

\subsection{Espérance et Variance}

\begin{definition}{Espérance}
L'espérance (moyenne) d'une variable aléatoire $X$ est :
\begin{equation}
    \E[X] = \begin{cases}
        \sum_x x \cdot p_X(x) & \text{(discret)} \\
        \int_{-\infty}^{\infty} x \cdot f_X(x) \, dx & \text{(continu)}
    \end{cases}
\end{equation}
\end{definition}

\textbf{Propriétés :}
\begin{itemize}
    \item Linéarité : $\E[aX + bY] = a\E[X] + b\E[Y]$
    \item $\E[X + c] = \E[X] + c$
\end{itemize}

\begin{definition}{Variance}
La variance mesure la dispersion autour de la moyenne :
\begin{equation}
    \Var(X) = \E[(X - \E[X])^2] = \E[X^2] - (\E[X])^2
\end{equation}
L'écart-type est $\sigma_X = \sqrt{\Var(X)}$.
\end{definition}

\textbf{Propriétés :}
\begin{itemize}
    \item $\Var(aX + b) = a^2 \Var(X)$
    \item Si $X, Y$ indépendants : $\Var(X + Y) = \Var(X) + \Var(Y)$
\end{itemize}

\textbf{Exemples :}
\begin{table}[h]
\centering
\begin{tabular}{lcc}
\toprule
\textbf{Loi} & \textbf{Espérance} & \textbf{Variance} \\
\midrule
Bernoulli$(p)$ & $p$ & $p(1-p)$ \\
Binomiale$(n, p)$ & $np$ & $np(1-p)$ \\
Poisson$(\lambda)$ & $\lambda$ & $\lambda$ \\
Uniforme$(a, b)$ & $\frac{a+b}{2}$ & $\frac{(b-a)^2}{12}$ \\
Normale$(\mu, \sigma^2)$ & $\mu$ & $\sigma^2$ \\
\bottomrule
\end{tabular}
\end{table}

\subsection{Loi Normale (Gaussienne)}

La loi normale est \textbf{fondamentale} en ML et statistiques.

\begin{theoreme}{Théorème Central Limite (TCL)}
Soit $X_1, \ldots, X_n$ des variables i.i.d. d'espérance $\mu$ et variance $\sigma^2$. Alors :
\begin{equation}
    \frac{\bar{X}_n - \mu}{\sigma/\sqrt{n}} \xrightarrow{n \to \infty} \mathcal{N}(0, 1)
\end{equation}
où $\bar{X}_n = \frac{1}{n}\sum_{i=1}^n X_i$.
\end{theoreme}

\textbf{Conséquence :} Beaucoup de phénomènes naturels suivent approximativement une loi normale grâce au TCL.

\textbf{Propriétés de la loi normale :}
\begin{itemize}
    \item Somme de gaussiennes = gaussienne
    \item 68\% des valeurs dans $[\mu - \sigma, \mu + \sigma]$
    \item 95\% dans $[\mu - 2\sigma, \mu + 2\sigma]$
    \item 99.7\% dans $[\mu - 3\sigma, \mu + 3\sigma]$
\end{itemize}

\begin{figure}[h]
\centering
\begin{tikzpicture}[scale=1.1]
    % Axes
    \draw[->] (-4.5,0) -- (4.5,0) node[right] {$x$};
    \draw[->] (0,0) -- (0,4.2) node[above] {Densité};

    % Courbe gaussienne (approximation)
    \draw[blue, very thick, smooth, domain=-4:4, samples=100]
        plot (\x, {3.5*exp(-0.5*\x*\x)});

    % Marques sur l'axe x
    \foreach \x/\label in {-3/$\mu-3\sigma$, -2/$\mu-2\sigma$, -1/$\mu-\sigma$, 0/$\mu$, 1/$\mu+\sigma$, 2/$\mu+2\sigma$, 3/$\mu+3\sigma$}
        \draw (\x,0.05) -- (\x,-0.05) node[below, font=\small] {\label};

    % Zone 68% (1 sigma)
    \fill[green!20, opacity=0.6, domain=-1:1, samples=50]
        plot (\x, {3.5*exp(-0.5*\x*\x)}) -- (1,0) -- (-1,0) -- cycle;
    \draw[green!60!black, very thick] (-1,0) -- (-1,2.12);
    \draw[green!60!black, very thick] (1,0) -- (1,2.12);
    \node[green!60!black, font=\bfseries] at (0, 1) {68\%};

    % Zone 95% (2 sigma)
    \fill[yellow!30, opacity=0.5, domain=-2:-1, samples=30]
        plot (\x, {3.5*exp(-0.5*\x*\x)}) -- (-1,0) -- (-2,0) -- cycle;
    \fill[yellow!30, opacity=0.5, domain=1:2, samples=30]
        plot (\x, {3.5*exp(-0.5*\x*\x)}) -- (2,0) -- (1,0) -- cycle;
    \draw[orange, very thick] (-2,0) -- (-2,0.47);
    \draw[orange, very thick] (2,0) -- (2,0.47);
    \node[orange, font=\small] at (-1.5, 0.7) {13.5\%};
    \node[orange, font=\small] at (1.5, 0.7) {13.5\%};

    % Zone 99.7% (3 sigma)
    \fill[red!20, opacity=0.4, domain=-3:-2, samples=30]
        plot (\x, {3.5*exp(-0.5*\x*\x)}) -- (-2,0) -- (-3,0) -- cycle;
    \fill[red!20, opacity=0.4, domain=2:3, samples=30]
        plot (\x, {3.5*exp(-0.5*\x*\x)}) -- (3,0) -- (2,0) -- cycle;
    \node[red!60!black, font=\tiny] at (-2.5, 0.3) {2.35\%};
    \node[red!60!black, font=\tiny] at (2.5, 0.3) {2.35\%};

    % Annotations
    \draw[<->, thick] (-1, 3.6) -- (1, 3.6);
    \node[above, font=\scriptsize] at (0, 3.6) {68\% des données};

    \draw[<->, thick] (-2, 4.4) -- (2, 4.4);
    \node[above, font=\scriptsize] at (0, 4.4) {95\% des données};

    % Légende
    \node[draw, fill=white, align=left, rounded corners] at (6, 2.5) {
        \textbf{Règle 68-95-99.7}\\[3pt]
        \textcolor{green!60!black}{$\blacksquare$} 68\% dans $[\mu-\sigma, \mu+\sigma]$\\
        \textcolor{orange}{$\blacksquare$} 95\% dans $[\mu-2\sigma, \mu+2\sigma]$\\
        \textcolor{red!60!black}{$\blacksquare$} 99.7\% dans $[\mu-3\sigma, \mu+3\sigma]$
    };

\end{tikzpicture}
\caption{Loi normale (Gaussienne): la règle 68-95-99.7 permet de quantifier rapidement les probabilités. Environ 68\% des données se trouvent à moins d'un écart-type de la moyenne, 95\% à moins de deux, et 99.7\% à moins de trois.}
\label{fig:gaussian}
\end{figure}

\subsection{Loi Normale Multivariée}

\begin{definition}{Loi normale multivariée}
Un vecteur aléatoire $\vect{X} \in \R^d$ suit une loi normale multivariée $\mathcal{N}(\vect{\mu}, \mat{\Sigma})$ si sa densité est :
\begin{equation}
    f_{\vect{X}}(\vect{x}) = \frac{1}{(2\pi)^{d/2} |\mat{\Sigma}|^{1/2}} \exp\left(-\frac{1}{2}(\vect{x} - \vect{\mu})^T \mat{\Sigma}^{-1} (\vect{x} - \vect{\mu})\right)
\end{equation}
où $\vect{\mu} \in \R^d$ est le vecteur moyenne et $\mat{\Sigma} \in \R^{d \times d}$ est la matrice de covariance.
\end{definition}

\textbf{Utilisations en ML :}
\begin{itemize}
    \item Gaussian Mixture Models (GMM)
    \item Analyse discriminante linéaire (LDA)
    \item Processus gaussiens
\end{itemize}

\section{Statistiques Descriptives}

\subsection{Mesures de Tendance Centrale}

\textbf{Moyenne empirique :}
\begin{equation}
    \bar{x} = \frac{1}{n} \sum_{i=1}^n x_i
\end{equation}

\textbf{Médiane :} Valeur centrale quand les données sont triées. Robuste aux outliers.

\textbf{Mode :} Valeur la plus fréquente.

\subsection{Mesures de Dispersion}

\textbf{Variance empirique :}
\begin{equation}
    s^2 = \frac{1}{n-1} \sum_{i=1}^n (x_i - \bar{x})^2
\end{equation}

\textbf{Écart-type :} $s = \sqrt{s^2}$

\textbf{Intervalle interquartile (IQR) :} $IQR = Q_3 - Q_1$ (robuste)

\subsection{Covariance et Corrélation}

\begin{definition}{Covariance}
La covariance entre deux variables $X$ et $Y$ mesure leur variation conjointe :
\begin{equation}
    \Cov(X, Y) = \E[(X - \E[X])(Y - \E[Y])] = \E[XY] - \E[X]\E[Y]
\end{equation}
Version empirique :
\begin{equation}
    \text{cov}(x, y) = \frac{1}{n-1} \sum_{i=1}^n (x_i - \bar{x})(y_i - \bar{y})
\end{equation}
\end{definition}

\begin{definition}{Coefficient de corrélation de Pearson}
\begin{equation}
    \rho_{XY} = \frac{\Cov(X, Y)}{\sigma_X \sigma_Y} \in [-1, 1]
\end{equation}
\end{definition}

\textbf{Interprétation :}
\begin{itemize}
    \item $\rho = 1$ : corrélation linéaire positive parfaite
    \item $\rho = -1$ : corrélation linéaire négative parfaite
    \item $\rho = 0$ : pas de corrélation linéaire
\end{itemize}

\textbf{Matrice de covariance :}
Pour un vecteur aléatoire $\vect{X} = (X_1, \ldots, X_d)^T$ :
\begin{equation}
    \mat{\Sigma} = \E[(\vect{X} - \E[\vect{X}])(\vect{X} - \E[\vect{X}])^T]
\end{equation}
avec $\Sigma_{ij} = \Cov(X_i, X_j)$.

\begin{astuce}
En Python (NumPy/Pandas) :
\begin{lstlisting}[language=Python]
# Matrice de covariance
cov_matrix = np.cov(X, rowvar=False)  # colonnes = variables

# Matrice de corrélation
corr_matrix = np.corrcoef(X, rowvar=False)
# ou avec pandas:
corr_matrix = df.corr()
\end{lstlisting}
\end{astuce}

\begin{miniprojet}{Classifier des emails (spam ou non)}
\textbf{Objectif :} Utiliser le théorème de Bayes pour détecter les spams.

\textbf{Données :} Sur 1000 emails :
\begin{itemize}
    \item 300 sont des spams (30\%)
    \item Le mot "gratuit" apparaît dans 80\% des spams
    \item Le mot "gratuit" apparaît dans 10\% des emails normaux
\end{itemize}

\textbf{Questions :}
\begin{enumerate}
    \item Calculer $P(\text{spam} \mid \text{"gratuit"})$ avec Bayes
    \item Si un email contient "gratuit" ET "urgent", quelle est la probabilité que ce soit un spam ?
    \item Simuler 1000 emails avec une loi de Bernoulli et calculer la précision du classifieur
    \item Tracer la courbe ROC (taux de vrais positifs vs faux positifs)
\end{enumerate}

\textbf{Indice :}
\begin{equation*}
    P(\text{spam} \mid \text{mot}) = \frac{P(\text{mot} \mid \text{spam}) \cdot P(\text{spam})}{P(\text{mot})}
\end{equation*}

\textbf{Code complet dans :} \texttt{01\_demo\_probabilites.ipynb}
\end{miniprojet}

% ===== PARTIE III: CALCUL DIFFÉRENTIEL =====
\part{Calcul Différentiel et Optimisation}

\begin{pourquoi}
L'optimisation est \textbf{LE CŒUR} du Machine Learning :
\begin{itemize}
    \item \textbf{Entraîner un modèle} = Minimiser une fonction de perte (loss function)
    \item \textbf{Gradient descent} = L'algorithme qui fait tout fonctionner (réseaux de neurones, régression, etc.)
    \item \textbf{Backpropagation} = Calcul automatique des gradients via la règle de la chaîne
\end{itemize}

\textbf{Exemple concret :}
\begin{itemize}
    \item Vous avez un modèle qui fait 30\% d'erreur sur vos données
    \item Le gradient vous dit : "Change ce poids de +0.5, cet autre de -0.2..."
    \item Après des millions d'itérations → Erreur tombe à 2\% !
\end{itemize}

\textbf{Sans gradient descent, pas de deep learning moderne !}
\end{pourquoi}

\section{Dérivées}

\subsection{Dérivée d'une Fonction Scalaire}

\begin{definition}{Dérivée}
La dérivée de $f: \R \to \R$ en $x$ est :
\begin{equation}
    f'(x) = \lim_{h \to 0} \frac{f(x+h) - f(x)}{h}
\end{equation}
\end{definition}

\textbf{Interprétation :} Pente de la tangente à la courbe en $x$.

\textbf{Règles de dérivation :}
\begin{align}
    (cf)' &= cf' \\
    (f + g)' &= f' + g' \\
    (fg)' &= f'g + fg' \quad \text{(règle du produit)} \\
    \left(\frac{f}{g}\right)' &= \frac{f'g - fg'}{g^2} \quad \text{(règle du quotient)} \\
    (f \circ g)' &= (f' \circ g) \cdot g' \quad \text{(règle de la chaîne)}
\end{align}

\textbf{Dérivées usuelles :}
\begin{table}[h]
\centering
\begin{tabular}{ll}
\toprule
$f(x)$ & $f'(x)$ \\
\midrule
$x^n$ & $nx^{n-1}$ \\
$e^x$ & $e^x$ \\
$\ln(x)$ & $1/x$ \\
$\sin(x)$ & $\cos(x)$ \\
$\cos(x)$ & $-\sin(x)$ \\
\bottomrule
\end{tabular}
\end{table}

\subsection{Gradient}

\begin{intuition}
Imaginez que vous êtes perdu en montagne dans le brouillard et voulez descendre :
\begin{itemize}
    \item La \textbf{fonction $f(\vect{x})$} = votre altitude à la position $\vect{x}$
    \item Le \textbf{gradient $\nabla f$} = une flèche qui pointe vers la montée la plus rapide
    \item \textbf{Descente de gradient} = Suivre la direction opposée ($-\nabla f$) pour descendre
\end{itemize}

\textbf{En ML :}
\begin{itemize}
    \item $f(\vect{w})$ = l'erreur de votre modèle avec les poids $\vect{w}$
    \item Vous voulez minimiser l'erreur (descendre)
    \item Le gradient vous dit comment ajuster chaque poids pour réduire l'erreur !
\end{itemize}
\end{intuition}

\begin{definition}{Gradient}
Pour une fonction $f: \R^n \to \R$, le gradient est le vecteur des dérivées partielles :
\begin{equation}
    \nabla f(\vect{x}) = \begin{pmatrix}
        \frac{\partial f}{\partial x_1} \\
        \frac{\partial f}{\partial x_2} \\
        \vdots \\
        \frac{\partial f}{\partial x_n}
    \end{pmatrix}
\end{equation}
\end{definition}

\textbf{Interprétation géométrique :}
\begin{itemize}
    \item Le gradient pointe dans la direction de \textbf{plus forte croissance}
    \item Sa norme $\|\nabla f\|$ mesure le taux de croissance
    \item $-\nabla f$ pointe vers la plus forte décroissance
\end{itemize}

\begin{exemple}{Gradient de fonctions courantes}
\begin{align}
    f(\vect{x}) &= \vect{a}^T \vect{x} = \sum_i a_i x_i \quad &\Rightarrow \quad \nabla f &= \vect{a} \\
    f(\vect{x}) &= \vect{x}^T \mat{A} \vect{x} \quad &\Rightarrow \quad \nabla f &= (\mat{A} + \mat{A}^T)\vect{x} \\
    f(\vect{x}) &= \|\vect{x}\|_2^2 = \vect{x}^T\vect{x} \quad &\Rightarrow \quad \nabla f &= 2\vect{x}
\end{align}
\end{exemple}

\subsection{Matrice Hessienne}

\begin{definition}{Hessienne}
Pour $f: \R^n \to \R$ deux fois différentiable, la matrice hessienne est la matrice des dérivées secondes :
\begin{equation}
    \mat{H}_f(\vect{x}) = \begin{pmatrix}
        \frac{\partial^2 f}{\partial x_1^2} & \frac{\partial^2 f}{\partial x_1 \partial x_2} & \cdots \\
        \frac{\partial^2 f}{\partial x_2 \partial x_1} & \frac{\partial^2 f}{\partial x_2^2} & \cdots \\
        \vdots & \vdots & \ddots
    \end{pmatrix}
\end{equation}
\end{definition}

\textbf{Propriétés :}
\begin{itemize}
    \item Si $f$ est $C^2$, alors $\mat{H}$ est symétrique (théorème de Schwarz)
    \item La hessienne mesure la \textbf{courbure} de $f$
    \item Utilisée dans les méthodes d'optimisation du second ordre (Newton)
\end{itemize}

\subsection{Jacobienne}

\begin{definition}{Jacobienne}
Pour une fonction vectorielle $\vect{f}: \R^n \to \R^m$, la jacobienne est la matrice des dérivées partielles :
\begin{equation}
    \mat{J}_{\vect{f}}(\vect{x}) = \begin{pmatrix}
        \frac{\partial f_1}{\partial x_1} & \cdots & \frac{\partial f_1}{\partial x_n} \\
        \vdots & \ddots & \vdots \\
        \frac{\partial f_m}{\partial x_1} & \cdots & \frac{\partial f_m}{\partial x_n}
    \end{pmatrix} \in \R^{m \times n}
\end{equation}
\end{definition}

\textbf{Application en Deep Learning :} La backpropagation utilise la règle de la chaîne avec des jacobiennes pour calculer les gradients.

\section{Optimisation}

\subsection{Conditions d'Optimalité}

\begin{theoreme}{Condition nécessaire du premier ordre}
Si $\vect{x}^*$ est un minimum local de $f: \R^n \to \R$ différentiable, alors :
\begin{equation}
    \nabla f(\vect{x}^*) = \vect{0}
\end{equation}
\end{theoreme}

\begin{theoreme}{Condition suffisante du second ordre}
Si $\nabla f(\vect{x}^*) = \vect{0}$ et la hessienne $\mat{H}_f(\vect{x}^*)$ est \textbf{définie positive} (toutes les valeurs propres $> 0$), alors $\vect{x}^*$ est un minimum local strict.
\end{theoreme}

\textbf{Cas de la hessienne :}
\begin{itemize}
    \item Définie positive $\Rightarrow$ minimum local
    \item Définie négative $\Rightarrow$ maximum local
    \item Indéfinie $\Rightarrow$ point-selle
\end{itemize}

\subsection{Convexité}

\begin{definition}{Fonction convexe}
Une fonction $f: \R^n \to \R$ est convexe si pour tous $\vect{x}, \vect{y}$ et $\lambda \in [0, 1]$ :
\begin{equation}
    f(\lambda \vect{x} + (1-\lambda) \vect{y}) \leq \lambda f(\vect{x}) + (1-\lambda) f(\vect{y})
\end{equation}
\end{definition}

\textbf{Critère différentiel :}
\begin{itemize}
    \item Si $\mat{H}_f(\vect{x})$ est semi-définie positive partout, alors $f$ est convexe
    \item Si $\mat{H}_f(\vect{x})$ est définie positive partout, alors $f$ est strictement convexe
\end{itemize}

\textbf{Propriété fondamentale :}
Pour une fonction convexe, \textbf{tout minimum local est global}.

\begin{exemple}{Fonctions convexes courantes}
\begin{itemize}
    \item Fonctions linéaires/affines : $f(\vect{x}) = \vect{a}^T\vect{x} + b$
    \item Norme $L^2$ au carré : $f(\vect{x}) = \|\vect{x}\|_2^2$
    \item Exponentielle : $f(x) = e^x$
    \item Logarithme négatif : $f(x) = -\ln(x)$ pour $x > 0$
\end{itemize}
\end{exemple}

\subsection{Descente de Gradient}

\begin{algorithm}[H]
\caption{Descente de Gradient}
\label{alg:gd}
\begin{algorithmic}[1]
\REQUIRE Fonction $f$, point initial $\vect{x}_0$, learning rate $\alpha > 0$, tolérance $\epsilon$
\ENSURE Minimum approximatif $\vect{x}^*$
\STATE $k \leftarrow 0$
\REPEAT
    \STATE Calculer le gradient : $\vect{g}_k = \nabla f(\vect{x}_k)$
    \STATE Mettre à jour : $\vect{x}_{k+1} = \vect{x}_k - \alpha \vect{g}_k$
    \STATE $k \leftarrow k + 1$
\UNTIL{$\|\vect{g}_k\| < \epsilon$}
\RETURN $\vect{x}_k$
\end{algorithmic}
\end{algorithm}

\begin{figure}[h]
\centering
\begin{tikzpicture}[scale=1.0]
    % Axes
    \draw[->] (0,0) -- (8,0) node[right] {$x_1$};
    \draw[->] (0,0) -- (0,6) node[above] {$x_2$};

    % Courbes de niveau (ellipses concentriques)
    \draw[gray, thick] (4,3) ellipse (3 and 2);
    \draw[gray, thick] (4,3) ellipse (2.2 and 1.5);
    \draw[gray, thick] (4,3) ellipse (1.5 and 1);
    \draw[gray, thick] (4,3) ellipse (0.8 and 0.5);
    \draw[gray, thick] (4,3) ellipse (0.3 and 0.2);

    % Minimum global
    \filldraw[red] (4,3) circle (3pt);
    \node[red, below] at (4,2.7) {$\vect{x}^*$ (minimum)};

    % Trajectoire de descente de gradient
    \coordinate (x0) at (1.5,5);
    \coordinate (x1) at (2.2,4.5);
    \coordinate (x2) at (2.7,4.1);
    \coordinate (x3) at (3.1,3.8);
    \coordinate (x4) at (3.4,3.5);
    \coordinate (x5) at (3.7,3.25);
    \coordinate (x6) at (3.9,3.1);

    % Points de la trajectoire
    \filldraw[blue] (x0) circle (2pt) node[above left] {$\vect{x}_0$};
    \filldraw[blue] (x1) circle (2pt);
    \filldraw[blue] (x2) circle (2pt);
    \filldraw[blue] (x3) circle (2pt);
    \filldraw[blue] (x4) circle (2pt);
    \filldraw[blue] (x5) circle (2pt);
    \filldraw[blue] (x6) circle (2pt);

    % Flèches entre les points
    \draw[->, very thick, blue] (x0) -- (x1);
    \draw[->, very thick, blue] (x1) -- (x2);
    \draw[->, very thick, blue] (x2) -- (x3);
    \draw[->, very thick, blue] (x3) -- (x4);
    \draw[->, very thick, blue] (x4) -- (x5);
    \draw[->, very thick, blue] (x5) -- (x6);
    \draw[->, very thick, blue] (x6) -- (4,3);

    % Gradient à x0
    \draw[->, very thick, green!60!black] (x0) -- (2.3,5.8);
    \node[green!60!black, right] at (2.3,5.8) {$\nabla f(\vect{x}_0)$};

    % Direction de descente à x0
    \draw[->, very thick, purple, dashed] (x0) -- (1.2,4.5);
    \node[purple, left, align=center, font=\small] at (1.2,4.5) {$-\nabla f$\\(descente)};

    % Légende
    \node[draw, fill=yellow!10, align=left, rounded corners] at (6.5, 5) {
        Courbes de niveau de $f$\\
        $\vect{x}_{k+1} = \vect{x}_k - \alpha \nabla f(\vect{x}_k)$\\[3pt]
        Learning rate $\alpha$ contrôle\\
        la taille des pas
    };

\end{tikzpicture}
\caption{Descente de gradient: la trajectoire suit la direction opposée au gradient ($-\nabla f$) pour descendre vers le minimum. Les courbes de niveau montrent les valeurs constantes de $f$. À chaque itération, on se rapproche du minimum $\vect{x}^*$.}
\label{fig:gradient_descent}
\end{figure}

\textbf{Intuition :} On se déplace itérativement dans la direction opposée au gradient (plus forte descente).

\textbf{Paramètres :}
\begin{itemize}
    \item \textbf{Learning rate $\alpha$} : Trop grand $\Rightarrow$ divergence ; trop petit $\Rightarrow$ convergence lente
    \item \textbf{Critère d'arrêt} : $\|\nabla f\| < \epsilon$ ou nombre max d'itérations
\end{itemize}

\begin{astuce}
\textbf{Variantes modernes (Deep Learning) :}
\begin{itemize}
    \item SGD (Stochastic Gradient Descent)
    \item Momentum
    \item Adam, RMSprop, AdaGrad
\end{itemize}
Voir Chapitre 06 pour les détails.
\end{astuce}

\subsection{Méthode de Newton}

L'idée est d'utiliser l'information du second ordre (hessienne) :
\begin{equation}
    \vect{x}_{k+1} = \vect{x}_k - \mat{H}_f(\vect{x}_k)^{-1} \nabla f(\vect{x}_k)
\end{equation}

\textbf{Avantages :}
\begin{itemize}
    \item Convergence quadratique (très rapide près du minimum)
    \item Pas besoin de tuner le learning rate
\end{itemize}

\textbf{Inconvénients :}
\begin{itemize}
    \item Coût : calcul et inversion de la hessienne ($O(n^3)$)
    \item Ne fonctionne que si $\mat{H}$ est définie positive
    \item Rarement utilisé en Deep Learning (trop coûteux)
\end{itemize}

\textbf{Compromis :} Méthodes quasi-Newton (L-BFGS) qui approximent la hessienne.

\begin{miniprojet}{Entraîner une régression linéaire}
\textbf{Objectif :} Implémenter la descente de gradient pour entraîner un modèle.

\textbf{Problème :} Prédire le salaire en fonction des années d'expérience.

\textbf{Données :} 5 personnes :
\begin{equation*}
    \text{Expérience (ans) : } [1, 2, 3, 4, 5] \quad \text{Salaire (k€) : } [30, 35, 45, 50, 60]
\end{equation*}

\textbf{Modèle :} $\hat{y} = w_1 x + w_0$ (régression linéaire)

\textbf{Loss function :} $L(\vect{w}) = \frac{1}{n}\sum_{i=1}^n (\hat{y}_i - y_i)^2$ (MSE)

\textbf{À faire :}
\begin{enumerate}
    \item Calculer le gradient : $\nabla L = \begin{pmatrix} \frac{\partial L}{\partial w_0} \\ \frac{\partial L}{\partial w_1} \end{pmatrix}$
    \item Initialiser $\vect{w} = \begin{pmatrix} 0 \\ 0 \end{pmatrix}$
    \item Appliquer la descente de gradient pendant 100 itérations avec $\alpha = 0.01$
    \item Tracer l'évolution de la loss et de la prédiction
    \item Comparer avec la solution analytique : $\vect{w}^* = (\mat{X}^T\mat{X})^{-1}\mat{X}^T\vect{y}$
\end{enumerate}

\textbf{Résultat attendu :} $w_1 \approx 7.5$ (chaque année d'expérience = +7.5k€), $w_0 \approx 22.5$ (salaire de base)

\textbf{Code complet dans :} \texttt{01\_demo\_optimisation.ipynb}
\end{miniprojet}

% ===== SECTION RÉSUMÉ =====
\section{Résumé du Chapitre}

\subsection{Points Clés}

\textbf{Algèbre Linéaire :}
\begin{itemize}
    \item Vecteurs et matrices sont omniprésents en ML (données, poids, transformations)
    \item Produit scalaire $\Rightarrow$ similarité ; norme $\Rightarrow$ distance
    \item SVD = décomposition universelle (PCA, compression, recommandation)
    \item Systèmes linéaires $\Rightarrow$ régression linéaire (moindres carrés)
\end{itemize}

\textbf{Probabilités et Statistiques :}
\begin{itemize}
    \item Loi normale = fondamentale (TCL, modèles génératifs)
    \item Théorème de Bayes = base de l'inférence bayésienne
    \item Covariance/corrélation $\Rightarrow$ dépendance entre variables
    \item Espérance, variance = résument une distribution
\end{itemize}

\textbf{Calcul Différentiel et Optimisation :}
\begin{itemize}
    \item Gradient = direction de plus forte croissance
    \item Descente de gradient = algorithme d'optimisation de base en ML
    \item Convexité $\Rightarrow$ minimum local = global
    \item Hessienne = courbure (méthodes du second ordre)
\end{itemize}

\subsection{Formules Essentielles}

\begin{tcolorbox}[colback=blue!5!white, colframe=blue!75!black, title=Formules à retenir]
\textbf{Algèbre linéaire :}
\begin{align}
    \text{Produit scalaire : } & \vect{u} \cdot \vect{v} = \sum_i u_i v_i \\
    \text{Norme $L^2$ : } & \|\vect{v}\|_2 = \sqrt{\sum_i v_i^2} \\
    \text{Moindres carrés : } & \vect{w}^* = (\mat{X}^T\mat{X})^{-1}\mat{X}^T\vect{y} \\
    \text{SVD : } & \mat{A} = \mat{U}\mat{\Sigma}\mat{V}^T
\end{align}

\textbf{Probabilités :}
\begin{align}
    \text{Bayes : } & P(A|B) = \frac{P(B|A)P(A)}{P(B)} \\
    \text{Loi normale : } & \mathcal{N}(x; \mu, \sigma^2) = \frac{1}{\sqrt{2\pi\sigma^2}} e^{-(x-\mu)^2/(2\sigma^2)} \\
    \text{Variance : } & \Var(X) = \E[X^2] - (\E[X])^2
\end{align}

\textbf{Optimisation :}
\begin{align}
    \text{Gradient : } & \nabla f = \left(\frac{\partial f}{\partial x_1}, \ldots, \frac{\partial f}{\partial x_n}\right)^T \\
    \text{Descente de gradient : } & \vect{x}_{k+1} = \vect{x}_k - \alpha \nabla f(\vect{x}_k)
\end{align}
\end{tcolorbox}

% ===== EXERCICES =====
\section{Exercices}

\subsection{Algèbre Linéaire}

\begin{enumerate}
    \item Calculer le produit scalaire de $\vect{u} = (1, 2, 3)$ et $\vect{v} = (4, -1, 2)$. Les vecteurs sont-ils orthogonaux ?

    \item Pour la matrice $\mat{A} = \begin{pmatrix} 2 & 1 \\ 1 & 3 \end{pmatrix}$, calculer :
    \begin{itemize}
        \item Les valeurs propres et vecteurs propres
        \item La décomposition spectrale
    \end{itemize}

    \item Appliquer la SVD à la matrice $\mat{A} = \begin{pmatrix} 3 & 1 \\ 1 & 3 \\ 1 & 1 \end{pmatrix}$ (en Python). Reconstruire $\mat{A}$ à partir de la SVD.

    \item Résoudre le système linéaire au sens des moindres carrés :
    \begin{equation}
        \begin{pmatrix} 1 & 1 \\ 1 & 2 \\ 1 & 3 \end{pmatrix} \begin{pmatrix} a \\ b \end{pmatrix} \approx \begin{pmatrix} 2 \\ 3 \\ 5 \end{pmatrix}
    \end{equation}
\end{enumerate}

\subsection{Probabilités et Statistiques}

\begin{enumerate}
    \item Un test médical détecte une maladie avec 99\% de précision (vrai positif). La prévalence de la maladie est 0.1\%. Si le test est positif, quelle est la probabilité d'être réellement malade ? (Bayes)

    \item Pour $X \sim \mathcal{N}(10, 4)$, calculer $P(8 \leq X \leq 12)$.

    \item Générer 1000 échantillons d'une loi normale $\mathcal{N}(5, 2)$ en Python. Calculer la moyenne et variance empiriques. Tracer l'histogramme.

    \item Calculer la matrice de covariance pour les données :
    \begin{equation}
        \mat{X} = \begin{pmatrix} 1 & 2 \\ 2 & 4 \\ 3 & 5 \end{pmatrix}
    \end{equation}
    Interpréter la corrélation entre les deux variables.
\end{enumerate}

\subsection{Calcul Différentiel et Optimisation}

\begin{enumerate}
    \item Calculer le gradient de $f(\vect{x}) = x_1^2 + 2x_1x_2 + 3x_2^2$.

    \item Montrer que $f(\vect{x}) = \|\mat{A}\vect{x} - \vect{b}\|_2^2$ est convexe. Calculer son gradient.

    \item Implémenter la descente de gradient pour minimiser $f(x) = (x-3)^2 + 5$ en partant de $x_0 = 0$. Tester différents learning rates.

    \item Minimiser $f(x_1, x_2) = x_1^2 + 4x_2^2$ par descente de gradient. Visualiser la trajectoire.
\end{enumerate}

\textit{Solutions détaillées dans} \texttt{01\_exercices.ipynb} \textit{(solutions intégrées dans le notebook)}

% ===== POUR ALLER PLUS LOIN =====
\section{Pour Aller Plus Loin}

\subsection{Lectures Recommandées}

\textbf{Livres :}
\begin{itemize}
    \item \textit{Linear Algebra and Its Applications} (4e éd., 2006) - Gilbert Strang
    \item \textit{Probability and Statistics for Engineers} (9e éd., 2016) - Montgomery \& Runger
    \item \textit{Convex Optimization} (2004) - Boyd \& Vandenberghe
    \item \textit{Mathematics for Machine Learning} (2020) - Deisenroth, Faisal, Ong (gratuit en ligne)
\end{itemize}

\textbf{Ressources en ligne :}
\begin{itemize}
    \item MIT OCW : Linear Algebra (18.06) - Gilbert Strang
    \item 3Blue1Brown : Essence of Linear Algebra (YouTube)
    \item Khan Academy : Probabilités et statistiques
\end{itemize}

\subsection{Outils Pratiques}

\textbf{NumPy (calcul numérique) :}
\begin{lstlisting}[language=Python]
import numpy as np

# Algèbre linéaire
A = np.array([[1, 2], [3, 4]])
eigenvalues, eigenvectors = np.linalg.eig(A)
U, Sigma, VT = np.linalg.svd(A)

# Statistiques
mean = np.mean(data)
std = np.std(data)
cov_matrix = np.cov(X, rowvar=False)
\end{lstlisting}

\textbf{SciPy (fonctions avancées) :}
\begin{lstlisting}[language=Python]
from scipy import stats, optimize

# Distributions
rv = stats.norm(loc=0, scale=1)  # N(0,1)
pdf_values = rv.pdf(x)

# Optimisation
result = optimize.minimize(f, x0, method='BFGS')
\end{lstlisting}

\subsection{Prochaines Étapes}

Chapitre suivant : \textbf{Chapitre 02 - Métriques d'Évaluation}

Ces fondamentaux seront utilisés tout au long du cours :
\begin{itemize}
    \item Régression linéaire (Ch. 03) : moindres carrés, gradient
    \item PCA (Ch. 05) : SVD, valeurs propres
    \item Réseaux de neurones (Ch. 06) : backpropagation, descente de gradient
    \item Probabilités : modèles génératifs, bayésiens
\end{itemize}

% ===== BIBLIOGRAPHIE =====
\section*{Références}

\begin{enumerate}
    \item Strang, G. (2006). \textit{Linear Algebra and Its Applications} (4e éd.). Cengage Learning.

    \item Deisenroth, M. P., Faisal, A. A., \& Ong, C. S. (2020). \textit{Mathematics for Machine Learning}. Cambridge University Press.

    \item Boyd, S., \& Vandenberghe, L. (2004). \textit{Convex Optimization}. Cambridge University Press.

    \item Bishop, C. M. (2006). \textit{Pattern Recognition and Machine Learning}. Springer.

    \item Murphy, K. P. (2022). \textit{Probabilistic Machine Learning: An Introduction}. MIT Press.

    \item Goodfellow, I., Bengio, Y., \& Courville, A. (2016). \textit{Deep Learning}. MIT Press.
\end{enumerate}

\end{document}
