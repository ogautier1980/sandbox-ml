% Chapitre 03 - Régression

\documentclass[11pt,a4paper]{article}

% ===== PACKAGES =====
\usepackage[utf8]{inputenc}
\usepackage[T1]{fontenc}
\usepackage[french]{babel}
\usepackage{lmodern}

% Mathématiques
\usepackage{amsmath, amssymb, amsthm}
\usepackage{mathtools}

% Mise en page
\usepackage[margin=2.5cm]{geometry}
\usepackage{parskip}
\usepackage{setspace}
\setstretch{1.15}

% Graphiques et couleurs
\usepackage{graphicx}
\usepackage{xcolor}

% ===== UNICODE CHARACTERS SUPPORT =====
\usepackage{newunicodechar}

% Emojis et symboles
\newunicodechar{✅}{\textcolor{green!60!black}{$\checkmark$}}
\newunicodechar{❌}{\textcolor{red!60!black}{$\times$}}
\newunicodechar{✓}{\textcolor{green!60!black}{$\checkmark$}}
\newunicodechar{✗}{\textcolor{red!60!black}{$\times$}}
\newunicodechar{⚠}{\textcolor{orange!80!black}{\textbf{/!\textbackslash}}}
\newunicodechar{💡}{\textcolor{blue!70!black}{\textbf{(i)}}}
\newunicodechar{🎯}{\textcolor{purple!70!black}{\textbf{$\star$}}}
\newunicodechar{📊}{\textcolor{blue!70!black}{\textbf{[=]}}}

% Étoiles (pour tableaux)
\newunicodechar{★}{\textcolor{orange!80!black}{$\star$}}
\newunicodechar{☆}{\textcolor{gray!50}{$\star$}}

% Flèches
\newunicodechar{→}{$\rightarrow$}
\newunicodechar{←}{$\leftarrow$}
\newunicodechar{↑}{$\uparrow$}
\newunicodechar{↓}{$\downarrow$}

\usepackage{tikz}
\usetikzlibrary{arrows.meta, positioning, shapes.geometric}

% Tableaux
\usepackage{booktabs}
\usepackage{longtable}
\usepackage{multirow}
\usepackage{tabularx}
\usepackage{colortbl}

% Code et algorithmes
\usepackage{listings}
\usepackage{algorithm}
\usepackage{algorithmic}

% Hyperliens
\usepackage{hyperref}
\hypersetup{
    colorlinks=true,
    linkcolor=blue,
    filecolor=magenta,
    urlcolor=cyan,
    citecolor=green,
    pdftitle={Chapitre 03 - Régression},
    pdfauthor={Cours ML},
}

% Boxes colorées
\usepackage{tcolorbox}
\tcbuselibrary{skins, breakable}


% ===== TCOLORBOX AVEC EMOJIS =====
\newtcolorbox{attention}{
    colback=red!5!white,
    colframe=red!75!black,
    fonttitle=\bfseries,
    title=⚠ Attention,
    breakable
}

\newtcolorbox{definition}{
    colback=blue!5!white,
    colframe=blue!75!black,
    fonttitle=\bfseries,
    title=Définition,
    breakable
}

\newtcolorbox{astuce}{
    colback=green!5!white,
    colframe=green!60!black,
    fonttitle=\bfseries,
    title=💡 Astuce,
    breakable
}

\newtcolorbox{remarque}{
    colback=yellow!5!white,
    colframe=orange!75!black,
    fonttitle=\bfseries,
    title=💡 Remarque,
    breakable
}

\newtcolorbox{important}{
    colback=purple!5!white,
    colframe=purple!75!black,
    fonttitle=\bfseries,
    title=⚠ Important,
    breakable
}

\newtcolorbox{exemple}{
    colback=gray!5!white,
    colframe=gray!75!black,
    fonttitle=\bfseries,
    title=Exemple,
    breakable
}

% En-têtes et pieds de page
\usepackage{fancyhdr}
\pagestyle{fancy}
\fancyhf{}
\fancyhead[L]{\small Chapitre 03 - Régression}
\fancyhead[R]{\small Cours Machine Learning}
\fancyfoot[C]{\thepage}

% ===== CONFIGURATION LISTINGS =====
\definecolor{codegreen}{rgb}{0,0.6,0}
\definecolor{codegray}{rgb}{0.5,0.5,0.5}
\definecolor{codepurple}{rgb}{0.58,0,0.82}
\definecolor{backcolour}{rgb}{0.95,0.95,0.92}

\lstdefinestyle{pythonstyle}{
    language=Python,
    backgroundcolor=\color{backcolour},
    commentstyle=\color{codegreen},
    keywordstyle=\color{blue},
    numberstyle=\tiny\color{codegray},
    stringstyle=\color{codepurple},
    basicstyle=\ttfamily\small,
    breakatwhitespace=false,
    breaklines=true,
    captionpos=b,
    keepspaces=true,
    numbers=left,
    numbersep=5pt,
    showspaces=false,
    showstringspaces=false,
    showtabs=false,
    tabsize=4,
    frame=single,
    rulecolor=\color{black}
}
\lstset{style=pythonstyle}

% ===== CONFIGURATION TCOLORBOX =====


\newtcolorbox{theoreme}[1]{
    colback=green!5!white,
    colframe=green!75!black,
    fonttitle=\bfseries,
    title=Théorème: #1,
    breakable
}







% ===== COMMANDES PERSONNALISÉES =====
\newcommand{\vect}[1]{\mathbf{#1}}
\newcommand{\mat}[1]{\mathbf{#1}}
\newcommand{\R}{\mathbb{R}}
\newcommand{\N}{\mathbb{N}}
\newcommand{\E}{\mathbb{E}}
\newcommand{\Var}{\operatorname{Var}}
\newcommand{\argmin}{\operatorname{argmin}}
\newcommand{\argmax}{\operatorname{argmax}}

% ===== DÉBUT DU DOCUMENT =====
\begin{document}

% ===== PAGE DE TITRE =====
\begin{titlepage}
    \centering
    \vspace*{2cm}

    {\Huge\bfseries Cours Machine Learning}\\[0.5cm]

    \vspace{1cm}

    {\LARGE Chapitre 03}\\[0.3cm]
    {\LARGE\bfseries Régression}\\[2cm]

    \vfill

    {\large
    \textbf{Objectifs d'apprentissage :}\\[0.5cm]
    \begin{itemize}
        \item Maîtriser la régression linéaire (simple, multiple, polynomiale)
        \item Comprendre les techniques de régularisation (Ridge, Lasso, Elastic Net)
        \item Diagnostiquer et valider un modèle de régression
        \item Implémenter et optimiser des modèles de régression
    \end{itemize}
    }

    \vfill

    {\large
    \textbf{Prérequis :} Chapitres 00, 01, 02\\[0.3cm]
    \textbf{Durée estimée :} 5-6 heures\\[0.3cm]
    \textbf{Notebooks :} \texttt{03\_demo\_*.ipynb}, \texttt{03\_exercices.ipynb}
    }

    \vfill

    {\large Cours ML - Sandbox-ML\\
    Version 1.0 - 2026}
\end{titlepage}

% ===== TABLE DES MATIÈRES =====
\tableofcontents
\newpage

% ===== SECTION 1: INTRODUCTION =====
\section{Introduction à la Régression}

\subsection{Problématique}

\begin{definition}{Régression}
La régression est une tâche d'apprentissage supervisé où l'objectif est de prédire une variable cible \textbf{continue} $y \in \R$ à partir de features $\vect{x} \in \R^d$.
\end{definition}

\textbf{Formulation mathématique :}

Étant donné un dataset $\{(\vect{x}_i, y_i)\}_{i=1}^n$, on cherche une fonction $f: \R^d \to \R$ telle que :
\begin{equation}
    \hat{y}_i = f(\vect{x}_i) \approx y_i
\end{equation}

\textbf{Différence avec la classification :}
\begin{itemize}
    \item \textbf{Régression :} $y$ est continu (ex: prix, température, âge)
    \item \textbf{Classification :} $y$ est discret (ex: spam/non-spam, classe 0/1/2)
\end{itemize}

\begin{exemple}{Applications de la régression}
\begin{itemize}
    \item \textbf{Immobilier :} Prédire le prix d'une maison (surface, nb pièces, localisation)
    \item \textbf{Finance :} Prédire le cours d'une action, le risque de défaut
    \item \textbf{Santé :} Prédire la durée d'hospitalisation, la progression d'une maladie
    \item \textbf{Marketing :} Prédire les ventes futures, le chiffre d'affaires
    \item \textbf{Météo :} Prédire la température, les précipitations
\end{itemize}
\end{exemple}

\subsection{Fonction de Coût}

La fonction de coût (loss) mesure l'erreur de prédiction. Pour la régression, les plus courantes sont :

\textbf{1. Mean Squared Error (MSE) :}
\begin{equation}
    \text{MSE} = \frac{1}{n} \sum_{i=1}^n (y_i - \hat{y}_i)^2
\end{equation}

\textbf{2. Mean Absolute Error (MAE) :}
\begin{equation}
    \text{MAE} = \frac{1}{n} \sum_{i=1}^n |y_i - \hat{y}_i|
\end{equation}

\textbf{3. Root Mean Squared Error (RMSE) :}
\begin{equation}
    \text{RMSE} = \sqrt{\frac{1}{n} \sum_{i=1}^n (y_i - \hat{y}_i)^2}
\end{equation}

\begin{astuce}
\textbf{Choix de la métrique :}
\begin{itemize}
    \item \textbf{MSE/RMSE :} Plus sensibles aux outliers (erreurs au carré), différentiables
    \item \textbf{MAE :} Plus robuste aux outliers, moins sensible aux valeurs extrêmes
    \item En pratique, MSE est la plus utilisée car différentiable et optimisable analytiquement
\end{itemize}
\end{astuce}

% ===== SECTION 2: RÉGRESSION LINÉAIRE SIMPLE =====
\section{Régression Linéaire Simple}

\subsection{Modèle}

\begin{definition}{Régression linéaire simple}
Modèle avec une seule feature $x$ :
\begin{equation}
    \hat{y} = w_1 x + w_0
\end{equation}
où $w_1$ est la pente et $w_0$ l'intercept (ordonnée à l'origine).
\end{definition}

\begin{figure}[h]
\centering
\begin{tikzpicture}[scale=1.0]
    % Axes
    \draw[->] (0,0) -- (10,0) node[right] {$x$ (Feature)};
    \draw[->] (0,0) -- (0,6) node[above] {$y$ (Cible)};

    % Points de données (nuage avec tendance linéaire + bruit)
    \foreach \x/\y in {1/1.2, 1.5/1.8, 2/2.1, 2.5/2.3, 3/2.9, 3.5/3.2, 4/3.4, 4.5/3.9, 5/4.2, 5.5/4.5, 6/4.7, 6.5/5.1, 7/5.3, 7.5/5.4, 8/5.8, 8.5/5.7, 9/6.0}
        \filldraw[blue] (\x,\y) circle (2pt);

    % Droite de régression y = 0.55x + 0.6
    \draw[red, very thick] (0.5,0.875) -- (9.5,5.825);
    \node[red, above right] at (9.5, 5.825) {$\hat{y} = w_1 x + w_0$};

    % Résidus (quelques exemples)
    \draw[green!60!black, dashed] (2,2.1) -- (2,1.7);
    \draw[green!60!black, dashed] (4,3.4) -- (4,2.8);
    \draw[green!60!black, dashed] (6,4.7) -- (6,3.9);
    \draw[green!60!black, dashed] (8,5.8) -- (8,5.0);

    % Annotations des résidus
    \node[green!60!black, right, font=\small] at (2.2,1.9) {$e_i = y_i - \hat{y}_i$};

    % Annotation pente
    \draw[<->, thick] (7,4.5) -- (9,4.5) node[midway, below] {$\Delta x = 2$};
    \draw[<->, thick] (9,4.5) -- (9,5.55) node[midway, right] {$\Delta y = w_1 \cdot 2$};
    \node[draw, fill=yellow!10, rounded corners, align=center] at (7.5, 6.5) {
        Pente $w_1 = \frac{\Delta y}{\Delta x}$
    };

    % Annotation intercept
    \draw[<->, thick] (0,0) -- (0,0.6);
    \node[left, font=\small] at (0, 0.3) {$w_0$};
    \filldraw[red] (0,0.6) circle (2pt);

    % Légende
    \node[draw, fill=white, align=left, rounded corners] at (2.5, 5.5) {
        \textcolor{blue}{$\bullet$} Données observées\\
        \textcolor{red}{---} Droite de régression\\
        \textcolor{green!60!black}{- - -} Résidus
    };

\end{tikzpicture}
\caption{Régression linéaire simple: la droite $\hat{y} = w_1 x + w_0$ minimise la somme des carrés des résidus (distances verticales entre les points et la droite). $w_1$ est la pente, $w_0$ est l'ordonnée à l'origine.}
\label{fig:linear_regression}
\end{figure}

\textbf{Notation alternative :} $\hat{y} = \beta_1 x + \beta_0$ ou $\hat{y} = ax + b$

\subsection{Solution Analytique}

Pour minimiser le MSE, on résout :
\begin{equation}
    \min_{w_0, w_1} \sum_{i=1}^n (y_i - w_1 x_i - w_0)^2
\end{equation}

En dérivant par rapport à $w_0$ et $w_1$ et en annulant, on obtient les \textbf{équations normales} :

\begin{align}
    w_1 &= \frac{\sum_{i=1}^n (x_i - \bar{x})(y_i - \bar{y})}{\sum_{i=1}^n (x_i - \bar{x})^2} = \frac{\text{Cov}(x, y)}{\Var(x)} \\
    w_0 &= \bar{y} - w_1 \bar{x}
\end{align}

où $\bar{x}$ et $\bar{y}$ sont les moyennes de $x$ et $y$.

\subsection{Coefficient de Détermination $R^2$}

\begin{definition}{Coefficient $R^2$}
Le coefficient de détermination mesure la proportion de variance expliquée par le modèle :
\begin{equation}
    R^2 = 1 - \frac{\sum_{i=1}^n (y_i - \hat{y}_i)^2}{\sum_{i=1}^n (y_i - \bar{y})^2} = 1 - \frac{\text{SS}_{\text{res}}}{\text{SS}_{\text{tot}}}
\end{equation}
\end{definition}

\textbf{Interprétation :}
\begin{itemize}
    \item $R^2 = 1$ : Modèle parfait (toutes les prédictions exactes)
    \item $R^2 = 0$ : Modèle aussi bon qu'une prédiction constante ($\hat{y} = \bar{y}$)
    \item $R^2 < 0$ : Modèle pire que la moyenne (rare, indique un problème)
\end{itemize}

\textbf{Relation avec la corrélation :} Pour la régression simple, $R^2 = \rho^2$ où $\rho$ est le coefficient de corrélation de Pearson.

% ===== SECTION 3: RÉGRESSION LINÉAIRE MULTIPLE =====
\section{Régression Linéaire Multiple}

\subsection{Modèle Vectoriel}

\begin{definition}{Régression linéaire multiple}
Modèle avec $d$ features :
\begin{equation}
    \hat{y} = w_1 x_1 + w_2 x_2 + \cdots + w_d x_d + w_0 = \vect{w}^T \vect{x} + w_0
\end{equation}
\end{definition}

En notation matricielle, pour $n$ instances :
\begin{equation}
    \vect{\hat{y}} = \mat{X} \vect{w}
\end{equation}
où $\mat{X} \in \R^{n \times (d+1)}$ inclut une colonne de 1 pour l'intercept, et $\vect{w} \in \R^{d+1}$ contient tous les poids.

\subsection{Solution par Moindres Carrés}

\textbf{Problème d'optimisation :}
\begin{equation}
    \vect{w}^* = \argmin_{\vect{w}} \|\mat{X}\vect{w} - \vect{y}\|_2^2
\end{equation}

\begin{theoreme}{Équations normales}
La solution optimale des moindres carrés est :
\begin{equation}
    \vect{w}^* = (\mat{X}^T\mat{X})^{-1}\mat{X}^T\vect{y}
\end{equation}
sous réserve que $\mat{X}^T\mat{X}$ soit inversible (rang plein).
\end{theoreme}

\textbf{Démonstration (esquisse) :}
\begin{align}
    \text{Minimiser } & f(\vect{w}) = \|\mat{X}\vect{w} - \vect{y}\|_2^2 \\
    \nabla_{\vect{w}} f &= 2\mat{X}^T(\mat{X}\vect{w} - \vect{y}) = \vect{0} \\
    \mat{X}^T\mat{X}\vect{w} &= \mat{X}^T\vect{y} \\
    \vect{w}^* &= (\mat{X}^T\mat{X})^{-1}\mat{X}^T\vect{y}
\end{align}

\begin{attention}
\textbf{Problèmes numériques :}
\begin{itemize}
    \item Si $\mat{X}^T\mat{X}$ est singulière ou mal conditionnée, l'inversion est instable
    \item Causes : multicollinéarité (features corrélées), $n < d$ (plus de features que d'instances)
    \item Solutions : régularisation (Ridge, Lasso), sélection de features, PCA
\end{itemize}
\end{attention}

\subsection{Implémentation}

\begin{lstlisting}[language=Python, caption=Régression linéaire avec scikit-learn]
from sklearn.linear_model import LinearRegression
from sklearn.model_selection import train_test_split
from sklearn.metrics import mean_squared_error, r2_score
import numpy as np

# Données
X = np.random.randn(100, 5)  # 100 instances, 5 features
y = 3*X[:, 0] + 2*X[:, 1] - X[:, 2] + np.random.randn(100)*0.5

# Split train/test
X_train, X_test, y_train, y_test = train_test_split(
    X, y, test_size=0.2, random_state=42
)

# Entraînement
model = LinearRegression()
model.fit(X_train, y_train)

# Prédiction
y_pred = model.predict(X_test)

# Évaluation
mse = mean_squared_error(y_test, y_pred)
r2 = r2_score(y_test, y_pred)

print(f"MSE: {mse:.4f}")
print(f"R²: {r2:.4f}")
print(f"Coefficients: {model.coef_}")
print(f"Intercept: {model.intercept_:.4f}")
\end{lstlisting}

% ===== SECTION 4: RÉGRESSION POLYNOMIALE =====
\section{Régression Polynomiale}

\subsection{Motivation}

La régression linéaire suppose une relation linéaire entre features et cible. Si la relation est non-linéaire, on peut utiliser des features polynomiales.

\begin{definition}{Régression polynomiale}
Pour une feature $x$, on crée des features polynomiales :
\begin{equation}
    \hat{y} = w_0 + w_1 x + w_2 x^2 + w_3 x^3 + \cdots + w_p x^p
\end{equation}
Le modèle reste linéaire en les poids $\vect{w}$, mais non-linéaire en $x$.
\end{definition}

\textbf{Généralisation à plusieurs features :}
Pour $d$ features, on peut créer tous les termes de degré $\leq p$ :
\begin{equation}
    \vect{\phi}(\vect{x}) = [1, x_1, x_2, x_1^2, x_1 x_2, x_2^2, \ldots]
\end{equation}

Le nombre de features polynomiales croît combinatoirement : $\binom{d+p}{p}$.

\subsection{Exemple}

Données 1D non-linéaires : $y = \sin(2\pi x) + \epsilon$

\begin{itemize}
    \item \textbf{Degré 1 (linéaire)} : sous-ajustement (underfitting)
    \item \textbf{Degré 3-5} : bon ajustement
    \item \textbf{Degré 15} : sur-ajustement (overfitting)
\end{itemize}

\begin{figure}[h]
\centering
\begin{tikzpicture}[scale=0.85]
    % Degré 1 - Underfitting
    \begin{scope}[xshift=0cm]
        \node at (2.5, 4.2) {\textbf{Degré 1 (Underfitting)}};
        \draw[->] (0,0) -- (5,0) node[right] {\small $x$};
        \draw[->] (0,0) -- (0,4) node[above] {\small $y$};

        % Vraie fonction (sinusoïde approximée)
        \draw[blue!30, dashed, thick, smooth, domain=0.2:4.8, samples=50]
            plot (\x, {2 + 1.5*sin(360*\x/2.5)});

        % Points
        \foreach \x/\y in {0.5/2.7, 1.0/3.2, 1.5/2.9, 2.0/1.8, 2.5/0.8, 3.0/1.2, 3.5/2.1, 4.0/3.0, 4.5/2.8}
            \filldraw[blue] (\x,\y) circle (1.5pt);

        % Modèle linéaire (trop simple)
        \draw[red, very thick] (0.5,2.8) -- (4.5,2.2);

        \node[align=center, font=\small] at (2.5, -0.7) {Trop simple\\Ne capture pas la courbe};
    \end{scope}

    % Degré 4 - Bon fit
    \begin{scope}[xshift=6cm]
        \node at (2.5, 4.2) {\textbf{Degré 4 (Bon Équilibre)}};
        \draw[->] (0,0) -- (5,0) node[right] {\small $x$};
        \draw[->] (0,0) -- (0,4) node[above] {\small $y$};

        % Vraie fonction
        \draw[blue!30, dashed, thick, smooth, domain=0.2:4.8, samples=50]
            plot (\x, {2 + 1.5*sin(360*\x/2.5)});

        % Points
        \foreach \x/\y in {0.5/2.7, 1.0/3.2, 1.5/2.9, 2.0/1.8, 2.5/0.8, 3.0/1.2, 3.5/2.1, 4.0/3.0, 4.5/2.8}
            \filldraw[blue] (\x,\y) circle (1.5pt);

        % Modèle polynomial degré 4 (suit bien la courbe)
        \draw[green!60!black, very thick, smooth, domain=0.5:4.5, samples=60]
            plot (\x, {2 + 1.4*sin(360*\x/2.5) + 0.1*(\x-2.5)*(\x-2.5)});

        \node[align=center, font=\small] at (2.5, -0.7) {Capture le pattern\\Généralise bien};
    \end{scope}

    % Degré 15 - Overfitting
    \begin{scope}[xshift=12cm]
        \node at (2.5, 4.2) {\textbf{Degré 15 (Overfitting)}};
        \draw[->] (0,0) -- (5,0) node[right] {\small $x$};
        \draw[->] (0,0) -- (0,4) node[above] {\small $y$};

        % Vraie fonction
        \draw[blue!30, dashed, thick, smooth, domain=0.2:4.8, samples=50]
            plot (\x, {2 + 1.5*sin(360*\x/2.5)});

        % Points
        \foreach \x/\y in {0.5/2.7, 1.0/3.2, 1.5/2.9, 2.0/1.8, 2.5/0.8, 3.0/1.2, 3.5/2.1, 4.0/3.0, 4.5/2.8}
            \filldraw[blue] (\x,\y) circle (1.5pt);

        % Modèle polynomial degré 15 (oscille entre les points)
        \draw[orange, very thick, smooth, domain=0.5:4.5, samples=80]
            plot (\x, {2 + 1.5*sin(360*\x/2.5) + 0.5*sin(720*\x) + 0.3*cos(540*\x)});

        \node[align=center, font=\small] at (2.5, -0.7) {M\'{e}morise le bruit\\Ne g\'{e}n\'{e}ralise pas};
    \end{scope}

    % Légende commune
    \node[draw, fill=white, align=left, rounded corners] at (9, -2) {
        \textcolor{blue!30}{- - -} Vraie fonction $f(x)$\\
        \textcolor{blue}{$\bullet$} Données observées\\
        \textcolor{red}{---} / \textcolor{green!60!black}{---} / \textcolor{orange}{---} Modèles ajustés
    };

\end{tikzpicture}
\caption{Régression polynomiale avec différents degrés: degré trop faible (underfitting), degré approprié (bon équilibre), et degré trop élevé (overfitting mémorisant le bruit).}
\label{fig:polynomial_regression}
\end{figure}

\begin{lstlisting}[language=Python, caption=Régression polynomiale]
from sklearn.preprocessing import PolynomialFeatures
from sklearn.pipeline import make_pipeline

# Créer un pipeline : features polynomiales + régression
degree = 3
model = make_pipeline(
    PolynomialFeatures(degree=degree),
    LinearRegression()
)

model.fit(X_train, y_train)
y_pred = model.predict(X_test)
\end{lstlisting}

\begin{attention}
\textbf{Risque d'overfitting :}
\begin{itemize}
    \item Augmenter le degré $p$ améliore le fit sur le train set
    \item Mais peut dégrader la généralisation (test set)
    \item Solution : validation croisée pour choisir $p$, régularisation
\end{itemize}
\end{attention}

% ===== SECTION 5: RÉGULARISATION =====
\section{Régularisation}

La régularisation pénalise la complexité du modèle pour éviter l'overfitting.

\subsection{Ridge Regression ($L^2$)}

\begin{definition}{Ridge (Régression $L^2$)}
Ajoute une pénalité sur la norme $L^2$ des poids :
\begin{equation}
    \vect{w}^* = \argmin_{\vect{w}} \left\{ \|\mat{X}\vect{w} - \vect{y}\|_2^2 + \lambda \|\vect{w}\|_2^2 \right\}
\end{equation}
où $\lambda \geq 0$ est le coefficient de régularisation.
\end{definition}

\textbf{Solution analytique :}
\begin{equation}
    \vect{w}^*_{\text{Ridge}} = (\mat{X}^T\mat{X} + \lambda \mat{I})^{-1}\mat{X}^T\vect{y}
\end{equation}

\textbf{Propriétés :}
\begin{itemize}
    \item Pénalise les poids de grande magnitude
    \item Réduit l'overfitting
    \item Solution toujours unique (même si $\mat{X}^T\mat{X}$ singulière)
    \item Les poids sont \textbf{rétrécis} (shrinkage) mais rarement nuls
\end{itemize}

\subsection{Lasso Regression ($L^1$)}

\begin{definition}{Lasso (Régression $L^1$)}
Ajoute une pénalité sur la norme $L^1$ des poids :
\begin{equation}
    \vect{w}^* = \argmin_{\vect{w}} \left\{ \|\mat{X}\vect{w} - \vect{y}\|_2^2 + \lambda \|\vect{w}\|_1 \right\}
\end{equation}
\end{definition}

\textbf{Propriétés :}
\begin{itemize}
    \item Pénalise les poids en valeur absolue
    \item Effectue une \textbf{sélection de features automatique} : certains poids deviennent exactement 0
    \item Pas de solution analytique simple, résolu par optimisation (coordinate descent, proximal gradient)
    \item Utile quand on suspecte que peu de features sont pertinentes
\end{itemize}

\subsection{Elastic Net}

\begin{definition}{Elastic Net}
Combinaison de Ridge et Lasso :
\begin{equation}
    \vect{w}^* = \argmin_{\vect{w}} \left\{ \|\mat{X}\vect{w} - \vect{y}\|_2^2 + \lambda_1 \|\vect{w}\|_1 + \lambda_2 \|\vect{w}\|_2^2 \right\}
\end{equation}
ou avec un ratio $\rho \in [0, 1]$ :
\begin{equation}
    \text{Pénalité} = \lambda \left[\rho \|\vect{w}\|_1 + \frac{1-\rho}{2} \|\vect{w}\|_2^2\right]
\end{equation}
\end{definition}

\textbf{Avantages :}
\begin{itemize}
    \item Combine les avantages de Ridge et Lasso
    \item Sélection de features (comme Lasso)
    \item Stabilité quand features corrélées (comme Ridge)
\end{itemize}

\subsection{Comparaison Ridge vs Lasso}

\begin{table}[h]
\centering
\caption{Ridge vs Lasso}
\begin{tabular}{lll}
\toprule
\textbf{Aspect} & \textbf{Ridge ($L^2$)} & \textbf{Lasso ($L^1$)} \\
\midrule
Pénalité & $\lambda \sum w_i^2$ & $\lambda \sum |w_i|$ \\
Solution analytique & Oui & Non \\
Sélection de features & Non & Oui \\
Poids à zéro & Jamais exactement & Oui \\
Features corrélées & Stable & Instable (choix arbitraire) \\
Interprétabilité & Moyenne & Élevée (sparse) \\
\bottomrule
\end{tabular}
\end{table}

\begin{figure}[h]
\centering
\begin{tikzpicture}[scale=1.0]
    % Ridge (L2) - gauche
    \begin{scope}[xshift=0cm]
        \node at (3, 4.8) {\textbf{Ridge (Régularisation $L^2$)}};

        % Axes
        \draw[->] (-1,0) -- (4,0) node[right] {$w_1$};
        \draw[->] (0,-1) -- (0,4) node[above] {$w_2$};

        % Contrainte L2 (cercle)
        \draw[blue!60!black, very thick] (1.5,1.5) circle (1.5);
        \node[blue!60!black, below right] at (3,1.5) {$w_1^2 + w_2^2 \leq t$};

        % Contours de la fonction de coût (ellipses)
        \draw[red!60!black, thick] (2.5,2.2) ellipse (1.2 and 0.8);
        \draw[red!60!black, thick] (2.5,2.2) ellipse (1.8 and 1.2);
        \draw[red!60!black, thick] (2.5,2.2) ellipse (2.4 and 1.6);

        % Optimum sans contrainte
        \filldraw[red!60!black] (2.5,2.2) circle (2pt);
        \node[red!60!black, above right] at (2.5,2.2) {$\hat{w}_{OLS}$};

        % Optimum avec Ridge (contact tangent cercle-ellipse)
        \filldraw[green!60!black] (2.1,1.05) circle (3pt);
        \node[green!60!black, below left] at (2.1,1.05) {$\hat{w}_{Ridge}$};

        % Annotation
        \node[align=center, font=\small] at (1.5, -1.5) {Contrainte circulaire\\Poids non nuls};
    \end{scope}

    % Lasso (L1) - droite
    \begin{scope}[xshift=8cm]
        \node at (3, 4.8) {\textbf{Lasso (Régularisation $L^1$)}};

        % Axes
        \draw[->] (-1,0) -- (4,0) node[right] {$w_1$};
        \draw[->] (0,-1) -- (0,4) node[above] {$w_2$};

        % Contrainte L1 (losange)
        \draw[blue!60!black, very thick] (0,2.1) -- (2.1,0) -- (0,-0.1) -- (-0.1,0) -- cycle;
        \draw[blue!60!black, very thick] (0,2.1) -- (2.1,0);
        \node[blue!60!black, below right] at (2.5,0.5) {$|w_1| + |w_2| \leq t$};

        % Contours de la fonction de coût (ellipses)
        \draw[red!60!black, thick] (2.5,2.2) ellipse (1.2 and 0.8);
        \draw[red!60!black, thick] (2.5,2.2) ellipse (1.8 and 1.2);
        \draw[red!60!black, thick] (2.5,2.2) ellipse (2.4 and 1.6);

        % Optimum sans contrainte
        \filldraw[red!60!black] (2.5,2.2) circle (2pt);
        \node[red!60!black, above right] at (2.5,2.2) {$\hat{w}_{OLS}$};

        % Optimum avec Lasso (contact au coin, w1=0)
        \filldraw[green!60!black] (0,1.55) circle (3pt);
        \node[green!60!black, left] at (0,1.55) {$\hat{w}_{Lasso}$};
        \node[green!60!black, left, font=\small] at (-0.3,0.8) {$w_1 = 0$};

        % Annotation
        \node[align=center, font=\small] at (1.5, -1.5) {Contrainte losange\\S\'{e}lection: $w_1=0$};
    \end{scope}

    % Légende
    \node[draw, fill=white, align=left, rounded corners] at (4, -3) {
        \textcolor{blue!60!black}{---} R\'{e}gion de contrainte ($\|\vect{w}\| \leq t$)\\
        \textcolor{red!60!black}{$\circ$} Contours de la fonction objectif (MSE)\\
        \textcolor{green!60!black}{$\bullet$} Solution optimale (minimise MSE + contrainte)
    };

\end{tikzpicture}
\caption{Géométrie de Ridge vs Lasso: la contrainte $L^2$ (cercle) réduit les poids mais ne les annule jamais exactement, tandis que la contrainte $L^1$ (losange) favorise les solutions aux coins, donc des poids à zéro (sélection de features).}
\label{fig:ridge_lasso_geometry}
\end{figure}

\begin{astuce}
\textbf{Quand utiliser quoi ?}
\begin{itemize}
    \item \textbf{Ridge :} Toutes les features potentiellement utiles, features corrélées
    \item \textbf{Lasso :} Peu de features importantes, besoin d'interprétabilité
    \item \textbf{Elastic Net :} Compromis, features corrélées + sélection
\end{itemize}
\end{astuce}

\subsection{Choix de $\lambda$ (Hyperparamètre)}

Le coefficient de régularisation $\lambda$ contrôle le trade-off biais-variance :
\begin{itemize}
    \item $\lambda = 0$ : Pas de régularisation (risque d'overfitting)
    \item $\lambda$ petit : Régularisation faible
    \item $\lambda$ grand : Régularisation forte (risque d'underfitting)
\end{itemize}

\textbf{Méthode de sélection :} Validation croisée

\begin{lstlisting}[language=Python, caption=Ridge avec validation croisée]
from sklearn.linear_model import RidgeCV

# Tester plusieurs valeurs de lambda (alpha en scikit-learn)
alphas = np.logspace(-3, 3, 50)
model = RidgeCV(alphas=alphas, cv=5)
model.fit(X_train, y_train)

print(f"Meilleur alpha: {model.alpha_:.4f}")
\end{lstlisting}

% ===== SECTION 6: DIAGNOSTIC =====
\section{Diagnostic et Analyse des Résidus}

\subsection{Résidus}

\begin{definition}{Résidus}
Les résidus sont les erreurs de prédiction :
\begin{equation}
    e_i = y_i - \hat{y}_i
\end{equation}
\end{definition}

\textbf{Hypothèses de la régression linéaire :}
\begin{enumerate}
    \item \textbf{Linéarité :} Relation linéaire entre $X$ et $y$
    \item \textbf{Indépendance :} Les résidus sont indépendants
    \item \textbf{Homoscédasticité :} Variance constante des résidus
    \item \textbf{Normalité :} Les résidus suivent une loi normale
    \item \textbf{Absence de multicollinéarité :} Features non (trop) corrélées
\end{enumerate}

\subsection{Graphiques de Diagnostic}

\textbf{1. Résidus vs Prédictions :}
\begin{itemize}
    \item Vérifier l'homoscédasticité
    \item Pattern : résidus dispersés aléatoirement autour de 0
    \item Problème : tendance, forme en entonnoir (hétéroscédasticité)
\end{itemize}

\textbf{2. Q-Q Plot :}
\begin{itemize}
    \item Vérifier la normalité des résidus
    \item Points alignés sur la diagonale = normalité
\end{itemize}

\textbf{3. Résidus vs Features :}
\begin{itemize}
    \item Détecter des relations non-linéaires manquées
\end{itemize}

\subsection{Multicollinéarité}

\begin{definition}{Multicollinéarité}
Corrélation forte entre plusieurs features. Rend l'estimation des coefficients instable.
\end{definition}

\textbf{Détection :}
\begin{itemize}
    \item Matrice de corrélation : $|\rho_{ij}| > 0.8$ problématique
    \item VIF (Variance Inflation Factor) : VIF $> 10$ indique multicollinéarité
\end{itemize}

\textbf{Solutions :}
\begin{itemize}
    \item Supprimer une des features corrélées
    \item PCA (réduction de dimensionnalité)
    \item Régularisation (Ridge)
\end{itemize}

% ===== SECTION 7: VALIDATION =====
\section{Validation et Sélection de Modèle}

\subsection{Train/Validation/Test Split}

\textbf{Procédure standard :}
\begin{enumerate}
    \item \textbf{Train set (70\%)} : Entraîner le modèle
    \item \textbf{Validation set (15\%)} : Tuner les hyperparamètres
    \item \textbf{Test set (15\%)} : Évaluation finale (une seule fois)
\end{enumerate}

\subsection{Validation Croisée (Cross-Validation)}

\begin{definition}{K-Fold Cross-Validation}
Diviser les données en $K$ folds. Pour chaque fold :
\begin{itemize}
    \item Entraîner sur $K-1$ folds
    \item Valider sur le fold restant
    \item Répéter $K$ fois, moyenner les scores
\end{itemize}
\end{definition}

\begin{lstlisting}[language=Python, caption=Validation croisée]
from sklearn.model_selection import cross_val_score

model = Ridge(alpha=1.0)
scores = cross_val_score(model, X, y, cv=5,
                          scoring='neg_mean_squared_error')
rmse_scores = np.sqrt(-scores)

print(f"RMSE: {rmse_scores.mean():.4f} (+/- {rmse_scores.std():.4f})")
\end{lstlisting}

\textbf{Avantages :}
\begin{itemize}
    \item Utilise toutes les données pour train et validation
    \item Estimation plus robuste de la performance
    \item Réduit le risque de chance (lucky/unlucky split)
\end{itemize}

\subsection{Courbe d'Apprentissage}

Tracer la performance (MSE, $R^2$) en fonction de la taille du train set.

\textbf{Diagnostic :}
\begin{itemize}
    \item \textbf{Underfitting :} Train et validation scores bas et proches
    \item \textbf{Overfitting :} Grand écart entre train et validation scores
    \item \textbf{Bon fit :} Scores convergent vers une valeur élevée
\end{itemize}

% ===== SECTION 8: EXTENSIONS =====
\section{Extensions et Variantes}

\subsection{Régression Robuste}

Pour gérer les outliers :
\begin{itemize}
    \item \textbf{Huber Regression :} Loss hybride (MSE + MAE)
    \item \textbf{RANSAC :} Fit sur inliers, ignore outliers
\end{itemize}

\subsection{Régression Non-Linéaire}

Au-delà des polynômes :
\begin{itemize}
    \item \textbf{Kernel Ridge Regression :} Kernels (RBF, polynomial)
    \item \textbf{Support Vector Regression (SVR)} : SVM pour la régression
    \item \textbf{Decision Trees, Random Forests :} Modèles non-linéaires
    \item \textbf{Gradient Boosting (XGBoost, LightGBM)} : État de l'art pour données tabulaires
    \item \textbf{Réseaux de neurones :} Deep Learning pour régression
\end{itemize}

\subsection{Régression Généralisée (GLM)}

Extension de la régression linéaire pour distributions non-gaussiennes :
\begin{itemize}
    \item Régression logistique (Bernoulli)
    \item Régression de Poisson
    \item Régression Gamma
\end{itemize}

% ===== RÉSUMÉ =====
\section{Résumé du Chapitre}

\subsection{Points Clés}

\begin{itemize}
    \item \textbf{Régression linéaire :} Modèle simple, interprétable, solution analytique
    \item \textbf{Équations normales :} $\vect{w}^* = (\mat{X}^T\mat{X})^{-1}\mat{X}^T\vect{y}$
    \item \textbf{$R^2$ :} Mesure de qualité du fit (0 à 1)
    \item \textbf{Régression polynomiale :} Capture des relations non-linéaires
    \item \textbf{Ridge ($L^2$) :} Régularisation, stabilité
    \item \textbf{Lasso ($L^1$) :} Sélection de features, sparsité
    \item \textbf{Diagnostic :} Analyse des résidus, détection de problèmes
    \item \textbf{Validation croisée :} Sélection d'hyperparamètres robuste
\end{itemize}

\subsection{Formules Essentielles}

\begin{tcolorbox}[colback=blue!5!white, colframe=blue!75!black, title=Formules à retenir]
\textbf{Régression linéaire simple :}
\begin{equation}
    w_1 = \frac{\text{Cov}(x, y)}{\Var(x)}, \quad w_0 = \bar{y} - w_1 \bar{x}
\end{equation}

\textbf{Régression multiple (équations normales) :}
\begin{equation}
    \vect{w}^* = (\mat{X}^T\mat{X})^{-1}\mat{X}^T\vect{y}
\end{equation}

\textbf{Ridge :}
\begin{equation}
    \vect{w}^* = (\mat{X}^T\mat{X} + \lambda\mat{I})^{-1}\mat{X}^T\vect{y}
\end{equation}

\textbf{Coefficient $R^2$ :}
\begin{equation}
    R^2 = 1 - \frac{\sum (y_i - \hat{y}_i)^2}{\sum (y_i - \bar{y})^2}
\end{equation}
\end{tcolorbox}

% ===== EXERCICES =====
\section{Exercices}

\subsection{Questions de Compréhension}

\begin{enumerate}
    \item Quelle est la différence entre régression et classification ?

    \item Pourquoi utilise-t-on le MSE plutôt que le MAE en régression linéaire ?

    \item Expliquez pourquoi la régression polynomiale de degré élevé peut causer de l'overfitting.

    \item Quelle est la différence principale entre Ridge et Lasso ?

    \item Comment détecte-t-on la multicollinéarité ? Quelles sont les solutions ?

    \item Pourquoi la validation croisée est-elle préférable à un simple train/test split ?
\end{enumerate}

\subsection{Exercices Pratiques}

\textit{Voir le notebook} \texttt{03\_exercices.ipynb} \textit{(solutions intégrées)}

% ===== POUR ALLER PLUS LOIN =====
\section{Pour Aller Plus Loin}

\subsection{Lectures Recommandées}

\begin{itemize}
    \item \textit{An Introduction to Statistical Learning} (2e éd., 2021) - James et al. (Ch. 3)
    \item \textit{The Elements of Statistical Learning} (2009) - Hastie et al. (Ch. 3)
    \item Scikit-learn documentation : Linear Models
\end{itemize}

\subsection{Prochaines Étapes}

Chapitre suivant : \textbf{Chapitre 04 - Classification Supervisée}

% ===== BIBLIOGRAPHIE =====
\section*{Références}

\begin{enumerate}
    \item Hastie, T., Tibshirani, R., \& Friedman, J. (2009). \textit{The Elements of Statistical Learning}. Springer.

    \item James, G., Witten, D., Hastie, T., \& Tibshirani, R. (2021). \textit{An Introduction to Statistical Learning} (2e éd.). Springer.

    \item Géron, A. (2022). \textit{Hands-On Machine Learning} (3e éd.). O'Reilly.
\end{enumerate}

\end{document}
