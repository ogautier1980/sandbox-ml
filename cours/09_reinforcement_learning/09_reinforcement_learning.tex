\documentclass[11pt,a4paper]{article}

% ===== PACKAGES =====
\usepackage[utf8]{inputenc}
\usepackage[T1]{fontenc}
\usepackage[french]{babel}
\usepackage{lmodern}
\usepackage{amsmath, amssymb, amsthm}
\usepackage{mathtools}
\usepackage[margin=2.5cm]{geometry}
\usepackage{parskip}
\usepackage{setspace}
\setstretch{1.15}
\usepackage{graphicx}
\usepackage{xcolor}
\usepackage{tikz}
\usetikzlibrary{arrows.meta, positioning, shapes.geometric}
\usepackage{booktabs}
\usepackage{listings}
\usepackage{algorithm}
\usepackage{algorithmic}
\usepackage{hyperref}
\hypersetup{
    colorlinks=true,
    linkcolor=blue,
    pdftitle={Chapitre 09 - Reinforcement Learning},
}
\usepackage{tcolorbox}
\tcbuselibrary{skins, breakable}
\usepackage{fancyhdr}
\pagestyle{fancy}
\fancyhf{}
\fancyhead[L]{\small Chapitre 09 - Reinforcement Learning}
\fancyhead[R]{\small Cours Machine Learning}
\fancyfoot[C]{\thepage}

\definecolor{codegreen}{rgb}{0,0.6,0}
\definecolor{codegray}{rgb}{0.5,0.5,0.5}
\definecolor{codepurple}{rgb}{0.58,0,0.82}
\definecolor{backcolour}{rgb}{0.95,0.95,0.92}

\lstdefinestyle{pythonstyle}{
    language=Python,
    backgroundcolor=\color{backcolour},
    commentstyle=\color{codegreen},
    keywordstyle=\color{blue},
    basicstyle=\ttfamily\small,
    breaklines=true,
    numbers=left,
    frame=single,
}
\lstset{style=pythonstyle}

\newtcolorbox{definition}[1]{
    colback=blue!5!white,
    colframe=blue!75!black,
    fonttitle=\bfseries,
    title=Définition: #1,
    breakable
}

\newtcolorbox{exemple}[1]{
    colback=orange!5!white,
    colframe=orange!75!black,
    fonttitle=\bfseries,
    title=Exemple: #1,
    breakable
}

\newtcolorbox{attention}{
    colback=red!5!white,
    colframe=red!75!black,
    fonttitle=\bfseries,
    title=⚠️ Attention,
    breakable
}

\newtcolorbox{astuce}{
    colback=yellow!10!white,
    colframe=yellow!75!black,
    fonttitle=\bfseries,
    title=💡 Astuce,
    breakable
}

\newcommand{\vect}[1]{\mathbf{#1}}
\newcommand{\mat}[1]{\mathbf{#1}}
\newcommand{\R}{\mathbb{R}}
\newcommand{\E}{\mathbb{E}}
\newcommand{\argmax}{\operatorname{argmax}}

\begin{document}

\begin{titlepage}
    \centering
    \vspace*{2cm}
    {\Huge\bfseries Cours Machine Learning}\\[0.5cm]
    \vspace{1cm}
    {\LARGE Chapitre 09}\\[0.3cm]
    {\LARGE\bfseries Reinforcement Learning}\\[2cm]
    \vfill
    {\large
    \textbf{Objectifs d'apprentissage :}\\[0.5cm]
    \begin{itemize}
        \item Comprendre le paradigme du Reinforcement Learning
        \item Maîtriser Q-Learning et Deep Q-Network (DQN)
        \item Découvrir Policy Gradient et Actor-Critic
        \item Appliquer le RL à des environnements pratiques
    \end{itemize}
    }
    \vfill
    {\large
    \textbf{Prérequis :} Chapitres 06 (MLP)\\[0.3cm]
    \textbf{Durée estimée :} 6-8 heures
    }
    \vfill
    {\large Cours ML - Sandbox-ML\\ Version 1.0 - 2026}
\end{titlepage}

\tableofcontents
\newpage

\section{Introduction au Reinforcement Learning}

\subsection{Paradigme RL vs Supervised Learning}

\begin{table}[h]
\centering
\begin{tabular}{lll}
\toprule
 & \textbf{Supervised Learning} & \textbf{Reinforcement Learning} \\
\midrule
Données & Labels explicites & Récompenses différées \\
Feedback & Immédiat & Delayed \\
Objectif & Prédire $y$ depuis $x$ & Maximiser récompense cumulative \\
Exemples & Classification, Régression & Jeux, Robotique \\
\bottomrule
\end{tabular}
\end{table}

\subsection{Composants du RL}

\begin{definition}{Markov Decision Process (MDP)}
Un MDP est défini par $(S, A, P, R, \gamma)$ :
\begin{itemize}
    \item $S$ : États (states)
    \item $A$ : Actions
    \item $P(s' | s, a)$ : Probabilités de transition
    \item $R(s, a, s')$ : Fonction de récompense
    \item $\gamma \in [0, 1]$ : Facteur de discount
\end{itemize}
\end{definition}

\textbf{Objectif :} Apprendre une \textbf{policy} $\pi(a|s)$ qui maximise la récompense cumulative :
\begin{equation}
    G_t = \sum_{k=0}^{\infty} \gamma^k R_{t+k+1}
\end{equation}

\section{Q-Learning}

\subsection{Q-Function}

\begin{definition}{Q-Function}
La Q-function représente la récompense cumulative attendue en prenant l'action $a$ dans l'état $s$ puis en suivant la policy $\pi$ :
\begin{equation}
    Q^\pi(s, a) = \E_\pi \left[ \sum_{k=0}^{\infty} \gamma^k R_{t+k+1} \mid S_t = s, A_t = a \right]
\end{equation}
\end{definition}

\textbf{Équation de Bellman :}
\begin{equation}
    Q(s, a) = R(s, a) + \gamma \max_{a'} Q(s', a')
\end{equation}

\subsection{Algorithme Q-Learning}

\begin{algorithm}[H]
\caption{Q-Learning}
\begin{algorithmic}[1]
\REQUIRE Environnement, learning rate $\alpha$, discount $\gamma$, exploration $\epsilon$
\STATE Initialiser $Q(s, a) = 0$ pour tout $s, a$
\FOR{chaque episode}
    \STATE Initialiser état $s$
    \REPEAT
        \STATE Choisir action $a$ : $\epsilon$-greedy sur $Q(s, \cdot)$
        \STATE Exécuter $a$, observer $r, s'$
        \STATE $Q(s, a) \leftarrow Q(s, a) + \alpha [r + \gamma \max_{a'} Q(s', a') - Q(s, a)]$
        \STATE $s \leftarrow s'$
    \UNTIL{$s$ terminal}
\ENDFOR
\end{algorithmic}
\end{algorithm}

\subsection{Implémentation}

\begin{lstlisting}[caption=Q-Learning simple]
import numpy as np

class QLearning:
    def __init__(self, n_states, n_actions, alpha=0.1, gamma=0.99, epsilon=0.1):
        self.Q = np.zeros((n_states, n_actions))
        self.alpha = alpha
        self.gamma = gamma
        self.epsilon = epsilon

    def choose_action(self, state):
        if np.random.rand() < self.epsilon:
            return np.random.randint(self.Q.shape[1])  # Explore
        return np.argmax(self.Q[state])  # Exploit

    def update(self, state, action, reward, next_state):
        target = reward + self.gamma * np.max(self.Q[next_state])
        self.Q[state, action] += self.alpha * (target - self.Q[state, action])
\end{lstlisting}

\section{Deep Q-Network (DQN)}

\subsection{Motivation}

Pour des espaces d'états larges/continus (images), stocker $Q(s, a)$ dans une table devient impossible. Solution : approximer $Q$ avec un réseau de neurones.

\subsection{Innovations DQN}

\begin{enumerate}
    \item \textbf{Experience Replay} : Stocker transitions dans buffer, échantillonner mini-batches
    \item \textbf{Target Network} : Réseau cible gelé pour stabiliser l'entraînement
\end{enumerate}

\subsection{Algorithme}

\begin{algorithm}[H]
\caption{Deep Q-Network (DQN)}
\begin{algorithmic}[1]
\STATE Initialiser réseau $Q(s, a; \theta)$ et réseau cible $Q(s, a; \theta^-)$
\STATE Initialiser replay buffer $D$
\FOR{chaque episode}
    \STATE Initialiser état $s_1$
    \FOR{$t = 1, \ldots, T$}
        \STATE $a_t = \epsilon$-greedy($s_t$)
        \STATE Exécuter $a_t$, observer $r_t, s_{t+1}$
        \STATE Stocker $(s_t, a_t, r_t, s_{t+1})$ dans $D$
        \STATE Échantillonner mini-batch de $D$
        \STATE Calculer target : $y = r + \gamma \max_{a'} Q(s', a'; \theta^-)$
        \STATE Gradient descent sur $(y - Q(s, a; \theta))^2$
        \STATE Mettre à jour $\theta^-$ périodiquement
    \ENDFOR
\ENDFOR
\end{algorithmic}
\end{algorithm}

\subsection{Implémentation PyTorch}

\begin{lstlisting}[caption=DQN avec PyTorch]
import torch
import torch.nn as nn
import torch.optim as optim
import random
from collections import deque

class DQN(nn.Module):
    def __init__(self, state_dim, action_dim, hidden_dim=128):
        super(DQN, self).__init__()
        self.fc = nn.Sequential(
            nn.Linear(state_dim, hidden_dim),
            nn.ReLU(),
            nn.Linear(hidden_dim, hidden_dim),
            nn.ReLU(),
            nn.Linear(hidden_dim, action_dim)
        )

    def forward(self, x):
        return self.fc(x)

class ReplayBuffer:
    def __init__(self, capacity=10000):
        self.buffer = deque(maxlen=capacity)

    def push(self, state, action, reward, next_state, done):
        self.buffer.append((state, action, reward, next_state, done))

    def sample(self, batch_size):
        return random.sample(self.buffer, batch_size)

    def __len__(self):
        return len(self.buffer)

class DQNAgent:
    def __init__(self, state_dim, action_dim):
        self.policy_net = DQN(state_dim, action_dim)
        self.target_net = DQN(state_dim, action_dim)
        self.target_net.load_state_dict(self.policy_net.state_dict())

        self.optimizer = optim.Adam(self.policy_net.parameters(), lr=1e-3)
        self.memory = ReplayBuffer()
        self.gamma = 0.99
        self.epsilon = 1.0
        self.epsilon_decay = 0.995
        self.epsilon_min = 0.01

    def select_action(self, state):
        if random.random() < self.epsilon:
            return random.randint(0, self.policy_net.fc[-1].out_features - 1)

        with torch.no_grad():
            state_tensor = torch.FloatTensor(state).unsqueeze(0)
            q_values = self.policy_net(state_tensor)
            return q_values.argmax().item()

    def train(self, batch_size=64):
        if len(self.memory) < batch_size:
            return

        batch = self.memory.sample(batch_size)
        states, actions, rewards, next_states, dones = zip(*batch)

        states = torch.FloatTensor(states)
        actions = torch.LongTensor(actions)
        rewards = torch.FloatTensor(rewards)
        next_states = torch.FloatTensor(next_states)
        dones = torch.FloatTensor(dones)

        # Q(s, a)
        q_values = self.policy_net(states).gather(1, actions.unsqueeze(1))

        # Target: r + gamma * max Q(s', a')
        with torch.no_grad():
            next_q_values = self.target_net(next_states).max(1)[0]
            targets = rewards + self.gamma * next_q_values * (1 - dones)

        # Loss et backprop
        loss = nn.MSELoss()(q_values.squeeze(), targets)

        self.optimizer.zero_grad()
        loss.backward()
        self.optimizer.step()

        # Decay epsilon
        self.epsilon = max(self.epsilon_min, self.epsilon * self.epsilon_decay)

    def update_target_network(self):
        self.target_net.load_state_dict(self.policy_net.state_dict())
\end{lstlisting}

\section{Policy Gradient}

\subsection{Policy directe}

Au lieu d'apprendre $Q$ puis dériver policy, apprendre directement $\pi_\theta(a | s)$.

\begin{definition}{Policy Gradient Theorem}
Le gradient de l'objectif $J(\theta) = \E_\pi [G_t]$ est :
\begin{equation}
    \nabla_\theta J(\theta) = \E_\pi [\nabla_\theta \log \pi_\theta(a|s) \cdot G_t]
\end{equation}
\end{definition}

\subsection{REINFORCE}

\begin{algorithm}[H]
\caption{REINFORCE}
\begin{algorithmic}[1]
\STATE Initialiser policy network $\pi_\theta$
\FOR{chaque episode}
    \STATE Générer épisode $\{s_1, a_1, r_1, \ldots, s_T, a_T, r_T\}$ en suivant $\pi_\theta$
    \FOR{$t = 1, \ldots, T$}
        \STATE $G_t \leftarrow \sum_{k=t}^T \gamma^{k-t} r_k$
        \STATE $\theta \leftarrow \theta + \alpha \nabla_\theta \log \pi_\theta(a_t | s_t) \cdot G_t$
    \ENDFOR
\ENDFOR
\end{algorithmic}
\end{algorithm}

\section{Actor-Critic}

Combiner policy gradient (actor) et value function (critic).

\begin{itemize}
    \item \textbf{Actor} : Policy $\pi_\theta(a|s)$
    \item \textbf{Critic} : Value function $V_\phi(s)$
\end{itemize}

\textbf{Advantage :}
\begin{equation}
    A(s, a) = Q(s, a) - V(s) \approx r + \gamma V(s') - V(s)
\end{equation}

\subsection{A3C (Asynchronous Advantage Actor-Critic)}

Version parallélisée avec plusieurs agents explorant en parallèle.

\section{Applications}

\begin{exemple}{Domaines d'application}
\begin{itemize}
    \item \textbf{Jeux} : AlphaGo, Atari (DQN), Dota 2, StarCraft
    \item \textbf{Robotique} : Manipulation d'objets, locomotion
    \item \textbf{Recommandation} : Systèmes adaptatifs
    \item \textbf{Finance} : Trading algorithmique
    \item \textbf{Ressources} : Optimisation énergétique, data centers
\end{itemize}
\end{exemple}

\section{Résumé}

\subsection{Points Clés}

\begin{itemize}
    \item \textbf{MDP} : États, actions, récompenses, transitions
    \item \textbf{Q-Learning} : Apprentissage de $Q(s, a)$ par équation de Bellman
    \item \textbf{DQN} : Approximation neuronale + replay buffer + target network
    \item \textbf{Policy Gradient} : Optimisation directe de la policy
    \item \textbf{Actor-Critic} : Combinaison policy + value function
\end{itemize}

\subsection{Formules Essentielles}

\begin{tcolorbox}[colback=blue!5!white, colframe=blue!75!black, title=Formules à retenir]
\textbf{Q-Learning Update :}
\begin{equation*}
    Q(s, a) \leftarrow Q(s, a) + \alpha [r + \gamma \max_{a'} Q(s', a') - Q(s, a)]
\end{equation*}

\textbf{Policy Gradient :}
\begin{equation*}
    \nabla_\theta J(\theta) = \E_\pi [\nabla_\theta \log \pi_\theta(a|s) \cdot G_t]
\end{equation*}
\end{tcolorbox}

\section{Pour Aller Plus Loin}

\subsection{Lectures}
\begin{itemize}
    \item Sutton \& Barto - \textit{Reinforcement Learning: An Introduction}
    \item Mnih et al. (2015) - "Human-level control through deep RL" (DQN)
    \item Silver et al. (2016) - "Mastering the game of Go with deep neural networks" (AlphaGo)
\end{itemize}

\subsection{Environnements}
\begin{itemize}
    \item OpenAI Gym : \texttt{gym.openai.com}
    \item Stable Baselines3 : Implémentations RL
\end{itemize}

\section{Notebooks Pratiques}

Ce chapitre est accompagné des notebooks suivants :

\begin{itemize}
    \item \texttt{09_demo_qlearning.ipynb} : Implémentation de Q-Learning sur FrozenLake
    \begin{itemize}
        \item Algorithme Q-Learning from scratch
        \item Environnement OpenAI Gym FrozenLake
        \item Construction et visualisation de la Q-table
        \item Analyse de la convergence
    \end{itemize}

    \item \texttt{09_demo_dqn.ipynb} : Deep Q-Network (DQN) sur CartPole
    \begin{itemize}
        \item Architecture DQN avec PyTorch
        \item Experience Replay et Target Network
        \item Entraînement et évaluation
        \item Visualisation des performances
    \end{itemize}
\end{itemize}

\end{document}
