% Chapitre 12 - Vision par Ordinateur Avancée
\documentclass[11pt,a4paper]{article}

% ===== PACKAGES =====
\usepackage[utf8]{inputenc}
\usepackage[T1]{fontenc}
\usepackage[french]{babel}
\usepackage{lmodern}

% Mathématiques
\usepackage{amsmath, amssymb, amsthm}
\usepackage{mathtools}

% Mise en page
\usepackage[margin=2.5cm]{geometry}
\usepackage{parskip}
\usepackage{setspace}
\setstretch{1.15}

% Graphiques et couleurs
\usepackage{graphicx}
\usepackage{xcolor}

% ===== UNICODE CHARACTERS SUPPORT =====
\usepackage{newunicodechar}

% Emojis et symboles
\newunicodechar{✅}{\textcolor{green!60!black}{$\checkmark$}}
\newunicodechar{❌}{\textcolor{red!60!black}{$\times$}}
\newunicodechar{✓}{\textcolor{green!60!black}{$\checkmark$}}
\newunicodechar{✗}{\textcolor{red!60!black}{$\times$}}
\newunicodechar{⚠}{\textcolor{orange!80!black}{\textbf{/!\textbackslash}}}
\newunicodechar{💡}{\textcolor{blue!70!black}{\textbf{(i)}}}
\newunicodechar{🎯}{\textcolor{purple!70!black}{\textbf{$\star$}}}
\newunicodechar{📊}{\textcolor{blue!70!black}{\textbf{[=]}}}

% Étoiles (pour tableaux)
\newunicodechar{★}{\textcolor{orange!80!black}{$\star$}}
\newunicodechar{☆}{\textcolor{gray!50}{$\star$}}

% Flèches
\newunicodechar{→}{$\rightarrow$}
\newunicodechar{←}{$\leftarrow$}
\newunicodechar{↑}{$\uparrow$}
\newunicodechar{↓}{$\downarrow$}

\usepackage{tikz}
\usetikzlibrary{arrows.meta, positioning, shapes.geometric}

% Tableaux
\usepackage{booktabs}
\usepackage{longtable}
\usepackage{multirow}
\usepackage{tabularx}
\usepackage{colortbl}

% Code et algorithmes
\usepackage{listings}
\usepackage{algorithm}
\usepackage{algorithmic}

% Hyperliens
\usepackage{hyperref}
\hypersetup{
    colorlinks=true,
    linkcolor=blue,
    filecolor=magenta,
    urlcolor=cyan,
    citecolor=green,
    pdftitle={Chapitre 12 - Vision par Ordinateur Avancée},
    pdfauthor={Cours ML},
}

% Boxes colorées
\usepackage{tcolorbox}
\tcbuselibrary{skins, breakable}


% ===== TCOLORBOX AVEC EMOJIS =====
\newtcolorbox{attention}{
    colback=red!5!white,
    colframe=red!75!black,
    fonttitle=\bfseries,
    title=⚠ Attention,
    breakable
}

\newtcolorbox{definition}{
    colback=blue!5!white,
    colframe=blue!75!black,
    fonttitle=\bfseries,
    title=Définition,
    breakable
}

\newtcolorbox{astuce}{
    colback=green!5!white,
    colframe=green!60!black,
    fonttitle=\bfseries,
    title=💡 Astuce,
    breakable
}

\newtcolorbox{remarque}{
    colback=yellow!5!white,
    colframe=orange!75!black,
    fonttitle=\bfseries,
    title=💡 Remarque,
    breakable
}

\newtcolorbox{important}{
    colback=purple!5!white,
    colframe=purple!75!black,
    fonttitle=\bfseries,
    title=⚠ Important,
    breakable
}

\newtcolorbox{exemple}{
    colback=gray!5!white,
    colframe=gray!75!black,
    fonttitle=\bfseries,
    title=Exemple,
    breakable
}

% En-têtes et pieds de page
\usepackage{fancyhdr}
\pagestyle{fancy}
\fancyhf{}
\fancyhead[L]{\small Chapitre 13 - Vision par Ordinateur Avancée}
\fancyhead[R]{\small Cours Machine Learning}
\fancyfoot[C]{\thepage}

% ===== CONFIGURATION LISTINGS (code Python) =====
\definecolor{codegreen}{rgb}{0,0.6,0}
\definecolor{codegray}{rgb}{0.5,0.5,0.5}
\definecolor{codepurple}{rgb}{0.58,0,0.82}
\definecolor{backcolour}{rgb}{0.95,0.95,0.92}

\lstdefinestyle{pythonstyle}{
    language=Python,
    backgroundcolor=\color{backcolour},
    commentstyle=\color{codegreen},
    keywordstyle=\color{blue},
    numberstyle=\tiny\color{codegray},
    stringstyle=\color{codepurple},
    basicstyle=\ttfamily\small,
    breakatwhitespace=false,
    breaklines=true,
    captionpos=b,
    keepspaces=true,
    numbers=left,
    numbersep=5pt,
    showspaces=false,
    showstringspaces=false,
    showtabs=false,
    tabsize=4,
    frame=single,
    rulecolor=\color{black}
}
\lstset{style=pythonstyle}

% ===== CONFIGURATION TCOLORBOX =====


\newtcolorbox{theoreme}[1]{
    colback=green!5!white,
    colframe=green!75!black,
    fonttitle=\bfseries,
    title=Théorème: #1,
    breakable
}







% ===== COMMANDES PERSONNALISÉES =====
\newcommand{\vect}[1]{\mathbf{#1}}
\newcommand{\mat}[1]{\mathbf{#1}}
\newcommand{\R}{\mathbb{R}}
\newcommand{\N}{\mathbb{N}}
\newcommand{\argmin}{\operatorname{argmin}}
\newcommand{\argmax}{\operatorname{argmax}}

% ===== DÉBUT DU DOCUMENT =====
\begin{document}

% ===== PAGE DE TITRE =====
\begin{titlepage}
    \centering
    \vspace*{2cm}

    {\Huge\bfseries Cours Machine Learning}\\[0.5cm]

    \vspace{1cm}

    {\LARGE Chapitre 13}\\[0.3cm]
    {\LARGE\bfseries Vision par Ordinateur Avancée}\\[2cm]

    \vfill

    {\large
    \textbf{Objectifs d'apprentissage :}\\[0.5cm]
    \begin{itemize}
        \item Maîtriser les architectures de détection d'objets (R-CNN, YOLO, Faster R-CNN)
        \item Comprendre la segmentation sémantique et d'instances (U-Net, Mask R-CNN)
        \item Découvrir les Vision Transformers (ViT) et leur application
        \item Comprendre les modèles vision-langage (CLIP)
        \item Implémenter des pipelines de détection et segmentation
    \end{itemize}
    }

    \vfill

    {\large
    \textbf{Prérequis :} Chapitres 06 (MLP), 07 (CNN), 08 (Transformers)\\[0.3cm]
    \textbf{Durée estimée :} 8-10 heures\\[0.3cm]
    \textbf{Notebooks :} \texttt{12\_demo\_*.ipynb}
    }

    \vfill

    {\large Cours ML - Sandbox-ML\\
    Version 1.0 - 2026}
\end{titlepage}

% ===== TABLE DES MATIÈRES =====
\tableofcontents
\newpage

% ===== SECTION 1: MOTIVATION =====
\section{Motivation}

La vision par ordinateur classique (chapitre 07) nous a permis de classifier des images avec des CNN. Cependant, de nombreuses applications réelles nécessitent bien plus qu'une simple classification :

\begin{exemple}{Applications nécessitant la vision avancée}
\begin{itemize}
    \item \textbf{Conduite autonome} : Détecter et localiser les voitures, piétons, panneaux
    \item \textbf{Imagerie médicale} : Segmenter précisément les tumeurs, organes
    \item \textbf{Surveillance} : Suivre et identifier les personnes dans une vidéo
    \item \textbf{Commerce électronique} : Rechercher des produits par image
    \item \textbf{Robotique} : Manipuler des objets détectés dans une scène
\end{itemize}
\end{exemple}

Ces problèmes requièrent trois capacités avancées :

\begin{enumerate}
    \item \textbf{Object Detection} : Où sont les objets ? (bounding boxes)
    \item \textbf{Semantic Segmentation} : Quel est le label de chaque pixel ?
    \item \textbf{Instance Segmentation} : Identifier chaque instance individuelle d'un objet
\end{enumerate}

Ce chapitre explore les architectures deep learning qui résolvent ces tâches complexes.

% ===== SECTION 2: OBJECT DETECTION =====
\section{Détection d'Objets (Object Detection)}

\subsection{Définition du Problème}

\begin{definition}{Object Detection}
La détection d'objets consiste à localiser et classifier simultanément plusieurs objets dans une image. Pour chaque objet, on prédit :
\begin{itemize}
    \item Une \textbf{bounding box} : $(x, y, w, h)$ où $(x,y)$ est le coin supérieur gauche, $w$ la largeur, $h$ la hauteur
    \item Une \textbf{classe} : probabilités $P(c_i | \text{box})$ pour chaque classe $c_i$
    \item Un \textbf{score de confiance} : probabilité qu'il y ait un objet dans la box
\end{itemize}
\end{definition}

\subsection{Métriques d'Évaluation}

\subsubsection{Intersection over Union (IoU)}

L'IoU mesure le chevauchement entre la box prédite et la ground truth :

\begin{equation}
\text{IoU} = \frac{\text{Aire}(\text{Box}_{\text{pred}} \cap \text{Box}_{\text{gt}})}{\text{Aire}(\text{Box}_{\text{pred}} \cup \text{Box}_{\text{gt}})}
\end{equation}

\begin{itemize}
    \item $\text{IoU} = 1.0$ : chevauchement parfait
    \item $\text{IoU} = 0.0$ : aucun chevauchement
    \item $\text{IoU} \geq 0.5$ : généralement considéré comme une détection correcte
\end{itemize}

\subsubsection{Mean Average Precision (mAP)}

La métrique standard pour évaluer les détecteurs :

\begin{enumerate}
    \item Pour chaque classe $c$, calculer la courbe Precision-Recall
    \item Calculer l'aire sous la courbe (Average Precision, AP)
    \item Faire la moyenne sur toutes les classes : $\text{mAP} = \frac{1}{C} \sum_{c=1}^{C} AP_c$
\end{enumerate}

\begin{itemize}
    \item \textbf{mAP@0.5} : seuil IoU = 0.5
    \item \textbf{mAP@0.5:0.95} : moyenne sur seuils IoU de 0.5 à 0.95 (par pas de 0.05)
\end{itemize}

\subsection{R-CNN (2014) - Région-based CNN}

\begin{definition}{R-CNN}
R-CNN propose de combiner :
\begin{enumerate}
    \item \textbf{Selective Search} : algorithme traditionnel proposant ~2000 régions candidates
    \item \textbf{CNN} : extraction de features pour chaque région
    \item \textbf{SVM} : classification de chaque région
    \item \textbf{Régression} : ajustement des bounding boxes
\end{enumerate}
\end{definition}

\subsubsection{Pipeline R-CNN}

\begin{algorithm}[H]
\caption{R-CNN}
\label{alg:rcnn}
\begin{algorithmic}[1]
\REQUIRE Image $I$
\ENSURE Liste de détections (boxes, classes, scores)
\STATE Générer ~2000 région proposals avec Selective Search
\FOR{chaque région $R$}
    \STATE Redimensionner $R$ en $227 \times 227$
    \STATE Extraire features : $f_R = \text{CNN}(R)$
    \STATE Classifier : $c = \text{SVM}(f_R)$
    \STATE Ajuster box : $b' = \text{Regressor}(f_R)$
\ENDFOR
\STATE Appliquer Non-Maximum Suppression (NMS)
\RETURN Détections filtrées
\end{algorithmic}
\end{algorithm}

\subsubsection{Limites de R-CNN}

\begin{itemize}
    \item ❌ \textbf{Très lent} : 47 secondes par image (2000 forward passes CNN)
    \item ❌ \textbf{Entraînement en 3 étapes} : CNN, SVM, régression (séparément)
    \item ❌ \textbf{Stockage important} : features extraites pour toutes les régions
\end{itemize}

\subsection{Fast R-CNN (2015)}

\textbf{Idée clé} : Ne calculer les features CNN qu'une seule fois pour toute l'image.

\begin{definition}{RoI Pooling}
Le \textbf{Region of Interest (RoI) Pooling} permet d'extraire des features de taille fixe depuis n'importe quelle région de la feature map :
\begin{enumerate}
    \item Projeter la région proposal sur la feature map
    \item Diviser la région en $H \times W$ sous-régions
    \item Appliquer max pooling sur chaque sous-région
\end{enumerate}
Résultat : un vecteur de features de taille fixe $H \times W \times C$ pour chaque RoI.
\end{definition}

\subsubsection{Architecture Fast R-CNN}

\begin{algorithm}[H]
\caption{Fast R-CNN}
\label{alg:fast-rcnn}
\begin{algorithmic}[1]
\REQUIRE Image $I$, régions proposals $\{R_i\}$
\ENSURE Détections
\STATE Calculer feature map : $F = \text{CNN}(I)$ \quad (une seule fois !)
\FOR{chaque région $R_i$}
    \STATE Extraire features : $f_i = \text{RoIPool}(F, R_i)$
    \STATE Prédire classe : $P(c | R_i) = \text{FC}(f_i)$
    \STATE Ajuster box : $\Delta b_i = \text{FC}(f_i)$
\ENDFOR
\STATE Appliquer NMS
\RETURN Détections
\end{algorithmic}
\end{algorithm}

\textbf{Améliorations} :
\begin{itemize}
    \item ✅ \textbf{25x plus rapide} que R-CNN (0.32s par image)
    \item ✅ \textbf{Entraînement end-to-end} : une seule loss combinée
    \item ✅ \textbf{Multi-task loss} : classification + régression de box
\end{itemize}

\subsection{Faster R-CNN (2015)}

\textbf{Idée clé} : Remplacer Selective Search par un réseau de neurones.

\begin{definition}{Region Proposal Network (RPN)}
Le RPN est un petit réseau fully convolutional qui prédit des région proposals directement depuis la feature map :
\begin{enumerate}
    \item Faire glisser une fenêtre $3 \times 3$ sur la feature map
    \item Pour chaque position, prédire $k$ anchors boxes (différentes tailles/ratios)
    \item Pour chaque anchor : prédire score objectness + ajustement de box
\end{enumerate}
\end{definition}

\subsubsection{Architecture Faster R-CNN}

\begin{enumerate}
    \item \textbf{Backbone CNN} : ResNet-50, VGG, etc. $\rightarrow$ feature map
    \item \textbf{RPN} : propose des régions candidates
    \item \textbf{RoI Pooling} : extrait features pour chaque proposition
    \item \textbf{Têtes de classification/régression} : prédictions finales
\end{enumerate}

\subsubsection{Loss Function}

Loss multi-task combinant RPN et détection :

\begin{equation}
L = L_{\text{RPN}}(\{p_i\}, \{t_i\}) + L_{\text{det}}(\{p_i'\}, \{t_i'\})
\end{equation}

où :
\begin{itemize}
    \item $L_{\text{RPN}} = L_{\text{cls}}(p_i, p_i^*) + \lambda L_{\text{reg}}(t_i, t_i^*)$ : loss du RPN
    \item $L_{\text{det}}$ : loss de détection (similaire)
    \item $p_i$ : probabilité objectness, $t_i$ : coordonnées box
\end{itemize}

\textbf{Améliorations} :
\begin{itemize}
    \item ✅ \textbf{10x plus rapide} que Fast R-CNN (0.2s par image, 5 FPS)
    \item ✅ \textbf{Entièrement appris} : plus besoin de Selective Search
    \item ✅ \textbf{État de l'art} en précision (mAP ~70\% sur COCO)
\end{itemize}

\subsection{YOLO (You Only Look Once)}

\textbf{Philosophie différente} : Faster R-CNN fait deux passes (RPN puis détection). YOLO fait tout en une seule passe forward !

\begin{definition}{YOLO}
YOLO divise l'image en une grille $S \times S$ et prédit directement, pour chaque cellule :
\begin{itemize}
    \item $B$ bounding boxes avec leurs coordonnées $(x, y, w, h)$
    \item Un score de confiance par box : $P(\text{object}) \times \text{IoU}$
    \item Des probabilités de classe : $P(c_i | \text{object})$
\end{itemize}
Le réseau produit un tenseur de taille $S \times S \times (B \cdot 5 + C)$.
\end{definition}

\begin{figure}[h]
\centering
\begin{tikzpicture}[scale=0.8, every node/.style={font=\small}]
    % Input image with grid overlay
    \draw[thick] (0,0) rectangle (6,6);
    \node[above] at (3, 6) {Image d'entrée ($448 \times 448$)};

    % Draw 7x7 grid
    \foreach \x in {0,0.857,...,6} {
        \draw[gray, thin] (\x, 0) -- (\x, 6);
    }
    \foreach \y in {0,0.857,...,6} {
        \draw[gray, thin] (0, \y) -- (6, \y);
    }

    % Highlight one cell
    \draw[red, very thick] (2.571, 3.429) rectangle (3.428, 4.286);
    \node[red, font=\scriptsize] at (3, 3.85) {Cell $(i,j)$};

    % Draw sample objects
    \draw[blue, ultra thick, rounded corners] (1.2, 4.2) rectangle (2.8, 5.5);
    \node[blue, font=\scriptsize, above] at (2, 5.5) {Chien};
    \fill[blue] (2, 4.85) circle (0.08);

    \draw[orange, ultra thick, rounded corners] (3.5, 1.5) rectangle (5.2, 3.2);
    \node[orange, font=\scriptsize, above] at (4.35, 3.2) {Chat};
    \fill[orange] (4.35, 2.35) circle (0.08);

    % Arrow to output tensor
    \draw[->, very thick] (7, 3) -- (9, 3);

    % Output tensor representation
    \begin{scope}[xshift=9.5cm, yshift=1cm]
        % 7x7 grid representation
        \foreach \i in {0,...,6} {
            \foreach \j in {0,...,6} {
                \draw[fill=gray!20] (\i*0.5, \j*0.5) rectangle ++ (0.5, 0.5);
            }
        }
        \node[below, font=\footnotesize, align=center] at (1.75, -0.3) {$7 \times 7$ grille};

        % Depth dimension
        \foreach \k in {1,...,4} {
            \draw[fill=blue!15, opacity=0.7] (0.15*\k, 0.15*\k) rectangle ++ (3.5, 3.5);
        }
        \node[right, font=\footnotesize, align=left] at (4.2, 2) {$B \cdot 5 + C = 30$\\(2 boxes,\\20 classes)};
    \end{scope}

    % Detailed cell output (zoom)
    \begin{scope}[yshift=-4cm]
        \node[align=center, font=\small] at (3, 0.5) {
            \textbf{Pour chaque cellule $(i,j)$ :}
        };

        % Box 1
        \draw[draw=purple!70!black, fill=purple!10, thick, rounded corners] (0, -0.5) rectangle (2.8, -3.5);
        \node[purple!70!black, font=\scriptsize, align=left] at (1.4, -1.5) {
            \textbf{Box 1:}\\
            $(x_1, y_1, w_1, h_1)$\\
            $\text{conf}_1$
        };

        % Box 2
        \draw[draw=purple!70!black, fill=purple!10, thick, rounded corners] (3.2, -0.5) rectangle (6, -3.5);
        \node[purple!70!black, font=\scriptsize, align=left] at (4.6, -1.5) {
            \textbf{Box 2:}\\
            $(x_2, y_2, w_2, h_2)$\\
            $\text{conf}_2$
        };

        % Class probabilities
        \draw[draw=green!60!black, fill=green!10, thick, rounded corners] (6.5, -0.5) rectangle (10.5, -3.5);
        \node[green!60!black, font=\scriptsize, align=left] at (8.5, -2) {
            \textbf{Classes (20):}\\
            $P(\text{chat} | \text{obj})$\\
            $P(\text{chien} | \text{obj})$\\
            $\vdots$
        };
    \end{scope}

    % Prediction visualization
    \begin{scope}[xshift=12cm]
        \draw[thick] (0,0) rectangle (6,6);
        \node[above] at (3, 6) {Prédictions finales};

        % Draw grid lightly
        \foreach \x in {0,0.857,...,6} {
            \draw[gray!30, thin] (\x, 0) -- (\x, 6);
        }
        \foreach \y in {0,0.857,...,6} {
            \draw[gray!30, thin] (0, \y) -- (6, \y);
        }

        % Predicted boxes with NMS
        \draw[blue, ultra thick, rounded corners] (1.15, 4.15) rectangle (2.85, 5.55);
        \node[blue, font=\scriptsize, fill=white, inner sep=1pt] at (1.5, 5.8) {Chien 0.92};

        \draw[orange, ultra thick, rounded corners] (3.45, 1.45) rectangle (5.25, 3.25);
        \node[orange, font=\scriptsize, fill=white, inner sep=1pt] at (4, 3.5) {Chat 0.87};

        % Show responsible cells
        \fill[blue!30, opacity=0.4] (0.857, 3.429) rectangle (1.714, 4.286);
        \fill[orange!30, opacity=0.4] (3.428, 1.714) rectangle (4.285, 2.571);

        \node[font=\tiny, align=center] at (3, -0.5) {après NMS\\(suppression\\non-max)};
    \end{scope}
\end{tikzpicture}
\caption{Principe de détection YOLO. \textbf{(Gauche)} L'image est divisée en grille $S \times S$ (ici $S=7$). Chaque cellule est responsable de détecter les objets dont le centre tombe dans cette cellule. \textbf{(Centre)} Chaque cellule prédit $B=2$ bounding boxes (coordonnées + confiance) et $C=20$ probabilités de classe, produisant un tenseur $7 \times 7 \times 30$. \textbf{(Droite)} Après NMS, on garde les détections avec haute confiance. Une seule passe forward permet de détecter tous les objets simultanément.}
\label{fig:yolo_detection_grid}
\end{figure}

\subsubsection{Architecture YOLO (v1, 2016)}

\begin{enumerate}
    \item 24 couches convolutionnelles (inspirées de GoogLeNet)
    \item 2 couches fully connected
    \item Sortie : $7 \times 7 \times 30$ (pour $S=7$, $B=2$, $C=20$)
\end{enumerate}

\subsubsection{Loss Function YOLO}

Loss complexe combinant 3 termes :

\begin{align}
L = &\lambda_{\text{coord}} \sum_{i=0}^{S^2} \sum_{j=0}^{B} \mathbb{1}_{ij}^{\text{obj}} [(x_i - \hat{x}_i)^2 + (y_i - \hat{y}_i)^2] \\
    &+ \lambda_{\text{coord}} \sum_{i=0}^{S^2} \sum_{j=0}^{B} \mathbb{1}_{ij}^{\text{obj}} [(\sqrt{w_i} - \sqrt{\hat{w}_i})^2 + (\sqrt{h_i} - \sqrt{\hat{h}_i})^2] \\
    &+ \sum_{i=0}^{S^2} \sum_{j=0}^{B} \mathbb{1}_{ij}^{\text{obj}} (C_i - \hat{C}_i)^2 \\
    &+ \lambda_{\text{noobj}} \sum_{i=0}^{S^2} \sum_{j=0}^{B} \mathbb{1}_{ij}^{\text{noobj}} (C_i - \hat{C}_i)^2 \\
    &+ \sum_{i=0}^{S^2} \mathbb{1}_{i}^{\text{obj}} \sum_{c \in \text{classes}} (p_i(c) - \hat{p}_i(c))^2
\end{align}

où $\mathbb{1}_{ij}^{\text{obj}}$ indique si un objet est présent dans la cellule $i$, box $j$.

\subsubsection{YOLOv2 et YOLOv3 (2017-2018)}

\textbf{YOLOv2 (YOLO9000)} :
\begin{itemize}
    \item Batch Normalization dans toutes les couches
    \item Anchor boxes (comme Faster R-CNN)
    \item Haute résolution : $416 \times 416$ au lieu de $448 \times 448$
    \item Multi-scale training
    \item Darknet-19 backbone (19 couches)
\end{itemize}

\textbf{YOLOv3} :
\begin{itemize}
    \item Darknet-53 backbone (53 couches + residual connections)
    \item Prédictions multi-échelles (3 échelles : $13 \times 13$, $26 \times 26$, $52 \times 52$)
    \item Meilleure détection des petits objets
    \item Logistic regression pour objectness
\end{itemize}

\subsubsection{YOLOv5, YOLOv8 (2020-2023)}

\textbf{YOLOv5} (Ultralytics) :
\begin{itemize}
    \item Implémentation PyTorch moderne
    \item CSPDarknet backbone
    \item Auto-anchor, auto-learning bounding box anchors
    \item Mosaic augmentation
    \item \textbf{Très rapide} : 140 FPS sur GPU
\end{itemize}

\textbf{YOLOv8} (2023) :
\begin{itemize}
    \item Architecture améliorée (anchor-free)
    \item Meilleure précision (mAP 53\% sur COCO)
    \item API simplifiée : \texttt{from ultralytics import YOLO}
    \item Support natif de la segmentation d'instances
\end{itemize}

\subsection{Comparaison R-CNN vs YOLO}

\begin{table}[h]
\centering
\caption{Comparaison des architectures de détection}
\label{tab:detection-comparison}
\begin{tabular}{lcccc}
\toprule
\textbf{Modèle} & \textbf{mAP (\%)} & \textbf{FPS} & \textbf{Approche} & \textbf{Temps réel} \\
\midrule
R-CNN & 66.0 & 0.02 & Two-stage & ❌ \\
Fast R-CNN & 70.0 & 3.1 & Two-stage & ❌ \\
Faster R-CNN & 73.2 & 5 & Two-stage & ❌ \\
YOLOv1 & 63.4 & 45 & One-stage & ✅ \\
YOLOv3 & 57.9 & 65 & One-stage & ✅ \\
YOLOv5 & 50.7 & 140 & One-stage & ✅ \\
YOLOv8 & 53.9 & 80 & One-stage & ✅ \\
\bottomrule
\end{tabular}
\end{table}

\begin{astuce}
\textbf{Quand utiliser quoi ?}
\begin{itemize}
    \item \textbf{Faster R-CNN} : Précision maximale, applications non temps réel (imagerie médicale)
    \item \textbf{YOLO} : Temps réel, vidéo, applications embarquées (conduite autonome, surveillance)
\end{itemize}
\end{astuce}

\subsection{Non-Maximum Suppression (NMS)}

Problème : Les détecteurs produisent souvent plusieurs boxes pour le même objet.

\begin{algorithm}[H]
\caption{Non-Maximum Suppression}
\label{alg:nms}
\begin{algorithmic}[1]
\REQUIRE Boxes $B = \{b_1, \dots, b_n\}$, scores $S = \{s_1, \dots, s_n\}$, seuil IoU $\tau$
\ENSURE Boxes filtrées $D$
\STATE $D \leftarrow \emptyset$
\WHILE{$B \neq \emptyset$}
    \STATE $b^* \leftarrow \argmax_{b \in B} S(b)$ \quad (box avec le score max)
    \STATE $D \leftarrow D \cup \{b^*\}$
    \STATE $B \leftarrow B \setminus \{b^*\}$
    \FOR{chaque box $b_i \in B$}
        \IF{$\text{IoU}(b^*, b_i) > \tau$}
            \STATE $B \leftarrow B \setminus \{b_i\}$ \quad (supprimer box chevauchante)
        \ENDIF
    \ENDFOR
\ENDWHILE
\RETURN $D$
\end{algorithmic}
\end{algorithm}

\textbf{Variantes} :
\begin{itemize}
    \item \textbf{Soft NMS} : Au lieu de supprimer, réduire le score proportionnellement à l'IoU
    \item \textbf{DIoU-NMS} : Utiliser la Distance-IoU au lieu de l'IoU standard
\end{itemize}

% ===== SECTION 3: SEGMENTATION SÉMANTIQUE =====
\section{Segmentation Sémantique}

\subsection{Définition du Problème}

\begin{definition}{Segmentation Sémantique}
La segmentation sémantique consiste à assigner une étiquette de classe à chaque pixel de l'image. Pour une image $I \in \R^{H \times W \times 3}$, on prédit une carte de segmentation $S \in \{1, \dots, C\}^{H \times W}$ où $S_{ij}$ est la classe du pixel $(i,j)$.
\end{definition}

\textbf{Différence avec la détection} :
\begin{itemize}
    \item Détection : boxes rectangulaires
    \item Segmentation sémantique : contours précis au pixel près
    \item Segmentation d'instances : distingue les instances individuelles d'une même classe
\end{itemize}

\subsection{Métriques d'Évaluation}

\subsubsection{Intersection over Union (IoU) par classe}

\begin{equation}
\text{IoU}_c = \frac{TP_c}{TP_c + FP_c + FN_c}
\end{equation}

où $TP_c$ = pixels correctement prédits de classe $c$, $FP_c$ = pixels faussement prédits, $FN_c$ = pixels manqués.

\subsubsection{Mean IoU (mIoU)}

Moyenne de l'IoU sur toutes les classes :

\begin{equation}
\text{mIoU} = \frac{1}{C} \sum_{c=1}^{C} \text{IoU}_c
\end{equation}

\subsubsection{Dice Coefficient}

Particulièrement utilisé en imagerie médicale :

\begin{equation}
\text{Dice} = \frac{2 \cdot |A \cap B|}{|A| + |B|} = \frac{2 \cdot TP}{2 \cdot TP + FP + FN}
\end{equation}

Le Dice est équivalent à la F1-score pour la segmentation.

\subsection{Fully Convolutional Networks (FCN, 2015)}

\textbf{Idée clé} : Remplacer les couches fully connected par des convolutions pour produire une carte de segmentation.

\begin{definition}{FCN}
Un FCN transforme un réseau de classification (VGG, ResNet) en réseau de segmentation :
\begin{enumerate}
    \item Encoder : extraire features avec convolutions + pooling
    \item Decoder : upsampling progressif pour retrouver la résolution originale
    \item Skip connections : combiner features haute et basse résolution
\end{enumerate}
\end{definition}

\subsubsection{Architecture FCN}

\begin{enumerate}
    \item \textbf{Convolutionalization} : Remplacer FC layers par convolutions $1 \times 1$
    \item \textbf{Upsampling} : Transposed convolutions (deconvolutions) pour augmenter la résolution
    \item \textbf{Skip connections} : Additionner les features du decoder avec celles de l'encoder
\end{enumerate}

\textbf{Variantes} :
\begin{itemize}
    \item \textbf{FCN-32s} : upsampling x32 en une seule étape
    \item \textbf{FCN-16s} : skip connection de pool4
    \item \textbf{FCN-8s} : skip connections de pool3 et pool4 (meilleur)
\end{itemize}

\subsection{U-Net (2015)}

\textbf{Architecture emblématique} pour la segmentation médicale.

\begin{definition}{U-Net}
U-Net a une architecture en U symétrique :
\begin{itemize}
    \item \textbf{Contracting path} (encoder) : Convolutions + max pooling $\downarrow$
    \item \textbf{Expansive path} (decoder) : Transposed convolutions $\uparrow$
    \item \textbf{Skip connections} : Concaténation (pas addition) des features
\end{itemize}
\end{definition}

\subsubsection{Architecture Détaillée}

\begin{verbatim}
Encoder (Contracting Path):
  Input (572x572x1)
    → Conv 3x3 ReLU (570x570x64)
    → Conv 3x3 ReLU (568x568x64)
    → MaxPool 2x2 (284x284x64)
    → Conv 3x3 ReLU (282x282x128)
    → Conv 3x3 ReLU (280x280x128)
    → MaxPool 2x2 (140x140x128)
    → ... (4 niveaux au total)

Bottleneck:
    → Conv 3x3 ReLU (28x28x1024)
    → Conv 3x3 ReLU (28x28x1024)

Decoder (Expansive Path):
    → UpConv 2x2 (56x56x512)
    → Concatenate avec skip connection de l'encoder
    → Conv 3x3 ReLU
    → Conv 3x3 ReLU
    → ... (4 niveaux au total)

Output:
    → Conv 1x1 (388x388xC) pour C classes
\end{verbatim}

\textbf{Points clés} :
\begin{itemize}
    \item \textbf{Skip connections par concaténation} : préserve mieux les détails que l'addition
    \item \textbf{Data augmentation intensive} : rotations, déformations élastiques
    \item \textbf{Weighted loss} : pondérer la loss aux frontières entre cellules
\end{itemize}

\begin{figure}[h]
\centering
\begin{tikzpicture}[scale=0.7, every node/.style={font=\scriptsize}]
    % Define styles
    \tikzstyle{conv}=[rectangle, draw, fill=blue!20, minimum width=1.2cm, minimum height=0.6cm]
    \tikzstyle{pool}=[rectangle, draw, fill=red!20, minimum width=1.2cm, minimum height=0.4cm]
    \tikzstyle{upconv}=[rectangle, draw, fill=green!20, minimum width=1.2cm, minimum height=0.6cm]
    \tikzstyle{concat}=[circle, draw, fill=orange!30, minimum size=0.5cm]

    % Encoder (left side, downward)
    % Level 0 (input)
    \node[conv] (e0a) at (0, 0) {Conv 3×3};
    \node[conv] (e0b) at (0, -0.8) {Conv 3×3};
    \node[above=0.1cm of e0a, font=\tiny] {572×572×1};
    \node[right=0.1cm of e0b, font=\tiny] {568×568×64};
    \node[pool] (p0) at (0, -1.5) {MaxPool};

    % Level 1
    \node[conv] (e1a) at (0, -2.5) {Conv 3×3};
    \node[conv] (e1b) at (0, -3.3) {Conv 3×3};
    \node[right=0.1cm of e1b, font=\tiny] {280×280×128};
    \node[pool] (p1) at (0, -4.0) {MaxPool};

    % Level 2
    \node[conv] (e2a) at (0, -5.0) {Conv 3×3};
    \node[conv] (e2b) at (0, -5.8) {Conv 3×3};
    \node[right=0.1cm of e2b, font=\tiny] {136×136×256};
    \node[pool] (p2) at (0, -6.5) {MaxPool};

    % Level 3
    \node[conv] (e3a) at (0, -7.5) {Conv 3×3};
    \node[conv] (e3b) at (0, -8.3) {Conv 3×3};
    \node[right=0.1cm of e3b, font=\tiny] {64×64×512};
    \node[pool] (p3) at (0, -9.0) {MaxPool};

    % Bottleneck (bottom)
    \node[conv, fill=purple!20] (b0) at (4, -10.0) {Conv 3×3};
    \node[conv, fill=purple!20] (b1) at (4, -10.8) {Conv 3×3};
    \node[below=0.1cm of b1, font=\tiny] {28×28×1024};

    % Decoder (right side, upward)
    % Level 3
    \node[upconv] (u3) at (8, -9.0) {UpConv 2×2};
    \node[concat] (c3) at (8, -8.3) {+};
    \node[conv] (d3a) at (8, -7.5) {Conv 3×3};
    \node[conv] (d3b) at (8, -6.8) {Conv 3×3};
    \node[left=0.1cm of d3b, font=\tiny] {64×64×512};

    % Level 2
    \node[upconv] (u2) at (8, -6.0) {UpConv 2×2};
    \node[concat] (c2) at (8, -5.3) {+};
    \node[conv] (d2a) at (8, -4.5) {Conv 3×3};
    \node[conv] (d2b) at (8, -3.8) {Conv 3×3};
    \node[left=0.1cm of d2b, font=\tiny] {136×136×256};

    % Level 1
    \node[upconv] (u1) at (8, -3.0) {UpConv 2×2};
    \node[concat] (c1) at (8, -2.3) {+};
    \node[conv] (d1a) at (8, -1.5) {Conv 3×3};
    \node[conv] (d1b) at (8, -0.8) {Conv 3×3};
    \node[left=0.1cm of d1b, font=\tiny] {280×280×128};

    % Level 0 (output)
    \node[upconv] (u0) at (8, 0) {UpConv 2×2};
    \node[concat] (c0) at (8, 0.7) {+};
    \node[conv] (d0a) at (8, 1.5) {Conv 3×3};
    \node[conv] (d0b) at (8, 2.3) {Conv 3×3};
    \node[conv, fill=yellow!30] (out) at (8, 3.1) {Conv 1×1};
    \node[above=0.1cm of out, font=\tiny] {388×388×C};

    % Arrows - Encoder path
    \draw[->, thick] (e0a) -- (e0b);
    \draw[->, thick] (e0b) -- (p0);
    \draw[->, thick] (p0) -- (e1a);
    \draw[->, thick] (e1a) -- (e1b);
    \draw[->, thick] (e1b) -- (p1);
    \draw[->, thick] (p1) -- (e2a);
    \draw[->, thick] (e2a) -- (e2b);
    \draw[->, thick] (e2b) -- (p2);
    \draw[->, thick] (p2) -- (e3a);
    \draw[->, thick] (e3a) -- (e3b);
    \draw[->, thick] (e3b) -- (p3);
    \draw[->, thick] (p3) -- (b0);
    \draw[->, thick] (b0) -- (b1);

    % Arrows - Decoder path
    \draw[->, thick] (b1) -- (u3);
    \draw[->, thick] (u3) -- (c3);
    \draw[->, thick] (c3) -- (d3a);
    \draw[->, thick] (d3a) -- (d3b);
    \draw[->, thick] (d3b) -- (u2);
    \draw[->, thick] (u2) -- (c2);
    \draw[->, thick] (c2) -- (d2a);
    \draw[->, thick] (d2a) -- (d2b);
    \draw[->, thick] (d2b) -- (u1);
    \draw[->, thick] (u1) -- (c1);
    \draw[->, thick] (c1) -- (d1a);
    \draw[->, thick] (d1a) -- (d1b);
    \draw[->, thick] (d1b) -- (u0);
    \draw[->, thick] (u0) -- (c0);
    \draw[->, thick] (c0) -- (d0a);
    \draw[->, thick] (d0a) -- (d0b);
    \draw[->, thick] (d0b) -- (out);

    % Skip connections (horizontal arrows with concatenation)
    \draw[->, thick, dashed, orange!70!black] (e3b.east) -- (c3.west) node[midway, above, font=\tiny] {concat};
    \draw[->, thick, dashed, orange!70!black] (e2b.east) -- (c2.west) node[midway, above, font=\tiny] {concat};
    \draw[->, thick, dashed, orange!70!black] (e1b.east) -- (c1.west) node[midway, above, font=\tiny] {concat};
    \draw[->, thick, dashed, orange!70!black] (e0b.east) -- (c0.west) node[midway, above, font=\tiny] {concat};

    % Labels
    \node[font=\small, blue!70!black] at (-1.5, -4.5) {\textbf{Encoder}};
    \node[font=\small, blue!70!black] at (-1.5, -5.0) {(Contracting)};
    \node[font=\small, purple!70!black] at (4, -11.5) {\textbf{Bottleneck}};
    \node[font=\small, green!60!black] at (9.5, -4.5) {\textbf{Decoder}};
    \node[font=\small, green!60!black] at (9.5, -5.0) {(Expansive)};

    % Dimension annotations
    \draw[<->, gray, thick] (-2, 0) -- (-2, -1.5) node[midway, left, font=\tiny] {↓ /2};
    \draw[<->, gray, thick] (-2, -2.5) -- (-2, -4.0) node[midway, left, font=\tiny] {↓ /2};
    \draw[<->, gray, thick] (-2, -5.0) -- (-2, -6.5) node[midway, left, font=\tiny] {↓ /2};
    \draw[<->, gray, thick] (-2, -7.5) -- (-2, -9.0) node[midway, left, font=\tiny] {↓ /2};

    \draw[<->, gray, thick] (10, -9.0) -- (10, -6.8) node[midway, right, font=\tiny] {↑ ×2};
    \draw[<->, gray, thick] (10, -6.0) -- (10, -3.8) node[midway, right, font=\tiny] {↑ ×2};
    \draw[<->, gray, thick] (10, -3.0) -- (10, -0.8) node[midway, right, font=\tiny] {↑ ×2};
    \draw[<->, gray, thick] (10, 0) -- (10, 2.3) node[midway, right, font=\tiny] {↑ ×2};
\end{tikzpicture}
\caption{Architecture U-Net. L'\textbf{encoder} (gauche) extrait des features à différentes échelles via convolutions et max pooling. Le \textbf{bottleneck} (bas) capture les features de plus haut niveau. Le \textbf{decoder} (droite) reconstruit la segmentation via upsampling et convolutions. Les \textbf{skip connections} (flèches orange) concatènent les features de l'encoder au decoder pour préserver les détails spatiaux fins. La forme en U caractéristique permet de combiner informations contextuelles (encoder) et locales (skip connections).}
\label{fig:unet_architecture}
\end{figure}

\subsubsection{Loss Function U-Net}

Weighted cross-entropy pour gérer le déséquilibre de classes :

\begin{equation}
L = \sum_{x \in \Omega} w(x) \log(p_{l(x)}(x))
\end{equation}

où :
\begin{itemize}
    \item $w(x)$ : poids du pixel $x$ (plus élevé aux frontières)
    \item $p_{l(x)}(x)$ : probabilité softmax de la classe $l(x)$ au pixel $x$
\end{itemize}

Poids calculé pour séparer les instances proches :

\begin{equation}
w(x) = w_c(x) + w_0 \cdot \exp\left(-\frac{(d_1(x) + d_2(x))^2}{2\sigma^2}\right)
\end{equation}

où $d_1(x)$, $d_2(x)$ sont les distances aux deux instances les plus proches.

\subsection{DeepLab (v1-v3, 2015-2018)}

\textbf{Idée clé} : Utiliser des \textbf{atrous convolutions} (dilated convolutions) pour augmenter le champ réceptif sans réduire la résolution.

\begin{definition}{Atrous Convolution}
Une convolution atrous avec taux de dilatation $r$ insère $r-1$ zéros entre chaque poids du filtre :
\begin{equation}
y[i] = \sum_{k} x[i + r \cdot k] \cdot w[k]
\end{equation}
Cela permet d'augmenter le champ réceptif de manière exponentielle sans augmenter le nombre de paramètres.
\end{definition}

\subsubsection{Atrous Spatial Pyramid Pooling (ASPP)}

Module clé de DeepLab v2/v3 :

\begin{enumerate}
    \item Appliquer des convolutions atrous avec différents taux : $r = \{6, 12, 18, 24\}$
    \item Appliquer global average pooling
    \item Concaténer toutes les features
    \item Convolution $1 \times 1$ pour fusion
\end{enumerate}

Cela capture le contexte à plusieurs échelles.

\subsubsection{DeepLab v3+}

Amélioration avec un decoder :

\begin{itemize}
    \item Encoder : ResNet + ASPP
    \item Decoder : Upsampling progressif avec skip connections
    \item État de l'art sur PASCAL VOC (mIoU 89\%)
\end{itemize}

\subsection{Mask R-CNN (2017)}

Extension de Faster R-CNN pour la segmentation d'instances.

\begin{definition}{Mask R-CNN}
Mask R-CNN ajoute une branche de segmentation parallèle à Faster R-CNN :
\begin{enumerate}
    \item Backbone + RPN + RoI Align (amélioration de RoI Pooling)
    \item Branche de classification + régression de box (comme Faster R-CNN)
    \item \textbf{Branche de masque} : FCN qui prédit un masque binaire pour chaque RoI
\end{enumerate}
\end{definition}

\subsubsection{RoI Align}

\textbf{Problème de RoI Pooling} : quantification qui cause un misalignment pixel-level.

\textbf{Solution RoI Align} :
\begin{itemize}
    \item Ne pas quantifier les coordonnées de la RoI
    \item Utiliser une interpolation bilinéaire pour échantillonner les features
    \item Précision au sub-pixel level
\end{itemize}

\subsubsection{Loss Function}

Multi-task loss :

\begin{equation}
L = L_{\text{cls}} + L_{\text{box}} + L_{\text{mask}}
\end{equation}

où :
\begin{itemize}
    \item $L_{\text{cls}}$ : classification loss
    \item $L_{\text{box}}$ : bounding box regression loss
    \item $L_{\text{mask}}$ : binary cross-entropy par pixel pour le masque
\end{itemize}

\textbf{Astuce} : La loss du masque est calculée uniquement pour la classe prédite, ce qui découple classification et segmentation.

\subsection{Comparaison des Architectures de Segmentation}

\begin{table}[h]
\centering
\caption{Comparaison des architectures de segmentation}
\label{tab:segmentation-comparison}
\begin{tabular}{lccc}
\toprule
\textbf{Modèle} & \textbf{Type} & \textbf{mIoU (\%)} & \textbf{Application} \\
\midrule
FCN-8s & Sémantique & 62.2 & Scènes générales \\
U-Net & Sémantique & 92.0 & Imagerie médicale \\
DeepLab v3+ & Sémantique & 89.0 & Scènes générales \\
Mask R-CNN & Instance & 37.1 & Détection + seg. \\
\bottomrule
\end{tabular}
\end{table}

% ===== SECTION 4: VISION TRANSFORMERS =====
\section{Vision Transformers (ViT)}

\subsection{Motivation}

Les CNN ont dominé la vision par ordinateur depuis 2012. Cependant, les Transformers (chapitre 08) ont révolutionné le NLP. Peut-on appliquer les Transformers à la vision ?

\textbf{Défis} :
\begin{itemize}
    \item Une image $224 \times 224$ a 50,176 pixels (vs ~100 tokens en NLP)
    \item Attention sur tous les pixels : complexité $O(n^2)$ prohibitive
\end{itemize}

\subsection{Architecture Vision Transformer (ViT, 2020)}

\begin{definition}{Vision Transformer}
ViT découpe l'image en patches et les traite comme des tokens :
\begin{enumerate}
    \item Diviser l'image en patches $16 \times 16$ (ou $32 \times 32$)
    \item Aplatir chaque patch en vecteur
    \item Embedding linéaire + positional encoding
    \item Transformer encoder standard
    \item Classification via un token [CLS]
\end{enumerate}
\end{definition}

\subsubsection{Patchification}

Pour une image $I \in \R^{H \times W \times C}$ et taille de patch $P$ :

\begin{enumerate}
    \item Nombre de patches : $N = \frac{H \cdot W}{P^2}$
    \item Chaque patch : $x_p \in \R^{P^2 \cdot C}$ (vecteur aplati)
    \item Embedding linéaire : $z_p = E \cdot x_p$ où $E \in \R^{D \times (P^2 \cdot C)}$
\end{enumerate}

\subsubsection{Architecture Complète}

\begin{algorithm}[H]
\caption{Vision Transformer (ViT)}
\label{alg:vit}
\begin{algorithmic}[1]
\REQUIRE Image $I \in \R^{H \times W \times 3}$
\ENSURE Prédiction de classe $y$
\STATE Découper $I$ en $N$ patches de taille $P \times P$
\STATE Aplatir chaque patch : $\{x_p^1, \dots, x_p^N\}$
\STATE Embedding linéaire : $z_p^i = E \cdot x_p^i$ pour $i=1, \dots, N$
\STATE Ajouter token [CLS] : $z_0 = z_{\text{cls}}$
\STATE Ajouter positional encoding : $z_i \leftarrow z_i + E_{pos}^i$
\STATE Passer dans $L$ couches de Transformer Encoder
\STATE Extraire $z_0^L$ (état final du token [CLS])
\STATE Classification : $y = \text{MLP}(z_0^L)$
\RETURN $y$
\end{algorithmic}
\end{algorithm}

\subsubsection{Positional Encoding}

Contrairement au NLP, on utilise des \textbf{positional embeddings appris} :

\begin{equation}
E_{pos} \in \R^{(N+1) \times D}
\end{equation}

Chaque position (patch) a son embedding de position appris durant l'entraînement.

\textbf{Alternative} : 2D positional encodings qui encodent séparément les coordonnées $x$ et $y$ du patch.

\subsection{Variantes de ViT}

\subsubsection{ViT-Base, ViT-Large, ViT-Huge}

\begin{table}[h]
\centering
\caption{Variantes de ViT}
\label{tab:vit-variants}
\begin{tabular}{lccc}
\toprule
\textbf{Modèle} & \textbf{Layers} & \textbf{Hidden Size} & \textbf{Heads} \\
\midrule
ViT-Base & 12 & 768 & 12 \\
ViT-Large & 24 & 1024 & 16 \\
ViT-Huge & 32 & 1280 & 16 \\
\bottomrule
\end{tabular}
\end{table}

\subsubsection{DeiT (Data-efficient image Transformers)}

Améliore l'entraînement de ViT :
\begin{itemize}
    \item \textbf{Distillation token} : apprendre d'un CNN teacher
    \item Augmentations agressives
    \item Entraînement plus efficace (moins de données)
\end{itemize}

\subsubsection{Swin Transformer (2021)}

\textbf{Idée clé} : Attention locale dans des fenêtres décalées (shifted windows).

\begin{enumerate}
    \item Diviser l'image en fenêtres non-chevauchantes
    \item Appliquer self-attention uniquement dans chaque fenêtre
    \item Décaler les fenêtres entre couches pour capturer interactions cross-window
    \item Hierarchical architecture (comme CNN) avec downsampling progressif
\end{enumerate}

\textbf{Avantages} :
\begin{itemize}
    \item ✅ Complexité linéaire : $O(H \cdot W)$ au lieu de $O((H \cdot W)^2)$
    \item ✅ Pyramidal features (multi-scale)
    \item ✅ État de l'art sur détection et segmentation
\end{itemize}

\subsection{CNN vs ViT}

\begin{table}[h]
\centering
\caption{Comparaison CNN vs ViT}
\label{tab:cnn-vs-vit}
\begin{tabular}{lcc}
\toprule
\textbf{Propriété} & \textbf{CNN} & \textbf{ViT} \\
\midrule
Inductive bias & Fort (localité, translation) & Faible \\
Données requises & Moins (ImageNet) & Plus (JFT-300M) \\
Précision (ImageNet) & 88.5\% (EfficientNet) & 90.4\% (ViT-Huge) \\
Complexité & $O(H \cdot W)$ & $O((H \cdot W)^2)$ \\
Interprétabilité & Filtres, feature maps & Attention maps \\
\bottomrule
\end{tabular}
\end{table}

\begin{astuce}
\textbf{Quand utiliser ViT ?}
\begin{itemize}
    \item Beaucoup de données disponibles (pré-entraînement sur large dataset)
    \item Besoin de capturer des dépendances globales (pas seulement locales)
    \item Ressources GPU importantes
\end{itemize}
\textbf{Quand utiliser CNN ?}
\begin{itemize}
    \item Dataset modeste (< 100k images)
    \item Contraintes computationnelles (edge devices)
    \item Tâches où la localité est importante
\end{itemize}
\end{astuce}

% ===== SECTION 5: VISION-LANGAGE (CLIP) =====
\section{Modèles Vision-Langage : CLIP}

\subsection{Motivation}

Les modèles de vision classiques apprennent à classifier des images en classes fixes. CLIP (Contrastive Language-Image Pre-training) apprend à associer images et textes libres.

\textbf{Applications} :
\begin{itemize}
    \item Zero-shot classification : classifier sans entraînement spécifique
    \item Image retrieval : chercher des images par description textuelle
    \item Vision-language reasoning
\end{itemize}

\subsection{Architecture CLIP}

\begin{definition}{CLIP}
CLIP entraîne conjointement un encodeur d'images et un encodeur de texte pour aligner leurs représentations dans un espace latent commun.
\end{definition}

\subsubsection{Composants}

\begin{enumerate}
    \item \textbf{Image Encoder} : ViT ou ResNet $\rightarrow$ vecteur $v_I \in \R^D$
    \item \textbf{Text Encoder} : Transformer $\rightarrow$ vecteur $v_T \in \R^D$
    \item \textbf{Contrastive Learning} : maximiser similarité cosinus pour paires correctes
\end{enumerate}

\subsubsection{Loss Contrastive}

Pour un batch de $N$ paires (image, texte) :

\begin{equation}
L = -\frac{1}{N} \sum_{i=1}^{N} \left[ \log \frac{\exp(v_I^i \cdot v_T^i / \tau)}{\sum_{j=1}^{N} \exp(v_I^i \cdot v_T^j / \tau)} + \log \frac{\exp(v_I^i \cdot v_T^i / \tau)}{\sum_{j=1}^{N} \exp(v_I^j \cdot v_T^i / \tau)} \right]
\end{equation}

où $\tau$ est une température apprise.

\textbf{Intuition} : Maximiser la similarité entre image $i$ et texte $i$, minimiser avec les autres.

\subsection{Zero-Shot Classification}

\begin{algorithm}[H]
\caption{CLIP Zero-Shot Classification}
\label{alg:clip-zeroshot}
\begin{algorithmic}[1]
\REQUIRE Image $I$, classes $\{c_1, \dots, c_K\}$
\ENSURE Classe prédite
\STATE Encoder l'image : $v_I = \text{ImageEncoder}(I)$
\FOR{chaque classe $c_k$}
    \STATE Créer prompt : $t_k = $ "A photo of a \{$c_k$\}"
    \STATE Encoder le texte : $v_T^k = \text{TextEncoder}(t_k)$
    \STATE Calculer similarité : $s_k = v_I \cdot v_T^k$
\ENDFOR
\STATE Softmax : $P(c_k) = \frac{\exp(s_k / \tau)}{\sum_j \exp(s_j / \tau)}$
\RETURN $\argmax_k P(c_k)$
\end{algorithmic}
\end{algorithm}

\textbf{Avantage} : Pas besoin de réentraîner pour de nouvelles classes, il suffit de changer les prompts !

\subsection{Résultats CLIP}

\begin{itemize}
    \item Entraîné sur 400M paires (image, texte) du web
    \item Zero-shot sur ImageNet : 76.2\% (comparable à ResNet-50 entraîné supervisé)
    \item Très robuste aux distribution shifts (ImageNet-A, ImageNet-R)
    \item Généralisation impressionnante à de nouveaux domaines
\end{itemize}

% ===== SECTION 6: IMPLÉMENTATION =====
\section{Implémentation}

\subsection{Détection d'Objets avec YOLOv8}

\begin{lstlisting}[language=Python, caption=YOLOv8 avec Ultralytics]
from ultralytics import YOLO
import cv2

# Charger modèle pré-entraîné
model = YOLO('yolov8n.pt')  # n, s, m, l, x

# Inférence sur une image
results = model('image.jpg')

# Afficher résultats
for r in results:
    boxes = r.boxes  # Bounding boxes
    for box in boxes:
        x1, y1, x2, y2 = box.xyxy[0]
        conf = box.conf[0]
        cls = int(box.cls[0])
        label = model.names[cls]
        print(f"{label} ({conf:.2f}): [{x1:.0f}, {y1:.0f}, {x2:.0f}, {y2:.0f}]")

# Entraîner sur custom dataset (COCO format)
model.train(
    data='custom_dataset.yaml',
    epochs=100,
    imgsz=640,
    batch=16,
    name='yolov8_custom'
)

# Évaluation
metrics = model.val()
print(f"mAP50: {metrics.box.map50}")
print(f"mAP50-95: {metrics.box.map}")
\end{lstlisting}

\subsection{Segmentation avec U-Net (PyTorch)}

\begin{lstlisting}[language=Python, caption=U-Net implémentation]
import torch
import torch.nn as nn

class UNet(nn.Module):
    def __init__(self, in_channels=3, out_channels=1):
        super().__init__()

        # Encoder
        self.enc1 = self.conv_block(in_channels, 64)
        self.enc2 = self.conv_block(64, 128)
        self.enc3 = self.conv_block(128, 256)
        self.enc4 = self.conv_block(256, 512)

        # Bottleneck
        self.bottleneck = self.conv_block(512, 1024)

        # Decoder
        self.upconv4 = nn.ConvTranspose2d(1024, 512, 2, stride=2)
        self.dec4 = self.conv_block(1024, 512)

        self.upconv3 = nn.ConvTranspose2d(512, 256, 2, stride=2)
        self.dec3 = self.conv_block(512, 256)

        self.upconv2 = nn.ConvTranspose2d(256, 128, 2, stride=2)
        self.dec2 = self.conv_block(256, 128)

        self.upconv1 = nn.ConvTranspose2d(128, 64, 2, stride=2)
        self.dec1 = self.conv_block(128, 64)

        # Output
        self.out = nn.Conv2d(64, out_channels, 1)

        self.pool = nn.MaxPool2d(2)

    def conv_block(self, in_ch, out_ch):
        return nn.Sequential(
            nn.Conv2d(in_ch, out_ch, 3, padding=1),
            nn.BatchNorm2d(out_ch),
            nn.ReLU(inplace=True),
            nn.Conv2d(out_ch, out_ch, 3, padding=1),
            nn.BatchNorm2d(out_ch),
            nn.ReLU(inplace=True)
        )

    def forward(self, x):
        # Encoder
        enc1 = self.enc1(x)
        enc2 = self.enc2(self.pool(enc1))
        enc3 = self.enc3(self.pool(enc2))
        enc4 = self.enc4(self.pool(enc3))

        # Bottleneck
        bottleneck = self.bottleneck(self.pool(enc4))

        # Decoder with skip connections
        dec4 = self.upconv4(bottleneck)
        dec4 = torch.cat([dec4, enc4], dim=1)
        dec4 = self.dec4(dec4)

        dec3 = self.upconv3(dec4)
        dec3 = torch.cat([dec3, enc3], dim=1)
        dec3 = self.dec3(dec3)

        dec2 = self.upconv2(dec3)
        dec2 = torch.cat([dec2, enc2], dim=1)
        dec2 = self.dec2(dec2)

        dec1 = self.upconv1(dec2)
        dec1 = torch.cat([dec1, enc1], dim=1)
        dec1 = self.dec1(dec1)

        return self.out(dec1)

# Entraînement
model = UNet(in_channels=3, out_channels=1).cuda()
criterion = nn.BCEWithLogitsLoss()
optimizer = torch.optim.Adam(model.parameters(), lr=1e-4)

for epoch in range(epochs):
    for images, masks in train_loader:
        images, masks = images.cuda(), masks.cuda()

        outputs = model(images)
        loss = criterion(outputs, masks)

        optimizer.zero_grad()
        loss.backward()
        optimizer.step()
\end{lstlisting}

\subsection{Vision Transformer avec timm}

\begin{lstlisting}[language=Python, caption=ViT avec timm]
import timm
import torch

# Charger modèle pré-entraîné
model = timm.create_model('vit_base_patch16_224', pretrained=True)

# Voir tous les modèles ViT disponibles
vit_models = timm.list_models('vit*')
print(f"Modèles ViT disponibles: {len(vit_models)}")

# Fine-tuning sur dataset custom
model = timm.create_model('vit_base_patch16_224',
                          pretrained=True,
                          num_classes=10)

# Geler l'encoder, entraîner seulement la tête
for param in model.parameters():
    param.requires_grad = False
for param in model.head.parameters():
    param.requires_grad = True

# Entraînement
optimizer = torch.optim.AdamW(model.head.parameters(), lr=1e-3)
criterion = torch.nn.CrossEntropyLoss()

model.train()
for images, labels in train_loader:
    outputs = model(images)
    loss = criterion(outputs, labels)

    optimizer.zero_grad()
    loss.backward()
    optimizer.step()

# Visualiser attention maps
attention_weights = model.blocks[-1].attn.get_attention_map()
\end{lstlisting}

\subsection{CLIP Zero-Shot}

\begin{lstlisting}[language=Python, caption=CLIP avec OpenAI]
import torch
import clip
from PIL import Image

# Charger modèle CLIP
device = "cuda" if torch.cuda.is_available() else "cpu"
model, preprocess = clip.load("ViT-B/32", device=device)

# Préparer image
image = preprocess(Image.open("cat.jpg")).unsqueeze(0).to(device)

# Définir classes possibles
text_prompts = [
    "a photo of a cat",
    "a photo of a dog",
    "a photo of a bird",
    "a photo of a car"
]
text = clip.tokenize(text_prompts).to(device)

# Inférence
with torch.no_grad():
    image_features = model.encode_image(image)
    text_features = model.encode_text(text)

    # Normaliser
    image_features /= image_features.norm(dim=-1, keepdim=True)
    text_features /= text_features.norm(dim=-1, keepdim=True)

    # Calculer similarités
    similarity = (100.0 * image_features @ text_features.T).softmax(dim=-1)
    values, indices = similarity[0].topk(4)

# Résultats
for value, index in zip(values, indices):
    print(f"{text_prompts[index]:20s}: {100 * value.item():.2f}%")
\end{lstlisting}

% ===== SECTION 7: AVANTAGES ET LIMITES =====
\section{Avantages et Limites}

\subsection{Object Detection}

\subsubsection{Avantages}
\begin{itemize}
    \item ✅ Localisation précise des objets dans l'image
    \item ✅ Performance temps réel avec YOLO (vidéo, applications embarquées)
    \item ✅ Datasets annotés disponibles (COCO, Pascal VOC)
    \item ✅ Transfert d'apprentissage efficace
\end{itemize}

\subsubsection{Limites}
\begin{itemize}
    \item ❌ Annotation coûteuse (bounding boxes pour chaque objet)
    \item ❌ Difficulté avec objets très petits ou occultés
    \item ❌ Trade-off précision vs vitesse
    \item ❌ Sensible aux variations d'échelle
\end{itemize}

\subsection{Segmentation}

\subsubsection{Avantages}
\begin{itemize}
    \item ✅ Précision au pixel près (vs boxes rectangulaires)
    \item ✅ U-Net très efficace en imagerie médicale
    \item ✅ Mask R-CNN : détection + segmentation en un modèle
    \item ✅ Interprétabilité : visualiser exactement les régions d'intérêt
\end{itemize}

\subsubsection{Limites}
\begin{itemize}
    \item ❌ Annotation pixel-level extrêmement coûteuse
    \item ❌ Plus lent que la détection (surtout segmentation d'instances)
    \item ❌ Difficulté avec les frontières ambiguës
    \item ❌ Mémoire GPU importante
\end{itemize}

\subsection{Vision Transformers}

\subsubsection{Avantages}
\begin{itemize}
    \item ✅ Capture des dépendances globales (pas seulement locales comme CNN)
    \item ✅ Meilleure précision avec beaucoup de données
    \item ✅ Attention maps interprétables
    \item ✅ Architecture unifiée pour vision et NLP
\end{itemize}

\subsubsection{Limites}
\begin{itemize}
    \item ❌ Requiert énormément de données (JFT-300M, ImageNet-21k)
    \item ❌ Complexité quadratique en la taille de l'image
    \item ❌ Moins bon que CNN avec peu de données
    \item ❌ Coût computationnel élevé
\end{itemize}

\subsection{CLIP}

\subsubsection{Avantages}
\begin{itemize}
    \item ✅ Zero-shot learning : pas besoin de réentraîner pour nouvelles classes
    \item ✅ Robuste aux distribution shifts
    \item ✅ Multimodal : image + texte
    \item ✅ Flexibilité : prompts textuels adaptables
\end{itemize}

\subsubsection{Limites}
\begin{itemize}
    \item ❌ Performances inférieures au fine-tuning sur tâches spécifiques
    \item ❌ Sensible à la formulation des prompts
    \item ❌ Biais des données web (400M paires non filtrées)
    \item ❌ Difficulté avec tâches nécessitant localisation précise
\end{itemize}

% ===== SECTION 8: HYPERPARAMÈTRES =====
\section{Hyperparamètres et Tuning}

\subsection{Détection d'Objets}

\begin{table}[h]
\centering
\caption{Hyperparamètres clés pour la détection}
\label{tab:hyperparams-detection}
\begin{tabular}{lll}
\toprule
\textbf{Paramètre} & \textbf{Valeurs typiques} & \textbf{Impact} \\
\midrule
learning\_rate & $10^{-4}$ à $10^{-2}$ & Convergence \\
image\_size & $416$, $640$, $1024$ & Précision vs vitesse \\
batch\_size & $8$ à $32$ & Stabilité gradient \\
IoU\_threshold & $0.5$ à $0.7$ & NMS agressivité \\
conf\_threshold & $0.25$ à $0.5$ & Rappel vs précision \\
anchor\_scales & Auto ou manuel & Détection multi-échelle \\
\bottomrule
\end{tabular}
\end{table}

\subsection{Segmentation}

\begin{table}[h]
\centering
\caption{Hyperparamètres clés pour la segmentation}
\label{tab:hyperparams-segmentation}
\begin{tabular}{lll}
\toprule
\textbf{Paramètre} & \textbf{Valeurs typiques} & \textbf{Impact} \\
\midrule
learning\_rate & $10^{-4}$ à $10^{-3}$ & Convergence \\
batch\_size & $2$ à $16$ & Mémoire GPU \\
image\_size & $256$, $512$, $1024$ & Détails vs vitesse \\
encoder\_depth & $4$ à $5$ & Complexité modèle \\
dropout & $0.1$ à $0.5$ & Régularisation \\
class\_weights & Auto ou manuel & Déséquilibre classes \\
\bottomrule
\end{tabular}
\end{table}

\subsection{Vision Transformers}

\begin{table}[h]
\centering
\caption{Hyperparamètres clés pour ViT}
\label{tab:hyperparams-vit}
\begin{tabular}{lll}
\toprule
\textbf{Paramètre} & \textbf{Valeurs typiques} & \textbf{Impact} \\
\midrule
patch\_size & $16$, $32$ & Nb tokens vs détails \\
num\_layers & $12$ à $32$ & Capacité modèle \\
hidden\_dim & $768$ à $1280$ & Capacité représentation \\
num\_heads & $12$ à $16$ & Attention multi-échelle \\
learning\_rate & $10^{-4}$ à $10^{-3}$ & Convergence \\
warmup\_steps & $500$ à $10000$ & Stabilité initiale \\
\bottomrule
\end{tabular}
\end{table}

\begin{astuce}
\textbf{Stratégies d'entraînement efficaces} :
\begin{itemize}
    \item \textbf{Transfer learning} : Toujours partir d'un modèle pré-entraîné (ImageNet, COCO)
    \item \textbf{Progressive resizing} : Commencer avec petites images, augmenter progressivement
    \item \textbf{Mixed precision} : FP16 pour accélérer et économiser mémoire
    \item \textbf{Data augmentation} : Mosaic, MixUp, CutMix pour la détection
    \item \textbf{Learning rate schedule} : Cosine annealing ou OneCycleLR
\end{itemize}
\end{astuce}

% ===== SECTION 9: APPLICATIONS =====
\section{Applications Pratiques}

\subsection{Conduite Autonome}

\begin{itemize}
    \item \textbf{Détection} : Voitures, piétons, vélos, panneaux de signalisation
    \item \textbf{Segmentation sémantique} : Route, trottoir, végétation, bâtiments
    \item \textbf{Segmentation d'instances} : Suivi de chaque voiture/piéton individuellement
    \item \textbf{Modèles} : YOLOv8, Mask R-CNN, DeepLab v3+
    \item \textbf{Datasets} : KITTI, Cityscapes, nuScenes
\end{itemize}

\subsection{Imagerie Médicale}

\begin{itemize}
    \item \textbf{Segmentation d'organes} : U-Net pour foie, reins, cerveau
    \item \textbf{Détection de tumeurs} : Faster R-CNN, RetinaNet
    \item \textbf{Segmentation cellulaire} : U-Net, Mask R-CNN
    \item \textbf{Modèles} : U-Net (référence), nnU-Net (auto-configuration)
    \item \textbf{Datasets} : Medical Segmentation Decathlon, LIDC-IDRI
\end{itemize}

\subsection{Reconnaissance Faciale}

\begin{itemize}
    \item \textbf{Détection de visages} : MTCNN, RetinaFace
    \item \textbf{Landmarks faciaux} : 68 points de repère (yeux, nez, bouche)
    \item \textbf{Segmentation} : Cheveux, peau, arrière-plan
    \item \textbf{Applications} : Sécurité, filtres AR, analyse d'émotions
\end{itemize}

\subsection{Commerce Électronique}

\begin{itemize}
    \item \textbf{Recherche visuelle} : CLIP pour "trouver des produits similaires"
    \item \textbf{Détection de produits} : Compter articles en rayon (retail)
    \item \textbf{Segmentation produits} : Extraction de produit pour montage
    \item \textbf{Modèles} : CLIP, YOLOv8, Mask R-CNN
\end{itemize}

\subsection{Surveillance et Sécurité}

\begin{itemize}
    \item \textbf{Détection d'intrusion} : YOLO temps réel
    \item \textbf{Suivi multi-objets} : DeepSORT avec détecteur
    \item \textbf{Détection d'anomalies} : Comportements inhabituels
    \item \textbf{Comptage de personnes} : Segmentation + tracking
\end{itemize}

% ===== SECTION 10: RÉSUMÉ =====
\section{Résumé du Chapitre}

\subsection{Points Clés}

\begin{itemize}
    \item \textbf{Object Detection} : Localiser et classifier des objets
    \begin{itemize}
        \item Two-stage : Faster R-CNN (précis, ~5 FPS)
        \item One-stage : YOLO (rapide, ~80 FPS)
        \item Métriques : IoU, mAP@0.5, mAP@0.5:0.95
    \end{itemize}

    \item \textbf{Segmentation Sémantique} : Classifier chaque pixel
    \begin{itemize}
        \item FCN : première architecture fully convolutional
        \item U-Net : architecture emblématique (médical)
        \item DeepLab : atrous convolutions + ASPP
        \item Métriques : mIoU, Dice coefficient
    \end{itemize}

    \item \textbf{Segmentation d'Instances} : Détecter et segmenter chaque instance
    \begin{itemize}
        \item Mask R-CNN = Faster R-CNN + branche de masque
        \item RoI Align pour précision pixel-level
    \end{itemize}

    \item \textbf{Vision Transformers} : Appliquer les Transformers à la vision
    \begin{itemize}
        \item ViT : découper en patches, self-attention
        \item Swin Transformer : attention locale + shifted windows
        \item Nécessite beaucoup de données
    \end{itemize}

    \item \textbf{Vision-Langage (CLIP)} : Aligner images et textes
    \begin{itemize}
        \item Contrastive learning sur 400M paires
        \item Zero-shot classification par prompts
        \item Applications : recherche, multimodal
    \end{itemize}
\end{itemize}

\subsection{Formules Essentielles}

\begin{tcolorbox}[colback=blue!5!white, colframe=blue!75!black, title=Formules à retenir]

\textbf{IoU (Intersection over Union)} :
\begin{equation*}
\text{IoU} = \frac{\text{Aire}(A \cap B)}{\text{Aire}(A \cup B)}
\end{equation*}

\textbf{Mean Average Precision} :
\begin{equation*}
\text{mAP} = \frac{1}{C} \sum_{c=1}^{C} AP_c
\end{equation*}

\textbf{Dice Coefficient} :
\begin{equation*}
\text{Dice} = \frac{2 \cdot TP}{2 \cdot TP + FP + FN}
\end{equation*}

\textbf{ViT Patch Embedding} :
\begin{equation*}
z_p = E \cdot x_p + E_{pos}, \quad x_p \in \R^{P^2 \cdot C}
\end{equation*}

\textbf{CLIP Contrastive Loss} :
\begin{equation*}
L = -\log \frac{\exp(v_I \cdot v_T / \tau)}{\sum_j \exp(v_I \cdot v_T^j / \tau)}
\end{equation*}

\end{tcolorbox}

\subsection{Tableau Récapitulatif}

\begin{table}[h]
\centering
\caption{Comparaison des approches de vision avancée}
\label{tab:recap}
\begin{tabular}{lccc}
\toprule
\textbf{Tâche} & \textbf{Modèle Recommandé} & \textbf{Précision} & \textbf{Vitesse} \\
\midrule
Détection (précision) & Faster R-CNN & ★★★ & ★☆☆ \\
Détection (temps réel) & YOLOv8 & ★★☆ & ★★★ \\
Seg. sémantique & DeepLab v3+ & ★★★ & ★★☆ \\
Seg. médicale & U-Net & ★★★ & ★★☆ \\
Seg. instances & Mask R-CNN & ★★★ & ★☆☆ \\
Classification & ViT-Large & ★★★ & ★☆☆ \\
Zero-shot & CLIP & ★★☆ & ★★★ \\
\bottomrule
\end{tabular}
\end{table}

% ===== SECTION 11: EXERCICES =====
\section{Exercices}

\subsection{Questions de Compréhension}

\begin{enumerate}
    \item Expliquer la différence entre R-CNN, Fast R-CNN et Faster R-CNN. Pourquoi chaque version est-elle plus rapide que la précédente ?

    \item Pourquoi YOLO est-il appelé "You Only Look Once" ? Quelle est sa principale différence philosophique avec Faster R-CNN ?

    \item Qu'est-ce que le RoI Pooling ? Pourquoi RoI Align est-il meilleur pour la segmentation ?

    \item Expliquer le rôle des skip connections dans U-Net. Pourquoi utiliser la concaténation plutôt que l'addition ?

    \item Comment fonctionne l'atrous convolution dans DeepLab ? Quel est son avantage ?

    \item Pourquoi ViT nécessite-t-il beaucoup plus de données d'entraînement que les CNN ?

    \item Comment CLIP permet-il la classification zero-shot ? Donner un exemple d'application.

    \item Quelle est la différence entre segmentation sémantique et segmentation d'instances ?
\end{enumerate}

\subsection{Exercices Pratiques}

\begin{enumerate}
    \item \textbf{Détection avec YOLOv8}
    \begin{itemize}
        \item Entraîner YOLOv8 sur un dataset custom (par ex. détecter des visages)
        \item Tester sur vidéo et mesurer le FPS
        \item Comparer YOLOv8n (nano) vs YOLOv8x (extra-large) : précision vs vitesse
        \item Notebook : \texttt{12\_demo\_object\_detection.ipynb}
    \end{itemize}

    \item \textbf{Segmentation médicale avec U-Net}
    \begin{itemize}
        \item Implémenter U-Net from scratch en PyTorch
        \item Entraîner sur un dataset de segmentation (par ex. cellules, tumeurs)
        \item Calculer Dice coefficient et mIoU
        \item Visualiser les prédictions
        \item Notebook : \texttt{12\_demo\_segmentation.ipynb}
    \end{itemize}

    \item \textbf{Vision Transformers}
    \begin{itemize}
        \item Fine-tuner ViT-Base sur CIFAR-10
        \item Comparer avec ResNet-50
        \item Visualiser les attention maps
        \item Tester DeiT avec distillation
        \item Notebook : \texttt{12\_demo\_vision\_transformers.ipynb}
    \end{itemize}

    \item \textbf{CLIP Zero-Shot}
    \begin{itemize}
        \item Utiliser CLIP pour classifier des images sans fine-tuning
        \item Tester différents prompts et mesurer l'impact
        \item Implémenter un moteur de recherche d'images par description textuelle
    \end{itemize}
\end{enumerate}

% ===== SECTION 12: POUR ALLER PLUS LOIN =====
\section{Pour Aller Plus Loin}

\subsection{Lectures Recommandées}

\subsubsection{Papers Fondateurs}

\begin{itemize}
    \item \textbf{R-CNN} : Girshick et al. (2014). "Rich feature hierarchies for accurate object detection and semantic segmentation"
    \item \textbf{Faster R-CNN} : Ren et al. (2015). "Faster R-CNN: Towards Real-Time Object Detection with Region Proposal Networks"
    \item \textbf{YOLO} : Redmon et al. (2016). "You Only Look Once: Unified, Real-Time Object Detection"
    \item \textbf{U-Net} : Ronneberger et al. (2015). "U-Net: Convolutional Networks for Biomedical Image Segmentation"
    \item \textbf{Mask R-CNN} : He et al. (2017). "Mask R-CNN"
    \item \textbf{ViT} : Dosovitskiy et al. (2020). "An Image is Worth 16x16 Words: Transformers for Image Recognition at Scale"
    \item \textbf{CLIP} : Radford et al. (2021). "Learning Transferable Visual Models From Natural Language Supervision"
\end{itemize}

\subsubsection{Papers Récents}

\begin{itemize}
    \item \textbf{YOLOv8} : Ultralytics (2023). Documentation officielle
    \item \textbf{Swin Transformer} : Liu et al. (2021). "Swin Transformer: Hierarchical Vision Transformer using Shifted Windows"
    \item \textbf{Segment Anything (SAM)} : Kirillov et al. (2023). "Segment Anything"
    \item \textbf{DINOv2} : Oquab et al. (2023). "DINOv2: Learning Robust Visual Features without Supervision"
\end{itemize}

\subsection{Ressources en Ligne}

\begin{itemize}
    \item \textbf{Ultralytics YOLOv8} : \url{https://docs.ultralytics.com/}
    \item \textbf{Detectron2} (Facebook) : \url{https://github.com/facebookresearch/detectron2}
    \item \textbf{MMDetection} : \url{https://github.com/open-mmlab/mmdetection}
    \item \textbf{Timm (PyTorch Image Models)} : \url{https://github.com/huggingface/pytorch-image-models}
    \item \textbf{OpenAI CLIP} : \url{https://github.com/openai/CLIP}
    \item \textbf{Hugging Face Transformers} : \url{https://huggingface.co/docs/transformers/}
\end{itemize}

\subsection{Datasets}

\begin{itemize}
    \item \textbf{COCO} (detection, segmentation) : \url{https://cocodataset.org/}
    \item \textbf{Pascal VOC} : \url{http://host.robots.ox.ac.uk/pascal/VOC/}
    \item \textbf{Cityscapes} (conduite) : \url{https://www.cityscapes-dataset.com/}
    \item \textbf{Medical Segmentation Decathlon} : \url{http://medicaldecathlon.com/}
    \item \textbf{Open Images} : \url{https://storage.googleapis.com/openimages/web/index.html}
\end{itemize}

\subsection{Outils et Bibliothèques}

\begin{itemize}
    \item \textbf{Ultralytics} : YOLO v5/v8 (PyTorch)
    \item \textbf{torchvision} : Modèles pré-entraînés (Faster R-CNN, Mask R-CNN)
    \item \textbf{segmentation\_models.pytorch} : U-Net, DeepLab, etc.
    \item \textbf{timm} : Vision Transformers
    \item \textbf{transformers} : CLIP, ViT (Hugging Face)
    \item \textbf{albumentations} : Augmentations avancées
\end{itemize}

\subsection{Prochaines Étapes}

\begin{itemize}
    \item \textbf{Chapitre 14 - GANs et Modèles Génératifs} : StyleGAN, Diffusion Models
    \item \textbf{Chapitre 15 - AutoML et NAS} : Architecture search automatique
    \item \textbf{Projets pratiques} : Appliquer ces techniques à vos propres données
\end{itemize}

% ===== BIBLIOGRAPHIE =====
\section*{Références}

\begin{enumerate}
    \item Girshick, R., Donahue, J., Darrell, T., \& Malik, J. (2014). Rich feature hierarchies for accurate object detection and semantic segmentation. \textit{CVPR}.

    \item Ren, S., He, K., Girshick, R., \& Sun, J. (2015). Faster R-CNN: Towards Real-Time Object Detection with Region Proposal Networks. \textit{NeurIPS}.

    \item Redmon, J., Divvala, S., Girshick, R., \& Farhadi, A. (2016). You Only Look Once: Unified, Real-Time Object Detection. \textit{CVPR}.

    \item Ronneberger, O., Fischer, P., \& Brox, T. (2015). U-Net: Convolutional Networks for Biomedical Image Segmentation. \textit{MICCAI}.

    \item He, K., Gkioxari, G., Dollár, P., \& Girshick, R. (2017). Mask R-CNN. \textit{ICCV}.

    \item Chen, L. C., Papandreou, G., Schroff, F., \& Adam, H. (2017). Rethinking Atrous Convolution for Semantic Image Segmentation. \textit{arXiv}.

    \item Dosovitskiy, A., et al. (2020). An Image is Worth 16x16 Words: Transformers for Image Recognition at Scale. \textit{ICLR 2021}.

    \item Liu, Z., et al. (2021). Swin Transformer: Hierarchical Vision Transformer using Shifted Windows. \textit{ICCV}.

    \item Radford, A., et al. (2021). Learning Transferable Visual Models From Natural Language Supervision. \textit{ICML}.

    \item Kirillov, A., et al. (2023). Segment Anything. \textit{ICCV}.
\end{enumerate}

\end{document}
