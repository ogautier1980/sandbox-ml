\documentclass[11pt,a4paper]{article}

\usepackage[utf8]{inputenc}
\usepackage[T1]{fontenc}
\usepackage[french]{babel}
\usepackage{lmodern,geometry,amsmath,hyperref,tcolorbox,enumitem,tikz}

\geometry{margin=2.5cm}
\usetikzlibrary{shapes,arrows,positioning}

\title{\textbf{Chapitre 5 : Communication Anonyme\\Mixnets, Onion Routing, Tor}}
\author{Cours de Cryptographie}
\date{\today}

\begin{document}
\maketitle
\tableofcontents
\newpage

%=============================================================================
\section{Introduction}
%=============================================================================

\subsection{Motivation}

Le chiffrement protège le \textbf{contenu} des messages mais pas les \textbf{métadonnées} :
\begin{itemize}
    \item Qui communique avec qui ?
    \item Quand ? À quelle fréquence ?
    \item Depuis où ? (adresses IP, localisation)
    \item Patterns de trafic (tailles, timings)
\end{itemize}

\textbf{Problème} : Les métadonnées révèlent énormément d'informations !

\textbf{Objectif} : Communiquer de manière \textbf{anonyme} (cacher l'identité) et/ou \textbf{non-traçable} (impossible de lier deux communications du même utilisateur).

\subsection{Applications}

\begin{itemize}
    \item Dissidents politiques, journalistes dans régimes autoritaires
    \item Whistleblowers (ex: Edward Snowden, WikiLeaks)
    \item Navigation web privée
    \item Contournement de censure
    \item Protection contre surveillance de masse
\end{itemize}

\subsection{Définitions}

\textbf{Anonymat de l'émetteur} : Cacher l'identité de celui qui envoie

\textbf{Anonymat du récepteur} : Cacher l'identité de celui qui reçoit

\textbf{Unlinkability} : Impossible de lier deux messages au même utilisateur

\textbf{Adversaire global} : Peut observer tout le réseau (NSA-level)

%=============================================================================
\section{Chaum's Mixnet (1981)}
%=============================================================================

\subsection{Principe}

\textbf{Idée} : Faire passer les messages par une cascade de serveurs (mixes) qui :
\begin{enumerate}
    \item Déchiffrent une couche de chiffrement
    \item Réordonnent les messages (batching)
    \item Transmettent au nœud suivant
\end{enumerate}

\subsection{Construction}

\textbf{Setup} : $n$ mixes $M_1, \ldots, M_n$ avec paires de clés $(pk_i, sk_i)$

\textbf{Alice envoie message $m$ à Bob} :
\begin{enumerate}
    \item Alice construit :
    \[
    c = \textsf{Enc}_{pk_1}(\textsf{Enc}_{pk_2}(\cdots \textsf{Enc}_{pk_n}(m, \text{Bob}) \cdots))
    \]
    \item Envoie $c$ à $M_1$
    \item $M_1$ déchiffre avec $sk_1$, obtient $c_2 = \textsf{Enc}_{pk_2}(\cdots)$, transmet à $M_2$
    \item $M_2$ déchiffre avec $sk_2$, obtient $c_3$, transmet à $M_3$
    \item \ldots
    \item $M_n$ déchiffre, obtient $(m, \text{Bob})$, délivre à Bob
\end{enumerate}

\subsection{Sécurité}

\textbf{Théorème (informel)} : Si \textbf{au moins un} mix est honnête, l'anonymat est garanti (adversaire ne peut pas lier Alice $\to$ Bob).

\textbf{Limitations} :
\begin{itemize}
    \item Nécessite batching $\Rightarrow$ latence élevée
    \item Vulnérable aux attaques de trafic (si adversaire observe entrées/sorties d'un mix)
    \item Nécessite serveurs de confiance
\end{itemize}

%=============================================================================
\section{Onion Routing}
%=============================================================================

\subsection{Principe}

\textbf{Différence avec mixnet} : Pas de batching, communication en temps réel

\textbf{Métaphore} : Pelure d'oignon - chaque nœud enlève une couche de chiffrement

\subsection{Construction}

\textbf{Circuit} : Alice choisit un chemin $R_1 \to R_2 \to R_3 \to \text{Destination}$

\textbf{Encryption layers} :
\begin{itemize}
    \item Layer 3 (extérieur) : Chiffré pour $R_1$
    \item Layer 2 : Chiffré pour $R_2$
    \item Layer 1 (intérieur) : Chiffré pour $R_3$
\end{itemize}

\textbf{Traversée} :
\begin{itemize}
    \item $R_1$ déchiffre layer 3, voit "transmettre à $R_2$", envoie
    \item $R_2$ déchiffre layer 2, voit "transmettre à $R_3$", envoie
    \item $R_3$ déchiffre layer 1, voit destination finale, envoie
\end{itemize}

\textbf{Propriété} : Chaque nœud connaît seulement le précédent et le suivant (pas la source ni la destination complète)

%=============================================================================
\section{Tor (The Onion Router)}
%=============================================================================

\subsection{Architecture}

\textbf{Composants} :
\begin{itemize}
    \item \textbf{Relays} : $\sim$7000 nœuds volontaires
    \item \textbf{Entry guards} : Premier nœud du circuit
    \item \textbf{Exit nodes} : Dernier nœud (celui qui sort vers Internet public)
    \item \textbf{Directory servers} : Maintiennent la liste des relays
    \item \textbf{Hidden services} (.onion) : Sites accessibles uniquement via Tor
\end{itemize}

\subsection{Fonctionnement}

\textbf{Établissement de circuit} :
\begin{enumerate}
    \item Client choisit 3 relays aléatoires : Guard $\to$ Middle $\to$ Exit
    \item Négocie clés symétriques avec chaque relay (Diffie-Hellman)
    \item Construit circuit avec 3 couches de chiffrement
\end{enumerate}

\textbf{Transmission} :
\begin{itemize}
    \item Données chiffrées 3 fois (onion)
    \item Chaque relay enlève une couche
    \item Exit node envoie en clair vers destination (peut voir contenu si pas HTTPS !)
\end{itemize}

\subsection{Sécurité et limitations}

\textbf{Menaces atténuées} :
\begin{itemize}
    \item Surveillance locale (FAI)
    \item Géolocalisation
    \item Censure par blocage IP
\end{itemize}

\textbf{Limitations} :
\begin{itemize}
    \item \textbf{Exit node eavesdropping} : Exit voit trafic non-HTTPS
    \item \textbf{Traffic correlation} : Adversaire global peut corréler entrées/sorties
    \item \textbf{Timing attacks} : Patterns temporels peuvent révéler liens
    \item \textbf{Compromission de guards} : Si guard malveillant, peut tracer utilisateur
    \item \textbf{Performance} : 3-5x plus lent que connexion directe
\end{itemize}

\subsection{Hidden Services (.onion)}

\textbf{Principe} : Serveur accessible uniquement via Tor, sans révéler son IP

\textbf{Fonctionnement} :
\begin{enumerate}
    \item Serveur choisit introduction points
    \item Publie descripteur avec clé publique dans DHT
    \item Client récupère descripteur
    \item Client et serveur se rencontrent au rendezvous point
    \item Communication via circuit Tor double (6 sauts !)
\end{enumerate}

\textbf{Exemples} : Facebook .onion, ProtonMail .onion, sites de whistleblowing

%=============================================================================
\section{Attaques sur Tor}
%=============================================================================

\subsection{Traffic Analysis}

\textbf{Attack} : Adversaire observe à la fois l'entrée (utilisateur $\to$ Guard) et la sortie (Exit $\to$ Destination)

\textbf{Corrélation} : Patterns temporels, tailles de paquets $\Rightarrow$ lien probable

\textbf{Défense} : Padding, dummy traffic (coût en bande passante)

\subsection{Website Fingerprinting}

\textbf{Attack} : Classifier le site visité à partir du pattern de trafic chiffré

\textbf{Résultats} : Accuracy > 90\% pour top-100 sites (recherche académique)

\textbf{Défense} : WTF-PAD (adaptive padding), mais overhead important

\subsection{Sybil Attacks}

\textbf{Attack} : Adversaire déploie beaucoup de relays malveillants

\textbf{Impact} : Augmente probabilité que circuit contienne nœud malveillant

\textbf{Défense} : Directory authorities vérifient relays, limitent influence nouveaux nœuds

%=============================================================================
\section{Alternatives et extensions}
%=============================================================================

\subsection{I2P (Invisible Internet Project)}

\textbf{Différence avec Tor} :
\begin{itemize}
    \item Réseau overlay complètement séparé (pas d'accès à Internet public par défaut)
    \item Circuits bidirectionnels (vs unidirectionnels Tor)
    \item Optimisé pour hidden services
\end{itemize}

\subsection{Mixminion (email anonyme)}

\textbf{Principe} : Mixnet pour emails avec réponses anonymes

\subsection{Private Information Retrieval (PIR)}

\textbf{Problème} : Récupérer un élément d'une base de données sans révéler lequel

\textbf{Solutions} :
\begin{itemize}
    \item PIR computationnel (chiffrement homomorphe)
    \item PIR information-théorique (multiple serveurs non-colluding)
\end{itemize}

%=============================================================================
\section{Notebooks pratiques}
%=============================================================================

\begin{itemize}
    \item \texttt{05_demo_onion_routing.ipynb} : Simulation onion routing simplifié
    \item \texttt{05_demo_mixnet.ipynb} : Implémentation mixnet avec batching
    \item \texttt{05_exercices.ipynb} : Exercices sur anonymat et traffic analysis
\end{itemize}

%=============================================================================
\section{Considérations éthiques et légales}
%=============================================================================

\textbf{Usage légitime} :
\begin{itemize}
    \item Protection de dissidents, journalistes
    \item Contournement de censure
    \item Vie privée numérique
\end{itemize}

\textbf{Usage illégal} :
\begin{itemize}
    \item Darknet markets (drogue, armes)
    \item Distribution de contenu illégal
    \item Cybercriminalité
\end{itemize}

\textbf{Position légale} :
\begin{itemize}
    \item Tor est légal dans la plupart des pays (USA, Europe)
    \item Utilisé par militaires US, journalistes, ONG
    \item Financé en partie par US Government (Bureau of Democracy, Human Rights and Labor)
    \item Mais interdit/bloqué dans certains pays (Chine, Iran)
\end{itemize}

%=============================================================================
\section{Conclusion}
%=============================================================================

\textbf{Points clés} :
\begin{itemize}
    \item Anonymat $\neq$ chiffrement (protège métadonnées, pas seulement contenu)
    \item Mixnets : Batching, haute latence, forte anonymat
    \item Onion Routing / Tor : Temps réel, latence acceptable, anonymat pratique
    \item Adversaire global reste une menace (traffic correlation)
    \item Trade-off fondamental : Anonymat vs Performance vs Usabilité
\end{itemize}

\textbf{Recherche active} :
\begin{itemize}
    \item Résistance aux adversaires globaux
    \item Anonymat post-quantique
    \item Systèmes à faible latence avec anonymat fort
    \item Blockchain et anonymat (Zcash, Monero)
\end{itemize}

\vfill

\begin{center}
\textit{``Privacy is necessary for an open society in the electronic age.''}\\
--- A Cypherpunk's Manifesto, Eric Hughes (1993)
\end{center}

\end{document}
